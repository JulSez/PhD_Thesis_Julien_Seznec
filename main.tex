\documentclass[12pt,fleqn]{book} % Default font size and left-justified equations

\input{structure}
\usepackage{amsthm}
\usepackage{amsmath}
\usepackage{amssymb}
\usepackage{bbm}
\usepackage{thm-restate}
\usepackage{nicefrac}       % compact symbols for 1/2, etc.x
\usepackage[htt]{hyphenat}
\AfterEndEnvironment{restatable}{\noindent\ignorespaces}
\usepackage{bm}
\usepackage{dsfont}
% !TEX root = main.tex

\DeclareMathOperator*{\argmax}{arg\,max}
\DeclareMathOperator*{\argmin}{arg\,min}
\DeclareMathOperator*{\arginf}{arg\,inf}
\DeclareMathOperator*{\sgn}{sgn}
%\DeclareMathOperator{\regret}{regret}
\DeclareMathOperator{\polylog}{polylog}
\DeclareMathOperator{\logloglog}{logloglog}
\DeclareMathOperator{\polyloglog}{polyloglog}


\newcommand*\diff{\mathop{}\!\mathrm{d}}
\newcommand*\Diff[1]{\mathop{}\!\mathrm{d^#1}}
\renewcommand{\d}[1]{\ensuremath{\operatorname{d}\!{#1}}}

\newcommand{\set}[1]{\left\{#1\right\}}
%\DeclarePairedDelimiter\ceil{\lceil}{\rceil}
%\DeclarePairedDelimiter\floor{\lfloor}{\rfloor}
\newcommand{\ceil}[1]{\left\lceil#1\right\rceil}
\newcommand{\floor}[1]{\left\lfloor#1\right\rfloor}

%\newcommand{\ceil}[1]{\lceil#1\rceil}

\newcommand{\II}[1]{\mathds{1}_{\left\{#1\right\}}}
\newcommand{\I}{{\mathds{1}}}


%arrows 
\newcommand{\ra}{\rightarrow}

%distributions 
\newcommand{\Bernoulli}{\mathrm{Bernoulli}}


\newcommand{\specialcell}[2][c]{%
 \begin{tabular}[#1]{@{}c@{}}#2\end{tabular}}

\newtheorem{assumption}{Assumption}
%\newtheorem{lemma}{Lemma}
%\newtheorem{theorem}{Theorem}
%\newtheorem{definition}{Definition}
%\newtheorem{corollary}{Corollary}
%\newtheorem{remark}{Remark}


% \newcommand{\R}{I\!\! R}
\newcommand{\R}{\mathbb{R}}
\newcommand{\realset}{\mathbb{R}}

\newcommand{\NN}{{\mathbb N}}
\newcommand{\1}{\mathds{1}}
\newcommand{\bOne}{{\bf 1}}
\newcommand{\bZero}{{\bf 0}}
\newcommand{\E}{\mathbb{E}}
\newcommand{\EE}[1]{\mathbb{E}\left[#1\right]}
\newcommand{\EEt}[1]{\mathbb{E}_t\left[#1\right]}
\newcommand{\EEs}[2]{\mathbb{E}_{#1}\left[#2\right]}
\newcommand{\EEc}[2]{\mathbb{E}\left[#1\left|#2\right.\right]}
\newcommand{\EEcc}[2]{\mathbb{E}\left[\left.#1\right|#2\right]}
\newcommand{\EEcct}[2]{\mathbb{E}_t\left[\left.#1\right|#2\right]}
\newcommand{\PP}[1]{\mathbb{P}\left[#1\right]}
\newcommand{\PPt}[1]{\mathbb{P}_t\left[#1\right]}
 \newcommand{\PPc}[2]{\mathbb{P}\left[#1\left|#2\right.\right]}
\newcommand{\PPcc}[2]{\mathbb{P}\left[\left.#1\right|#2\right]}
\newcommand{\PPct}[2]{\mathbb{P}_t\left[#1\left|#2\right.\right]}
\newcommand{\PPcct}[2]{\mathbb{P}_t\left[\left.#1\right|#2\right]}
\newcommand{\EEempty}{\mathbb{E}}
\newcommand{\PPempty}{\mathbb{P}}
%parens
\newcommand{\pa}[1]{\left(#1\right)}
\newcommand{\sqpa}[1]{\left[#1\right]}
\newcommand{\ac}[1]{\left\{#1\right\}}
\newcommand{\ev}[1]{\left\{#1\right\}}
\newcommand{\card}[1]{\left|#1\right|}


\newcommand{\normtwo}[1]{\|#1\|_2}
\newcommand{\norm}[1]{\left\|#1\right\|}
\newcommand{\onenorm}[1]{\norm{#1}_1}
\newcommand{\infnorm}[1]{\norm{#1}_\infty}

\newcommand{\abs}[1]{\left|#1\right|}

\newcommand*{\MyDef}{\mathrm{\tiny def}}
\newcommand*{\eqdefU}{\ensuremath{\mathop{\overset{\MyDef}{=}}}}% Unscaled version
\newcommand*{\eqdef}{\mathop{\overset{\MyDef}{\resizebox{\widthof{\eqdefU}}{\heightof{=}}{=}}}}
%\newcommand{\eqdef}{\stackrel{{\rm def}}{=}}
%\newcommand{\eqdef}{\stackrel{{\rm \tiny def}}{=}}
\newcommand{\transpose}{^\mathsf{\scriptscriptstyle T}}

%Calligraphic Shorthands
\newcommand{\cA}{\mathcal{A}}
\newcommand{\cB}{\mathcal{B}}
\newcommand{\cC}{\mathcal{C}}
\newcommand{\cD}{\mathcal{D}}
\newcommand{\cE}{\mathcal{E}}
\newcommand{\F}{\mathcal{F}}
\newcommand{\cF}{\mathcal{F}}
\newcommand{\cG}{\mathcal{G}}
\newcommand{\cH}{\mathcal{H}}
\newcommand{\cI}{\mathcal{I}}
\newcommand{\cJ}{\mathcal{J}}
\newcommand{\cK}{\mathcal{K}}
\newcommand{\cL}{\mathcal{L}}
\newcommand{\calL}{\cL}
\newcommand{\cM}{\mathcal{M}}
\newcommand{\cN}{\mathcal{N}}
\newcommand{\cO}{\mathcal{O}}
\newcommand{\tcO}{\widetilde{\cO}}
\newcommand{\OO}{\mathcal{O}}
\newcommand{\tOO}{\wt{\OO}}
\newcommand{\cP}{\mathcal{P}}
\newcommand{\cQ}{\mathcal{Q}}
\newcommand{\cR}{\mathcal{R}}
\newcommand{\Sw}{\mathcal{S}}
\newcommand{\cS}{\mathcal{S}}
\newcommand{\cT}{\mathcal{T}}
\newcommand{\T}{\cT}
\newcommand{\cU}{\mathcal{U}}
\newcommand{\cV}{\mathcal{V}}
\newcommand{\cW}{\mathcal{W}}
\newcommand{\cX}{\mathcal{X}}
\newcommand{\X}{\cX}
\newcommand{\cY}{\mathcal{Y}}
\newcommand{\cZ}{\mathcal{Z}}

%Bolds Shorthands
\newcommand{\bA}{{\bf A}}
\newcommand{\bb}{{\bf b}}
\newcommand{\bB}{{\bf B}}
\newcommand{\bc}{{\bf c}}
\newcommand{\bC}{{\bf C}}
\newcommand{\bD}{{\bf D}}
\newcommand{\bg}{{\bf g}}
\newcommand{\bG}{{\bf G}}
\newcommand{\bI}{{\bf I}}
\newcommand{\bM}{{\bf M}}
\newcommand{\bO}{\boldsymbol{O}}
\newcommand{\bp}{\boldsymbol{p}}
\newcommand{\bP}{{\bf P}}
\newcommand{\br}{{\bf r}}
\newcommand{\bR}{{\bf R}}
\newcommand{\bQ}{{\bf Q}}
\newcommand{\be}{{\bf e}}
\newcommand{\bff}{{\bf f}}
\newcommand{\bi}{{\bf i}}
\newcommand{\bk}{{\bf k}}
\newcommand{\bK}{{\bf K}}
\newcommand{\bL}{{\bf L}}
\newcommand{\bs}{{\bf s}}
\newcommand{\bq}{{\bf q}}
\newcommand{\bu}{{\bf u}}
\newcommand{\bU}{{\bf U}}
\newcommand{\bv}{{\bf v}}
\newcommand{\bV}{{\bf V}}
\newcommand{\bw}{{\bf w}}
\newcommand{\bW}{{\bf W}}
\newcommand{\by}{{\bf y}}
\newcommand{\bx}{{\bf x}}
\newcommand{\bX}{{\bf X}}
\newcommand{\bZ}{{\bf Z}}

\newcommand{\eps}{\varepsilon}
\renewcommand{\epsilon}{\varepsilon}
\renewcommand{\hat}{\widehat}
\renewcommand{\tilde}{\widetilde}
\renewcommand{\bar}{\overline}

\newcommand{\balpha}{{\boldsymbol \alpha}}
\newcommand{\talpha}{\widetilde{\indn}}
\newcommand{\btheta}{{\boldsymbol \theta}}
\newcommand{\tTheta}{{\widetilde\Theta}}
\newcommand{\bdelta}{{\boldsymbol \delta}}
\newcommand{\bDelta}{{\boldsymbol \Delta}}
\newcommand{\bLambda}{{\boldsymbol \Lambda}}
\newcommand{\bSigma}{{\boldsymbol \Sigma}}
\newcommand{\Bmu}{{\boldsymbol \mu}}
\newcommand{\bxi}{{\boldsymbol \xi}}
\newcommand{\bell}{\boldsymbol \ell}

\newcommand{\nothere}[1]{}
\newcommand{\moveb}{\\ \bigskip}

% bandits
\newcommand{\hloss}{\hat\ell}
\newcommand{\bloss}{\boldsymbol  \ell}
\newcommand{\hbl}{\hat{\bloss}}
\newcommand{\hbL}{\wh{\bL}}
\newcommand{\wh}{\widehat}
\newcommand{\ti}{_{t,i}}
\newcommand{\wt}{\widetilde}



%% from single papers, but merged here
% algo names
\usepackage{xspace}
\renewcommand{\ttdefault}{lmtt}
%other algos
\newcommand{\LP}{\texttt{LP}\xspace}
\newcommand{\CMG}{\texttt{CMG}\xspace}
%bandits
\newcommand{\FPL}{\texttt{FPL}\xspace}
\newcommand{\TS}{\texttt{TS}\xspace}
\newcommand{\UCB}{\texttt{UCB}\xspace}
\newcommand{\MOSS}{\texttt{MOSS}\xspace}
\newcommand{\UCBE}{\texttt{UCB-E}\xspace}
\newcommand{\ImprovedUCB}{\texttt{ImprovedUCB}\xspace}
\newcommand{\klucb}{\texttt{KL-UCB}\xspace}
\newcommand{\CUCB}{\texttt{CUCB}\xspace}
\newcommand{\EXP}{\texttt{Exp3}\xspace}
\newcommand{\exph}{\EXP}
%linear bandits
\newcommand{\LinearTS}{\texttt{LinearTS}\xspace}
\newcommand{\ThompsonSampling}{\texttt{ThompsonSampling}\xspace}
\newcommand{\SpectralEliminator}{\texttt{\textcolor[rgb]{0.5,0.2,0}{SpectralEliminator}}\xspace}
\newcommand{\LinearEliminator}{\texttt{\textcolor[rgb]{0.5,0.2,0}{LinearEliminator}}\xspace}
\newcommand{\LinUCB}{\texttt{LinUCB}\xspace}
\newcommand{\LinRel}{\texttt{LinRel}\xspace}
\newcommand{\KernelUCB}{\texttt{\textcolor[rgb]{0.5,0.2,0}{KernelUCB}}\xspace}
\newcommand{\SupKernelUCB}{\texttt{\textcolor[rgb]{0.5,0.2,0}{SupKernelUCB}}\xspace}
\newcommand{\GPUCB}{\texttt{GP-UCB}\xspace}
\newcommand{\OFUL}{\texttt{OFUL}\xspace}
\newcommand{\OPM}{\texttt{\textcolor[rgb]{0.5,0.2,0}{OPM}}\xspace}
%graph bandits
\newcommand{\CLUB}{\texttt{CLUB}\xspace}
\newcommand{\GOBLin}{\texttt{GOB.Lin}\xspace}
\newcommand{\UCBN}{\texttt{UCB-N}\xspace}
\newcommand{\UCBmaxN}{\texttt{UCB-MaxN}\xspace}
\newcommand{\GraphMOSS}{\texttt{\textcolor[rgb]{0.5,0.2,0}{GraphMOSS}}\xspace}
\newcommand{\SpectralUCB}{\texttt{\textcolor[rgb]{0.5,0.2,0}{SpectralUCB}}\xspace}
\newcommand{\CheapUCB}{\texttt{\textcolor[rgb]{0.5,0.2,0}{CheapUCB}}\xspace}
\newcommand{\SpectralTS}{\texttt{\textcolor[rgb]{0.5,0.2,0}{SpectralTS}}\xspace}
\newcommand{\SupLinRel}{\texttt{SupLinRel}\xspace}
\newcommand{\SupLinUCB}{\texttt{SupLinUCB}\xspace}
\newcommand{\imb}{\texttt{\textcolor[rgb]{0.5,0.2,0}{IMLinUCB}}\xspace}
\newcommand{\NetBandits}{\texttt{NetBandits}\xspace}
\newcommand{\BARE}{\texttt{\textcolor[rgb]{0.5,0.2,0}{BARE}}\xspace}
\newcommand{\ELP}{\texttt{ELP}\xspace}
\newcommand{\ELPP}{\texttt{ELP.P}\xspace}
\newcommand{\expix}{\texttt{\textcolor[rgb]{0.5,0.2,0}{Exp3-IX}}\xspace}
\newcommand{\expset}{\texttt{Exp3-SET}\xspace}
\newcommand{\expdom}{\texttt{Exp3-DOM}\xspace}
\newcommand{\expg}{\texttt{Exp3.G}\xspace}
\newcommand{\fplbgr}{\texttt{FPL-BGR}\xspace}
\newcommand{\fplix}{\texttt{\textcolor[rgb]{0.5,0.2,0}{FPL-IX}}\xspace}
\newcommand{\comphedge}{\texttt{Component\-Hedge}\xspace}
\newcommand{\hedge}{\texttt{Hedge}\xspace}
\newcommand{\expxxx}{\texttt{\textcolor[rgb]{0.5,0.2,0}{Exp3-WIX}}\xspace}
\newcommand{\expwix}{\texttt{\textcolor[rgb]{0.5,0.2,0}{Exp3-WIX}}\xspace}
\newcommand{\expixa}{\texttt{Exp3-IXa}\xspace}
\newcommand{\expixb}{\texttt{Exp3-IXb}\xspace}
\newcommand{\expixt}{\texttt{Exp3-IXt}\xspace}
\newcommand{\expcoop}{\texttt{Exp3-Coop}\xspace}
\newcommand{\expres}{\texttt{\textcolor[rgb]{0.5,0.2,0}{Exp3-Res}}\xspace}

%continuous bandits
\newcommand{\StoSOO}{\texttt{\textcolor[rgb]{0.5,0.2,0}{StoSOO}}\xspace}
\newcommand{\POO}{\texttt{\textcolor[rgb]{0.5,0.2,0}{POO}}\xspace}
\newcommand{\DOO}{\texttt{DOO}\xspace}
\newcommand{\SOO}{\texttt{SOO}\xspace}
\newcommand{\Zooming}{\texttt{Zooming}\xspace}
\newcommand{\UCT}{\texttt{UCT}\xspace}
\newcommand{\HCT}{\texttt{HCT}\xspace}
\newcommand{\SHOO}{\POO}
\newcommand{\HOO}{\texttt{HOO}\xspace}
\newcommand{\ATB}{\texttt{ATB}\xspace}
\newcommand{\TZ}{\texttt{TaxonomyZoom}\xspace}
\newcommand{\Direct}{\texttt{DiRect}\xspace}
%extreme bandits
\newcommand{\SiRI}{\texttt{\textcolor[rgb]{0.5,0.2,0}{SiRI}}\xspace}
%planning "bandits"
\newcommand{\olop}{\texttt{OLOP}\xspace}
\newcommand{\stopalgo}{\texttt{StOP}\xspace}
\newcommand{\metagrill}{\texttt{\textcolor[rgb]{0.5,0.2,0}{TrailBlazer}}\xspace}
%polymatroid bandits
\newcommand{\greedy}{\texttt{Greedy}\xspace}
\newcommand{\opm}{\texttt{\textcolor[rgb]{0.5,0.2,0}{OPM}}\xspace}


\newcommand{\maxn}{\texttt{max}\xspace}
\newcommand{\avgn}{\texttt{avg}\xspace}


%kernelUCB
\newcommand{\reg}{\gamma}
\newcommand{\hmu}{\hat{\mu}}
\newcommand{\bmu}{\bar{\mu}}
%\newcommand{\hm}{\hat{\mu}}
\newcommand{\hw}{\hat{w}}
\newcommand{\hth}{\hat{\theta}}
\newcommand{\hs}{\hat{\sigma}}
\newcommand{\hepsilon}{\hat{\epsilon}}


%graph bandit notation
\newcommand{\etat}{\eta_t}
\newcommand{\gammat}{\gamma_t}
\newcommand{\nodes}{{\textcolor[rgb]{0.3,0.8,0.0}{N}}}
\newcommand{\rounds}{{\textcolor[rgb]{0.3,0.0,0.8}{T}}}
\newcommand{\td}{{\textcolor[rgb]{0.6,0.0,0.6}{\tilde{d}}}}
\newcommand{\matL}{{\textcolor[rgb]{0.6,0.0,0.6}{L}}}
\newcommand{\matK}{{\textcolor[rgb]{0.6,0.0,0.6}{K}}}
\newcommand{\effd}{{\textcolor[rgb]{0.6,0.0,0.6}{d}}}
\newcommand{\effD}{{\textcolor[rgb]{0.6,0.0,0.6}{D}}}
\newcommand{\indn}{{\textcolor[rgb]{0.6,0.0,0.6}{\alpha}}}
\newcommand{\indnstar}{{\textcolor[rgb]{0.6,0.0,0.6}{\alpha^\star}}}
\newcommand{\cliquen}{{\textcolor[rgb]{0.6,0.0,0.6}{\chi}}}
\newcommand{\erdosr}{{\textcolor[rgb]{0.6,0.0,0.6}{r}}}
\newcommand{\detD}{{\textcolor[rgb]{0.6,0.0,0.6}{D}}}
\newcommand{\detDstar}{{\textcolor[rgb]{0.6,0.0,0.6}{D_\star}}}
\newcommand{\infibeta}{{\textcolor[rgb]{0.6,0.0,0.6}{\beta}}}
\newcommand{\mas}{{\textcolor[rgb]{0.6,0.0,0.6}{\texttt{mas}}}}
\newcommand{\nodeset}{\cV}
\newcommand{\edgeset}{\cE}
\newcommand{\regret}{R_\rounds}
\newcommand{\cgamma}{c_\gamma}
\newcommand{\cgammat}{c_{\gamma_t}}
\newcommand{\sumT}{\sum_{t = 1}^\rounds}
\newcommand{\sumt}{\sum_{t=1}^\rounds}
\newcommand{\sumtl}{\sum\limits_{t=1}^\rounds}
\newcommand{\sumj}{\sum_{j\in \nodes_i^-}}
\newcommand{\sumtj}{\sum_{j\in \nodes_{t,i}^-}}
\newcommand{\sumi}{\sum_{i=1}^{\nodes}}
\newcommand{\sumji}{\sum_{j\in \{\nodes_i^-\cup\{i\}\}}}
\newcommand{\sumtji}{\sum_{j\in \{\nodes_{t,i}^-\cup\{i\}\}}}
\newcommand{\dti}{d_{t,i}^-}
\newcommand{\hdi}{\hat{d}_i^-}
\newcommand{\hdk}{\hat{d}_k^-}
\newcommand{\hd}{\hat{d}^-}
\newcommand{\tti}{_{t+1,i}}
\newcommand{\tj}{_{t,j}}
\newcommand{\ji}{_{j,i}}
\newcommand{\Ii}{_{I_t,i}}
\newcommand{\pti}{p\ti}
\newcommand{\pta}{p_{t,a}}
\newcommand{\qti}{q\ti}
\newcommand{\hpti}{\hat{p}\ti}
\newcommand{\hpi}{\hat{p}_i}
\newcommand{\hp}{\hat{p}}
\newcommand{\hqti}{\hat{q}\ti}
\newcommand{\ptj}{p_{t,j}}
\newcommand{\qtj}{q_{t,j}}
\newcommand{\hptj}{\hat{p}_{t,j}}
\newcommand{\hqtj}{\hat{q}_{t,j}}
\newcommand{\oti}{o\ti}
\newcommand{\Oti}{O\ti}
\newcommand{\loss}{\ell}
\newcommand{\hLoss}{\hat{L}}
\newcommand{\hL}{\wh{L}}
\newcommand{\noise}{\epsilon}
%spectral
\newcommand{\dold}{\effd_{\scriptsize\mbox{old}}}
\newcommand{\dnew}{\effd_{\scriptsize\mbox{new}}}
%wix
\newcommand{\gweight}{s}
\newcommand{\avgalpha}{\indnstar_{\text{avg}}}
%bare
\newcommand{\rkdual}{r_k^{\circ}}
\newcommand{\rktdual}{r_{k,t}^{\circ}}
\newcommand{\rkprimetdual}{r_{k',t}^{\circ}}
\newcommand{\rkttdual}{r_{k,t+1}^{\circ}}
\newcommand{\rkprimettdual}{r_{k',t+1}^{\circ}}
\newcommand{\ridual}{r_i^{\circ}}
\newcommand{\rstardual}{r_\star^{\circ}}
\newcommand{\Ddual}{D^{\circ}}
\newcommand{\Ddualset}{\mathcal D^{\circ}}
%continuous bandit notation
%\newcommand{\node}[2]{\circ[#1,#2]}
\newcommand{\node}[2]{(#1,#2)}
\newcommand{\CommaBin}{\mathbin{\raisebox{0.5ex}{,}}}


%%% todos
\usepackage[colorinlistoftodos, textwidth=30mm, shadow, textsize=small]{todonotes}
\definecolor{babyblue}{rgb}{0.54, 0.81, 0.94}
\definecolor{citrine}{rgb}{0.89, 0.82, 0.04}
\definecolor{misocolor}{rgb}{0.16,0.27,0.86}
\newcommand{\todom}[1]{\todo[color=misocolor!30]{#1}\xspace}
\newcommand{\todomi}[1]{\todo[inline,color=misocolor!30]{#1}}
\newcommand{\wb}[1]{\overline{#1}}



\newcommand{\ind}{\mathbb{I}}
\newcommand{\narms}{K}
\newcommand{\currentTime}{t}
\newcommand{\lipschitz}{L}
\newcommand{\timeEnd}{T}
\newcommand{\currentNumPull}{N_{\pi(t)}(t)}
\newcommand{\numPullEnd}{N_{\pi(T}(T)}
\newcommand{\rewardFunction}{r^i}
\newcommand{\subgaussian}{\sigma}
\newcommand{\arm}{i}
\newcommand{\armCount}{N}
\newcommand{\policy}{\pi}
\newcommand{\reward}{\mu}
\newcommand{\obs}{o}
\newcommand{\totalReward}{J}
\newcommand{\expectation}{\mathop{\mathbb{E}}}
\newcommand{\possibleArms}{\mathcal{K}}
\newcommand{\arms}{\mathcal{K}}
\newcommand{\window}{h}
\newcommand{\estReward}{\hat{\reward}}
\newcommand{\expestReward}{\bar{\reward}}
\newcommand{\historyt}{\mathcal{H}_\currentTime}
\newcommand{\history}{\mathbf{H}}
\newcommand{\historyn}{\mathcal{H}_\timeEnd}
\newcommand{\underpullSet}{\textsc{up}}
\newcommand{\overpullSet}{\textsc{op}}
\newcommand{\HPevent}{\xi^\alpha_t}
\newcommand{\HPeff}{\xi^\alpha_{t,\, \texttt{eff}}}
\newcommand{\HPSWA}{\xi_{\rm SWA}}
\newcommand{\policySet}{\Pi}
\newcommand{\rewardSet}{\mathcal{L}_L}
\newcommand{\BBxSet}{\mathcal{B}_{B,x}}
\newcommand{\BBSet}{\mathcal{B}_{B}}
\newcommand{\BSet}{\mathcal{B}}
\newcommand{\stationarySet}{\mathcal{S}_L}
\newcommand{\myAlgorithm}{\normalfont \texttt{FEWA}\xspace}

\newcommand{\GO}{\pi_{\rm O}}
\newcommand{\Azero}{\mathcal{A}_{\rm 0}}
\newcommand{\Atwo}{\mathcal{A}_{\rm 2}}
\newcommand{\GB}{\pi_{\rm G}}
\newcommand{\FEWA}{\normalfont \texttt{FEWA}\xspace}
\newcommand{\EWA}{\policy_{\rm F}}
\newcommand{\piF}{\policy_{\rm F}}
\newcommand{\wSWA}{\normalfont \texttt{wSWA}\xspace}
\newcommand{\SWA}{\normalfont \texttt{SWA}\xspace}
\newcommand{\piSWA}{\pi_{\rm SWA}}

\newcommand{\DUCB}{\normalfont \texttt{D-UCB}\xspace}
\newcommand{\SWUCB}{\normalfont \texttt{SW-UCB}\xspace}
\newcommand{\UCBone}{\normalfont \texttt{UCB1}\xspace}
\newcommand{\EFF}{\normalfont {\texttt{EFF\_UPDATE}}\xspace}
\newcommand{\EFFP}{\policy_{\rm EFF-FEWA}}
\newcommand{\FILTER}{\normalfont \texttt{FILTER}\xspace}
\newcommand{\UPDATE}{\normalfont \texttt{UPDATE}\xspace}
\newcommand{\EFFFEWA}{\normalfont \texttt{EFF-FEWA}\xspace}
\newcommand{\EUCB}{\normalfont \texttt{RAW-UCB}\xspace}
\newcommand{\XUCB}{\normalfont \texttt{RAW-UCB}\xspace}
\newcommand{\RUCB}{\normalfont \texttt{RAW-UCB}\xspace}
\newcommand{\RAW}{\normalfont \texttt{RAW-UCB}\xspace}
\newcommand{\piX}{\policy_{\rm R}}
\newcommand{\piR}{\policy_{\rm R}}
\newcommand{\EFFU}{\normalfont \texttt{EFF-RAW-UCB}\xspace}
\newcommand{\EFFX}{\normalfont \texttt{EFF-RAW-UCB}\xspace}
\newcommand{\EFFRAW}{\normalfont \texttt{EFF-RAW-UCB}\xspace}
\newcommand{\piEFFE}{\normalfont \policy_{\rm ERUCB}\xspace}
\newcommand{\piEXU}{\normalfont \policy_{\rm ERUCB}\xspace}
\newcommand{\piEFFR}{\normalfont \policy_{\rm ERUCB}\xspace}
\newcommand{\piER}{\normalfont \policy_{\rm ER}\xspace}
\newcommand{\piEF}{\normalfont \policy_{\rm EF}\xspace}
\newcommand{\EFFKLR}{\normalfont \texttt{EFF-kl-RAW-UCB}\xspace}
\newcommand{\EXPS}{\normalfont \texttt{Exp3.S}\xspace}
\newcommand{\EXPR}{\normalfont \texttt{Exp3.R}\xspace}
\newcommand{\ADSWITCH}{\normalfont \texttt{AdSwitch}\xspace}
\newcommand{\GLRKLUCB}{\normalfont \texttt{GLR-klUCB}\xspace}
\newcommand{\GLRUCB}{\normalfont \texttt{GLR-UCB}\xspace}
\newcommand{\MUCB}{\normalfont \texttt{M-UCB}\xspace}
\newcommand{\CUSUMUCB}{\normalfont \texttt{CUSUM-UCB}\xspace}
\newcommand{\ADAILTCBplus}{\normalfont \texttt{ADA-ILTCB+}\xspace}

\newcommand{\PiO}{\Pi_{\rm O}}
\newcommand{\PiL}{\Pi_{\rm L}}
\newcommand{\Nit}{N_{i,\,t}}
\newcommand{\Nitmone}{N_{i,\,t-1}}
\newcommand{\NiT}{N_{i,\,T}}
\newcommand{\Njt}{N_{j,\,t}}
\newcommand{\Njtmone}{N_{j,\,t-1}}
\newcommand{\NjT}{N_{j,\,T}}
\newcommand{\Nitpi}{N_{i,\,t}^\pi}
\newcommand{\Nitmonepi}{N_{i,\,t-1}^\pi}
\newcommand{\NiTpi}{N_{i,\,T}^\pi}
\newcommand{\Njtpi}{N_{j,\,t}^\pi}
\newcommand{\Njtmonepi}{N_{j,\,t-1}^\pi}
\newcommand{\NjTpi}{N_{j,\,T}^\pi}
\newcommand{\hit}{h_{i,\,t}}
\newcommand{\hitmone}{h_{i,\,t-1}}
\newcommand{\hiT}{h_{i,\,T}}
\newcommand{\hjt}{h_{j,\,t}}
\newcommand{\hjtmone}{h_{j,\,t-1}}
\newcommand{\hjT}{h_{j,\,T}}
\newcommand{\hitpi}{h_{i,\,t}^\pi}
\newcommand{\hiTpi}{h_{i,\,T}^\pi}
\newcommand{\hjtpi}{h_{j,\,t}^\pi}
\newcommand{\hjTpi}{h_{j,\,T}^\pi}
\newcommand{\PPv}{\mathbb{P}}
\newcommand{\ie}{\emph{i.e.}\xspace} 
\newcommand{\piFRSet}{\left\{\piF, \piR\right\}}
\newcommand{\Him}{H_{i,\,m}}
\newcommand{\Hitm}{H_{i_t,\,m}}
\newcommand{\hmueff}{\hmu_{i,\texttt{eff}}^{h_j}}
\newcommand{\hmuiteff}{\hmu_{i_t,\texttt{eff}}^{h_j}}
\newcommand{\bmueff}{\bmu_{i,\texttt{eff}}^{h_j}}
\newcommand{\bmuiteff}{\bmu_{i_t,\texttt{eff}}^{h_j}}
\newcommand{\neff}{n_i^{h_j}}
\newcommand{\peff}{p_i^{h_j}}
\newcommand{\ist}{i^\star_t}
\newcommand{\tteff}{\texttt{eff}}


\usepackage{algorithmicx}
\usepackage{algorithm,algpseudocode}
\usepackage{footnote}
\usepackage{subcaption}
\usepackage{float}
\usepackage{wrapfig}
\usepackage{afterpage}
\usepackage{placeins}
\usepackage{svg}
\newcommand{\bookboxx}[1]{\small
\par\medskip\noindent
\framebox[0.46\textwidth]{
\begin{minipage}{0.44\dimexpr\textwidth-\parindent\relax} {#1} \end{minipage} } \par\medskip }

\setlength\parindent{0pt}
\setlength\parskip{1em plus 2pt}
%\setlength{\parskip}{1em}

\newcommand{\citep}{\parencite}
\newcommand{\citet}{\textcite}
\hypersetup{pdftitle={Thèse de Doctorat, Julien SEZNEC},pdfauthor={Julien SEZNEC}}


\begin{document}
\newpage
~\vfill
\thispagestyle{empty}
{\fontfamily{ptm}\selectfont 

 \vspace{-3cm} \hspace{-1cm}\fbox{
\begin{minipage}[c]{16cm}
\vspace{.2cm}
\begin{center}
{\large\sc\bf\sffamily TH\`ESE DE DOCTORAT DE L'UNIVERSIT\'E DE LILLE} 
\\{\bf {\sffamily École Doctorale Sciences Pour l'Ingénieur}}\\
{\sffamily Sp{\'e}cialit{\'e} :} {\bf\sffamily Informatique}

\vspace{.2cm}
\end{center}
\end{minipage}
}

%\vspace{1.0cm}
%\begin{center}
%  \includegraphics[width=0.2\columnwidth]{fig/ens.pdf}
%  %\hspace{5cm}
%  %\includegraphics[width=0.18\columnwidth]{fig/inria.pdf}
%\end{center}
%\vspace{0.5cm}

\begin{center}
{\sffamily \vspace{.4cm} Thèse de Doctorat pr\'esent\'ee par \vspace{.4cm}}

{\large \bf\sffamily  Julien SEZNEC} \\
\end{center}

\vspace{.4cm}
\begin{center}
\hspace{-0.3cm}\hrulefill \hspace{.2cm}

\vspace{.2cm}

\hspace{-.5cm}{\bf \Large \sffamily{Sequential Machine Learning for Adaptive Educational Systems}}

\vspace{-.1cm}

\hspace{-0.3cm}\hrulefill \\ \hspace{.2cm}

{\sffamily
sous la direction de MM. Michal Valko et Alessandro Lazaric, \\ et l'encadrement de M. Jonathan Banon.
}
\end{center}



\vspace{1.cm}

{\sffamily

\begin{tabular}{lllll}
{\bf\sffamily Rapporteurs :} 
& {M.}   & Aur\' elien & {\bf\sffamily GARIVIER} & ENS de Lyon \\
& {M.}   & Gilles & {\bf\sffamily STOLTZ}     & Universit\'e Paris Saclay \& CNRS \\
\end{tabular}

{\sffamily\vspace{1.cm} Soutenue le {\bf\sffamily 15 décembre 2020} devant le jury
compos{\'e} de}
}
\bigskip

\bigskip

%\hspace{-1.5cm}
{\sffamily
\begin{tabular}{llllll}
 {M.}   & Gilles & {\bf\sffamily STOLTZ}     & Univ. Paris Saclay \& CNRS & Rapporteur\\
{M.}   & Aur\' elien & {\bf\sffamily GARIVIER} & ENS de Lyon & Rapporteur \\
{M.}   & Steffen & {\bf\sffamily GRÜNEWÄLDER}     &  University of Lancaster & Examinateur\\
{M.}   & Manuel & {\bf\sffamily LOPES}     &  Instituto Superior Tecnico & Examinateur\\
{Mme}   & Mathilde & {\bf\sffamily MOUGEOT}     & Univ. Paris Saclay \& ENSIEE & Examinatrice\\
{M.}   & Michal & {\bf\sffamily VALKO}     &  INRIA Lille \& Deepmind & Directeur\\
{M.}   & Alessandro & {\bf\sffamily LAZARIC}     &  INRIA Lille \& FAIR  & Co-Directeur\\
{M.}   & Jonathan & {\bf\sffamily BANON}     &  Lelivrescolaire.fr & Encadrant\\
\end{tabular}
}
\begin{center}
\small\sffamily{Centre de Recherche en Informatique, Signal et Automatique de Lille (CRIStAL),\\
UMR 9189 Équipe SequeL, 59650, Villeneuve d’Ascq, France}
\end{center}
\makebox[\textwidth][c]{
\includegraphics[width=0.36\columnwidth]{fig/logo_lls.pdf}
\includegraphics[width=0.288\columnwidth]{fig/logo_univ-lille.png}
\includegraphics[width=0.264\columnwidth]{fig/logo_cristal.pdf}
\includegraphics[width=0.288\columnwidth]{fig/logo_inria.pdf}
}
%!TEX root = ./main.tex
\cleardoublepage
\hspace{0pt}
\vfill
\begin{flushright}
\emph{À mes grands-parents: Denise, Jean et Germaine}.
\end{flushright}
\vfill
\hspace{0pt}
\cleardoublepage
\section*{R\'esum\'e}
Proposer des séquences adaptatives d'exercices dans un Environnement informatique pour l'Apprentissage Humain (EIAH) nécessite de caractériser les lacunes de l'élève et d'utiliser cette caractérisation dans une stratégie pédagogique adaptée. Puisque les élèves ne font que quelques dizaines de questions dans une session de révision, ces deux objectifs sont en compétition. L'apprentissage automatique appelle \emph{problème de bandits} ces dilemmes d'exploration-exploitation dans les prises de décisions séquentielles. Dans cette thèse, nous étudions trois problèmes de bandits pour une application dans les systèmes éducatifs adaptatifs.

Les \emph{bandits décroissants au repos} sont un problème de décision séquentiel dans lequel la récompense associée à une action décroît lorsque celle-ci est sélectionnée. Cela modélise le cas où un élève progresse quand il travaille et l'EIAH cherche à sélectionner le sujet le moins maîtrisé pour combler les plus fortes lacunes. Nous présentons de nouveaux algorithmes et nous montrons que pour un horizon inconnu $T$ et sans aucune connaissance sur la décroissance des $K$ bras, ces algorithmes atteignent une borne de regret dépendante du problème $\cO(\log{T}),$ et une borne indépendante du problème $\tcO(\sqrt{KT})$. Nos résultats améliorent substantiellement l'état de l'art, ou seule une borne minimax $\tcO(K^{\nicefrac{1}{3}}T^{\nicefrac{2}{3}})$ avait été atteinte. Ces nouvelles bornes sont à des facteurs polylog des bornes optimales sur le problème stationnaire, donc nous concluons: les bandits décroissants ne sont pas plus durs que les bandits stationnaires.

Dans les \emph{bandits décroissants sans repos}, la récompense peut décroître à chaque tour pour toutes les actions. Cela modélise des situations différentes telles que le vieillissement du contenu dans un système de recommandation. On montre que les algorithmes conçus pour le problème "au repos" atteignent les bornes inférieures agnostiques au problème et une borne dépendante du problème $\cO(\log{T})$. Cette dernière est inatteignable dans le cas général ou la récompense peut croître. Nous concluons : l'hypothèse de décroissance simplifie l'apprentissage des bandits sans repos. 

Viser le sujet le moins connu peut être intéressant avant un examen, mais pendant le cursus - quand tous les sujets ne sont pas bien compris - cela peut mener à l'échec de l'apprentissage de l'étudiant. On étudie un Processus de Décision Markovien Partiellement Observable (POMDP, selon l'acronyme anglais) dans lequel on cherche à maîtriser le plus de sujets le plus rapidement possible. On montre que sous des hypothèses raisonnables sur l'apprentissage de l'élève, la meilleure stratégie oracle sélectionne le sujet le plus connu sous le seuil de maîtrise. Puisque cet oracle optimal n'a pas besoin de connaitre la dynamique de transition du POMDP, nous proposons une stratégie apprenante avec des outils "bandits" classiques, en évitant ainsi les méthodes gourmandes en données de l'apprentissage de POMDP.
\newpage

\section*{Abstract}
Designing an adaptive sequence of exercises in Intelligent Tutoring Systems (ITS) requires to characterize the gaps of the student and to use this characterization in a relevant pedagogical strategy. Since a student does no more than a few tens of exercises in a session, these two objectives compete. Machine learning called these exploration-exploitation trade-offs in sequential decision making the \emph{bandits problems}. In this thesis, we study different bandits setups for intelligent tutoring systems.%TODO

The \emph{rested rotting bandits} are a sequential decision problem in which the reward associated with an action may decrease when it is selected. It models the situation where the student improves when he works and the ITS aims the least known subject to fill the most important gaps.  We design new algorithms and we prove that for an unknown horizon $T$, and without any knowledge on the decreasing behavior of the $K$ arms, these algorithms achieve problem-dependent regret bound of $\cO(\log{T}),$ and a problem-independent one of $\tcO(\sqrt{KT})$. Our result substantially improves over existing algorithms, which suffers minimax regret $\tcO(K^{\nicefrac{1}{3}}T^{\nicefrac{2}{3}})$. These bounds are at a polylog factor of the optimal bounds on the classical stationary bandit; hence our conclusion: rotting bandits are not harder than stationary ones. 

In the \emph{restless rotting bandits}, the reward may decrease at each round for all the actions. They model different situations such as the obsolescence of content in recommender systems. We show that the rotting algorithms designed for the rested case match the problem-independent lower bounds and a $\cO(\log{T})$ problem-dependent one. The latter was shown to be unachievable in the general case where rewards can increase. We conclude: the rotting assumption makes the restless bandits easier.

Targeting the least known topic may be interesting before an exam but during the curriculum - when all the subjects are not yet understood - it can lead to failure in the learning of the student. We study a Partially Observable Markov Decision Process in which we aim at mastering as many topics as fast as possible. We show that under relevant assumptions on the learning of the student, the best oracle policy targets the most known topic under the mastery threshold. Since this optimal oracle does not need to know the transition dynamics of the POMDP, we design a learning policy with classical bandits tools, hence avoiding the data-intensive methods of POMDP learning. 
\newpage

\section*{Acknowledgments}
Je remercie Aurélien Garivier et Gilles Stoltz pour leurs relectures de cette version du manuscrit. Je n'ignore pas la quantité de travail que cela représente, et je suis sincèrement touché qu'ils aient accepté d'être Rapporteurs.

I also would like to thank the examiners - Mathilde Mougeot, Manuel Lopes, and Steffen Grünewälder -  for accepting my invitation.

More acknowledgments to come in the final version.

\cleardoublepage
\chapterimage{chapter_head/0_content.jpg} % Table of contents heading image

\pagestyle{empty} % Disable headers and footers for the following pages

\tableofcontents % Print the table of contents itself
\chapterimage{chapter_head/0_figures.jpg}
% list of items
\begingroup
\makeatletter
\chapter*{List of items}
\section*{Figures}
\@starttoc{lof}
\let\clearpage\relax
%\listoftables
\section*{Tables}
\@starttoc{lot}
\section*{Algorithms}
\@starttoc{loa}
\makeatother
\endgroup


\cleardoublepage % Forces the first chapter to start on an odd page so it is not on the right side of the book

\pagestyle{fancy} % Enable headers and footers again

%%----------------------------------------------------------------------------------------
%%	GLOSSARY
%%----------------------------------------------------------------------------------------
%\clearpage
%
%\printglossary[numberedsection,type=symbols,style=list,nogroupskip]
%
%%\printglossary[type=\acronymtype, style=myList]
%\printglossary[type=\acronymtype]
%\printglossaries
\clearpage
%!TEX root = ../main.tex
\part{Introduction}
\input{2Literature/0Afterclasse}
%!TEX root = ../main.tex
\chapterimage{chapter_head/2_147romantisme.jpg} 
\chapter{Exploration in online learning}
\vspace{-2.5cm}
{\emph{Il était tard lorsque K. arriva. Une neige épaisse couvrait le village. La colline était cachée par la brume et par la nuit, nul
rayon de lumière n’indiquait le grand Château. K. resta longtemps sur le pont de bois qui menait de la grand-route au village, les yeux levés vers ces hauteurs qui semblaient vides.} \\ \vspace{-1.2cm}
\begin{flushright}
\emph{Franz Kafka,} Le Château, \emph{Chapitre Premier}.
\end{flushright}

\label{ch:exploration}
\section{The multi-armed bandits model}

The multi-armed bandits (MAB) model is a sequential decision process in which the machine learner faces many possible actions. At each round $t \in \left\{ 1, \dots, T\right\}$, it selects one of these actions $i_t \in \arms$ (also called "arms") and receives an observation  $r_{i_t, t}$ which measures the benefits of this action (also called "reward"). A common goal is to maximize the sum of the rewards collected at the last round,
\[
J_T = \sum_{t=1}^T r_{i_t,t}.
\]

In order to do so, the learner should try the different options and discover which action yields the largest rewards. The more the learner try different actions, the more they will be accurate in the future. Yet, there is an inherent cost of "trying" options. Multi-armed bandits methods focus on solving this \textit{exploration-exploitation dilemma}. 

The model was first studied in 1933 by \citet{thompson1933likelihood}. The denomination "multi-armed bandits" was coined in the 80's in reference to the slot machines. Indeed, in a casino, a gambler may face several machines and wonder which one is the most profitable. Of course, the model aims at optimizing more interesting or useful trial and error processes like clinical trials \citep{villar2015bandit}, recommender systems \citep{traca2015regulating} or intelligent tutoring systems \citep{clement2015multi, pikeburke2019phd}.

Yet, before going any further in the modelization, we should stress the assumptions that we already made. First, what we observe is connected to the action we took. In particular, we don't observe the reward associated with other actions. This is known as \emph{bandits feedback}. Second, the observation is revealed just after the action choice. Third, the observation measures how good the action is. Last, the sum of the observations is our final objective. It means that rewards are exchangeable, we can trade-off reward in the present for reward in the future.

\section{Stochastic bandits}
\label{sec:stoch-bandits}
\subsection{Regret minimization}
Up to this point, we did not precise how the environment generates rewards. A popular assumption associates each arm $i$ to a stochastic distribution with mean $\mu_i$. Each time an action is selected, the environment outputs an independent reward sample from the arm's distribution. The mean of the distribution can be seen as the intrinsic value of the action. This intrinsic value is only accessible to the learner through the noisy reward.

For instance, in clinical trials, consider many patients which are affected by the same disease. The different actions are the different drugs that could heal the patients. The goal could be to cure as many patients as possible. The learner observes if a patient heals or not.  Each drug has its own probability of success that we don't know \emph{before testing}. 

If the learner would know in advance the means, he would select the arm with the largest $\mu_i$ to maximize the cumulative reward in expectation. How the learner can compare to this oracle strategy? In order to answer to this question, we define the (expected) regret after $T$ rounds, which is the expected difference between the cumulative reward of the oracle strategy and the cumulative reward gathered by the learner,
\[
R_T(\pi) \triangleq \EE{\sum_{t=1}^T \mu_\star - \mu_{i_t}}
\]
with $\mu_\star = \max_{i\in \arms} \mu_i$. The expected regret is positive, as the oracle policy obtains the best possible performance in expectation. 

How small the regret of the learner can be? In fact, a policy that selects always arm $1$ will have zero regret as soon as arm $1$ is optimal. However, this policy suffers a regret which scales linearly with the number of rounds $T$ when arm $1$ is suboptimal. Thus, this kind of policy is not adaptive at all.  

What do we mean by \emph{adaptive}? In fact, for any arms' distributions set, we would like the policy to make fewer and fewer mistakes as it receives feedback. More formally, the regret per round should decrease with the number of rounds. Or, equivalently, the regret should scale sublinearly with the horizon $T$. A policy is called \emph{consistent} when it has a sublinear regret rate on any possible bandits game. 

What is the cost of being adaptive? \citet{lai1985asymptotically} and \citet{burnetas1996optimal} show that the expected regret per suboptimal arm $i$ for consistent policies is lower bounded asymptotically by $\cO\pa{\frac{\log{T}}{\pa{\mu_\star -\mu_i}}}$. This is the minimal cost on each bandit game to be quite good (\ie consistent) on every one. This is a \emph{problem-dependent} bound because it depends on the value of the arms' means. Later, we will describe famous policies that are proven to get this logarithmic rate (asymptotically) on each bandit game. These policies are called asymptotic optimal because one cannot get better asymptotic performance on any bandit game without suffering very large regret on another problem. 

This logarithmic rate is optimal only asymptotically. In fact, when the gap $\Delta_i = \mu_\star -\mu_i$ tends to zero, the rate diverges at finite horizon $T$. Yet, we cannot have infinite regret as we cannot do more than $T$ mistakes of size $\Delta_i$, \ie at most $T\Delta_i$ regret for arm $i$. Hence, when $\Delta_i$ tends to zero, the regret at finite-time also tends to zero. When $\Delta_i$ is large, the cost of each mistake is large, but a good learner can quickly learn from these mistakes and reach the logarithmic asymptotic regime. In between, we have difficult problems, where the learner struggles to detect significant differences between arms and yet suffers a rather large error at each mistake. 

What is the worst possible regret a good learner can get for finite-horizon $T$?  \citet{auer2002nonstochastic} give a quantitative version of the last argument. With $K$ arms, they design a bandit problem where the best arm's mean is separated from the others by a distance of $\cO\pa{\sqrt{\nicefrac{K}{T}}}$. Then, they show that this difference is small enough so the learner does not see significant differences between arms. Hence, in expectation, the optimal arm cannot be pulled a lot more than the others, which is $\nicefrac{T}{K}$ times. Thus, we do roughly $\cO\pa{T}$ mistakes of size $\cO\pa{\sqrt{\nicefrac{K}{T}}}$ in this setting, \ie a worst-case (or \emph{problem-independent}) regret rate of at least $\cO\pa{\sqrt{KT}}$. 

We will later present some algorithms which match this rate in the worst-case (with an increased constant factor compared to the lower bound). These algorithms are called minimax optimal.  The denomination "minimax" comes from game theory, where a player tries to maximize its performance knowing that its adversary will later try to minimize it. Here, the adversary is the environment, which chooses the worst possible gaps between arms. 

We presented two types of performance criteria, one which depends on the specific parameter of the bandits we are considering and the other which holds in the worst case. Another point of view is to consider the weighted average performance across multiple bandits games. The weight used in the average is called the \emph{prior} probability distribution across bandit games. This prior represents how likely a bandit game is according to our belief before the game has started. One may recognize the language of Bayesian statistics, and this objective measure is called the Bayesian regret. Bayesian regret is weaker than the problem-dependent bound in the sense that we can deduce a Bayesian regret bound from the problem-dependent bound by averaging. Also, the worst-case regret upper bounds the Bayesian regret.


\subsection{Upper confidence bound methods}
In stochastic bandits, we know that arms have intrinsic values. Each time we pull an arm, we get an observation which is useful in two different ways: first, it is an instantaneous reward; second, it brings some information about the intrinsic value of the arm. The ultimate goal is the cumulative reward the learner gathers, so we would like to estimate how much the extra information is worth in terms of future reward. With this estimation, we could estimate the value of pulling an arm by adding the reward with the value of information. 

We call \emph{index policies}, the policies which computes a value for each arm based on the arm's history and selects the arm with the largest value. The \UCB algorithm uses as index an upper confidence bound on the value of the arm. For instance, if arms are gaussians with known variance $\sigma^2$, \UCB computes the following indexes,
\begin{equation}
\label{eq:ucb}
\text{ind}(i) = \hmu_{i,t} + \sqrt{\frac{2\sigma^2 \log{\nicefrac{1}{\delta}}}{N_{i,t}}}.
\end{equation}
with $\hmu_{i,t}$ the average of the $N_{i,t}$ values of arm $i$ at each round $t$.  The average can be seen as the estimate of the instantaneous reward we should get, and the Hoeffding confidence bound term as the value we are willing to pay for the information that the $N_{i,t}+1$-th reward sample should bring. 

Yet, this estimated value depends on a parameter $\delta$: how should we tune it? It is possible to show that \UCB with $\delta = \nicefrac{1}{t}$ is asymptotic optimal in the case of gaussian arms with known variance. However, for other distributions, how should we set $\sigma$ in Equation~\ref{eq:ucb}? One possibility is to upper-bound the variance. For instance, for Bernoulli distribution, we can use Equation~\ref{eq:ucb} with $\sigma^2 = \nicefrac{1}{4}$. By doing so, we can get a near-optimal logarithmic regret rate, \ie a regret rate which is at a constant factor of the \citet{lai1985asymptotically}'s lower bound.

Indeed, upper-bounding the variance means that we "buy" new information at a higher price than what it is worth. For instance, for Bernoulli distribution with a small probability $p\sim 0$, the variance is $p(1-p) \sim p \sim 0$ which is much smaller than $\nicefrac{1}{4}$ when $p=\nicefrac{1}{2}$. \UCBV \citep{audibert2009ucbv} is an extension of \UCB which estimates the variance empirically. While \UCBV shows improved results over classical \UCB in the general case, it is not asymptotic optimal. 

In order to get the asymptotic optimal rate, we need better statistical tools. \KLUCB \citep{cappe2013klucb} uses the Kullback-Leibler divergence which measures how plausible is a distribution $p'$ given that data are generated with an other distribution $p$. More precisely, it computes as index of an arm, 
\begin{equation}
\label{eq:klucb}
\text{ind}(i) = \sup \left\{\mu \in[0,1]\ \bigg{|}\  \mathbb{KL}\left(\hat{\mu}_{i,t}, \mu\right) \leqslant \frac{\log\pa{t} + c\log\pa{\log\pa{t}}}{N_{i,t}} \right\}
\end{equation}
 The expression of the KL-divergence depends on the family of distributions which are considered. When the distributions are gaussians with fixed and known variance, \KLUCB is equivalent to \UCB. Yet, in general, the index of \KLUCB cannot be computed with a closed formula, and we need to use standard optimization software to approximate the index.

\KLUCB is asymptotic optimal but only near-minimax optimal (even for the simple gaussian bandits' case) as we can show a minimax regret rate of $\cO\pa{\sqrt{KT\log T}}$. The extra $\sqrt{\log T}$ factor has a clear interpretation: \UCB buys information at a $\cO\pa{\sqrt{\log t}}$ price. This cost will be paid off asymptotically, but at finite-time, when arms are too close to each other to be distinguished, this information is rather useless. In the early work of \citet{lai1987adaptive}, they suggest to use a refined confidence level $\delta = \nicefrac{N_{i,t}}{t}$ in the ucb such that we do not buy information for the most pulled arms. Yet, when the $K$ arms are close to each other $N_{i,t} \sim \nicefrac{t}{K}$, so we still buy information at a $\cO\pa{\sqrt{\log K}}$ cost. 

The Minimax Optimal Stochastic Strategy \MOSS \citep{audibert2009minimax, degenne2016anytime} suggests to use $\delta = \nicefrac{K N_{i,t}}{t}$. As its name suggests, \MOSS is minimax optimal. It is also asymptotic optimal for the gaussian case \citep{lattimore2020banditbook}. \citet{menard2017klucb++} suggested \KLUCBpp, an algorithm which is minimax and asymptotic optimal for many famous distributions (the single-parameter exponential family). This algorithm uses the tuning of the confidence levels of \MOSS with the KL divergence upper-confidence bound of \KLUCB.

\citet{lattimore2018refining} suggests that asymptotic and minimax optimality may not be enough. When there are many arms, but only one suboptimal arm is close to the optimal value (with a distance $\Delta$), it is effectively a two-arm bandit problem. The other arms weigh very little in terms of both regret and number of pulls (for a good policy).  Yet, \MOSS tunes $\delta$ with $K$. In particular, the exploration bonus is canceled after $\nicefrac{T}{K}$ pulls, which only guarantees (with high probability) to pull the optimal arm $\cO\pa{\nicefrac{T}{K}}$ times at the beginning of the game. By contrast, \UCB keeps exploring the two arms such that they are pulled $\nicefrac{T}{2}$ at the beginning of the game. During this starting phase, the two arms' values are not well identified by the algorithm, and the expected regret is linear. This linear phase ends once each arm has been pulled $\cO\pa{\nicefrac{1}{\Delta^2}}$, hence it is $K$ times longer for \MOSS than for \UCB. In fact, at the end of this phase for \MOSS, its expected regret is $K$ times larger than \UCB. That is why \citet{lattimore2018refining} suggests the sub-UCB criteria, which ensures that the policy is at a constant factor of the performance of \UCB at any round $t$. He also suggests the policy \ADAUCB, which computes for each arm the number of other arms that are "competing" with this arm, and they plug this number instead of $K$ in the confidence level tuning. \ADAUCB is proven to be sub-ucb, asymptotic optimal, and minimax optimal. 

We have discussed how optimistic strategies based on upper-confidence bound indexes can achieve multiple optimality criteria. However, the main advantage of \UCB could be its simplicity. Indeed, it is a deterministic algorithm, that is, an algorithm that outputs always the same action given the same data. Arguably, this is a desirable property for explainability as well as for an implementation purpose. It is worth noticing that one of the most quoted paper \citep{auer2002finite} in the bandit literature studies a suboptimal version of \UCB (namely \UCBone)  with $\delta = \nicefrac{1}{t^4}$. It gives a simple proof that leads to a finite-time and problem-dependent regret bound which holds with high-probability and from which we can derive near-optimal minimax and asymptotic bounds. From a research perspective, this simplicity is desirable as it gives a simple starting point when one studies a more complex setup than the stochastic stationary multi-armed bandits.


\subsection{Bayesian methods}
\label{ss:bayes}

In his early work, \citet{thompson1933likelihood} suggests pulling an arm according to its probability of being the best given the data. It is difficult to compute this probability directly. Hence, we compute for each arm the probability of the parametric distribution beyond it given the data and a prior. Then, we sample a model for each arm according to this distribution and we select the arm with the best mean according to this sampling. This procedure is known as Thompson Sampling (\TS).

Though \TS is very old, it was only shown recently \citep{kaufmann2012ts, agrawal2013finite} that it is asymptotic optimal (when it is fed with an uninformative prior). Borrowing the idea of canceling the exploration for arms with $N_{i,t} = \nicefrac{T}{K}$ from \MOSS, \citet{jin2020mots} suggested the Minimax Optimal Thompson Sampling (\MOTS) which clipped the posterior distribution at a quantile $\delta = \nicefrac{T}{KN_{i,t}}$. \MOTS is minimax and asymptotic optimal. 

\BayesUCB \citep{kaufmann2012bayesian} is another asymptotic optimal Bayesian algorithm. It computes an optimistic index based on an optimistic quantile of the posterior. \BayesUCB and \TS have empirical performance very similar to \KLUCB.

The posterior distribution can sometimes be computed explicitly, for instance with the Beta distribution for Bernoulli reward. When it is not possible, one can use Markov-Chain Monte-Carlo (MCMC, \citet{andrieu2003introduction}). This technique can sample from a probability distribution $p$, if we know the probability ratio $p(x)/p(y)$ for all $x$ and $y$. Indeed, when we use the Bayes rules, we often have an unknown normalization factor which can be hard to compute.
 

\section{Adversarial bandits}
\label{sec:adv-bandits}
\subsection{Pseudo-regret}
Another popular assumption is to consider the environment fully adversarial \citep{auer2002nonstochastic}, which means that rewards are generated by an adversary who wants to maximize our regret. But how do we define the regret in this setting? In adversarial bandits, it is not possible to compete with the oracle who would know in advance what reward is beyond each arm at every step. Indeed, let's consider an adversary who rewards one arm uniformly at random at every step, and set the reward of the other arms to zero. An oracle can select the right arm at every step, but a learning policy can only try to guess what is the right arm. "Guessing" an independent random variable cannot be improved with past feedback (by definition of independent), and hence the learner suffers a linear regret rate compared to the best possible sequence.

Thus,  we will target a more reasonable objective: we will compare to the best arm in hindsight, \ie we take as reference the best policy (for this reward sequence) among the ones which select always the same arm. Formally, with $r_{i,t}$ the reward of arm $i$ at each round $t$, we define the pseudo-regret,
\[
R_T(\pi) = \max_{i \in \arms} \left( \sum_{t=1}^T  r_{i,t} - r_{i_t,t}\right).
\]

The adversarial multi-armed bandit may look much harder than the stochastic bandits due to the latitude the adversary has to trick us. However, \citet{auer2002nonstochastic} have designed \EXP (Exponential weight for exploration-exploitation), an algorithm with a proven worst-case regret upper bound of $\cO\pa{\sqrt{KT\log\pa{K}}}$. This rate was further refined by \INF \citep{audibert2009minimax} to $\cO\pa{\sqrt{KT}}$. It shows that stochastic bandits are not much easier than adversarial bandits from the minimax perspective. More recently, \citet{zimmert2018tsallis} designed a variant \TsallisINF which is minimax optimal in both adversarial and stochastic settings and near-asymptotic optimal in the stochastic setting. They also show relevant results in intermediate settings. It tends to show that we can have simultaneously the best of both worlds (without knowing in advance in which world the learner is).  Yet, we emphasize that \TsallisINF is not completely asymptotic optimal as it does not recover the right multiplicative constant in the regret rate. 

The adversarial bandit framework is a bit odd: on the one hand, the learner tries to compare to the best arm in hindsight; on the other hand, there is no mechanism beyond the reward generation of each arm which guarantees any coherence in the sequence. Let's go back to the casino: if the gambler acknowledges that slot machines are just some black boxes the casino uses to diminish its performance, why would they care about comparing to the best machine in hindsight? 

There is no fully satisfying answer to this question. An important point is that the learner has to believe in something (because they will suffer linear regret in the worst-case if they compare to any possible sequence of actions), and the meaning of this belief is not included in the model. A popular extension of the adversarial bandits computes the regret against the best policy in a predefined set of $E$ experts.  \citet{auer2002nonstochastic} suggests \EXPfour which is proven to achieve a regret rate of $\cO\pa{\sqrt{\min\pa{K, E} T\log E}}$. Notice the logarithmic dependence with the number of experts: we can have a rather high number of experts, but if we consider all the possible sequences of choices, \ie $ E = K^T$, the upper bound rate becomes linear with $T$. 

The learner may believe that there is an inner mechanism beyond each arm, such that it makes sense to compare to the best arm. In an old-time casino, each machine may have an independent non-stochastic mechanism such that one is more rewarding than the others. Yet, the mechanism may be complex to model and the learner may be lazy and assume the reward adversarial. The aforementioned "best of both worlds" results may encourage him in that way. However, one should be cautious: low regret compared to bad policies can mean low reward. For instance, if the arms have periodic and synchronous rewards (the reward of arm $1$ is low when the reward of arm $2$ is high) competing against the best fixed-arm policy may be much less rewarding than competing against experts which are aware of the periodicity.


\subsection{Adversarial methods}

Adversarial games are very different from stochastic games. In the stochastic setting, when we observe the reward for all the actions (a.k.a the full information setting), the learner can follow the actions with the largest current average reward. Indeed, the learner does not need to explore like in the bandit setting, and \emph{Follow the Leader} (\FTL) is guaranteed to do less than a constant regret (with respect to $T$). Yet, in the adversarial full-information setting, \FTL suffers a linear regret. Indeed, the adversary can alternate the reward between two arms such as the current "leader" is never rewarded. 

In fact, in the adversarial setting, every deterministic policy (like \UCB) would fail because a good adversary may know our strategy. Hence, it can set to zero the reward of the action we select. Hence, we need to design probabilistic strategies that output a probability distribution across actions. We already presented \TS, a probabilistic policy. Yet, this policy suffers linear regret in the adversarial setting. Indeed, it is fairly easy to trick optimistic strategies: during the first quarter of the game, we may reward only one arm such that an optimistic stationary policy is very confident that it is the best arm. Then, the adversary can increase the reward of another arm. This arm will be pulled only at a logarithmic pace and even when it is pulled the high reward will be averaged with older lower rewards such that it will take a very long time to realize that something has changed. Recently, \citet{zimmert2018tsallis} empirically show that \TS suffers near-linear regret even in an intermediate setup called "stochastically constrained adversarial regime". In this setup, the rewards are generated stochastically but the probability distributions beyond arms change a few times during the game without changing the best arm identity. Once again, the key is to exploit the "inertia" of this stationary bandit policy, which average rewards from different distributions.

In adversarial games, the output probability distribution needs to take into account the data while being sufficiently unpredictable for the adversary. This is the spirit of the Follow the Regularized Leader (\FTRL) policy. This full-information policy selects the probability distribution which maximizes the expected performance (according to the current data) plus a regularization term that penalized probability distributions that are too concentrated. More formally, with $p_t$ the output probability distribution on arms at each round $t$, $D_t$ the sum of the observed reward for each arm at each round $t$, and $L$ a regularizing function,

\begin{equation}
\label{eq:ftrl}
p_t \in \argmax_p \left\{ < p | D_t > + L(p)\right\}.
\end{equation}

In the bandit setting, we do not have access to $D_t$, the sum of the reward for each arm from the beginning of the game to round $t$. We can estimate $D_t$ with importance weighted estimator, that is, we add $\hat{r}_{i,t} = \mathbbm{1}\left[i_t =i \right] \nicefrac{r_{i,t}}{p_{i,t}}$ to the sum at each round. This quantity is equal to zero for all the arms which are not selected and for which we don't know $r_{i,t}$. For the arm which is selected, the reward $r_{i,t}$ is normalized by the probability of selecting the arm. This weighting strategy is unbiased in the sense that $\EE{\hat{r}_{i,t}} = r_{i,t}$ (the expectation is taken on the algorithm randomization conditionally on the observed history before round $t$).

This estimator is unbiased but has a large variance when $p_{i,t}$ is small and $r_{i,t}$ is large. Indeed, in this case, $\hat{r}_{i,t}$ will have a very different value depending on whether we pull arm $i$ at the round $t$ or not. This variance will be transmitted to $\hat{D}_{t,i} = \sum_{s=1}^t \hat{r}_{i,s}$ that we want to use instead of $D_{t,i}$ in Equation~\ref{eq:ftrl}. It means that when we observe a good reward for an arm that is pulled with low probability, it can squash all the other probabilities to almost zero. Then, the algorithm may never recover because it will keep selecting this arm and adding a positive weighted reward to $\hat{D}_{t,i}$. Yet, if the algorithm did not pull the arm $i$ at the round $t$ in the first place, it would have very different behavior for the same data sequences generated by the adversary. 

The solution is to work with losses instead of rewards. We can define the losses $l_{i,t} = 1 - r_{i,t}$, the importance-weighted estimator of the losses $\hat{l}_{i,t}  = \mathbbm{1}\left[i_t =i \right] \nicefrac{l_{i,t}}{p_{i,t}}$ and the estimated sum of reward $\hat{D}_{t,i} = \sum_{s=1}^t 1 - \hat{l}_{i,s}$. In that case, the variance can also be high but when $p_{i,t}$ is small and $r_{i,t}$ is small. If we select arm $i$ at a round $t$, $\hat{l}_{i,t}$ will be very large and it will reduce $\hat{D}_{t,i} = \sum_{s=1}^t 1 - \hat{l}_{i,s}$. Hence, according to Equation~\ref{eq:ftrl}, it will squash $p_{i,t+1}$ to zero. This is arguably better for the stability of the algorithm than squashing all the other probabilities to zero. Moreover, arm $i$ may recover after few rounds because $\hat{D}_{t,i}$ is increased by $1$ every time arm $i$ is not pulled. 

Up to this point, we did not precise what regularizer $L$ we should use. A good $L$ will penalize probability vectors which are too predictable. In information theory, a classical measure of how predictable is a probability distribution is its (Shannon) entropy: $- \sum_{i \in \arms} p_i \log\pa{p_i}$. The larger is the entropy the more unpredictable it is. Hence, we could use the negentropy as a regularizer. 

We now have all the ingredients beyond the aforementioned \EXP algorithm \citep{auer2002nonstochastic}. \EXP is equivalent to \FTRL (see Equation~\ref{eq:ftrl}) where we use the loss-based importance weighted estimator $\hat{D}_{t,i} = \sum_{s=1}^t 1 - \hat{l}_{i,s}$ and the unormalized negentropy as regularizer $F(p) = \sum_{i \in \arms} p_i \log\pa{p_i} - p_i$. Notice that with this regularizer, there exists a closed-form formula for $p_t$ instead of the implicit formulation in Equation~\ref{eq:ftrl}. This expression is useful for implementation but it hides the main ideas beyond \EXP.

Most of the recent adversarial algorithms use slight (but powerful) modifications of the aforementioned ideas. For instance, we already advertised $\TsallisINF$ \citep{zimmert2018tsallis}, which improves over \EXP from the minimax adversarial perspective and recovers logarithmic asymptotic bound for the stochastic stationary bandits' case. \TsallisINF uses Online Mirror Descent (\OMD) instead of \FTRL. Without going into the details, the two algorithms share deep similarities. In fact, they are even equivalent for some regularizers \citep{mcmahan2011ftrl}. They also use a different regularizing function known as Tsallis entropy \citep{tsallis1988possible} and they finally discussed another unbiased estimation scheme of the losses.

\section{Non-stationary bandits}
\label{sec:non-stationary}
Since the early stages of the research in bandits \citep{thompson1933likelihood,whittle1980multi}, one of the most desirable properties for a learner would be to adapt to actions whose \textit{value changes over time} \citep{whittle1988restless}, as it happens in non-stationary environments. In fact, from applications in medical trials (where the patient can become more resistant to antibiotics) to a modern applications in recommender systems \citep{chapelle2011empirical,traca2015regulating}, assuming that the environment is \textit{stationary is very limiting}. 

In the adversarial bandit setting, rewards do not have to be generated by a stationary stochastic process. However, the objective is strongly stationary as the pseudo-regret definition competes against a \emph{fixed} set of policies (e.g. the stationary policies which select always the same arm). As in stationary bandits, we would like to define the regret against the best oracle, or, at least, a good enough one. Indeed, depending on the non-stationarity, it can be challenging to compute the best oracle (also called the offline policy), especially when the choice of the learner impacts the non-stationarity. 

With bandit feedback, it can be meaningful to assume that the arms evolve only when they are pulled. In that \emph{rested} case, the learner observes (often with noise) every value. \citet{bouneffouf2016multi-armed} consider the case where all the arms are evolving with a known trend which depends on the number of pulls. They compare to the greedy oracle which selects the largest available reward at each round. They design a variant of \UCB which uses as index the product of the classical ucb index by the trend. This algorithm achieves a logarithmic asymptotic bound similar to \UCB's on stationary bandits. \citet{heidari2016tight} consider the two monotonic rested cases without noise in the observation. \citet{levine2017rotting} consider the parametric and non-parametric decreasing (or rotting) rested case with noise. We give a detailed review of their result on the non-parametric rotting case in Chapter~\ref{ch:rested}. 

In the \emph{restless} setting, the arms can evolve even when they are not pulled. Hence, the learner does not know the last reward state beyond each arm. \citet{whittle1988restless} first consider a very general restless setting: arms are associated with Markov chains with different transition probabilities depending on whether an arm is selected. While \citet{whittle1988restless} suggested a heuristic known as the \emph{Whittle's index} policy, the restless bandits problem was later shown to be \texttt{PSPACE}-hard even to approximate \citep{papadimitriou1994complexity}. 

Assuming that the transition probabilities do not depend on the action of the user simplifies the restless setup. Indeed, in that case, the optimal oracle is straightforward: one should pull the arm with the current largest expected reward. When the evolution is deterministic, \ie the Markov chains are replaced by functions of the round, it is possible to approximate this optimal oracle with an online bandit policy when the changes are either not too frequent \citep{garivier2011upper-confidence-bound} or not too big \citep{besbes2014stochastic}. We give a detailed literature review of the restless bandits with independent evolution in Chapter~\ref{ch:restless}.

While the general restless bandits are unlearnable, some authors studied some specific instances of the restless bandits. For instance,  \citet{immorlica2018recharging, pikeburke2019recovering, cella2020stochastic} studied different models of recharging bandits, where arms' rewards decrease when they are selected, and increase back when the arm has not been pulled for a while. In these problems, the optimal oracle policy for the full horizon regret is hard to compute and the authors often consider approximated oracle policies \citep{immorlica2018recharging, cella2020stochastic} or weakened regret definition \citep{pikeburke2019recovering}.

\section{Contextual bandits}
\label{sec:contextual}
The contextual bandits framework \citep{tewari2017contextual} assumes that at each round contextual information is given to the learner. The reward is associated with the action and context such that an action can be good for a given context and bad in another one. Of course, if we have to explore from scratch every time we receive a new context, it can be quite expensive. 

A classical assumption is that actions and contexts can be embedded in a vector space such that the reward is smooth enough in that space - e.g. it is a linear form \citep{abe1999associative, auer2002using, abbasi2011improved, lattimore2017end} though more complex structures were also considered \citep{filippi2010parametric, valko2013finite, valko2014spectral}. It allows for a potentially infinite number of contexts and actions, as long as there is a finite number of unknown parameters that determine the rewards. Interestingly, while optimistic strategies were shown to perform quite well in this setting \citep{abbasi2011improved}, \citet{lattimore2017end} recently advocates that they could not reach asymptotic optimality, unlike in the multi-armed bandits setup. 

\section{Beyond bandits: Reinforcement Learning}
\label{sec:rl}
Reinforcement learning \citep{sutton1998book} extends the former \emph{contextual bandits} so that the context (renamed "state") is controlled by the learner through the actions. The goal is not only to find and exploit the function which relates states and actions to rewards but also to discover the relation between actions and states.

Formally, the Markov Decision Process (MDP, \cite{howard1960dynamic}) models this situation as a quadruplet $\left\{\mathcal{S}, \mathcal{A}, \mathcal{T},  \mathcal{R}\right\}$ where $\mathcal{S}$ is the states space, $\mathcal{A}$ the actions space, $\mathcal{T}(s, a)$ the transition operator which associates an origin state and an action to a probability density over the destination states, $\mathcal{R}(s, a ,s')$ which associates a probability density over the reward to a transition from $s$ to $s'$ after choosing action $a$. It is often easier and more meaningful to consider the discounted cumulative reward rather than the cumulative reward at finite-horizon $T$. Indeed, in some setups, the termination rule may not be known and the learner may discount the reward to take into account the probability of termination.

The exact state of the learner may be observed partially. For instance, the knowledge of students revising on an intelligent tutoring system is not directly observable: each answer gives only limited information about what they know or do not know. In order to model this situation, Partially Observable Markov Decision Process (POMDP, \cite{astrom1965optimal}) adds a set of observation ($\Omega$) and a probability distribution over this observation set for each states-action transition ($\mathcal{O}(s,a,s')$).

% Planning
Unlike in stationary bandits, finding an optimal policy when we know the MDP parameters is not straightforward. Dynamic programming \citep{bellman1966dynamic} is a general method which uses a recursive relation -  the Bellman equation - on the state value, which is the cumulative value that the agent can expect in a given state if s/he follows a given policy.  

%Learning criteria
Reinforcement learning aims at finding the optimal policy when the reward function $\mathcal{R}$ and transition probabilities $\mathcal{T}$ are not known. There are several quantitative objectives associated with the maximization of the reward. As in the bandit case, we can define the regret for the optimal policy. However, this objective is ambitious in RL as in some problems a single mistake can send irreversibly the learner in a sub-optimal region of the state space. A weaker objective is to minimize the sample complexity, which is the number of rounds after which the policy behaves near optimally with high probability. 

%Limitation for bandits
The framework of RL models much more complex situations than bandits. However, it is possible to adapt the \UCB strategy and its optimistic paradigm for RL. \UCRLtwo \citep{auer2009near} selects the most optimistic model in a confidence region built around the empirical means and then uses classical dynamic programming method to get the optimal policy. Then, it runs this policy for a while and restarts the procedure. Being optimistic about the transition probabilities is not as straightforward as it is for the reward parameters. Indeed, for the reward, we can simply increase the reward with the confidence bound. For the transition, we cannot simply increase each transition, because, 1) the probabilities would not be normalized; and 2) increasing the probability to reach a low reward region of the state space is a pessimistic choice. Yet, we can find with an optimization software (with complexity $\cO\pa{|\mathcal{S}|}$) an optimistic MDP whose transition probabilities lies within a confidence band around the empirical average and maximize the reward that the learner can get given the optimistic estimation of the reward function. If the diameter of the MDP is finite - that is, one can (with the right policy) reach any state from any other state in a finite expected number of rounds - \UCRLtwo recovers near-optimal regret bounds. 

\UCRLtwo models the environment by fitting the environment (that is, $\mathcal{R}$ and $\mathcal{T}$). This type of method is called \emph{model-based} RL. By contrast, model-free reinforcement learning directly tries to fit the policy without modeling the environment. For instance, \QLearning \citep{watkins1989learning} plays a behavior policy and tries to measure the total values that the learner can get from selecting each action in each state (the q-value function). At the end of the learning phase, it outputs a target policy that selects in each state the action with the largest q-value. The fact that the policy which is used in the learning phase differs from the output one is called \emph{off-policy learning}. The value of a state-action pair is the sum of the expected instantaneous reward - for which we get a noisy yet objective sample - and the value of the destination state when we play the optimal policy. Of course, the optimal policy is unknown but we can approximate the aforementioned value by considering the maximal state-action value we have estimated for the destination state. The fact that we reinforce our estimated values with other estimated values is called \emph{bootstrapping}. Notice that at the beginning of the learning, the values are just a random guess, and bootstrapping may propagate the error to other nodes. Yet, the q-values converge with high probability under mild conditions on the learning rate \citep{watkins1992q}.

There exist many other policies than the two we have quickly described. Yet, none of them can learn anything but toy models without extra assumptions. Indeed, there are a priori $|\mathcal{A}||\mathcal{S}|^2$ transition parameters and $|\mathcal{A}||\mathcal{S}|$ reward parameters. When the state space is not very small, it is much more than the $K = |\mathcal{A}|$ reward parameters in the $K$-armed bandit case ($|\mathcal{S}| = 1$). Hence, it will take thousands of rounds to \UCRLtwo to get a basic understanding of a fairly small environment with ten states and ten actions. The problem is even worse for \emph{model-free} methods like \QLearning. Indeed, model-based methods use every sample to estimate the model. In model-free learning, samples are forgotten either because they were only used to evaluate one policy (\emph{on-policy} learning) or because bootstrapping updates by using the (inaccurate) current belief. Hence, they need multiple visits of each state-action pair to converge. 

Deep Reinforcement Learning tries to mitigate this issue by using deep neural networks to generalize the experience the learner obtains. For instance, Deep Q-network \citep{mnih2013playing} uses deep networks to learn the q-value function. Yet, using supervised predictive methods in an online active environment is nothing but straightforward. Indeed, in the supervised learning setting, we learn a function that maps observations $X$ to results $Y$. The way $X$ is generated is assumed to be stationary between the training and production phases. Moreover, $Y$ is assumed to be an objective value that is given to the learner. In the online setting, the observations $X$ are heavily dependent on the policy which is played. With off-policy learning methods, if the policy which is used in the learning phase output a very different $X$ proportion than the optimal policy, then it will bias the neural network. On the other hand, with bootstrapping, the target values $Y$ do not correspond to purely objective values. Indeed, they are constructed using the current belief of the model on the value destination state. The combination of bootstrapping, off-policy learning, and (supervised) function approximation was called \emph{the deadly triad} because it can lead to unstable algorithms that do not converge to the optimal policy.

Surprisingly, using sophisticated deep networks instead of more classical and simple supervised models leads to more stable algorithms. Indeed, deep networks are trained with mini-batches: we only use a subset of the data to estimate the network's parameters gradient. This is arguably a key feature for online learning applications where incoming data are natural mini-batches for continuous training. However, this feature alone is not sufficient to solve the whole deadly triad problem. 

We will not review in detail all the ideas (experience replay, double-Q-network ...) which have improved the stability of DRL methods. Yet, we advertise that this line of work led to superhuman performances in many complex games such as the board game of Go \citep{silver2016mastering, silver2017mastering} or Starcraft II \citep{vinyals2019alphastar}. It shows that given a potentially infinite source of data - and enough computational power to process it - DRL methods can learn very complex tasks. However, these methods are still sample-inefficient: AlphaZero \citep{silver2017mastering} played several million games before reaching superhuman performance in both Chess and Go. For many real-life applications, one may not have a simulator that can produce a tremendous amount of accurate and cheap data.

Improving the sample efficiency is a hot research topic \citep{yu2018towards, yarats2019improving}, and there exist more efficient methods than the ones which have been designed for applications with accurate simulators. However, one should notice that the interaction of planning and exploration makes the methods much more data-intensive than in bandits.  In a small data situation, it is preferable to frame a given problem as a bandit than to rely on the too general reinforcement learning paradigm.
\chapterimage{chapter_head/3_429hero.jpg} 
%!TEX root = ../main.tex
\chapter{Applications to Intelligent Tutoring Systems}
\label{ch:its}
\vspace{-2.5cm}
\emph{Là-haut, le Château, déjà étrangement sombre, que K. avait
espéré atteindre dans la journée, recommençait à s’éloigner.
Mais, comme pour saluer K., à l’occasion de ce provisoire adieu,
le Château fit retentir un son de cloche, un son ailé, un son
joyeux, qui faisait trembler l’âme un instant : on eût dit – car il
avait aussi un accent douloureux – qu’il vous menaçait de
l’accomplissement des choses que votre cœur souhaitait obscurément. }\\ \vspace{-1.2cm}
\begin{flushright}\emph{Franz Kafka,} Le Château, \emph{Chapitre Premier}.
\end{flushright}
\section{Shortcomings in the bandits model}
When we ask a question to a student, we observe their answer to this particular question. This is a good example of \emph{bandits feedback}. Facing this partial feedback, the machine learner has to explore the different options to understand what to do. Handling this exploration is the main question of the bandits' literature. This is quite relevant for Adaptive Intelligent Tutoring Systems: if we think that different students should have different learning paths, we have to characterize to which extends a student is different. 

Of course, a good \emph{exploration} strategy depends on what we want to achieve. In the previous section, we have presented the most famous objective, \ie the cumulative reward maximization. The main objective is to balance between gathering new information and using this information to collect rewards. This exploration-exploitation dilemma is relevant for ITS: characterizing the student is only a tool to improve the learning. A good ITS should size the effort spent on characterizing the student versus the estimated benefits of such characterization. 

Yet, the cumulative reward maximization makes strong assumptions about the benefits. These assumptions strongly orientate the answers the bandits' community gives to this exploration-exploitation dilemma.
In the following, we will discuss four limits of applying classical bandits methods to Intelligent Tutoring Systems.

\subsection{Observation is reward.}
In the cumulative reward maximization setup, there is an identification between observation and reward. For an ITS, it is rather unclear what is the reward that can be associated with the student answer. One should be careful: the reward measures how well the ITS behaves and not how well the student answers the question. If we reward the ITS for the success of the student, the ITS will find very easy questions for the students, which is arguably not the best pedagogical strategy. In Section~\ref{sec:bandits4ITS}, we will present different rewards that were used for ITS applications. 

We advertise some objectives in the bandits' literature that are different from cumulative reward maximization. The Best Arm Identification (BAI) \citep{audibert2010bai, gabillon2012bai} is a pure exploration objective where the learner should output the best arm at the end of the game. There are several quantitative objectives associated with the best arm identification. In the fixed budget setting, one may want to minimize the \emph{simple regret} \citep{audibert2010bai}, that is, the difference between the true best arm's and the identified arm's values. Another possibility is to target the probability of outputting the best arm \citep{carpentier2016tight}. In the fixed confidence setting \citep{garivier2016optimalbai, kaufmann2016complexity}, the learner outputs as fast as possible with high-probability $1- \delta$ an arm at a residual distance $\epsilon$ from the best arm. The algorithms designed for cumulative regret do not work very well in the BAI setting. Indeed, at least from the asymptotical perspective, these algorithms spend $\cO\pa{\log{T}}$ in exploration and most of their budget in exploiting the identified best arm. 

The Best Arm Identification still considers observation as "reward", in the sense that the motivation of targeting the arm with the largest observation is based on the identification "large observation = good". This is not the case for Thresholding Bandits \citep{locatelli2016thresholding, garivier2016thresholding, mukherjee2017thresholding} where the learner wants to separate the arms according to their position relative to a threshold. Interestingly, \citet{locatelli2016thresholding} found a near-optimal algorithm which is fully agnostic. This problem is interesting from an educational perspective: if we have several topics with corresponding sets of related questions, we may want to know which topics are mastered by a student. We could define a threshold above which the topic is considered as mastered and use a Thresholding Bandits algorithm.

More generally, \emph{exploration} can be intrinsically interesting for ITS if the goal is to send information to the teacher. That is why \citet{liu2014trading, erraqabi2017trading} considered a setup where the objective combines the cumulative reward and the error the learner do on the estimation of each arm. \citet{erraqabi2017trading} show that naive \UCB algorithm fails in this setting. They describe \ForcingBalance, an algorithm that directly targets the optimal allocation of pulls for this problem. 


\subsection{Comparing to the best action.}
In both the stationary and the adversarial setups, the performance is compared with the policy which selects always the "best" action. For ITS applications, we believe that the best thing to do is not to recommend always the same type of exercises. 

In Sections~\ref{sec:non-stationary} and~\ref{sec:contextual}, we presented two lines of work where the optimal action is not always the same during the game. These approaches have some limits. Contextual bandits need a meaningful representation for the context such that the reward is a simple function in this space. This representation can be learned using offline data, but it breaks the online paradigm. Moreover, contextual algorithms often use more sophisticated techniques. For instance, the simple averages which are used in the classical multi-armed bandits' framework (e.g., for computing the ucb) are replaced by regression methods which are computationally more expensive. Non-stationary bandits also have some important drawbacks. They often require much more exploration than in the stationary case. This is especially true in the restless case, when arms which are not pulled can change. Indeed, this type of non-stationarity is particularly challenging with bandit feedback. 

\subsection{Actions do not impact observations}
In the classical adversarial and stochastic setting, the learner has no impact on the observations (rewards) output by the arms. This is a strong limitation for tutoring systems, as we expect a teaching strategy to modify the student's knowledge. 

Reinforcement Learning (Section~\ref{sec:rl}) models a more general setup where the learner has a state which is changed by the action. In some non-stationary bandits setups (Section~\ref{sec:non-stationary}), arms reward is changing when the arm is pulled. It could be reformulated as a "state" which is impacted by the actions of the learner. There is a difference of perspective between RL and non-stationary bandits: bandits methods often focus on tracking the changing rewards to target the (often short-sighted) best action while RL methods design policies that monitor the state's dynamics to remain in rewarding regions of the state space. 

\subsection{Learning is quite slow.}
The last paragraphs suggest increasing the complexity of the classical bandits model with context, state, or non-stationarity. However, the stationary stochastic bandits are already quite hard to learn when the horizon is small. Indeed, students often do no more than a few tens (or hundreds) questions. By contrast, bandits algorithm are often evaluated for longer horizon $T > 10^4$ (e.g., \citet{chapelle2011empirical}). 

From the theoretical perspective, the asymptotic rate $\cO\pa{\nicefrac{\log T}{\Delta_i}}$ is larger than the maximal regret $T\Delta_i$ for many gaps $\Delta_i < 0.2$ when the horizon is small ($T=100$). It means that the asymptotic analysis is not meant for such small horizons (except when the gap is very large). Even the minimax rate $\cO\pa{\sqrt{KT}}$ is not very different from the worst possible rate $T$ for small $T$.

This \emph{small data} situation is particularly challenging. Special care should be taken to overcome this issue: The learning problem should not be too ambitious, the setup should be correctly designed. In particular, the number of arms (or unknown reward parameters for contextual bandits) should be reduced. Prior information should be included in the model, algorithms should target finite-time and empirical performance.

\section{Literature review on exploration methods in Adaptive Intelligent Tutoring Systems}
\label{sec:bandits4ITS}

In this section, we will review previous work which involves bandits and reinforcement learning methods in Intelligent Tutoring Systems.  We will focus on the work where the action-observation loop corresponds to the sequence of question-answer of a single student. In these setups, the goal is to explore the student's knowledge and exploit this knowledge to improve educational actions.

Notice that there are also different exploration scenarios where the feedback loop is the sequence of incoming students. The goal is to refine the instructional policy from one student to another. The objective can be to choose the courses that maximize the final grade of the student \citep{xu2016personalized, lan2016contextual}, or to find the teaching examples sequence that maximizes the score at the test \citep{lindsey2013optimizing}.

\subsection{Target the largest improvement}
\citet{clement2015multi} suggest an ITS which selects sequentially a question linked to a knowledge component (KC) and receives the answer of the student. They suggest selecting the action which leads to the largest increase in student's performance. Besides maximizing the learning gain, it is also the action that motivates the most the learner \citep{gottlieb2013information}. 

They present two similar algorithms. Each algorithm has two components: the first one computes a \emph{Zone of Proximal Developement} (ZPD, \citet{luckin2001designing}), the second selects one knowledge component in the ZPD. The ZPD aims to exclude the KCs on which the student is either too good, too bad, or the ones on which s/he does not progress.  These algorithms don't use any model nor parametric assumption on the way the student progresses. They compute non-parametric statistics to estimate the current level or the current progress of the student on a KC.

Once the ZPD is set, a bandit algorithm selects a KC. The reward is a difference between the last samples and the before last ones. In the first algorithm, they use the average of the $d$ last samples minus the $d$ before last. In the second algorithm, they use the last sample minus a discounted average of the previous ones. These two statistics measure the recent progress of the students.

They claim to use a variant of \EXPfour \citep{auer2002nonstochastic}. Like \EXPfour, this algorithm is probabilistic. The output probability distribution pulls arm according to weights, which are a weighted sum of rewards. This probability distribution also has a uniform component, like the vanilla \EXPfour. Notice that this component was later proven to be unnecessary \citep{bubeck2012regret}, even to recover high probability guarantees \citep{neu2015explore}.

This bandit algorithm is rather far from the ideas beyond \EXPfour. First, there are no experts recommendations which are a necessary input of \EXPfour. Hence, this algorithm is closer to \EXP. In Subsection~\ref{sec:adv-bandits}, we presented the three ideas beyond \EXP: Follow the regularized leader, a specific regularization, and an unbiased estimation scheme based on importance weight. The specific regularization is responsible for the exponential weights, which are absent from the algorithm of \citet{clement2015multi}. The importance weights of the loss -  the loss is divided by the probability of pulling the arm - are replaced by fixed weights on the reward. These fixed weights are closer to discounted statistics used in non-stationary bandits (like \DUCB, \citet{kocsis2006discounted, garivier2011upper-confidence-bound}).

\EXP guarantees low regret compared to the sum of the reward for the policy which selects always the same arm. This is a debatable choice for two reasons. First, the sum of the rewards, which are themselves weighted differences of past observations, is a weird quantity. Indeed, there are telescoping effects in the sum of differences. For some choices of parameters, this telescoping can be total such that the sum of the rewards is simply the last observation minus the first. Second, comparing to the policy which selects always the same KCs may not be meaningful. 

However, this algorithm is not really \EXP. It targets a more pragmatic goal: selecting randomly the KCs (to ensure diversity in the tasks) while favoring smoothly the KCs which demonstrate recent progress. Yet, one should notice that there are also some telescoping effects in the weighted sum of the rewards used in KC's probability computation. Again, rewards samples are not independent, they are differences of the same observations. The discounted sum of these differences is an uncontrolled statistic. It seems unclear if it is a statistically interesting way to measure recent progress.

\citet{clement2015multi} provides empirical evidence of the benefits of their algorithms. In a simulated experiment, they show that their algorithms are more adaptable than an expert sequence to the profile of some simulated students. In an in-class experiment on real students, they show that students who were learning with their algorithms achieve more balanced performance between KCs than a control group that was using the expert sequence. They also demonstrate qualitative differences between the behavior of their algorithm and the expert sequence.

As noticed by \citet{pikeburke2019phd}, this paper is arguably one of the most advanced works using bandits in ITS. The objective - targeting the topic on which the student progresses - is very appealing. The ZPD design allows some timely exploration by unlocking progressively the most advanced topics. The experiments bring many insightful comparisons between the studied algorithms and the expert sequence.

In our opinion, the black spot of the method is the inadequacy between the goal of the paper and the cumulative reward paradigm and associated algorithms. The definition of rewards leads to cumbersome telescoping issues when they are summed. The \EXP inspiration leads to a complex KC choosing algorithm with four parameters: one for the computation of the reward, two for the discounted sum of rewards, and one for the uniform exploration. If the goal is to select the KC at random while favoring the recent progress, why not using a single statistic for the estimation of recent progress and adding some small random exploration in an \epsilongreedy way? This approach would reduce the complexity of the algorithm while getting rid of the telescoping problems.

The work of \citet{clement2015multi} was further extended by \citet{mu2018combining} to take into account the forgetting of the student and the learning of the ZPD structure. 

We also advertise the works of \citet{rollinson2015predictive, kaser2016stop}, which also try to track the progress of the student. These works do not aim at choosing the knowledge component among several possibilities. Instead, they try to decide when one should stop the work on a given skill. \citet{rollinson2015predictive} suggest stopping when there is a sufficient probability that the prospective learning gain associated with the next question is below a threshold. The prospective learning gains are estimated with a student model. Notice that these models are trained with the data of many other students such as it reflects the "average" student.  The models assume a specific shape of the progression. Hence, different models with the same input sequence can lead to different stopping times, even when they have comparable predictive performance. The issue is that the predictive performance is evaluated on several students, the goal is to predict correctly on average. However, when they are used in instructional policies, these models are required to explain and predict quantitatively the learning of a specific student given a small amount of data. This instructional policy was further extended by \citet{kaser2016stop} to be able to stop when a student's performance diverges from the model (for instance, for wheel-spinning student) and to include more complex student models such as deep belief network. 

\subsection{Target the least known subject}
\label{ss:less-known}
\citet{melesko2019computer} suggest targeting the less known subjects. The idea is that the student has more to learn from their mistakes than from their successes. Hence, they suggest rewarding the failed questions and to not reward the succeeded ones. 

Rewarding the system for finding the failed exercises has some limits. Some skills are harder to get, and it could be useful to start with the simplest one. It can also be the case that there are some prerequisite dependencies between the different skills. From the motivational point of view, recommending too hard questions may disengage the student. 

Yet, consider a student that is learning some geography facts. S/he wants to check if s/he know their lesson. The different topics in the lesson are as hard to learn \emph{a priori}. Yet, the student could have studied a lot the first part of the course and did not spend too much effort on the other parts. The goal of the ITS could be to try to spot the weaker part of the course and teach them with some questions.

Another motivation highlighted by \citet{melesko2019computer} is the pure-exploration setup, where the goal of the ITS is to find the weakest topic to send the information to the teacher. 

\citet{melesko2019computer} suggest using the classical \UCB algorithm. They carry many very small data experiments where the number of topics (arms of the bandits) is of the same order of magnitude as the number of questions (horizon). In this context, they recommend the usage of smaller confidence intervals than classical \UCB. The experiments show improved performance for \UCB compared to the random strategy. 

In their work, \citet{melesko2019computer} neglect the impact of the questions on the knowledge of the student. The goal is not really to teach through questions, but to find the least understood topic. However, it is surprising that they use an exploration-exploitation algorithm instead of a specialized algorithm from the best arm identification literature. 

\citet{teng2018interactive} also target to find the least known questions with multi-armed bandits methods. They suggest an algorithm adapted from the linear bandits' literature where the reward depends linearly on an embedding. The algorithm uses several graphs structuring questions, users, and concepts. These graphs are used to infer a vector representation of the users and questions. They bring theoretical and empirical evidence of the performance of their algorithm. 

\subsection{Target faster learning}
\citet{rafferty2016faster} suggest minimizing the time the student spends to understand a concept. Hence, the cost (negative reward) associated with each pedagogical action is the time the action takes to be completed. They formulate their problem as a Partially Observable Markov Decision Process. Indeed, the student has a knowledge state which is only partially observable by the teacher. The teacher has several actions: some \emph{examples} which teaches the concept to the student, some \emph{quizzes} which retrieves information about the knowledge state of the student, and some \emph{questions with feedback} which do both at the same time.  The goal of the learner is to track the state of the student (which encodes what the student does not know) with questions to show some relevant examples. 

They test this framework with several student models and several learning scenarios. The algorithm shows significant time reduction compared to random policies. Some student models are better than others. In particular, modeling long-term memory improves performance compared to models that react only to the last seen example.


%!TEX root = ../main.tex
\part{Rotting bandits}

\chapter{Rested rotting bandits are not harder than stationary ones}
%!TEX root = ../main.tex


\section{Rested rotting bandit : model and preliminaries}
\label{Model}
\subsection{Model}
\subsubsection*{Feedback loop}
At each round $t$, an agent chooses an arm $i_t \in \possibleArms \triangleq \left\{ 1, ... , K\right\} $ and receives a noisy reward $o_t$. The reward associated to each arm $i$ is a $\subgaussian^2$-sub-Gaussian random variable with expected value of $\mu_i(n)$, which depends on the number of times $n$ it was pulled before; $\mu_i(0)$ is the initial expected value. %\footnote{Our definition slightly differs from the one of~\citet{levine2017rotting}. 
We use $\mu_i(n)$ for the expected value of arm~$i$ \textit{after $n$ pulls} instead of when it is pulled \textit{for the $n$-th time}. 
Let $\historyt \triangleq \left\{ \left\{ i_s, o_s \right\}, \forall s < t \right\}$ be the sequence of arms pulled and rewards observed until round $t$, then 
%
\begin{equation}
\label{eq:rested-feedback}
o_{t} \triangleq \mu_{i_t}(N_{i_t,t-1}) + \noise_t
 \;\; \text{with}\; \EE{ \noise_t | \historyt }= 0 \;\; \text{and} \; \forall \lambda \in \R, \; \EE{ e^{\lambda\noise_t}} \leq e^{\frac{\subgaussian\lambda^2}{2}},
\end{equation}
%
where $N_{i,t}\triangleq \sum_{s=1}^{t} \mathbb{I}\{i_s = i\}$ is the number of times arm $i$ is pulled after round $t$. %We use $r_i(n)$ to denote the random reward of arm $i$ when pulled for the $n+1$-th time, i.e., $r_{i_t,t} = \obs_{i_t}(N_{i_t,t})$. 
%
\begin{definition}\label{def:rew-bounded-decay} 
We introduce $\rewardSet$, the set of non-increasing reward functions with bounded decay $L$,
\[ 
\rewardSet \triangleq \left\{ \mu : \left\{0, \dots, T-1 \right\} \rightarrow \left[- L\pa{T -1},  L\right] \;\big{|}\; 0 \leq \mu(n) - \mu(n+1)  \leq L \text{ and } \mu(0) \in \left[0,  L\right] \right\}.
\]
\end{definition}

\begin{remark}
\label{rem:stationary-is-rotting}
We define the set of constant reward function in $\left[0, L\right]$ : 
\[ 
\stationarySet \triangleq \left\{ \mu : \left\{0, \dots, T-1 \right\} \rightarrow \left[0,  L\right] \;\big{|}\;  \mu(n) = \mu_i  \right\}.
\]
We have that $\stationarySet \subset \rewardSet$. Hence, we can conclude that the rotting bandits model include all the stationary bandits problems.
\end{remark}

%\begin{remark}
%Notice that we have that $\mathcal{B}_L \in \rewardSet$ and $\rewardSet \in \mathcal{B}_{LT}$. 
%Therefore, any upper bound on the performance of an algorithm on $\rewardSet$ holds on $\mathcal{B}_L$. Reciprocally, any upper bound derived on $\mathcal{B}_{LT}$ hold on $\rewardSet$. 
%\end{remark}



%
\subsubsection*{Online and offline objectives}
In this chapter, we will only consider deterministic agents which output an arm $i$ at each round $t$. They are degenerate cases of probabilistic agent, which outputs a probability distribution over arm at each round. For the sake of simplicity, we present only the deterministic formalism.   

We will distinguish two types of policies. On the one hand, an offline (or oracle) policy~$\pi \in \PiO$ is a function which maps the round $t$ and the set of reward functions $\mathbb{\mu} \triangleq \left\{ \mu_i \right\}_{i \in \possibleArms}$ to arms, i.e. $\pi(t, \mathbb{\mu}) \in \possibleArms$.  On the other hand, an online (or learning) policy~$\pi \in \PiL$ is a function from the history of observations at time $t$ (which includes the knowledge of the round $t$) to arms, i.e., $\pi(\mathcal{H}_t) \in \mathcal{K}$. For both types of policies, we often use the shorter notation $\pi(t)$, where the dependencies on $\mu$ or $\mathcal{H}_t$ is implicit. 

For a policy $\pi$, let $N_{i,t}^\pi \triangleq \sum_{s=1}^{t} \mathbb{I}\{\pi(s) = i\}$ be the number of pulls of arm $i$ at the end of round $t$. The performance of a policy $\pi$ is measured by the (expected) rewards accumulated over time, 
%
\begin{equation}
\label{eq:cumul-reward}
J_T(\pi) \triangleq \sum_{t=1}^T \mu_{\pi(t)}\pa{N_{\pi(t),t-1}} = \sum_{i \in \possibleArms} \sum_{n=0}^{N_{i,T}^\pi-1} \mu_i(n).
\end{equation}
\begin{remark}
\label{rem:pull-allocation}
The cumulative reward depends only on the number of pull of each arm at the horizon $T$: it does not depend on the specific pulling order of the arms. Hence, two distinct policies with the same pulling allocation at the horizon $T$, \emph{i.e.} $N_{i,T}^{\pi_1} = N_{i,T}^{\pi_2}$ for all $i$, have the same cumulative reward.
\end{remark}

We notice that $\pi \in \PiL$ depends on the (random) history observed over time, and $J_T(\pi)$ is also random for learning policies. The goal of the learning agent is to maximize the expected reward $\EE{J_T(\pi)}$.
%
On the contrary,  oracle policies do not depend on the (random) history. They can be computed entirely before the start of the game. Hence, finding $\pi^\star \in \argmax_{\pi \in \PiO} J_T(\pi)$ is called the \textit{offline problem}. For a given problem $\mathbbm{\mu}$, there is a finite number ($K^T$) of policies, hence the maximum always exists and it could be found by brute-force with infinite computational power.

We set a policy $\pi^\star\in \argmax_{\pi \in \PiO} J_T(\pi)$. Calling $J_T^\star = J_T(\pi^\star)$ the largest cumulative reward achievable, one can measure the regret of any policy (learning or oracle) compared to the optimal one, 
\begin{align}\label{eq:regret}
\regret(\pi) \triangleq J^\star - J_T(\pi).
\end{align}

Let $N_{i,T}^\star \triangleq N_{i,T}^{\pi^\star}$ be the number of times that arm~$i$ is pulled by the oracle policy $\pi^\star$ up to time~$T$ (excluded).  Using Equation~\ref{eq:cumul-reward},  we can conveniently rewrite the regret as
%
\begin{align}
\!\regret(\pi) &= \sum_{i\in\possibleArms}\left( \sum_{n=0}^{N_{i,T}^\star-1}  \reward_{i}(n)  - \sum_{n=0}^{N_{i,T}^\pi-1}  \mu_{i}(n) \right) \nonumber\\ 
& = \sum_{i \in \underpullSet}\sum_{n=N_{i, T}^{\pi}}^{N_{i, T}^{\star}-1} \mu_i(n) - \sum_{i \in \overpullSet} \sum_{n=N_{i, T}^{\star}}^{N_{i, T}^{\pi}-1} \mu_i(n),\label{eq:regret2}
\end{align}

where we define $\underpullSet \triangleq \left\{ \arm \in \possibleArms | N_{i, T}^{\star} > N_{i, T}^{\pi} \right\}$ and likewise $\overpullSet \triangleq \left\{ i\in \possibleArms | N_{i, T}^{\star} < N_{i, T}^{\pi}\right\}$ as the sets of arms that are respectively under-pulled and over-pulled by~$\pi$ with respect to the optimal policy.


\begin{remark}
The regret is measured against an optimal allocation over arms rather than a fixed-arm policy as it is a case in adversarial and stochastic bandits. Therefore, even the adversarial algorithms that one could think of applying in our setting (e.g., \EXP of \citet{auer2002finite}) are not known to provide any guarantee for our definition of regret. Moreover, for constant $\mu_i(n)$-s, our problem and definition of regret reduce to the one of stationary stochastic bandits. 
\end{remark}

We give an upperbound on the regret that holds for any policy and will be used in the analysis of all the presented learning policies. First, we upper-bound all the rewards in the first double sum - the underpulls - by their maximum $\reward^+_T(\pi) \triangleq \max_{i\in\possibleArms} \reward_i(N_{i,T}^\pi)$. Indeed, for any overpulls $\mu_i(n_i) $ (with  $n_i > N_{i,T}^\pi$), we have that
\[
\mu_i(n_i) \leq \mu_i(N_{i,T}^\pi) \leq \max_{i\in\possibleArms} \reward_i(N_{i,T}^\pi),
\]
where the first inequality follows by the non-increasing property of $\mu_i$s; and the second by the defintion of the maximum operator. Second, we notice that there are as many underpulls than overpulls (terms of the second double sum) because there both policies $ \pi$ and $\pi^\star$ pull $T$ arms. Notice that this does \emph{not} mean that for each arm $i$, the number of overpulls equals to the number of underpulls, which cannot happen anyway since an arm cannot be simultaneously underpulled and overpulled. Therefore, we keep only the second double sum,
\begin{equation}
\label{eq:regret-first-bound}
\regret(\pi) \leq \sum_{i\in \overpullSet}   \sum_{n=N_{i,T}^\star}^{N_{i,T}^\pi-1} \pa{\mu^+_T(\pi) - \mu_i(n)}.
\end{equation}

The \textit{online problem} is to find a learning policy which maximizes the expected cumulative reward (or equivalently minimizes the expected regret). In the next sections, we will present the main results of \citet{heidari2016tight}, which has solved the offline problem and the online problem in the absence of noise, and \citet{levine2017rotting}, which has presented the first learning policy with non trivial guarantees for rotting bandits with noise. 

\subsection{The offline problem \citep{heidari2016tight}}
We consider the greedy policy $\GO$ (Alg.~\ref{alg:greedy-oracle}) which at each round selects the arm with the best value.

\begin{minipage}{\textwidth}
\renewcommand*\footnoterule{}
\begin{savenotes}
\begin{algorithm}[H]
\caption{Greedy Oracle $\GO$ (or $\Azero$, \citet{heidari2016tight})}
\label{alg:greedy-oracle}
\begin{algorithmic}[1]
	\Require $\arms$, $\left\{\mu_i\right\}_{i \in \possibleArms}$
	\State Initialize $N_i \leftarrow 0$ for all $i \in \possibleArms$
	\For{$t \gets 1, 2, \dots \do $}
		\State \textsc{Pull}  $i_t \in \argmax_{i \in \possibleArms} \mu_i(N_{i})$\footnote{One can choose the tie break selection rule arbitrarily, e.g. by selecting the arm with the smallest index.}
		\State $N_{i_t} \leftarrow N_{i_t} + 1$
	\EndFor
\end{algorithmic}
\end{algorithm}
\end{savenotes}
\end{minipage}
\begin{proposition}[\citet{heidari2016tight}]
For any reward functions $\mu \in \rewardSet^K$ and any horizon $T$, $\GO \in \argmax_{\pi \in \PiO} J_T(\pi)$.
\end{proposition}%
\begin{proof}

At each round $t$, $\GO$ collects the largest reward that can be available in the future, \textit{i.e.} 
\[
\forall i \in \possibleArms, \ \forall n_i \geq \Nit^{\GO}, \ \mu_{\GO(t)}\pa{N_{\GO(t),\,t}^{\GO}} \geq\mu_{i}\pa{\Nit^{\GO}}  \geq \mu_i(n_i).
\]

The first inequality is due to the selection rule of the policy; the second is due to the decreasing reward functions. 

A direct consequence is that, at round $T$, $\GO$ has selected the $T$ largest reward sample among the $KT$ possible ones. Therefore, any other policy which would select other reward samples can only have worse or equal cumulative reward. 
\end{proof}

According to Remark~\ref{rem:pull-allocation}, for a given horizon $T$, all the policies with the same number of pulls of each arm than $\GO$ at round $T$ have the optimal cumulative reward. Yet, we show in the following Proposition that $\GO$ is the only \emph{anytime} optimal policy.
\begin{proposition}
Let $\pi$ such that $\pi(t) \notin \argmax_{i\in \possibleArms} \mu_i(\Nitpi)$.
\[\text{Then, } J_t(\pi) < \max_{\pi \in \PiO} J_t(\pi).\]
\end{proposition}
\begin{proof}
Let $i^\star_t \in \argmax_{i\in \possibleArms} \mu_i(N_{i,t}^\pi)$. We consider the policy $\pi^+$ which selects the same arm than $\pi$ during the $t-1$ first rounds and selects $i^\star_t$ at round $t$. Therefore, the two policies $\pi$ and $\pi^+$ collects the same rewards except the last one. Notice that before the last round $t$, the two policies have the same pulling allocation $N_{j,\,t-1}^\pi = N_{j,\,t-1}^{\pi^+}$ for all $j \in \possibleArms$.  Hence, there is only a difference between the two last reward samples,
\[ 
J_t(\pi^+) - J_t(\pi) =  \mu_{i^\star_t}(N_{i^\star_t,\,t-1}^{\pi^+}) - \mu_{\pi(t)}(N_{\pi(t),\,t-1}^{\pi}) = \mu_{i^\star_t}(N_{i^\star_t,\,t-1}^{\pi}) - \mu_{\pi(t)}(N_{\pi(t),\,t-1}^{\pi}) > 0.
\]

The inequality follows from $\pi(t) \notin \argmax_{i\in \possibleArms} \mu_i(N_{i,t}^\pi)$ and $i^\star_t \in \argmax_{i\in \possibleArms} \mu_i(N_{i,t}^\pi)$.
\end{proof}

\begin{remark}
\textbf{Complexity.} We have already highlighted that the offline problem is a computational problem. Indeed, the optimal solution can always be computed by brute force by iterating all the possible policies, i.e. with exponential time complexity per round $\cO(K^T)$. By contrast, $\GO$ can be computed with space complexity $\cO(K)$ and time complexity per round $\cO(\log{K})$. Indeed, at each round one should find the maximum among $K$ values. Yet, from one round to another, there is only one value which changes : the value of the last selected arm. Thus, one can store a sorted list of the $K$ arm's value and change one element at each round which costs $\cO(\log{K})$. Then, accessing the first element of the sorted list is a $\cO(1)$ operation.
\end{remark}

To conclude, $\GO$ solves the offline problem in the sense that it provides a cheap way to compute the optimal policy without any knowledge of the horizon $T$. Interestingly, $\GO$ takes the optimal decision by being greedy on the current values. It shows that there is no planning aspect in this problem : the learner never has to sacrifice rewards in the present to get more reward in the future.

\subsection{The noise-free online problem \citep{heidari2016tight}}
In the online problem, the learner does not have access to the current value of the arms. Can they track the best current value using only the observed past values ?  \citet{heidari2016tight} first studied the simpler noise-free problem ($\sigma =0$), where the learner observes the true value of an arm after selecting it (instead of a noisy sample). They suggested the greedy bandit $\GB$ (Alg.~\ref{alg:greedy-bandit}), a policy which selects greedily the arm with the largest last observed value. Indeed, instead of looking at the (unavaible) current values as $\GO$, $\GB$ looks at the closest past. 

\begin{minipage}{\textwidth}
\renewcommand*\footnoterule{}
\begin{savenotes}
\begin{algorithm}[H]
\caption{Greedy Bandit $\GB$ (or $\Atwo$, \citet{heidari2016tight})}
\label{alg:greedy-bandit}
\begin{algorithmic}[1]
\Require $\arms$
\State Initialize $\hat{\mu}_{i}^1 \leftarrow + \infty$ for all $i \in \possibleArms$
	\For{$t \gets 1, 2, \dots \do $}
		\State \textsc{Pull} $i_t \in \argmax_{i \in \possibleArms} \hat{\mu}_{i}^1$\footnote{One can choose the tie break selection rule arbitrarily, e.g. by selecting the arm with the smallest index.}; \textsc{Receive} $o_{t}$
		\State $\hat{\mu}_{i_t}^1 \leftarrow o_{t}$
	\EndFor
\end{algorithmic}
\end{algorithm}
\end{savenotes}
\end{minipage}


\begin{proposition}[\citet{heidari2016tight}]
\label{prop:GB-ub}
For any problem $\mu \in \rewardSet^K$ and any horizon $T$, 
\[\regret (\GB) \leq (K-1)L. \]
\end{proposition}
Surprisingly, the worst case regret is upper-bounded by a constant with respect to $T$. %TODO EXAMPLE.
\begin{proof}
We start from Equation~\ref{eq:regret-first-bound} applied to policy $\GB$,
\begin{equation}
\label{eq:regret-first-bound-GB}
\regret(\GB) \leq \sum_{i\in \overpullSet}   \sum_{n=\NiT^\star}^{\NiT^{\GB}-1} \pa{\mu^+_T(\GB) - \mu_i(n)}.
\end{equation}
%
Let $i \in \arms$ an arm which is pulled at least twice at the end of the game $\NiT^{\GB} \geq 2$. We call $t_i \triangleq \min\left\{t\leq T\; |\; N_{i,t} = N_{i,T}\right\}$ the last round at which $i$ is pulled. For any arm  $j \in \arms$ pulled at least once at the end of the game $\NjT^{\GB} \geq 1$, and for all $n_i \leq \NiT^{\GB} -2$, 
\begin{equation}
\label{eq:overpull-GB1}
\mu_i(n_i) \geq \mu_i(\NiT^{\GB} - 2 ) = \mu_i(N_{i,\,t_i -1}^{\GB} - 1 ) \geq \mu_j(N_{j,\,t_i-1}^{\GB} - 1).
\end{equation}
The first inequality follows by the non-increasing hypothesis on the reward function. The equality follows by definition of $t_i$. The last inequality is by definition of the policy : at time $t_i$, $\GB$ selects $i \in \argmax_{j \in \arms} \mu_j(N_{j,\,t_i-1}^{\GB}-1)$, the largest last observed sample. 

We choose $j$ such that $ \mu_j(\NjT^{\GB}) = \mu^+_T(\GB) \pa{\triangleq \max_{j'\in \possibleArms} \mu_{j'}(N_{j',\,T}^{\GB}}$. \\Since $t_i \leq T$, $N_{j,\,t_i-1}^{\GB} - 1 < \NjT^{\GB}$. By the rotting assumption, 
\begin{equation}
\label{eq:overpull-GB2}
 \mu_j(N_{j,\,t_i-1}^{\GB} - 1) \geq \mu_j(\NjT^{\GB}) = \mu^+_T(\GB).
\end{equation}
%
Gathering Equations~\ref{eq:overpull-GB1} and~\ref{eq:overpull-GB2}, we have that 
\begin{equation}
\label{eq:overpull-GB3}
\forall n_i \leq \NiT^{\GB} \!-\! 2, \;\;  \mu(n) \geq \mu^+_T(\GB).
\end{equation}
Therefore,  we can upper-bound all the before last terms in each second sum in Equation~\ref{eq:regret-first-bound-GB} by zero. Hence, 
\begin{align*}
\regret(\GB) &\leq \sum_{i\in \overpullSet} \pa{\mu^+_T(\GB) - \mu_i(\NiT^{\GB}-1)}\\
&\leq \sum_{i\in \overpullSet} \pa{\mu^+_T(\GB) - \pa{\mu_i(\NiT^{\GB}-2) - L}}\\
&\leq |\overpullSet| L \\
&\leq \pa{K-1} L
\end{align*}
In the second inequality, we used $\mu_i \in \rewardSet$ (see Definition~\ref{def:rew-bounded-decay}). The third inequality follows from Equation~\ref{eq:overpull-GB3}. We can conclude by noticing that they are at most $K-1$ overpulled arm. Indeed, there are as many overpulls than underpulls since the two policies $\pi^\star$ and $\GB$ both pull $T-1$ sample. Hence, if there is at least one overpulled arm, there is necessary at least one underpulled arm. 
\end{proof}

In the next proposition, we state that this rate is minimax optimal at the first order in $\frac{K}{T}$.

\begin{proposition}[\citet{heidari2016tight}]
\label{prop:lb-noisefree}
For any policy $\pi\in \PiL$ and any horizon $T \geq K-1$, there exists a stationary problem $\mu \in \stationarySet \subset \rewardSet$ (see Remark~\ref{rem:stationary-is-rotting}) , 
\[\regret (\pi) \geq (K-1)L \pa{1-\frac{K-1}{T}}. \]
\end{proposition}
We highlight that our proposition is more precise than the one of \citet{heidari2016tight}. Indeed, while they show only a $\cO(K)$ worst case rate, we show that $\pi_G$ is minimax optimal up to a second order term in $\cO\pa{\frac{K}{T}}$. Moreover, we show that this lower bound holds for the easier stationary problem. Hence, it shows that, without noise, rotting bandits are not harder than stationary ones.

\begin{proof}
We consider a set of $K$ problems where 
\begin{itemize}
\item the first arm has always a constant value equals to $L\pa{1- \frac{K-1}{T}}$;
\item problem $p =1$ has all the other arms with a value $0$;
\item problem $p \in \left\lbrace 2, \dots, K \right\rbrace$ has arm $p$ with value $L$ and the other arms $i\in \possibleArms \smallsetminus \left\lbrace 1, i \right\rbrace$ with a value $0$.
\end{itemize}
The learner can distinguish between problem $p \in \left\lbrace 2, \dots, K \right\rbrace$ and problem $1$ only by pulling arm $p$ once. If the learner $\pi \in \PiL$ pulls every arm $i \in \left\lbrace 2, \dots, K \right\rbrace$ once, it suffers on problem $1$,
\[\regret^1\pa{\pi} \geq \pa{K-1}L\pa{1- \frac{K-1}{T}}.\]
If there exists an arm $i \in \left\lbrace 2, \dots, K \right\rbrace$ which is never pulled, $\pi$ suffers on problem $i$,
\[\regret\pa{\pi}^i \geq T \pa{L - L\pa{1- \frac{K-1}{T}}}= L\pa{K-1}.\]
Therefore, we have that for any $\pi$, there exists a stationary problem $\mu \in \stationarySet$ such that,
\[\regret\pa{\pi} \geq \pa{K-1}L\pa{1- \frac{K-1}{T}}\]
\end{proof} 
\begin{remark}
\citet{heidari2016tight} have also studied rested bandits with increasing and concave reward function (without noise).The offline analysis shows that the optimal policy selects always the same arm. This is very different from the rotting case, where the optimal allocation may pull several arms. They suggest an online policy which plays Round-robin on an active set of arms. An arm is excluded from this active set if the optimistic projection of its total available reward untill the end of the game (which can be computed thanks to the concavity assumption) is lower than the pessimistic projection of any other arm (i.e. the arm stays constant). They prove a $o\pa{T}$ regret bound (in the noise-less case !) for this algorithm. While they do not provide a lower bound, it suggests that this problem is harder than the rotting case, where the minimax rate is only in $\cO\pa{KL}$.
\end{remark}





\subsection{\citet{levine2017rotting} : {\wSWA}, a first policy for the noisy problem}
\subsubsection{Sliding-Window Average ({\SWA})}
When the feedback is noisy ($\sigma > 0$), selecting greedily on the last observed reward may be very risky. Indeed, a sample from an optimal pull could be underestimated by $\cO(\sigma)$. $\GB$ may not pull this good underestimated arm for a long time, because it only estimates the value of the arm with the last sample. This behaviour may cause a regret of $\cO(\sigma T)$ which could be much larger from the noise-free rate $\cO(KL)$ depending on the parameters.

\citet{levine2017rotting} suggested to use the Sliding-Window Average (\SWA) policy, a policy which selects the arm with the largest average of its $h$ last sample. Averaging in the presence of noise is a straightforward idea. Yet, it is unclear how the learner should choose $h$. Before going through the detailed analysis, we give the high-level idea. First, we notice that when $h=1$, \SWA reduces to $\GB$. Indeed, intuitively, the smaller the noise, the less averaging we need. On the one hand, with a window $h$, the learner should expect to do $\cO(h)$ overpulls for an arm which abruptly decay at $N_{i,T}^\star$. Indeed, its estimator $\hat{\mu}_i^h$ will be positively bias during the next $h$ pulls. Hence, the learner may suffer up to $\cO(KLh)$ due to this bias. On the other hand, the learner will also take decision based on estimators with variance $\tcO(\frac{\sigma}{\sqrt{h}})$ which may cost $\tcO(\frac{\sigma T}{\sqrt{h}})$ on the long run. Choosing $h = \tcO\pa{\frac{ \sigma T}{KL}}^{2/3}$, we get the regret rate of $\tcO\pa{L^{1/3} \sigma^{2/3} K^{1/3} T^{2/3}}$. 

\begin{minipage}{\textwidth}
\renewcommand*\footnoterule{}
\begin{savenotes}
\begin{algorithm}[H]
\caption{\SWA \citep{levine2017rotting} }
\label{alg:SWA}
\begin{algorithmic}[1]
\Require $\arms$, $h$
\State Initialize $\hat{\mu}_{i}^h \leftarrow + \infty$ for all $i \in \possibleArms$
\State Initialize $\history(i) \leftarrow []$ for all $i \in \possibleArms$
	\For{$t \gets 1, 2, \dots, Kh \do $}
	 	\State \textsc{Pull Round-Robin}  $i_t \gets t \% h $; \textsc{Receive} $o_{t}$
	 	\State $\history(i_t) \leftarrow \history(i_t)\text{.append}(o_{t})$
	\EndFor
	\For{$t \gets Kh + 1, Kh + 2, \dots \do $}
		\State \textsc{Pull}  $i_t \in \argmax_{i \in \possibleArms} \hat{\mu}_{i}^h$\footnote{One can choose the tie break selection rule arbitrarily, e.g. by selecting the arm with the smallest index.}; \textsc{Receive} $o_{t}$
		\State $\history(i_t) \leftarrow \history(i_t)\text{.append}(o_{t})$
		\If{$\text{len}(\history(i_t)) \geq h$}
		\State $\hat{\mu}_{i_t}^h \leftarrow \textsc{Mean}(\history(i_t)[-h:])$
		\EndIf
	\EndFor
\end{algorithmic}
\end{algorithm}
\end{savenotes}
\end{minipage}
\begin{remark}
\SWA uses a rested sliding-window mechanism. Indeed, the window of arm $i$ slides only when arm $i$ is selected. Notice the difference with the restless sliding-window of \SWUCB \citep{garivier2011upper-confidence-bound}, which slides for all arms at every round.
\end{remark}
%
\subsubsection*{Analysis}

The analysis of \citet{levine2017rotting} uses the set of bounded decaying function instead of $\rewardSet$. 

\begin{definition}\label{def:rew-bounded} 
Let $\BBxSet$, the set of non-increasing reward functions with bounded amplitude $B$,
\[ 
\BBxSet \triangleq \left\{ \mu : \left\{0, \dots, T-1 \right\} \rightarrow \left[x , x +B\right] \;\big{|}\; \mu(n) \geq \mu(n+1)  \right\}.
\]
The choice of origin $x$ is not important. Without loss of generality, we will carry the analysis on $\BBSet \triangleq \BSet_{B,0}$. 
\end{definition}
\begin{remark}
\label{rem:BBvsLL}
We have that $\BSet_L \subset \rewardSet$. Hence, any guarantee of any algorithm on $\rewardSet$ applies on $\BBSet$ by setting $L := B$. We also have that $\rewardSet \subset \BSet_{LT, -L\pa{T-1}}$. Hence, any guarantee of any algorithm on $\BBxSet$ applies on $\rewardSet$ by setting $B := LT$.
\end{remark}

\paragraph{Estimators}  
For policy $\pi$, we define the average of the last $h$ observations of arm $i$ at time $t$ as
\begin{equation}
\label{eq:def-hmu}
\widehat{\mu}_i^h(t,\pi) \triangleq \frac{1}{h}\sum_{s=1}^{t-1} \mathbbm{1}\pa{\pi\pa{s}\! =\! i \land N_{i,s}^\pi\!>\! N_{i,t-1}^\pi\! -\! h } o_{s}
\end{equation}
and the average of the associated means as
\begin{equation}
\label{eq:def-bmu}
\bar{\mu}_i^h(t,\pi) \!\triangleq\! \frac{1}{h}\sum_{s=1}^{t-1} \mathbbm{1}\pa{\pi\pa{s}\! =\! i \land N_{i,\,s}^\pi\!>\! N_{i,\,t-1}^\pi\! -\! h } \mu_{i}(N_{i,s-1}^\pi)\,.
\end{equation}

We notice that $\bar{\mu}_i^h(t,\pi) = \frac{1}{h}\sum_{h'=1}^{h} \mu_i(N_{i,\,t-1}^\pi-h') = \bar{\mu}_i^h(N_{i,\,t-1}^\pi)$ . With a slight abuse of notation, we will also use $\hat{\mu}_i^h(\Nit^\pi) \triangleq \hat{\mu}_i^h(t,\pi)$. Indeed, the average of the observations depends on the realization of the noise $\epsilon_t$ at time $t$. Yet, these $h$ samples of noise are i.i.d.\,and thus do not perturb the analysis. 


%Recall Chernoff Hoeffding bound

\paragraph{A favorable event}

\begin{proposition}
\label{prop:prb_favorable_event_SWA}
For a confidence level $\delta_{T} \triangleq 2T^{-3}$
, let
\begin{equation}
\!\HPSWA \! \triangleq\! \Big\{\forall t\!\in\!\left\{Kh +1, \dots, T\right\}, \forall i\!\in\!\mathcal{K},\ \forall n\in \left\{h, \dots, t-1\right\}, \ \big| \ \hmu_i^h(n) - \bmu_i^h(n) \big| \!\leq\! c(\window, \delta_{T}) \!\Big\}
\label{eq:def_favorable_event_SWA}
\end{equation}
be the event under which all the possible estimates constructed at round $t$ are all accurate up to $c(h,\delta_{T}) \triangleq \sqrt{2 \subgaussian^2\log(2/\delta_T)/h}$. Then, for a policy which pulls every arm $h$ times at the beginning (like \SWA),
\[
\PPempty\Big[\bar{\HPSWA}\Big] \leq\frac{K}{T}\,\cdot
\]
\end{proposition} 

\begin{proof}
We want to upper bound the probability
\[
\PP{\bar{\HPSWA}} = \PP{\exists t \in \left\{Kh +1, \dots, T\right\},\exists i \in \arms,\,\exists n\in \left\{h, \dots, t-1\right\}, \big| \hmu_i^h(n) - \bmu_i^h(n)\big|>c(h,\delta_T) }\,.
\]
For $N_{i,\,t-1}^{\piSWA} = n$, we have that, 
\[
 \hmu_i^h(n) - \bmu_i^h(n)= \frac{1}{h} \sum_{s=1}^{t-1}\mathbbm{1}\pa{i_s = i \, |\, N_{i,s}> n - h }\epsilon_s\,.
\]
By Doob's optional skipping (e.g. see \citet{chow1997probability}, Section 5.3) there exists a sequence of random independent variable $(\epsilon'_l)_{l\in\NN}$ , $\sigma^2$ sub-Gaussian such that 

\[\hmu_i^h(n) - \bmu_i^h(n)= \frac{1}{h} \sum_{s=1}^{t-1}\mathbbm{1}\pa{i_s = i \, |\, N_{i,s}> n - h }\epsilon_s=  \frac{1}{h} \sum_{l=n-h+1}^n \epsilon'_l \triangleq \hepsilon^h_n. \]

Hence, 
\begin{align*}
    &\PP{\exists t \in \left\{Kh +1, \dots, T\right\},\exists i \in \arms,\,\exists n \in \left\{h, \dots, t-1\right\}, \big| \hmu_i^h(n) - \bmu_i^h(n)\big|>c(h,\delta_T) }\\
    &\qquad= \PP{\exists t\leq T,\exists i \in \arms,\,\exists n \in \left\{h, \dots, t-1\right\},|\hepsilon^h_n|>c(h,\delta_T) }\\
    &\qquad\leq \sum_{t=Kh + 1}^{T}\sum_{i \in \arms} \sum_{n=h}^{t-1} \PP{|\hepsilon^h_n|>c(h,\delta_T)} \\
    &\qquad\leq  \frac{KT(T-1)}{2}\cdot \delta_T  \\
    &\qquad \leq \frac{K}{T}\,,
\end{align*}
where we used the Chernoff inequality at the before last line and $\delta_{T} = 2T^{-3}$ at the last one. 
\end{proof}


\begin{remark}
\label{rem:uncorrect-levine}
Notice that \citet{levine2017rotting} suggests to use $\delta_T = \frac{1}{T^2}$ to recover the same probability $\frac{K}{T}$. They argue that \SWA only uses less than $KT$ statistics along a trajectory. Yet, this argument is wrong. Indeed, $\Nit^{\piSWA}$ is a random variable which depends on the past observations. When an arm  %TODO
\end{remark}


 
\paragraph{Regret upper-bound}

\begin{proposition}[\citet{levine2017rotting}]
\label{prop:SWA}
 For a problem $\Bmu \in \BBSet^K$, the expected regret of \SWA tuned with $h$ is bounded as
 \[
\EE{ \regret(\piSWA)} \leq 2\sigma T\cdot\sqrt{\frac{6\log\pa{T}}{h}} + K\pa{h+1}B
 \]
\end{proposition}

\begin{proof}
We split the regret on the events $\HPSWA$ and $\bar{\HPSWA}$, 
\[
\EE{\regret(\piSWA)} \leq \EE{\mathbbm{1} \Big[\HPSWA\Big] \regret(\piSWA)} + \EE{\mathbbm{1} \left[\bar{\HPSWA}\right] \regret(\piSWA)}.
\]
The regret on the unfavorable event $\mathbbm{1} \left[\bar{\HPSWA}\right]$ can be bounded by the maximal regret $LT$ (since $\mu \in \BBSet^K$), 
\[
\EE{\regret(\piSWA)} \leq  \EE{\mathbbm{1} \Big[\HPSWA\Big] \regret(\piSWA)} + \PP{\bar{\HPSWA}} BT.
\]

Using Proposition~\ref{prop:prb_favorable_event_SWA}, we get,
\begin{equation}
\label{eq:regret-unfav-event-SWA}
\EE{\regret(\piSWA)} \leq  \EE{\mathbbm{1} \Big[\HPSWA\Big]  \regret(\piSWA)} + KB.
\end{equation}


We will now bound the regret on the favorable event,
\[
\regret(\piSWA | \HPSWA) \triangleq \mathbbm{1} \Big[\HPSWA\Big]  \regret(\piSWA)
\]

We start from Equation~\ref{eq:regret-first-bound} applied to policy $\SWA$,
\begin{equation}
\label{eq:regret-first-bound-SWA}
\regret(\piSWA| \HPSWA) \leq  \mathbbm{1} \Big[\HPSWA\Big] \sum_{i\in \overpullSet}    \sum_{n=\NiT^\star}^{\NiT^{\piSWA}-1} \pa{\mu^+_T(\piSWA) - \mu_i(n)}.
\end{equation}
%
The remaining of the proof is similar to the proof of Proposition~\ref{prop:GB-ub} about algorithm $\GB$. Instead of showing that the before last terms in the sums are equals to zeros, we will show that the terms before $h$ last one are less than $2c(h, \delta_T)$. Let $i \in \arms$ an arm which is pulled at least $h+1$ times at the end of the game $\NiT^{\piSWA} \geq h+1$. We call $t_i \triangleq \min\left\{t\leq T\; |\; \Nit^{\piSWA} = \NiT^{\piSWA}\right\}$ the last round at which $i$ is pulled. For any arm $j \in \arms$ pulled at least $h$ at the end of the game $\NjT^{\piSWA} \geq h$, and for all $n_i \leq \NiT^{\piSWA} -(h+1)$, 
\begin{align}
\label{eq:overpull-SWA1}
\mu_i(n_i) &\geq \mu_i(\NiT^{\piSWA} - (h+1) )\nonumber\\
 &\geq \bmu_i^h(N_{i,\,t_i -1}^{\piSWA}) \nonumber\\
 &\geq \hmu_i^h(N_{i,\,t_i -1}^{\piSWA}) - c(h, \delta_T) \nonumber\\
& \geq \hmu_j^h(N_{j,\,t_i -1}^{\piSWA}) - c(h, \delta_T)  \nonumber\\
& \geq \bmu_j^h(N_{j,\,t_i -1}^{\piSWA}) - 2c(h, \delta_T). 
\end{align}
The first inequality follows by the non-increasing hypothesis on the reward function. The second inequality is because $\bmu_i^h(N_{i,\,t_i -1}^{\piSWA})$ is the average of $h$ reward sample of arm $i$ after the $\NiT^{\piSWA} - (h+1)$-th. The third and fifth one use the concentration of all the possible estimates on the event $\HPSWA$.  The fourth  inequality follows by definition of the policy : at time $t_i$, $\piSWA$ selects $i \in \argmax_{j \in \arms} \hmu_j^h(N_{j,\,t_i -1}^{\piSWA})$, the largest last observed sample. 

We choose $j$ such that $ \mu_j(\NjT^{\piSWA}) = \mu^+_T(\piSWA) \pa{\triangleq \max_{j'\in \possibleArms} \mu_{j'}(N_{j',\,T}^{\piSWA})}$. \\Since $t_i \leq T$, by the rotting assumption, 
\begin{equation}
\label{eq:overpull-SWA2}
 \bmu_j^h(N_{j,\,t_i -1}^{\piSWA}) \geq \mu_j(\NjT^{\piSWA}) = \mu^+_T(\piSWA).
\end{equation}
%
Gathering Equations~\ref{eq:overpull-SWA1} and~\ref{eq:overpull-SWA2}, we have that 
\begin{equation}
\label{eq:overpull-SWA3}
\forall n_i \leq \NiT^{\piSWA} \!-\! \pa{h+1}, \;\;   \pa{\mu^+_T(\piSWA) - \mu_i(n_i)} \leq 2c(h, \delta_T).
\end{equation}
Therefore,  in Equation~\ref{eq:regret-first-bound-SWA}, we can split the sum on $\NiT^{\piSWA} \!-\! h$. Hence, 

\begin{align}
\regret(\piSWA| \HPSWA) \leq&  \mathbbm{1} \Big[\HPSWA\Big] \sum_{i\in \overpullSet}    \sum_{n=\NiT^\star}^{\NiT^{\piSWA}-1} \pa{\mu^+_T(\piSWA) - \mu_i(n)}\nonumber\\
=&  \mathbbm{1} \Big[\HPSWA\Big] \sum_{i\in \overpullSet}    \sum_{n=\NiT^\star}^{\NiT^{\piSWA}-\pa{h+1}} \pa{\mu^+_T(\piSWA) - \mu_i(n)} \nonumber\\ 
&+ \mathbbm{1} \Big[\HPSWA\Big] \sum_{i\in \overpullSet} \sum_{n=\NiT^{\piSWA} - h }^{\NiT^{\piSWA}-1} \pa{\mu^+_T(\piSWA) - \mu_i(n)}\nonumber\\
\leq& 2Tc\pa{h,\delta_T} + KhB.
\label{eq:regret-fav-event-SWA}
\end{align}
In the last inequality, we used Equation~\ref{eq:overpull-SWA3} and that there is less than $T$ overpulls in the first sums. We also use $\mu \in \BBSet$ to bound each term in the second sum by $B$. Finally,  we can conclude by plugging Equation~\ref{eq:regret-fav-event-SWA} in Equation~\ref{eq:regret-unfav-event-SWA} and by using the definition of $c\pa{h,\delta_T}$ and $\delta_T= 2T^{-3}$ in Proposition~\ref{prop:prb_favorable_event_SWA},
\[
\EE{\regret(\piSWA)} \leq 2\sigma T\cdot\sqrt{\frac{6\log\pa{T}}{h}} + K\pa{h+1}B
\]
\end{proof}
\begin{corollary}[\citet{levine2017rotting}]
\label{cor:SWA}
For $C$ such that $h:= \ceil{C \pa{\frac{\sigma T}{KB}}^{2/3}\pa{6\log\pa{T}}^{1/3}}$, 
\[
\regret(\piSWA) \leq \pa{\frac{2}{C^{1/2}} + C} \pa{6 \sigma^2 B K T^2 \log\pa{T}}^{1/3} + 2KB. 
\]

Hence, if the learner knows $T$ and the ratio $\frac{\sigma}{B}$, they can set $h:= \ceil{\pa{\frac{\sigma T}{KB}}^{2/3}\pa{6\log\pa{T}}^{1/3}}$ (i.e. $C=1$) and be guaranteed to perform 
\[
\regret(\piSWA) \leq 6 \pa{\sigma^2 B K T^2 \log\pa{T}}^{1/3} + 2KB. 
\]

\end{corollary}

\begin{remark}
\label{rem:comparaison-levine}
We highlighted in Remark~\ref{rem:uncorrect-levine} that we have to use tighter confidence level $\delta_T$  in the analysis that what \citet{levine2017rotting} suggest. It slightly impacts the theoretical optimal choice of the window as they recommand  $h:= \ceil{\pa{\frac{\sigma T}{KB}}^{2/3}\pa{4\log\pa{\sqrt{2}T}}^{1/3}}$.
\end{remark}

\subsubsection*{Empirical evaluation of the anytime version {\wSWA}}
The theoretical window choice require the knowledge of the horizon $T$, the subgaussian parameter $\sigma$ and the reward range $B$ (or at least the ratio $\frac{B}{\sigma}$). \citet{levine2017rotting} suggest \wSWA, which wraps \SWA  with the doubling trick. The algorithm is initialized with a first (small) guess of the horizon. When the horizon is reached, the algorithm is fully reinitialized with a doubled horizon. This is a classic trick in the litterature : it is known to recover the problem-independent rate of a given algorithm (with a worse constant factor), but the empirical performance is often significantly reduced \citep{besson2018}. In the case of \wSWA, the doubling trick erases all the history $\history_t$ and increases the window. In Algorithm~\ref{alg:wSWA}, we reproduce the version suggested by \citet{levine2017rotting} (without the small modification of the tuning $h$) . 

\begin{minipage}{\textwidth}
\renewcommand*\footnoterule{}
\begin{savenotes}
\begin{algorithm}[H]
\caption{\wSWA \citep{levine2017rotting} }
\label{alg:wSWA}
\begin{algorithmic}[1]
\Require $\alpha$, $\sigma$, $T_0 \gets 1$
\State $T \gets T_0$
\State $h \gets \ceil{\alpha\pa{\frac{4\sigma T}{K}}^{2/3}\pa{\log\pa{\sqrt{2}T}}^{1/3}}$
\For{$t \gets 1, 2, \dots, T \do $}
		\State \textsc{Run} \SWA(h)
	\EndFor
\State \textsc{Clean \SWA's \textsc{Memory}}
\State \wSWA($\alpha$, $\sigma$, $2T_0$) 
\end{algorithmic}
\end{algorithm}
\end{savenotes}
\end{minipage}

We notice that the parameter $\alpha$ of \wSWA hide the dependency in $B$. Indeed, the best theoretical tuning corresponds to $\alpha := \pa{2B}^{-2/3}$. In their experimental section, \citet{levine2017rotting} select $\alpha:= 0.2$ by grid-search on one problem. Yet, the reader should not forget that the tuning of $\alpha$ is dependent on $B$, and more generally on which $\Bmu \in \BBSet$. 



%TODO figures

\subsection{Open problems}

\subsubsection{Minimax rate}
We report existing regret bounds for two special cases. First, in Proposition~\ref{prop:lb-noisefree}, \citet{heidari2016tight} show that in the absence of noise, the regret is lower bounded by $\cO\pa{KL}$. Second, we recall the minimax regret lower bound for stochastic stationary bandits.

\begin{proposition}{\cite[Thm.\,5.1]{auer2002nonstochastic}}
\label{stochastic-LB}
For any learning policy $\policy$ and any horizon $T$, there exists a stochastic stationary problem $\left\{ \mu_i (n) \triangleq \mu_i\right\}_i$ with $K$ $\sigma$-sub-Gaussian arms such that $\pi$ suffers a regret
\begin{equation*}
%\max_{\left\{ \mu_i \in [0,L] \right\}_i}
 \mathbb{E}[\regret(\policy)] \geq \frac{\sigma}{10}\min\pa{\sqrt{\narms\timeEnd},\timeEnd}.
\end{equation*}
where the expectation is w.r.t.\ both the randomization
over rewards and algorithm's internal randomization.
\end{proposition}

Any problem in the two settings above is a rotting problem with parameters ($\sigma$, $L$). Therefore, the performance of any algorithm on the general rotting problem is also bounded by these two lower bounds. For reward functions in $\BBSet$, \SWA is guaranteed to achieve $\cO\pa{T^{2/3}}$ regret rate. Yet, \citet{levine2017rotting} do not provide a lower bound while they suggest it could be an interesting future work direction.

\subsubsection{Problem-dependent rate}
\SWA starts by pulling every arm $h$ times. It means that even for simple stationary problem with large difference $\Delta_i > \sigma$ between suboptimal and optimal arms, \SWA does $h = \cO\pa{T^{2/3}}$ mistakes per suboptimal arms which is much more than the standard $\cO\pa{\frac{\sigma\log\pa{T}}{\Delta_i^2}}$.

More generally, it is an open-question whether it is possible to get problem-dependent guarantees - which depends on the values $\mu_i(n)$ - while keeping 


\subsubsection{Agnostic algorithm}
\SWA requires the knowledge of the horizon $T$, the subgaussian parameter $\sigma$ and the reward range $B$  to tune the window $h$. We showed empirically that the doubling trick leads to large regret increases at each restart. We also showed that the tuning of $h$



\subsubsection{Global budget or Budget per round}
The guarantee

%!TEX root = ../main.tex 

\section{{\FEWA} and {\RAW} : Two adaptive window algorithms}

\subsection{}
Since the expected rewards $\mu_i$ change from one pull to another, the main difficulty in the rested rotting bandits is that we cannot rely on all samples observed until time~$t$ to predict which arm is likely to return the highest reward in the future. In fact, the older a sample, the less representative it is for future rewards. This suggests constructing estimates using the more recent samples. Nonetheless, discarding older rewards reduces the number of samples used in the estimates, thus increasing their variance. 

\SWA chooses a window which balances the cost due to variance and the cost due to bias. %TODO Comment figure 1 and explain why it is stupid.


%TODO : A Favorable event 

\subsection{{\FEWA}: Filtering on expanding window average}%\label{Algorithm}

 In Alg.\,\ref{EWA} we introduce \myAlgorithm (or~$\EWA$) that at each round $t$, relies on estimates using windows of increasing length to filter out arms that are suboptimal with high probability and then pulls the least pulled arm among the remaining arms. 

We first describe the subroutine {\small\textsc{Filter}} in Alg.\,\ref{filter}, which receives a set of active arms $\mathcal{K}_h$, a window~$h$, and a confidence parameter $\delta$ as input and returns an updated set of arms $\mathcal{K}_{h+1}$. For each arm~$i$ that has been pulled~$n$ times, the algorithm constructs an estimate $\estReward^\window_\arm(n)$ that averages the $h \leq n$ most recent rewards observed from~$i$. %The estimator is well defined only for  and the construction of the set $\mathcal{K}_h$ and the stopping condition at Line~\ref{algline:condition} in Alg.\,\ref{EWA} guarantee that $\estReward^\window_\arm(\armCount_{\arm,\currentTime})$ are always well defined for the arms in $\mathcal{K}_h$. 
The subroutine {\small\textsc{Filter}} discards all the arms whose mean estimate (built with window~$h$) from $\mathcal{K}_h$  is lower than the empirically best arm by more than twice a threshold $c(\window, \delta_\currentTime)$ constructed by standard Hoeffding's concentration inequality (see Prop.\,\ref{prop:heoffding}). %TODO Update reference


\begin{algorithm}[t]
\caption{\myAlgorithm}
\label{EWA}
\begin{algorithmic}[1]
\REQUIRE $\subgaussian$, $\possibleArms$, $\delta_0$, $\alpha$
	\STATE pull each arm once, collect reward, and initialize $N_{\arm,K} \leftarrow 1$ 
	\FOR{$\currentTime \gets K+1, K+2, \dots \do $}
		\STATE $\delta_t \leftarrow \delta_0/(t^\alpha)$
		\STATE $\window \leftarrow 1$ 
		{\footnotesize \COMMENT{\emph{initialize bandwidth}}}
		\STATE $\possibleArms_1 \leftarrow \possibleArms$ 
		{\footnotesize \COMMENT{\emph{initialize with all the arms}}}
		\STATE $\arm(t) \gets {\tt none}$
		\WHILE{$\arm(t)$ is  ${\tt none}$}
			\STATE $\possibleArms_{\window+1} \leftarrow {\textsc{Filter}}(\possibleArms_{\window} ,\window, \delta_\currentTime)$
			\STATE $\window \leftarrow \window+1$ \label{algline:window}
			\IF{$\exists \arm \in \possibleArms_{\window}$ such that $\armCount_{\arm, t}=h$}
			\label{algline:condition}
			\STATE $\arm(t) \leftarrow \arg\min_{i\in\possibleArms_{\window}} N_{i,t}$
			\ENDIF
		\ENDWHILE
		\STATE  receive $\obs_\arm(\armCount_{\arm,\currentTime +1 }) \leftarrow \obs_{\arm(\currentTime),\currentTime}$
		\STATE $\armCount_{\arm(\currentTime),\currentTime} \leftarrow \armCount_{\arm(\currentTime),\currentTime-1} +1$
		\STATE $\armCount_{j,\currentTime} \leftarrow \armCount_{j, \currentTime-1}, \quad \forall j \neq \arm(\currentTime)$
	\ENDFOR
\end{algorithmic}

\end{algorithm}

%\end{minipage}
%
%\hfill
%\begin{minipage}{0.5\textwidth}
\begin{algorithm}[t]
\caption{{\textsc{Filter}}}
\label{filter}
\begin{algorithmic}[1]
\REQUIRE $\possibleArms_{\window}$, $\window$, $\delta_\currentTime$
\STATE $c(\window, \subgaussian, \delta_\currentTime) \leftarrow \sqrt{(2\subgaussian^2/\window) \log{(1/\delta_\currentTime)}}$
\FOR{$ \arm \in \possibleArms_{\window}$}
\STATE $\estReward^\window_\arm(\armCount_{\arm,\currentTime}) \leftarrow \frac{1}{\window} \sum_{j=1}^\window \obs_\arm(\armCount_{\arm,\currentTime} -j)$
\ENDFOR
\STATE $\estReward^\window_{\max,\currentTime}  \leftarrow \max_{\arm \in \possibleArms_{\window}}\estReward^\window_\arm(\armCount_{\arm,\currentTime})$
\FOR{$ \arm \in \possibleArms_{\window}$}
	\STATE $\Delta_\arm \leftarrow  \estReward^\window_{\max,\currentTime}  - \estReward^\window_\arm(\armCount_{\arm,\currentTime})$
	\IF{$\Delta_\arm \leq 2c(\window, \subgaussian, \delta_\currentTime) $}
	\STATE add $\arm$ to $\possibleArms_{\window+1}$
	\ENDIF
\ENDFOR
\ENSURE $\possibleArms_{\window+1}$
\end{algorithmic}
\end{algorithm}
%\end{minipage}



The {\small\textsc{Filter}} subroutine is used in \myAlgorithm to incrementally refine the set of active arms, starting with a window of size $1$, until the condition at Line~\ref{algline:condition} is met. As a result, $\mathcal{K}_{h+1}$ only contains arms that passed the filter for all windows from $1$ up to $h$. Notice that it is important to start filtering arms from a small window and to keep refining the previous set of active arms. % instead of completely recomputing them for every new window $h$. 
In fact, the estimates constructed using a small window use recent rewards, which are closer to the future value of an arm. As a result, if there is enough evidence that an arm is suboptimal already at a small window $h$, it should be directly discarded. On the other hand, a suboptimal arm may pass the filter for small windows as the threshold $c(\window, \subgaussian, \delta_\currentTime)$ is large for small $h$ (i.e., as few samples are used in constructing $\estReward^\window_\arm(\armCount_{\arm,\currentTime})$, the estimation error may be high). Thus, \myAlgorithm keeps refining $\mathcal{K}_{h}$ for larger windows in the attempt of constructing more accurate estimates and discard more suboptimal arms. This process stops when we reach a window as large as the number of samples for at least one arm in the active set $\mathcal{K}_{h}$ (i.e., Line~\ref{algline:condition}). At this point, increasing $h$ would not bring any additional evidence that could refine $\mathcal{K}_{h}$ further (recall that $\estReward^\window_\arm(\armCount_{\arm,\currentTime})$ is not defined for $h > \armCount_{\arm,\currentTime}$). Finally,  \myAlgorithm selects the active arm $i(t)$ whose number of samples matches the current window, i.e., the least pulled arm in $\mathcal{K}_{h}$. The set of available rewards and the number of pulls are then updated accordingly. 

\paragraph{Active set} We derive an important lemma that provides support for the arm selection process obtained by a series of refinements through the {\small \textsc{Filter}} subroutine. Recall that at any round $t$, after pulling arms $\{ \armCount^{\EWA}_{\arm,\currentTime} \}_i$ the greedy (oracle) policy would select an arm 
%
\begin{align*}
\arm^\star_\currentTime \pa{\left\{ \armCount^{\EWA}_{\arm,\currentTime} \right\}_i}  \in  \argmax_{\arm \in \possibleArms} \reward_\arm \left( \armCount^{\EWA}_{\arm,\currentTime}\right).
\end{align*}
%
We denote by $\reward^+_t(\EWA) \triangleq \max_{\arm \in \possibleArms} \reward_\arm ( \armCount^{\EWA}_{\arm,\currentTime}),$ the reward obtained by pulling~$\arm^\star_\currentTime.$ The dependence on $\EWA$ in the definition of $\reward^+_t(\EWA)$ stresses the fact that we consider what the oracle policy would do at the state reached by $\EWA$.
%In the following, we drop the dependency on the number of pulls and we use $i^\star_t$ to denote the greedy arm at round $t$. 
While \myAlgorithm cannot directly match the performance of the oracle arm, the following lemma shows that the reward averaged over the last $h$ pulls of any arm in the active set is close to the performance of the oracle arm up to four times $c(\window,  \delta_\currentTime)$.

\begin{restatable}{lemma}{restafundamentallemma}
\label{fundamental-lemma}
On the favorable event $\HPevent_t$, if an arm~$\arm$ passes through a filter of window $\window$ at round $\currentTime$, i.e., $i\in\ \mathcal{K}_h$, then the average of its $\window$ last pulls satisfies
%
\begin{equation}\label{eq:fundamental.eq}
\expestReward^{\window}_\arm(\armCount_{\arm,\currentTime}^{\EWA} ) \geq  \reward^+_t(\EWA) - 4 c(\window, \delta_\currentTime).
\end{equation}
%
\end{restatable}
This result  relies heavily on the non-increasing assumption of rotting bandits. In fact, for any arm $i$ and any window $h$, we have
%
\begin{equation*}
\wb\mu_i^h(N_{i,t}^{\EWA}) \geq \wb\mu_i^1(N_{i,t}^{\EWA}) \geq \mu_i(N_{i,t}^{\EWA}).
\end{equation*}
%
While the inequality above for $i_t^*$ trivially satisfies Eq.\,\ref{eq:fundamental.eq}, Lem.\,\ref{fundamental-lemma} is proved by integrating the possible errors introduced by the filter in selecting active arms due to the error of the empirical estimates.

\begin{restatable}{corollary}{restafundamentalcorrelary}\label{fundamental-correlary}
	%\begin{corollary}
	Let $\arm \in \overpullSet$ be an arm overpulled by {\FEWA} at round $t$ and $\window_{\arm,t} \triangleq \armCount_{\arm, t}^{\EWA} - \armCount_{\arm, t}^{\policy^\star} \geq 1$ be the difference in the number of pulls w.r.t.\,the optimal policy $\pi^\star$ at round $t$. On the favorable event $\HPevent_t$,  we  have
	\begin{align}
	\reward^+_t(\EWA) - \expestReward^{\window_{\arm,t}}_i(\armCount_{\arm,t}) \leq  4 c(\window_{\arm,t}, \delta_t).
	\end{align}
\end{restatable}


\subsection{The {\EUCB} algorithm}
\label{sec:algo}


\paragraph{A general index policy}
We will study a single class of policies which select at each round $t$ the arm with the maximal index of the form
\vspace{-4pt}
\begin{align}
\label{eq:xindex}
\operatorname{ind}(i,t, \delta_{t}) \triangleq \min_{h\leq N_{i,t-1}} {\widehat{\mu}_i^h(t-1,\pi) + c(h,\delta_{t})}.
\end{align}
We set $\delta_{t} \triangleq \frac{1}{t^\alpha}$ and  call this algorithm Rotting Adaptive Window UCB (\EUCB). There is  a bias-variance trade-off for the window choice: more variance for smaller size of the window $h$ and more bias for larger $h$. The goal of \XUCB is to adaptively select the right window to compute the tightest UCB. \XUCB uses the indexes of \UCBone computed on all the slices of each arm's history which include the last pull. When the rewards are rotting, all these indexes are upper confidence bounds on the \textit{next value}.  Thus, \XUCB simply selects the tightest (minimum) one as index of the arm: it is a pure UCB-index algorithm. By contrast, when reward can increase, the learner can only derive upper-confidence bound on past values which are loosely related to the next value. Hence, all the UCB-index algorithms in the restless non-stationary literature need to add change-detection sub-routine, active random exploration or passive forgetting mechanism. 

\begin{figure}
\vspace{-10pt}
\bookboxx{
\begin{algorithmic}[1]
\REQUIRE$(\delta_t)_{t\geq 1}$
	\STATE pull each arm once
	\STATE initialize $N_{\arm,K} \leftarrow 1$
	\FOR{$\currentTime \gets K+1, K+2, \dots \do $}
		\STATE $\arm_t \gets \argmax_i\operatorname{ind}(i,t, \delta_{t})$ 
		{\footnotesize \COMMENT{\emph{cf.\,\eqref{eq:xindex}}}}
		\STATE  receive reward $\obs_t$%$\obs_\arm(\armCount_{\arm,\currentTime +1 }) \leftarrow \obs_{\arm(\currentTime),\currentTime}$
		\STATE $\armCount_{\arm_\currentTime,\currentTime} \leftarrow \armCount_{\arm_\currentTime,\currentTime-1} +1$
		\STATE $\armCount_{j,\currentTime} \leftarrow \armCount_{j, \currentTime-1}, \quad \forall j \neq \arm_\currentTime$
	\ENDFOR
\end{algorithmic}
\caption{The \EUCB algorithm} \label{algo:xucb}}
\end{figure}

\begin{restatable}{lemma}{restafundamentallemma}
\label{fundamental-lemma}
At round $t$ on favorable event $\HPevent_t$, if arm~$i_{t}$ is selected, for any $h \leq N_{i,t-1}$,  the average of its $\window$ last pulls cannot deviate significantly from the best available arm at that round, i.e.,
%
\vspace{-4pt}
\begin{equation*}
\bar{\mu}^{h}_{i_{t}}(t-1,\pi) \geq \max_{i \in \possibleArms}\mu_i(t,N_{i,t-1}) - 2 c(h, \delta_{t}).
\end{equation*}
\end{restatable}

This fundamental guarantee is comparable with Corollary~ %TODO corollary link
about the algorithm \FEWA. \FEWA uses the same statistics than \XUCB but in a rather complex expanding filtering mechanism. \EUCB has tighter guarantees than \FEWA (2 versus 4 confidence bands), which is the benefit of upper confidence bounds index policies over confidence bound filtering policies. 

%!TEX root = ../main.tex 
\section{Regret Analysis}\label{sec:theory}

We first give problem-independent regret bound for \FEWA and \RAWUCB and sketch its proof in Sect.\,\ref{sketch}. Then, we derive problem-dependent guarantees in Sect.\,\ref{ss:dep}.

%TODO : we only use lemmas

\subsection{Sketch of the proof}
\label{sketch}
%TODO 3 ingredients : 
% 1) The regret upperbound  
% 2) Proposition low probability event
% 3) Lemmas

%TODO General Lemma 


\subsection{Problem-independent bound}
\begin{restatable}{theorem}{restaalgoindepub}
\label{independent_theorem}
For any rotting bandit scenario with means $\{\mu_i(n)\}_{i,n}$ satisfying Asm.\,\ref{assum-Lipschitz} with bounded decay~$L$ and any time horizon $T$, {\myAlgorithm} run with $\alpha= 5$ and $\delta_\currentTime= 1/( t^5),$ suffers an expected regret\,\footnote{See Corollary~\ref{cor:HP-minimax} and~\ref{cor:HP-PD} for the high-probability result.} of
\begin{equation*}
\mathbb{E}[\regret(\EWA)] \leq 13\subgaussian(\sqrt{KT} + K)\sqrt{\log(T)}+ 2KL.
\end{equation*}%
\end{restatable}%
\paragraph{Comparison to \citet{levine2017rotting}} The regret of \SWA is bounded by $\tcO(\mu_{\max}^{1/3}K^{1/3} T^{2/3})$ for rotting functions with range in $[0,\mu_{\max}]$. In our setting, we do not restrict rewards to stay positive but we bound the per-round decay by $L$, thus leading to rotting functions with range in $\left[-LT, L\right]$. As a result, when applying \SWA to our setting, we should set $\mu_{\max}=L(T+1)$, which leads to $\cO(T)$ regret, thus showing that according to its original analysis, \SWA may not be able to learn in our general setting. On the other hand, we could use \myAlgorithm in the setting of \citet{levine2017rotting} by setting $L = \mu_{\max}$ as the largest drop that could occur. In this case, \myAlgorithm suffers a regret of $\tcO(\sqrt{KT})$, thus significantly improving over \SWA. The improvement is mostly due to the fact that \myAlgorithm exploits filters using moving averages with increasing windows to discard arms that are suboptimal w.h.p. Since this process is done at each round, \myAlgorithm smoothly tracks changes in the value of each arm, so that if an arm becomes worse later on, other arms would be recovered and pulled again. On the other hand, \SWA relies on a fixed exploratory phase where all arms are pulled in a round-robin way and the tracking is performed using averages constructed with a fixed window. Moreover, \myAlgorithm is anytime, while the fixed exploratory phase of \SWA requires either to know $T$ or to resort to a doubling trick, which often performs poorly in practice. 
%Algorithms (such as \myAlgorithm) with direct anytime guarantees show a practical advantage over the doubling-trick ones, that often give a suboptimal empirical performance.
\paragraph{Comparison to deterministic rotting bandits}
For $\sigma=0$, our upper bound reduces to $KL$, thus matching the prior (upper and lower) bound of~\citet{heidari2016tight} for deterministic rotting bandits. Moreover, the additive decomposition of regret shows that there is \emph{no coupling} between the stochastic problem and the rotting problem as terms depending on the noise level $\sigma$ are separated from the terms depending on the rotting level $L$, while in \SWA these are coupled by a $L^{1/3}\sigma^{2/3}$ factor in the leading term. 
\paragraph{Comparison to stochastic bandits}
The regret of \FEWA matches the worst-case optimal regret bound of the standard stochastic bandits (i.e., $\mu_i(n)$s are constant) up to a logarithmic factor. Whether an algorithm can achieve $\cO(\sqrt{KT})$ regret bound is an open question. On one hand, \FEWA needs confidence bounds to hold for different windows at the same time, which requires an additional union bound and thus larger confidence intervals w.r.t.\,\UCBone. On the other hand, our worst-case analysis shows that some of the difficult problems that reach the worst-case bound of Thm.\,\ref{independent_theorem} are realized with constant 
functions, which is the standard stochastic bandits, for which \MOSS-like~\citep{audibert2009minimax} algorithms achieve regret guarantees without the $\log T$ factor. Thus, the necessity of the extra $\log T$ factor for the worst-case regret of rotting bandits remains an open problem.



\subsection{Problem-dependent guarantees}
\label{ss:dep}
\vspace{-0.1in}

Since our setting generalizes the standard stochastic bandit setting, a natural question is whether we pay any price for this generalization. While the result of~\citet{levine2017rotting} suggested that learning in rotting bandits could be more difficult, in Thm.\,\ref{independent_theorem} we actually proved that \myAlgorithm nearly matches the problem-independent regret $\tcO(\sqrt{\narms\timeEnd})$. We may wonder whether this is true for the \emph{problem-dependent} regret as well.
% As illustrated in the next remark, we show that up to constants, \myAlgorithm performs as well as \UCB on any stochastic problem.

\begin{remark}\label{remarkUCB}
Consider a stationary stochastic bandit setting with expected rewards $\{\mu_i\}_i$ and $\reward_\star \triangleq \max_\arm \reward_\arm$. Corollary~\ref{fundamental-correlary} guarantees that for $\delta_\currentTime \geq 1/\timeEnd^\alpha,$ 
\begin{align}
\reward_\star - \reward_\arm \leq 4c\pa{\window_{\arm,T}-1,  \delta_\currentTime} = 4\sqrt{\frac{2\alpha\subgaussian^2 \log(\timeEnd)}{\window_{\arm,T} -1}}
%\CommaBin
\nonumber\\
\text{\ or equivalently,\ }
%\begin{equation}
\label{eq:LaiRob}
\window_{\arm,T} \leq 1+ \frac{32\alpha \subgaussian^2 \log(\timeEnd)}{(\reward_{\star} - \reward_\arm) ^2}\cdot
\end{align}
\end{remark}
Therefore, our algorithm matches the lower bound of~\citet{lai1985asymptotically} up to a constant, thus showing that learning in the rotting bandits are never harder than in the stationary case. Moreover, this upper bound is at most $\alpha$ larger than the one for \UCBone~\citep{auer2002finite}.\footnote{To make the results comparable to the one of~\citet{auer2002finite}, we need to replace $2\subgaussian^2$ by $\nicefrac{1}{2}$ for sub-Gaussian noise.} The main source of suboptimality is the use of a confidence bound filtering instead of an upper-confidence index policy. Selecting the less pulled arm in the active set is conservative as it requires uniform exploration until elimination, resulting in a factor 4 in the confidence bound guarantee on the selected arm (vs.\,2 for \UCB), which implies 4 times more overpulls than \UCB (see Eq.\,\ref{eq:LaiRob}). We conjecture that this may not be necessarily needed and it is an open question whether it is possible to derive either an index policy or a better selection rule. The other source of suboptimality w.r.t.\,\UCB is the use of larger confidence bands because of the higher number of estimators computed at each round ($Kt^2$ instead of $Kt$ for \UCB).


Remark~\ref{remarkUCB} also reveals that Corollary~\ref{fundamental-correlary} can be used to derive a general problem-dependent result in the rotting case.

In particular, with Corollary~\ref{fundamental-correlary} we upper-bound 
the maximum number of overpulls by a problem dependent quantity
\begin{equation}
\label{eq:hit+}
\window_{\arm,T}^+ \triangleq \max \left\{ \window \leq 
1 + \frac{32\alpha \subgaussian^2 \log(T)}{\Delta_{i,h-1}^2} \right\}\CommaBin
\end{equation}
\[
\text{\qquad where \ } \Delta_{i,h} \triangleq \min_{j \in \possibleArms} \reward_j\pa{N_{j,T}^\star -1} - \bar{\reward}_i^h\left( N_{i,t}^\star+h \right).
\]
%If we reach this upperbound, we know with high probability that the arm is currently suboptimal. 
We then use Corollary~\ref{fundamental-correlary} again to upper-bound the regret caused by $h_{\arm,T}^+$ overpulls for each arm, leading to Corollary~\ref{dependent_theorem} (see the full proof in App.\,\ref{sec:proofdep}). 

\begin{restatable}{corollary}{restaalgoub}\label{dependent_theorem}
For $\delta_\currentTime \triangleq 1/(\currentTime^5)$ and $C_\alpha\triangleq 32\alpha\subgaussian^2$, the regret of \FEWA is  bounded as
\begin{align*}
\mathbb{E}\left[R_T(\EWA)\right]  \leq \sum_{\arm\in \possibleArms} \pa{\frac{C_{5}\log(T)}{\Delta_{i,h_{i,T}^+-1}}+ \sqrt{C_{5}\log(T)} + 2L}.
%\\ 
%\text{\ with  and $h_{i,T}^+$ defined in Equation~\ref{eq:hit+}.}
\end{align*}

\end{restatable}





%!TEX root = ../main.tex 
\section{Experimental benchmarks}
\label{sec:rested-experiment}
We use the two benchmarks described in Subsection~\ref{subsec:rested-experiment1}.

\subsection{Simulated benchmark $\#$1 (2 arms).}
\begin{figure*}[ht]
\centering
\includegraphics[clip, width= 0.51\textwidth]{2.1Rested/fig/fig1A_main.pdf}
\includegraphics[clip, width= 0.49\textwidth]{2.1Rested/fig/fig1B_main.pdf}
\includegraphics[clip, width= 0.49\textwidth]{2.1Rested/fig/fig1C_main.pdf}
\caption{\textbf{Top:} Regret at the end of the game for different values of $L$. \textbf{Bottom:} Regret across time for two values of $L$. Average over 1000 runs. We highlight the $\left[10\%, 90\%\right]$ confidence region.}
\label{fig:rested-exp1}
\end{figure*}

\paragraph{Algorithms.} We display the performance of \RAWUCB and \FEWA for two versions of each algorithm: with the theoretical tuning $\alpha = 4$; and with the empirical tuning $\alpha_{\mathrm{R}} = 1.4$ and $\alpha_{\mathrm{F}} = 0.06$. These two values are selected by grid-search. Though there are 30 different problems (for different $L$), the best tuning of $\alpha$ is the same for all the considered problem. We also include the three versions of \wSWA that we displayed in Subsection~\ref{subsec:rested-experiment1}.


\paragraph{Results - {\RAWUCB} versus {\FEWA}.} We compare \RAWUCB and \FEWA both for theoretical and empirical tuning. For theoretical tuning, we see in Figure~\ref{fig:rested-exp1} (top), that \RAWUCB outperforms \FEWA on all sizes of decays by a factor $\sim 4$ which is predicted by our theory. Indeed, there is also a factor 4 between the two problem-dependent upper-bounds (Theorem~\ref{th:rested-PD}). 

Surprisingly, for empirical tuning, the average performances of the two algorithms are much closer. We also notice that there is a larger variance in \FEWA's result compared to \RAWUCB. This is not surprising because we had to drastically reduce the confidence bounds to make \FEWA practical. It means that empirical \FEWA filters arms based only on a handful of samples. This bet leads to both very good and very bad runs. Last, Figure~\ref{fig:rested-exp1} (bottom) shows that \RAWUCB outperforms \FEWA at almost any time $t$, both on easy ($L=4.24$) and difficult ($L=0.233$) problems. The only round at which \FEWA shows better performance than \RAWUCB is after the regret decay. It is because \FEWA was less good at identifying the best arm in the first part of the game. Hence, just after the decay, it pulls more the other arm - which has become optimal. 

In the following, we will compare \RAWUCB with \wSWA. Notice that a similar comparison can hold for \FEWA ($\alpha=0.06$).

\paragraph{Results - Problem dependent performance and the impact of L.} \RAWUCB with the best empirical tuning improves over \wSWA on each problem (Figure~\ref{fig:rested-exp1} (top)). \RAWUCB with the theoretical tuning recovers quite good performance as well. 

In this setting, $L$ has two different meanings. It is the maximum decay per round (noted as $L$ in the theoretical section) and the gap between arms $\Delta_{2,h} = \nicefrac{L}{2}$ (for any $h$). According to our problem-dependent bound in Theorem~\ref{th:rested-PD}, the regret bound converges to $\cO\pa{KL}$ when $L$ and $\Delta_{2,h}$ are large with respect to $\sigma$. It tends to show that setup where arms are well separated from each other are easy problems for \FEWA and \RAWUCB. It is indeed confirmed in Figure~\ref{fig:rested-exp1} (top), where the regret of \FEWA and \RAWUCB converges to $\nicefrac{L}{2}$ when $L$ is large.

\paragraph{Results - Worst-case improvement.}
In Figure~\ref{fig:rested-exp1} (top), the worst regret for any of the two versions of \RAWUCB is smaller than the worst regret of any of the three versions \wSWA. Moreover, we remark that the regret at the round $T$ has one maximum for the variation of $L$ for \RAWUCB. This is not the case for \wSWA where the regret increases again for large values of $L$.

It confirms our analysis. Indeed, Theorem~\ref{th:rested-PI} shows a larger regret rate than Proposition~\ref{prop:SWA}. Moreover, the analysis shows that the worst cases for \RAWUCB correspond to cases where the learner does $\cO\pa{T}$ mistakes of intermediate size  $\cO\pa{\sqrt{\nicefrac{K}{T}}}$ which corresponds to the single maximum in Figure~\ref{fig:rested-exp1} (top).  

\paragraph{Results - Tuning and agnostic algorithms.}
Figure~\ref{fig:rested-exp1} (top) confirms that \FEWA and \RAWUCB do not rely on the knowledge of $L$. Indeed, the optimal tuning is the same for all the 30 problems. By contrast, the performance of \wSWA depends critically on the prior knowledge of  $L$: each of the three displayed tunings is the best for a specific range of $L$. 

Figure~\ref{fig:rested-exp1} (bottom) shows the advantage of anytime algorithms compared to the doubling trick. Indeed, the periodic restarts are quite expensive for \wSWA.  

\paragraph{Results - High-probability.}
We see that the variance of \wSWA is quite large for intermediate values of $L$. It confirms the analysis of \wSWA which shows two sources of the regret: the variance and the bias of the index. The regrets caused by variance has itself a large variance. Indeed, the sub-optimal arms are often correctly estimated, and hence not pulled by the index policy. It leads to many good runs of \wSWA. However, there are still many runs on which there is a sufficient deviation in the indexes which leads to very large regret. 

By contrast, the variance in the results is much more controlled by \RAWUCB and \FEWA. Indeed, when the statistics of these algorithms are not significant enough they tend to explore which leads to less large deviation of the regret. 

\subsection{Simulated benchmark $\#$2 (10 arms).}

\begin{figure*}[t]
\centering
\includegraphics[width = 0.99 \textwidth]{2.1Rested/fig/fig2_main.pdf}
\caption{\textbf{Left:} Regret at the end of the game for different values of $L$. \textbf{Middle, Right:} Regret across time for two values of $L$. Average over 1000 runs. We highlight the $\left[10\%, 90\%\right]$ confidence region.}
\label{fig:rested-exp2}
\end{figure*}

\paragraph{Algorithms.} We display the same two versions of \FEWA and \RAWUCB. We also show the three best algorithms presented in Subsection~\ref{subsec:rested-experiment1}: two versions of \wSWA with $\alpha\in \left\{0.002, 0.02\right\}$ and \GLRUCB with no exploration. 

\paragraph{Results.}
The comparison between \RAWUCB, \FEWA, and \wSWA leads to a similar conclusion than for the two-arm bandit experiment. \RAWUCB and \FEWA show superior performance, except for the theoretical tuning of \FEWA which is too conservative. 

In particular, these algorithms show a better adaptation to each arm's gap. Indeed, the regret per arm is more controlled, especially for large values of the gaps, on which \wSWA suffers a large regret. There is also less deviation in the regret and we see the benefits of avoiding the doubling trick. 

In the two-arm setup with a single decay, it is possible to find a value of $\alpha$ for which \wSWA is correctly tuned for the specific decay. For instance, for $L\in \left[ 1, 3\right]$, \wSWA with $\alpha = 0.02 $ has almost the same performance than \RAWUCB (Fig.~\ref{fig:rested-exp1}). In the ten-arm setup with multiple decays, this is not possible anymore. Indeed, since there are several dropping values for each arm, there exists at least one arm on which the fixed window of \wSWA is not correctly tuned. For instance, for $\alpha = 0.002$ , \wSWA suffers a large regret on the arm with $\Delta_i = 0.3$. For \wSWA with $\alpha = 0.02$, the regret is large when $\Delta_i = 10$.

\RAWUCB and \FEWA also improve over \GLRUCB when their confidence bounds are tuned. We recall that \GLRUCB is an algorithm that uses a classical UCB index with a change detection procedure. When the change-detection procedure triggers, it erases the history of the changing arm. Notice that the confidence bounds of the index of \GLRUCB are already well-tuned, as they use the same confidence bounds as the asymptotic optimal tuning of \UCB. \GLRUCB shows sub-optimal performance on two arms $\Delta_i \in \left\{0.1, 10\right\}$. \GLRUCB suffers from the late restart for $\Delta_i = 0.1$.  Indeed, the change-point is hard to detect, and the index of the sub-optimal value is positively biased while it has not restarted. For $\Delta_i = 10$, the large regret of \GLRUCB is due to an implementation artefact. Indeed, we used the fast implementation for the change detector (by default in \citep{SMPyBandits}). It speeds up the algorithm but it can delay the change-detection scheme (by 10 pulls in this case). This delay leads to large regret when the mistake associated with each arm is large (as it is the case for $\Delta_i=10$).

\paragraph{Running time.}
\begin{table}[H]
\centering
\begin{tabular}{|c|c|}
\hline
\textbf{Policy} &\textbf{Running time (s)} \\ \hline
\FEWA($\alpha = 0.06$)    & 91                      \\ 
\FEWA($\alpha = 4$)      & 780                     \\ \hline
\RAWUCB ($\alpha = 1.4$) & 27                      \\
\RAWUCB($\alpha = 4$)    & 25                      \\ \hline
\wSWA($\alpha = 0.002$)   & 1                       \\ 
\wSWA($\alpha = 0.02$)    & 1                       \\ \hline
\GLRUCB          & 46 \\ \hline
\end{tabular}
  \caption{Average running time for the 10-arms experiment in seconds.}
  \label{tab:time-fig2}
\end{table}

In Table~\ref{tab:time-fig2}, we display the running time for this experiment. The computational experiments were conducted using the Grid’5000 experimental testbed \citep{grid5000}. For meaningful comparison, all the algorithms run on the same "Grenoble/dahu" cluster (2 CPUs Intel Xeon Gold 6130, 16 cores/CPU, 192GB RAM, 223GB SSD, 447GB SSD, 3726GB HDD, 1 x 10Gb Ethernet, 1 x 100Gb Omni-Path). 

\RAWUCB runs 25 times slower than \wSWA. We will provide a computational analysis in the next section but we can already relate this increased running time with the higher number of statistics \RAWUCB update and compare at each round.

The $\alpha$ parameter of \FEWA has a large impact on the running time. Indeed, the larger the $\alpha$, the less aggressive are the filters, the longer it takes to reach the end of the filtering process. Yet, even when $\alpha$ is small, \FEWA is slower than \RAWUCB. This is a consequence of the simplicity of the index policy over the filtering procedure. Indeed, in Python, we can use the fast C++ implementation of the scientific computing library Numpy to perform the most classical operations. Hence, for \RAWUCB, we only use the NumPy functions $\argmax$ and $\min$ to choose the next arm. For \FEWA, the comparison part is more custom: we had to implement the while-loop at Line~\ref{algline:fewa-while} with a Python loop, which is known to be quite slow. Notice that since the two algorithms use the same statistics we use the same function \UPDATE in both algorithms.

\GLRUCB is slower than \RAWUCB. Notice that it is already a fast version of \GLRUCB which runs the change-detection subroutine sparsely (approximately 10 to 100 times faster than the original \GLRUCB).
\begin{remark}
We emphasize the better characteristics of \RAWUCB over \FEWA: better bounds, better empirical performances, easier and faster implementation, a better agreement between theory and practice, closer to the classical \UCB. For these reasons, we will focus our future empirical investigation on \RAWUCB.
\end{remark}
%!TEX root = ../main.tex 
\section{Efficient algorithms}
\label{app:efficient_alg}

In this section, we refine the efficient trick of \cite{seznec2019rotting}. The core idea is to store and update only few statistics for each arm.
Let $m$ be a parameter in $(1,2]$. We define $s_{i,j}^{c}$ and $s_{i,j}^p$ the $j$-th current and pending statistics of arm $i$. 



Algorithm~\ref{EFFUPDATE} is quite technical. We give here a short description. At each pull, we replace $s_{i,1}^{c}$ and create $s_{i,2}^{p}$ if it is \texttt{None}.
The incoming reward at time $t$ are added to the existing pending statistics. We define $h_j$ recursively : $h_{j+1} \triangleq \ceil{m*h_{j}}$. When the pending statistics contains $h_j$ sample, we set $s_{i,j}^{c} \leftarrow s_{i,j}^{p}$. If $s_{i,j+1}^{p}$ is \texttt{None}, we set $s_{i,j+1}^{p} \leftarrow s_{i,j}^{p}$. After that, we set $s_{i,j}^{p} \leftarrow \texttt{None}$.

By doing that, we ensure that $s_{i,j}^{c}$ contains the empirical average of $h_j$ pulls among the $2h_j-1$ ones. 

We define \EFFU the index policy which selects
\[\argmax_{i \in \possibleArms} \left\{\min_{j}\left(s^c_{i,j} + c(h_j, \delta_t)\right)\right\}\cdot
\]


\begin{restatable}{lemma}{restafundamentallemmaefficient}
\label{fundamental-lemma_efficient}
At round $t$ on favorable event $\HPevent_t$, if arm~$i_{t}$ is selected by $\EFFU$ (m=2), for any $h \leq N_{i,t-1}$,  the average of its $\window$ last pulls cannot deviate significantly from the best available arm at that round, i.e.,
%
\vspace{-4pt}
\begin{equation*}
\bar{\mu}^{h}_{i_{t}}(t-1,\pi) \geq \max_{i \in \possibleArms}\mu_i(t,N_{i,t-1}) - \frac{2\sqrt{2}}{\sqrt{2}-1} c(h, \delta_{t}).
\end{equation*}
\end{restatable}
\begin{proof}
We denote by $\bar{\mu}^{hh'}_i(t-1,\pi)$ and $\hat{\mu}^{hh'}_i(t-1,\pi)$ the true mean and empirical average associated to the $h'-h$ samples between the $h$-th last one (included) and the $h'$-th last one (excluded). Let $j_h$ such that :
$2^{j_h} -1 \leq  h < 2^{j_h+1}-1$.
\begin{align*}
\bar{\mu}^{h}_{i_t}(t-1,\pi) &\geq \bar{\mu}^{2^{j_h}-1}_{i_t}(t-1,\pi)\\ & = \sum_{j=0}^{j_h-1} \frac{2^j}{2^{j_h}-1} \bar{\mu}^{2^{j}2^{j+1}}_{i_t}(t-1,\pi)\\ & \geq \sum_{j=0}^{j_h-1} \frac{2^j}{2^{j_h}-1} \left(s_{i_tj}^c - c(2^j, \delta_t)\right)\\
& \geq \min_j\left(s_{i_tj}^c + c(2^j, \delta_t)\right) - \sum_{j=0}^{j_h-1} \frac{2^{j+1}}{2^{j_h}-1}c(2^j, \delta_t)\\
& = \min_j\left(s_{i_tj}^c + c(2^j, \delta_t)\right) - \frac{2c(1, \delta_t)}{2^{j_h}-1}\sum_{j=0}^{j_h-1} 2^{\frac{j}{2}}\\
& = \min_j\left(s_{i_tj}^c + c(h_j, \delta_t)\right) - \frac{2c(1, \delta_t)}{2^{j_h}-1} \frac{2^{\frac{j_h}{2}}-1}{\sqrt{2}-1}\\
& \geq \min_j\left(s_{i_tj}^c + c(h_j, \delta_t)\right) - \frac{2\sqrt{2}c(2^{j_h+1}, \delta_t)}{\sqrt{2}-1}\\
& \geq \min_j\left(s_{i_tj}^c + c(h_j, \delta_t)\right) - \frac{2\sqrt{2}c(h_j, \delta_t)}{\sqrt{2}-1}\\
& \geq \max_{i\in\possibleArms}\mu_i(t,N_{i,t-1})  - \frac{2\sqrt{2}c(h_j, \delta_t)}{\sqrt{2}-1} 
\end{align*}

The first inequality is due to the decreasing nature of the reward. The second inequality is because, on $\xi_t$, $s_{ij}^c$ concentrates near a value which is smaller than $\bar{\mu}^{2^{j}2^{j+1}}_i(t-1,\pi)$  because it is an average from a sequence of consecutive reward which is newer than $2h_j \leq 2^{j+1}$ (when $m \leq 2$). The third inequality holds by selecting the minimum. The fourth one is standard algebra. The fifth one hold because $c( \cdot, \delta_t)$ decreases with h and $h_j < 2^{j_h+1}$. For the last one, we use the concentration on $\xi_t$ and the decreasing assumption. 


\end{proof}

\begin{algorithm}

\caption{{\small\sc EFF\_Update}}
\begin{algorithmic}[H]
\label{EFFUPDATE}
\REQUIRE $i$, $r$, $t$, $m$
\STATE $\left\{h_j\right\} \leftarrow  \left\{\, h_{j+1} = \ceil{mh_j}\, |\, h_0 =1 \,\right\}$
\STATE $\armCount_{\arm(\currentTime),\currentTime} \leftarrow \armCount_{\arm(\currentTime),\currentTime-1} +1$
\STATE $R^{\rm total}_i \leftarrow R^{\rm total}_i + r$ 
{\footnotesize \COMMENT{\emph{keep track of total reward}}}
\IF{$\exists j$ such that $N_{i,t} = h_j$}
\STATE $s_{i,j}^{\,c} \leftarrow R_i^{\rm total}/N_{i,t}$ 
{\footnotesize \COMMENT{\emph{initialize new statistics}}}
\STATE $s_{i,j}^{\,p} \leftarrow 0$
\STATE $n_{i,j}\leftarrow 0$
\ENDIF
\FOR{$j \in  \{ j\ |\  s_{i,j}^{\,p} \neq \texttt{None}\}$}
\STATE $n_{i,j} \leftarrow n_i +1$
\STATE $s_{i,j}^{\,p} \leftarrow s_{i,j}^{\,p} + r$
\IF{$ n_{i,j} = h^j $}
\IF{$s_{i,j+1}^p =\texttt{None}$}
\STATE $s_{i,j+1}^{\,p} \leftarrow s_{i,j}^{\,p}$
\STATE $n_{i,j+1} \leftarrow n_{i,j}$
\ENDIF
\STATE $s_{i,j}^{\,c} \leftarrow s_{i,j}^{\,p}/h_j$
\STATE $n_{i,j} \leftarrow 0$
\STATE $s_{i,j}^{\,p} \leftarrow \texttt{None}$
\ENDIF
\ENDFOR
\end{algorithmic}
\end{algorithm}
\newpage
%!TEX root = ../main.tex 
\section{How harder are rotting bandits ?}
\label{sec:howhard}
In the last sections, we presented \RAWUCB, an algorithm which extends the results of \UCBone \citep{auer2002finite} on stationary bandits to the more general rotting bandits setup. Hence, we conclude that rotting bandits are not much harder than stationary ones. 

Yet, \UCBone is only near asymptotic and minimax optimal. In Section~\ref{sec:stoch-bandits}, we explain that a better tuning of the confidence levels allows \UCB variant to match the asymptotic and minimax rates for gaussian bandits.

This section investigates the impact of confidence levels tuning on \RAWUCB. How does it compare with \UCB on stationary bandits?  Does it improve the performance of \RAWUCB on our rotting benchmarks? 

\subsection{{\RAWUCBpp}}
We introduce \RAWUCBpp, an algorithm which uses the \RAWUCB procedure (Alg.\ref{alg:RAWUCB}) with a new index,
\begin{multline}
\label{eq:rawpp_index}
\operatorname{ind}(i,t, \delta_{t,h}) \triangleq \min_{h\leq N_{i,t-1}}\pa{ {\hmu}_i^h(\Nitmone) + \sqrt{\frac{2\sigma^2\log_+\pa{\nicefrac{2}{\delta_{t,h}}}}{h}}}\\ \text{ with } \; \delta_{t,h} \triangleq \frac{2\pa{\nicefrac{Kh}{t}}^\alpha}{\pa{1+ \log_+\pa{\nicefrac{t}{Kh}}}^\beta}\CommaBin
\end{multline}

with $\log_+\pa{\cdot} \triangleq \max\pa{\log\pa{\cdot}, 0}$. The main difference with the index of \RAWUCB in Equation~\ref{eq:raw_index} is the more complex confidence level. First, we multiply our confidence level by $Kh$ and replace $\log$ by $\log_+$. This is similar to the $\delta = \nicefrac{KN_{i,t}}{t}$ of \MOSSa  \citep{degenne2016anytime}. We replace $N_{i,t}$ - the number of pulls of arm $i$ at the round $t$- by $h$, the number of sample in the associated average. Indeed, let us consider a two-arm bandit problem where the first arm has a much larger value $\mu_2 + 100 \sigma$ than the second one (with value $\mu_2$) at the beginning of the game. Hence, at the beginning of the game, $N_{1,t} \sim t$ because \RAWUCBpp can quickly identify arm $1$ as the current best arm. After $\frac{T}{2}$ pulls, arm $1$ abruptly decay to a value $\mu_2 + \sigma$. If we do not replace $N_{i,t}$ by $h$ in the confidence levels in Equation~\ref{eq:rawpp_index}, the exploration bonus would be canceled until the end of the game for all the UCB of arm $1$ because $\frac{KN_{i,t}}{t} > 1$. Without the exploration bonus, there is a large enough probability that the index of arm $1$ takes a value below $\mu_2$. Indeed, since we take the minimum across indexes, if the first reward sample after the decay is below $\mu_2$, then the meta-index will be below $\mu_2$. In this case, \RAWUCB may pull arm $2$ until the end of the game and suffer at least $\cO\pa{\sigma T}$ regret. Replacing $N_{i,t}$ by $h$ restore the exploration bonus for arms which have recently decay. 

Second, we add a logarithmic exploration inflation factor. Notice that we also divide $t$ by $Kh$ in the inner logarithm, as it is done for \KLUCBpp \citep{menard2017klucb++}. When the noise is not gaussian, the concentration results are slightly less tight and the asymptotic optimality proof often needs this factor. For instance, \citet{cappe2013klucb} use a factor $\log\pa{t}^{-3}$ in their theory, but they recommend to not use it in practice. However, for \RAWUCB, we believe that extra-exploration is needed in practice. Indeed, we find our best experimental performance for $\alpha=1.4$ which is larger than the asymptotic optimal tuning for \UCB $\alpha=1$ \citep{lattimore2020banditbook}. In our theory in Section~\ref{sec:theory}, we increase $\alpha$ by one compared to \UCBone to ensure that the $t$ (instead of $K$) constructed statistics were into the confidence levels. 




\subsection{Experiments}
\subsubsection{Stationary Experiment}

\paragraph{Setup.} We consider a stationary bandits with two arms with $\mu_1= 0$ and $\mu_2 = \Delta$. We consider two different values of $\Delta \in \left\{0.01, 1\right\}$. The rewards are then generated by applying a Gaussian i.i.d.\,noise $\mathcal{N}\left(0,\subgaussian = 1\right)$. We run the experiment with the horizon $T=10^6$.

\paragraph{Algorithms.} We consider \UCB and \MOSSa \citep{degenne2016anytime}. We tune \UCB with asymptotic optimal confidence level $\sqrt{\nicefrac{2\log\pa{t}}{\Nit}}$ \citep{lattimore2020banditbook}. For \MOSSa, we use $\sqrt{\nicefrac{2\log\pa{\nicefrac{t}{K\Nit}}}{\Nit}}$, which corresponds to a tuning of its parameter $\alpha = 3$. We test \RAWUCBpp with many different values, but we display two different sets of values $\alpha = 1$ and $\beta = 3.5$ or $\alpha = 2$ and $\beta=0$. These two sets of values give the most consistent performance on the two problems. We add \RAWUCB with $\alpha = 1.4$ for comparison. For \RAWUCB and \RAWUCBpp , we use the efficient version with $m=1.01$ which performs similarly than the classical algorithm Subsection~\ref{ss:eff-exp}.

\begin{figure*}[ht]
\centering
\includegraphics[clip, width= 0.49\textwidth]{3Rested/fig/fig_asy0,01.pdf}
\includegraphics[clip, width= 0.49\textwidth]{3Rested/fig/fig_asy1.pdf}
\caption{Stationary experiments}
\label{fig:stationary-experiment}
\end{figure*}
\paragraph{Results.}  \RAWUCBpp seems to improve slightly the results compare to \RAWUCB. Yet, the improvement is not as significant than between \MOSSa and \UCB. On the $\Delta=1$ experiment, we see that the tuning with the logarithm ($\beta=3.5$) seems to enjoy better asymptotic guarantee than the tuning with $\alpha = 2$. Yet, it is not clear if \RAWUCBpp ($\beta=3.5$) is asymptotic optimal with respect to the Lai and Robbin's lower bound. However, at finite horizon, the different parameters are quite close to each other.



\subsubsection{Rotting Experiments $\#$1 (2 arms) and $\#$2 (10 arms).}
\paragraph{Setup and Algorithms.} We study the two benchmarks described in Subsection~\ref{subsec:rested-experiment1} and Section~\ref{sec:rested-experiment}. In Figures~\ref{fig:rested-eff1} and~\ref{fig:rested-eff2}, we compare \EFFRAWpp ($\alpha=2$, $m=1.01$) with \EFFRAW  ($\alpha=1.4$, $m=1.01$). The parameters $\alpha$ were selected according to previous experiments.

\paragraph{Results.} \EFFRAWpp performs slightly better than \EFFRAW for almost any experiments and at almost any rounds. A noticeable exception is when $L$ is large in the two-arms experiment: the result of \EFFRAWpp is slightly worse than for \EFFRAW. Overall, the results suggest that the aggressive confidence tuning technique of stationary bandits also improves the rotting adaptivity. Yet, we notice that the confidence levels with $\alpha =2$ are less tight than the tuning of \MOSSa (which would correspond to $\alpha=1$).

\begin{figure*}[ht]
\centering
\includegraphics[clip, width= 0.51\textwidth]{3Rested/fig/fig1A_pp.pdf}
\includegraphics[clip, width= 0.49\textwidth]{3Rested/fig/fig1B_pp.pdf}
\includegraphics[clip, width= 0.49\textwidth]{3Rested/fig/fig1C_pp.pdf}
\caption{\textbf{Top:} Regret at the end of the game for different values of $L$. \textbf{Bottom:} Regret across time for two values of $L$. Average over 1000 runs. We highlight the $\left[10\%, 90\%\right]$ confidence region.}
\label{fig:rested-pp1}
\end{figure*}


\begin{figure*}[ht]
\centering
\includegraphics[width = 0.99 \textwidth]{3Rested/fig/fig2_pp.pdf}
\caption{\textbf{Left:} Regret at the end of the game for different values of $L$. \textbf{Middle, Right:} Regret across time for two values of $L$. Average over 1000 runs. We highlight the $\left[10\%, 90\%\right]$ confidence region.}
\label{fig:rested-pp2}
\end{figure*}

\subsection{Towards a theoretical analysis of {\RAWUCBpp}}
Analyzing \UCB with tight confidence levels in stationary bandits is already a challenging task \citep{degenne2016anytime, menard2017klucb++, lattimore2018refining}. The analysis of \RAWUCBpp faces two additional difficulties: on the one hand, \RAWUCB 's meta-index is more complex than \UCB 's; on the other hand, rotting bandits are more difficult to analyze than stationary ones. 

First, we can ignore the second part of the problem and try to analyze \RAWUCBpp on stationary problems. Tight analysis of \UCB usually bound the number of pulls of the suboptimal arms. A classical trick is to set a threshold $\mu_\star - \epsilon_i$ and notice that a necessary condition to pull a suboptimal arm $i$ is that either the index of the optimal arm is below the threshold, or the index of the suboptimal arm is above the threshold,

\[ \NiT \leq \sum_{t=1}^T \1 \left[ \texttt{ind}\pa{i_\star, t, \delta_{t,h}} < \mu_\star - \epsilon_i \right] + \1 \left[ \texttt{ind}\pa{i, t, \delta_{t,h}} > \mu_\star - \epsilon_i \right].
\]

The upper deviation of suboptimal arms' indexes is not more difficult to control for \RAWUCB than for \UCB. Indeed, since we take the minimum across confidence bounds, the indexes of \RAWUCB are smaller than the indexes of \UCB (when the confidence levels are the same). 

Controlling the lower deviation of the optimal arm's index is more challenging.  Indeed, at each round $t$, we have to control the probability that any ucb associated with any $h$ last pulls after any $N_{i,t}$ pulls is below the threshold. Compared to \UCB where there is only a scan on the possible values of $N_{i,t}$, we have to handle a double scan on $N_{i,t}$ and $h$. In Section~\ref{sec:theory}, we handle the multiple windows with a crude union bound which leads to a fairly large decrease of the confidence levels. A tighter analysis would probably require better statistical engineering than a union bound or a simple peeling argument. For instance, \citet{maillard2019sequential} develop new concentration results for similar scan statistics for sequential change-point detectors (with some applications to bandits). The difficulty is that the quantity to be bounded is not a sub/super-martingale. Yet, it is quite uncertain that a tighter analysis is actually possible in our case. Indeed, the empirical tuning of \RAWUCB (resp. \RAWUCBpp) increases $\alpha$ by 0.4 (resp. 1) compared to the confidence levels of \UCB (resp. \MOSSa). This is comparable with the theoretical increase of $1$ due to the union bound over all the possible windows.
%%!TEX root = ../main.tex 

\section{Linear rotting bandits are impossible to learn}



\subsection{Linear rested rotting bandits}
In this section, we present our rotting linear bandit framework which recovers 1) the linear model of %TODO name paper
 as soon as the reward is stationnary; and 2) the rotting multi-armed bandits model as soon as $\mathcal{X}$ contains exclusively canonical basis vectors. 

We introduce $d$ non-increasing and $L$-Lipschitz functions $\mu_i : \realset \rightarrow \realset$. These functions satisfies Assumption~ %TODO
 but while there were $K$ reward functions defined on $\NN$ in the rotting MAB model, we now have $d$ functions defined on $\realset$. Indeed, in the linear setup the number of reward parameter is $d$ and we expect this value to replace $K$ in the regret bound.

We call $N_{i,t} \triangleq \sum_{t'=0}^t (X_{t'})_i$, which quantifies the amount of pull of direction $i$. We then define the reward : 
\[
o_t(X) = \sum_{i\leq d} \int_{N_{i,t}}^{N_{i,t+1}} \mu_i(x)dx. +\eta_t 
 = \int_{\bm{N_{t}}}^{\bm{N_{t+1}}} \bm{\mu}(\bm{n})^\intercal  d\bm{n} + \eta_t\]

The total reward can thus be writen:  
\[ J(\pi, T) = \int_{\bm{0}}^{\bm{N_{T}}} \bm{\mu}(\bm{n})^\intercal d\bm{n} .\]

Hence, we found a model which extends both rotting MAB model (when the actions are encoded by canonical vectors) and linear bandit model (when the reward is stationnary, i.e. $\bm{\mu}$ is a constant vector function). Moreover, for any vector $\bm{X}$, the reward associated to $\bm{X}$ is decreasing along the pulls while the cumulative reward is totally determined by the number of pull $\bm{N_T}$ and the knowledge of $\bm{\mu}$.

However, one can note that the number of pulls $N_{i,t}$ in the rotting MAB setup has two meaningfull equivalent in the rotting linear setup : $\sum_t x_{i,t}$ and $\sum_t x_{i,t}^2$. The first one is useful in the integral to have linear dependence of the reward with X. The second is usefull from an information theoretic point of view (least square regression) to quantify how much we pulled each direction.

Bandits problems are often considered as the RL problems without state. However, in this setup we do have a state as the next reward depends on the matrix $A_T$. In the  rotting MAB framework, we overcome this issue by showing that the greedy oracle strategy is optimal. Hence, there is no need for planning and the stochastic learning problem is reduced to a pure exploration-exploitation problem where one needs to determine the action which currently performs the best. Therefore, we would like to show that the greedy oracle policy (ie. the policy which selects $\int_{\bm{N_{t}}}^{\bm{N_{t+1}}} \bm{\mu}(\bm{n})^\intercal  d\bm{n}$) is optimal in the rotting linear bandit problem. In the next section, we will show that the greedy oracle policy is not optimal and hence that there is no anytime optimal policy. 

\section{The non-optimality of the greedy oracle policy }
\label{Optimal}
\begin{theorem}
The greedy oracle strategy $\pi_G$ is not optimal. More precisely, for any horizon $T$, there exists a reward vector function $\vec{\mu}$ such as the performance compared to the optimal policy for horizon $T \geq 2$ $\pi_{O_T}$ is :
\[
J(\pi_{O_T}, T) - J(\pi_G, T) \geq \frac{L(T-1)}{8}
\]
\end{theorem}
\begin{proof}
We consider $d = 2$, $\mathcal{X} = \left\{ X_1, X_2 \right\}$ with $X_1 = (1,0)^\intercal$ and $X_2 = (\frac{1}{\sqrt{2}},\frac{1}{\sqrt{2}})^\intercal$. For any horizon $T$, we consider the following reward functions :

\[\mu_1(x) = L \text{ if } x < \frac{T}{2} \text{ else } 0 \qquad \text{and} \qquad  \mu_2(x) = \frac{L}{2}.
\]

The greedy strategy will therefore select $X_1$ until $\floor{\frac{T}{2}}$ and then $X_2$ untill the end of the game. Hence  :
\[J(\pi_G, T) = \int_0^{\floor{\frac{T}{2}} + \ceil{\frac{T}{2}}/2} \mu_1(x)dx + \int_0^{\ceil{\frac{T}{2}}/2}  \mu_2(x)dx = \frac{T}{2}  L + \ceil{\frac{T}{2}} \frac{L}{4} \leq \frac{5LT + L }{8}.\]
We now consider the policy $\pi_2$ which always selects arm 2. At the end of the game, it gathers the reward : 
\[
J(\pi_2, T) = \int_0^{\frac{T}{2}} \mu_1(x)dx + \int_0^{\frac{T}{2}} \mu_2(x)dx = \frac{T}{2} L + \frac{T}{2} \frac{L}{2} = \frac{3LT}{4}
\]

Hence, since optimal policy $\policy_T^\star$ has larger reward than  $\policy_2$ at horizon $T$ (by definition), we have that 
\[
J(\policy_T^\star, T) - J(\pi_G, T) \geq J(\policy_2, T) - J(\pi_G, T) \geq \frac{L(T-1)}{8}
\]

\end{proof}

Hence the greedy policy can be as bad as 8th the regret of the worst performance possible on the problem sets. This is surprising as the greedy oracle strategy was optimal for the rotting MAB problem.  One can note that the vectors used in the proof have the same $L_2$-norm and that the vector function $\vec{\mu}$ is bounded in $[0, L]^2$. The overall setup is simple. We do not need complex decays nor vectors with different "pulling amount" to have a suboptimal performance of the greedy policy. The suboptimality comes from the fact that we do not have access to all the canonical vectors. Hence, when the greedy algorithm has collected all the reward it can get from direction 1, it will start focusing on collecting reward in the second direction. When it pulls the second vector to take advantage of the second direction it also pulls the first direction which is now useless. Here comes some "regret" : the algorithm could have started directly collecting direction 2 as it would have got all the direction 1 benefits anyway. The following corollary underlines that the failure of the greedy oracle strategy implies the necessity of planning.

\begin{lemma}
For a cumulative reward exploration exploitation problem, the only possible anytime optimal oracle strategy is the greedy oracle one.
\end{lemma}
\begin{proof}
Let's assume $\pi_{O_a}$ an anytime optimal strategy which does not select the greedy action at time $T$.
Let's consider $\pi_{G_T}$ a strategy which copies $\pi_{O_a}$ for the $T-1$ round and is greedy at round $T$.
\[
J(\pi_{O_a}, T) - J(\pi_{G_T},T) = r_T(\pi_{O_a}(T)) - r_T(\pi_{G_T}(T)) < 0
\]
where the last inequality comes from the fact that $r_T(\pi_{O_a}(T))$ is below the best reward available for that time. 
\end{proof}

\begin{corollary}
There is no anytime optimal oracle strategy for the rotting linear bandit model. 
\end{corollary}

What is the regret of a short-sighted oracle strategy which sees F steps in the future (ie . knows $\mu_i$ from $0$ up to $a_{ii,t}^2 + F \max_d X_{d,i}^2$?)
\begin{theorem}
Any strategy which can anticipate the future up to $F$ steps in advance has a worst case regret which scales at least with $O(T-2F)$. More precisely : 
\[
\max_\mu R(\pi, T) \geq  \frac{L(T-2F)}{12} - \frac{L}{6}
\]
\end{theorem}
\begin{proof}
We still consider $d = 2$, $\mathcal{X} = \left\{ X_1, X_2 \right\}$ with $X_1 = (1,0)^\intercal$ and $X_2 = (\frac{1}{\sqrt{2}},\frac{1}{\sqrt{2}})^\intercal$. For any horizon $T$, we consider the following reward functions :

\[\mu_1^1(x) = L \quad \text{and} \quad\mu_1^2(x) = L \text{ if } x < \frac{T}{2} \text{ else } 0 \quad \text{and} \quad  \mu_2(x) = \frac{L}{2}.
\]

The optimal strategy associated to $\mu_1^1$ (respectively $\mu_1^2$) is $\pi_O \triangleq \pi_1$  (resp. $\pi_O \triangleq\pi_2$), the policy which always pulls the first (resp. second) arm, and it gathers the cumulative reward $J_1(\pi_O, T)$ (resp. $J_2(\pi_O, T)$). We have by simple calculations:
\[
J_1(\pi_O, T) = LT \quad \text{and} \quad J_2(\pi_O, T) = \frac{3TL}{4} 
\]

Depending on whether $\mu_1$ is $\mu_1^1$ or $\mu_1^2$, we can express the regret as a function of $N_{1,T}$ or $N_{2,t}$.
\begin{align}
R_1(\pi, T) \triangleq J_1(\pi_O, T) - J_1(\pi_{t_f},T) = LT -  L (T- N_{2,T}) -  N_{2,T} \frac{3L}{4}  = \frac{LN_{2,T}}{4}\\
R_2(\pi, T) \triangleq J_2(\pi_O, T) - J_2(\pi_{t_f},T) = \frac{3LT}{4} -  \frac{LT}{2}  -  (T- N_{1,T}) \frac{L}{4}  = \frac{LN_{1,T}}{4}
\end{align}


 
We call $t_f$ the first time such that $||\epsilon_1 ||_{A_{t_f}} \geq \frac{T}{2} - F$. After $t_f$ the learner entirely knows which reward functions she faced and before $t_f$ the two reward functions are undistinguishable to the learner.  Note that for any policy, $||\epsilon_1 ||_{A_{T}} \geq \frac{T}{2}  > \frac{T}{2} - F$, hence $t_f$ exists for any policy.  We have that
\begin{align}
& N_{1, t_f} + \frac{N_{2, t_f}}{2} \geq \frac{T}{2} - F  \\
& N_{1, t_f} + \frac{N_{2, t_f}}{2} \leq \frac{T}{2} - F + 1 \\
& N_{1, t_f} + N_{2, t_f} = t_f
\end{align}

Hence, we have the following lowerbound for $N_{i,T}$:
\begin{align}
& N_{1,T} \geq N_{1, t_f} \geq T - t_f - 2F  \\
& N_{2,T} \geq N_{2, t_f} \geq 2 t_f - T + 2(F-1) 
\end{align}

Hence, worst case regret is :
\[
\max_\mu R(\pi, T) \geq \max( R_1(\pi,T), R_2(\pi, T)) = \frac{L}{4} \max_{0 \leq t_f \leq T}(T - t_f - 2F, 2 t_f - T + 2(F-1) ) \geq  \frac{L(T-2F)}{12} - \frac{L}{6}
\]

\end{proof}

Note we can slightly modify the proof to get $O(T -\alpha F)$ for $\alpha >1$.
%!TEX root = ../main.tex
\chapterimage{chapter_head/5_361Sarajevo.jpg} 
\chapter{The rotting assumption makes restless bandits easier}
\label{ch:restless}
\vspace{-3cm}
\begin{flushright}
\emph{A non-rotting restless bandit. Who would say he is not tough?}
\end{flushright}
\vspace{.85cm}
%!TEX root = ../main.tex
\section{Restless rotting  bandits}
\label{sec:restless-model}
\subsection{Restless bandits model}
\subsubsection*{Feedback loop}
At each round $t$, an agent chooses an arm $i_t \in \possibleArms \triangleq \left\{ 1, ... , K\right\} $ and receives a noisy reward $o_t$. The reward associated to each arm $i$ is a $\subgaussian^2$-sub-Gaussian random variable with expected value of $\mu_i(t)$, which depends on the number of rounds $t$. Let $\historyt \triangleq \left\{ \left\{ i_s, o_s \right\}, \forall s < t \right\}$ be the sequence of arms pulled and rewards observed until round $t$, then 
%
\begin{equation}
\label{eq:restless-feedback}
o_{t} \triangleq \mu_{i_t}(t) + \noise_t
 \;\; \text{with}\; \EE{ \noise_t | \historyt }= 0 \;\; \text{and} \; \forall \lambda \in \R, \; \EE{ e^{\lambda\noise_t}} \leq e^{\frac{\subgaussian\lambda^2}{2}}.
\end{equation}
%

\subsubsection*{Objective}
We will only consider deterministic agents which output an arm $i$ at each round $t$. Like in the previous chapter, we distinguish offline (or oracle) policies~$\pi \in \PiO$ - which are functions which map the round $t$ and the set of reward functions to arms - from online (or learning) policies~$\pi \in \PiL$ - which are functions from the history of observations $\mathcal{H}_t$ at a round $t$ to arms. For both types of policies, we often use the shorter notation $\pi(t)$, where the dependency on $\mu$ or $\mathcal{H}_t$ is implicit. The performance of a policy $\pi$ is measured by the (expected) rewards accumulated over time, 
%
\begin{equation}
\label{eq:cumul-reward-restless}
J_T(\pi) \triangleq \sum_{t=1}^T \mu_{\pi(t)}\pa{t}.
\end{equation}
%

\begin{proposition}
The characterization of the optimal oracle policies is straightforward,
\[\pi^\star \in \argmax_{\pi \in \PiO} J_T(\pi) \iff \forall t \leq T,  \pi^\star(t) \in \argmax_{i \in \arms} \mu_i\pa{t}.\]
In the following, we call $i^\star_t \in \argmax_{i \in \arms} \mu_i\pa{t}$ one of the best arm at the round $t$, and $\mu_\star(t) \triangleq \max_{i \in \arms} \mu_i\pa{t}$ the corresponding best value.
\end{proposition}
Notice that there may be several optimal policies if, at a given round $t$, there are several arms with maximal value. However, all these policies get the same cumulative reward at every round; thus, the tie-break rule can be chosen arbitrary without impacting the performance.  We set a policy $\pi^\star\in \argmax_{\pi \in \PiO} J_T(\pi)$. Calling $J_T^\star = J_T(\pi^\star)$ the largest cumulative reward achievable, one can measure the regret of any policy (learning or oracle) compared to the optimal one, 
\begin{align}\label{eq:restless-regret}
\regret(\pi) \triangleq J^\star_T - J_T(\pi).
\end{align}
%
\begin{remark}
Like in the rested setup, the regret is measured against the optimal oracle policy rather than a fixed-arm policy as it is a case in adversarial bandits. Moreover, for constant $\mu_i(t)$-s, the problem, and definition of regret reduce to the ones of stationary stochastic bandits (where the regret is measured against the best fixed-arm policy which is also the optimal oracle policy). 
\end{remark}
%

\subsection{Piece-wise stationary bandits}
\label{subsec:piecewise}
\citet{garivier2011upper-confidence-bound} study the restless bandits case, where rewards are piece-wise stationary. 
 \begin{assumption}\label{assum:piece-wise}
Let $V$ be a positive constant and $\Upsilon_T$ a positive integer.  $\mu_i : \NN^\star \rightarrow [- V, 0]$\footnote{We could choose any interval $\left[x, x+V\right]$. Yet, with the upcoming decreasing assumption, choosing $[- V, 0]$ instead of $[0, V]$ emphasizes that the learner cannot infer parameter $V$ from the first pulls. Notice that we will never use that our rewards are negative in our analysis.} are piecewise stationary non-increasing functions of the time $t$ with at most $\Upsilon_T-1$ breakpoints. Formally, 
\[\sum_{t=1}^{T-1} \mathbbm{1}\left(\exists i\!\in\! \arms, \mu_i(t) \!\neq\! \mu_i(t\!+\!1)\right)\leq \Upsilon_T\!-\!1.\]
\end{assumption}
%
 We call $\left\{t_k\right\}_{k \leq \Upsilon-1}$ the set of breakpoints with $t_0 = 0$, $\mu_i^k$ the value of $\mu_i(t)$ for $t \in \left\{t_k+1 , \dots, t_{k+1}\right\}$. We call $i^\star_k \in \argmax_{i \in \arms}{\mu_i^k}$ (one of) the best arm in batch $k$, $\mu_\star^k \in \max_{i \in \arms}{\mu_i^k}$ the corresponding best value, and $\Delta_{i,k} \triangleq \mu_{\star}^k - \mu_i^k$ the gap to the best arm for arm $i$ during batch $k$.
%
\subsubsection{Lower Bounds}

\begin{proposition}[\citet{auer2002nonstochastic}]
\label{prop:piecewise_lb}
For any strategy $\pi$, there exists a piece-wise stationary bandit scenario with means $\{\mu_i(t)\}_{i,t}$ satisfying Assumption~\ref{assum:piece-wise} such that,
\[
    \mathbb{E}\left[R_T(\pi)\right] \geq \frac{\sigma}{32}\sqrt{ \Upsilon_T KT}\,.
\]
\end{proposition}

This bound is not surprising as it shows that piece-wise stationary bandits with $\Upsilon_T$ change-points are at least as hard as $\Upsilon_T$ stationary problems with horizon $\frac{T}{\Upsilon_T}$ \citep{auer2002nonstochastic}. We will show a slightly stronger result in Subsection~\ref{subsec:restless-rotting}.

\citet{garivier2011upper-confidence-bound} shows a self-bonding property of the regret. They build a problem $\mu'$ on which the reward function equals the reward on a stationary problem $\mu$ except on a period $\tau$ (see Figure~\ref{fig:garivier-lb}). During this time span, the best arm of $\mu$ keeps its value while the worst arm \textit{increases} to become optimal. The size of $\tau$ is chosen inversely proportional to the average pulling rate of the bad arm in $\mu$. Indeed, the lower the pulling rate of the bad arm, the longer the adversary can increase its value in $\mu'$ without being noticeable by the learner (which can be quantified thanks to Lemma~5.1, \citet{auer2002nonstochastic}). Since the pulling rate of the bad arm in $\mu$ is proportional to $R_T(\mu)$, we get a lower bound proportional to $\tau \sim \frac{T}{R_T(\mu)}$. We reproduce the version of the theorem in \citet{lattimore2020banditbook}. 

\begin{figure*}[h]
\centering
\includegraphics[clip, width= 0.99\textwidth]{4Restless/fig/garivier_lb.png}
\caption{The reward functions $\mu$ and $\mu'$. A policy with low regret on $\mu$ cannot achieve low regret on $\mu'$.}
\label{fig:garivier-lb}
\end{figure*}

\begin{proposition}[Theorem~31.2, \citet{lattimore2020banditbook}]
\label{prop:garivier-lb}
If a policy $\pi$ performs a regret $R_T(\pi, \mu)$ on a 2-arm stationary instance $\mu$, one can find a piece-wise stationary instance $\mu'$ with only two breakpoints such that, for a sufficiently long horizon $T$, the regret is lower bounded by 
\[\EE{R_T(\pi, \mu')} \geq \frac{T}{22R_T(\pi, \mu)}\,\cdot\]  
\end{proposition}

\begin{corollary}
\label{cor:garivier-lb}
Let $\pi$ a minimax optimal policy on the piece-wise stationary setups. Then, for a sufficiently large horizon $T$, there exists a universal constant $C$ such that for all the 2-arm stationary problems $\mu$, 
\[
\EE{R_T(\pi,\mu)} \geq C\sqrt{T}.
\]
\end{corollary}

These results state that one cannot have simultaneously a near-optimal problem-dependent regret rate $\cO\pa{\log{T}}$ on stationary instances and the minimax optimal piece-wise stationary rate $\cO\pa{\sqrt{T}}$. It is very different from the stationary case (or even with the rested rotting bandits presented in the last section) where some algorithms are shown to perform optimally both problem-dependent and problem-independent wise \citep{lattimore2018refining, menard2017klucb++}.

\subsubsection{Policies for piece-wise stationary bandits.}
\paragraph{Softmax policies.} For any sequence generated by an oblivious adversary, \EXPS \citep{auer2002nonstochastic} - an extension of \EXP - is guaranteed to achieve $\cO\pa{\sqrt{K\Upsilon_T T\log\pa{KT}}}$ regret against the best policy among the ones which change arms at most $\Upsilon_T -1$ times. The bound holds in the special case where the adversary generates the reward with noisy piece-wise stationary functions. In that case, the pseudo-regret definition is equivalent to the piece-wise stochastic regret defined in Equation~\ref{eq:restless-regret}. Indeed, the optimal policy is included in the set of $\cO\pa{\pa{KT}^{\Upsilon_T}}$ policies with at most $\Upsilon_T-1$ change of arms. 

\paragraph{Passive forgetting policies.} \DUCB \citep{kocsis2006discounted} and \SWUCB \citep{garivier2011upper-confidence-bound} are two ucb index policies which forget the older sample either by a discount factor or by a sliding window mechanism. The confidence interval increases when an arm has not been pulled for many rounds. When they are adequately tuned, these policies achieve respectively $\cO\pa{\sqrt{K\Upsilon_T T}\log{T}}$ and $\cO\pa{\sqrt{K\Upsilon_T T \log{T}}}$ minimax regret rate. While these policies do not improve the rate of \EXPS, they are deterministic and more explainable. 

\paragraph{Change-detection policies.} Instead of throwing away old samples at a fixed pace, one could remove samples from the index only when they notice a change in the arm's mean. This is the spirit of the Change-Detection ucb algorithms. These algorithms have three components: an ucb index, a change-detection subroutine, and a fixed active exploration rate (either deterministic or random pulls). The active exploration rate is meant to detect the arms which change from suboptimal value to optimal ones (like in Figure~\ref{fig:garivier-lb}). The optimal budget dedicated to active exploration scales with $\cO\pa{\sqrt{K\Upsilon_T T}}$. %When $\Upsilon_T$ is unknown, one can set a count $\Upsilon$ to $1$ and increases it at each detected change. 

\MUCB \citep{cao2019nearly} uses a simple change detector which compares the average of the last $\nicefrac{w}{2}$ samples with the average of the before last $\nicefrac{w}{2}$ ones and check whether the difference is significant or not. The optimal tuning of the parameter $w$ depends on the value of $\Upsilon_T$: if changes are large and frequent, one should choose a small value of $w$; if changes are small and sparse, one should choose a large value of $w$.

\CUSUMUCB \citep{liu2018change-detection} uses a change detector which constructs two random walks based on the upper and lower deviation of the new samples compared to the mean of the $M$ first ones. If one of the random walks reaches a threshold $h$, then the change detector triggers. The random walks are negatively biased with a small value $\epsilon$ to prevent the natural deviation to trigger the change detector. Again, the optimal value of the parameters $M$, $\epsilon$ and $h$ depends on the number of changes $\Upsilon_T$.

\GLRUCB \citep{besson2019generalized} uses the Gaussian Likelihood Ratio change detector. This change detector scans all the samples to detect any size of change on any period with high probability. The probability parameter only needs the knowledge of the horizon $T$ to achieve near-optimal minimax bound. \citet{mukherjee2019distribution} introduces a very similar algorithm but study the assumption where all the arms change their value significantly at each breakpoint. With this assumption, they do not need active exploration and recover problem-dependent bound $\cO\pa{\log{T}}$.

On the theoretical side, the analysis often assumes that each change is large enough to be detected before the next change. Indeed, after the detection of the breakpoint, they use the analysis of \UCB on each stationary batch. Before the change detection, they do not provide any non-trivial bound on the quality of the selected arm. 


\paragraph{Agnostic policies.}
\citet{auer2019adaptively} consider the problem with no assumption on the change-point detectability. They propose \ADSWITCH, which also uses a parameter-free change-detection subroutine but with an elimination policy: it pulls arms in a round-robin way a refined set of good arms. Arms are excluded from this set when they demonstrate with high probability that they underperform. The bad arms are also actively explored with consecutive sampling: the algorithm selects at random an arm and a deviation size $\Delta$ and pulls the arm the right number of rounds to detect if there is a change of size $\Delta$ in the arm's value. \citet{chen2019new} extend this technique to the contextual bandits problem.

A previous attempt \citep{cheung2019new} to solve this problem uses an expert aggregation bandit algorithm (e.g. \EXPfour) to select between different tuning of \SWUCB . Yet expert aggregation of bandit algorithm is problematic \citep{agarwal2017corralling, besson2018aggregation}, and \citet{cheung2019new} has to run each copy by batch with full restart. This technique leads to a suboptimal rate $\tcO\pa{\sqrt{K \max\pa{\Upsilon_T, \sqrt{T}} T}}$.


\subsection{Variation budget bandits}
\label{subsec:variation}
\citet{besbes2014stochastic} introduce the limited variation budget bandits, a restless setting where at each round Nature can modify the reward value of any arm but with a limited total variation budget $V_T$ at the round $T$. 

\begin{assumption}
\label{assum:variation}
$\mu_i : \NN^\star \rightarrow [- V_T, 0]$ are functions of the time $t$ with $V_T$ a positive constant. Moreover, we have that 
\begin{equation}
\label{eq:defbudget}
    \sum_{t=1}^{T-1} \sup_{i \in \arms} |\mu_i(t+1) - \mu_i(t) | \leq V_T\,.
\end{equation}
\end{assumption}


\paragraph{Lower Bound}
\begin{proposition}[\citet{besbes2014stochastic}]
\label{prop:variation_lb}
For any strategy $\pi$, there exists a  variation budget bandit scenario with means $\{\mu_i(t)\}_{i,t}$ satisfying Assumption~\ref{assum:variation} with a budget $V_T \geq \sigma \sqrt{\frac{K}{8T}}$ such that
%
\[
    \mathbb{E}\left[R_T(\pi)\right] \geq \frac{1}{16\sqrt{2}} \pa{\sigma^2 V_T KT^2}^{1/3}.
\]
\end{proposition}

In the next section, we prove a stronger statement, using only non-increasing reward functions. Yet, there is no additional difficulty. While the two Assumptions~\ref{assum:piece-wise} and~\ref{assum:variation} leads to different regret rate (see Proposition~\ref{prop:piecewise_lb}), the proof (see e.g. Lemma~\ref{lemma:lb} in the next subsection) shows that there is a strong similarity between the two problems, at least from a minimax perspective.

\paragraph{Policies for variation budget bandits.}
Most of the algorithms presented for the piece-wise stationary case are also near-optimal for the variation budget case. Indeed, \citet{besbes2014stochastic} show that \EXPS also learns in the variation budget setup. They also present \REXP, an algorithm based on \EXP with periodic restart which recovers a similar guarantee than \EXPS. \citet{cheung2019new} and \citet{russac2019weighted} extend \SWUCB and \DUCB to the linear bandit setting with variation budget. \citet{chen2019new} proves that \ADSWITCH is also optimal in the variation budget setting. However, change-detection ucb algorithms are not proved to perform well in the variation budget setting. Indeed, their proofs use the proof of \UCB on each stationary batch. In the variation budget setup, there is no stationary batch, which makes these algorithms harder to analyze. 

\subsection{The restless rotting assumption}
\label{subsec:restless-rotting}
\begin{assumption}
\label{assum:restless_rotting}
Reward functions $\left\{\mu_i \right\}_i$ are non-increasing with $t$.
\end{assumption}
We use this Assumption with Assumption~\ref{assum:piece-wise} and~\ref{assum:variation}.
\begin{remark}
\label{remark:budget}
With the rotting assumption, the variation budget assumption is very similar to the bounded assumption. Indeed, any set of decreasing functions $\mu_i : \NN^\star \rightarrow [- V, 0]$ satisfies Equation~\ref{eq:defbudget} with $V_T = KV$. Reciprocally, any set of functions satisfying Equation~\ref{eq:defbudget} with $\mu_i(1) \in [- V_T, 0]$ are bounded in $[- 2V_T, 0]$. 
\end{remark}

\paragraph{Lower bounds.} We show that our additional decreasing assumption does not change the minimax rates of the two settings. This is an adaptation of the proof of \citet{besbes2014stochastic} where we only use rotting functions.

\begin{restatable}{proposition}{restapiecewiselb}
\label{prop:piecewise_lb2}
For any strategy $\pi$, there exists a \underline{rotting} piece-wise stationary bandit scenario with means $\{\mu_i(t)\}_{i,t}$ \underline{satisfying Assumptions~\ref{assum:piece-wise} and~\ref{assum:restless_rotting}} with $\Upsilon_T\! \leq \!\pa{\!\frac{32V^2T }{K\sigma^2}\!}^{\!\nicefrac{1}{3}\!}\!$ such that,
\[
    \mathbb{E}\left[R_T(\pi)\right] \geq \frac{\sigma}{32}\sqrt{ \Upsilon_T KT}\,.
\]
\end{restatable}
\begin{restatable}{proposition}{restavariationlb}
\label{prop:variation_lb2}
For any strategy $\pi$, there exists a \underline{rotting} variation budget bandit scenario with means $\{\mu_i(t)\}_{i,t}$ \underline{satisfying Assumptions~\ref{assum:variation} and~\ref{assum:restless_rotting}} with a budget $V_T \geq \sigma \sqrt{\frac{K}{8T}}$ such that,
%
\[
    \mathbb{E}\left[R_T(\pi)\right] \geq \frac{1}{16\sqrt{2}} \pa{\sigma^2 V_T KT^2}^{\nicefrac{1}{3}}.
\]
\end{restatable}

The condition on $\Upsilon_T$ in Proposition~\ref{prop:piecewise_lb2} follows from Remark~\ref{remark:budget}: if $V$ is too small compared to $\Upsilon_T$, then we have a budget constraint - with associated lower bound in Proposition~\ref{prop:variation_lb2} - rather than a breakpoint constraint.


\paragraph{Proof}
 Our proof build a set of rotting piece-wise stationary problems with an evenly spaced set of $\Upsilon -1$ breakpoints. The adversary can choose the distance between arms $\Delta=\frac{1}{4} \sqrt{\frac{\sigma^{2} K \Upsilon}{2 T}}$ at the maximum such that the best arm is barely identifiable between two breakpoints (see Lemma~5.1, \citet{auer2002nonstochastic}). At each breakpoint, each arm's value decreases by $\Delta$ or $2\Delta$. Even if the set of breakpoints would be known, the learner does not know which arm is the best on each stationary part. Hence, in the worst case, she suffers at least the sum of the minimax regret of $\Upsilon$ stationary bandits problems with horizon $\frac{T}{\Upsilon}$, \textit{i.e.}  $\cO \pa{\sqrt{K\Upsilon T}}$. In the piece-wise stationary setting, we can simply identify $\Upsilon = \Upsilon_T$. In the variation budget setting, the adversary has a constraint over $\Upsilon \Delta = \frac{1}{4} \sqrt{\frac{\sigma^{2} K \Upsilon^3}{2 T}}=  \cO\pa{V_T}$. Hence, when the budget is limited, the adversary can choose up to $\Upsilon = \cO\pa{ T^{1/3}}$ breakpoints such that the suboptimal arms are "sufficiently" far from the best one (\textit{i.e} at $\Delta$). This dependence on $T$ leads to the increased regret rate of $\cO\pa{T^{\nicefrac{2}{3}}}$.
 
\begin{lemma}\label{lemma:lb}
Let $\Upsilon \in \left\{1,\dots, T\right\}$ and $\left\{\tau_k \triangleq \ceil{\frac{T}{ \Upsilon}} \text{ if } k \leq T \bmod{\Upsilon} \text{ else } \floor{\frac{T}{ \Upsilon}}\right\}_{k\leq \Upsilon}$. We call $t_k = \sum_{k'=1}^k \tau_{k'}$ and $t_0 = 0$.  Consider a family of piece-wise stationary bandits indexed by a vector $i^\star\in (\{0\}\cup \arms)^{\Upsilon}$ as follows: arm $i$ is a Gaussian distribution $\mathcal{N}\pa{\mu_i(t), \sigma}$ such that 
\[
\forall k \in \left\{0 , \dots, \Upsilon -1 \right\}, \ \forall t \in \left\{t_{k-1}+1,\dots, t_{k}\right\}, \ 
\mu_i(t) = 
\begin{cases}
-k \Delta \text{ if } i = i^\star_k\\
-(k+1)\Delta  \text{ else.}
\end{cases}
\]
We denote by $\EEempty_{i^\star}$ the expectation under the problem indexed by $i^\star$. Then, if $\Delta = \frac{1}{4}\sqrt{\frac{\sigma^2K\Upsilon}{2T}}$, for any policy $\pi$ :
\[
 \exists i^\star\in (\{0\}\cup \arms)^{\Upsilon}, \  \EEempty_{i^\star}\big[R_T(\pi)\big]  \geq  \frac{\sqrt{\sigma^2KT\Upsilon}}{32}\cdot
\]
\end{lemma}
\begin{proof}
Note that when $i^\star_k = 0$ then all the arms share the same means. We also define the vector $i^\star_{-k}$ equals to $i^\star$ with the coordinate $k$ empty and for $i\in\arms$ the vector $(i^\star_{-k},i)$ as the vector where we fill the empty coordinate with $i$.  We fix a policy $\pi$ and we will lower bound its average regret on the bandits problem indexed by $i^\star \in \arms^\Upsilon$ 
\begin{align*}
    \frac{1}{K^{\Upsilon}} \sum_{i^\star\in \arms^{\Upsilon}}  \EEempty_{i^\star}\big[R_T(\pi)\big] &= \frac{1}{K^{\Upsilon}} \sum_{i^\star\in \arms^{\Upsilon}} \sum_{k=1}^{\Upsilon}\Delta \EEempty_{i^\star}[\tau_k - N_{i^\star_k}^k] \\
    &=\Delta \left(T - \frac{1}{K^{\Upsilon}} \sum_{i^\star\in \arms^{\Upsilon}} \sum_{k=1}^{\Upsilon} \EEempty_{i^\star}[N_{i^\star_k}^k]\right),
\end{align*}
where $N_i^k$ is the number of pulls of arm $i$ during epoch $k$. Thus we need to upper bound the following quantity
\[
\frac{1}{K^{\Upsilon}} \sum_{i^\star\in \arms^{\Upsilon}} \sum_{k=1}^{\Upsilon} \EEempty_{i^\star}[N_{i^\star_k}^k] = \sum_{k=1}^{\Upsilon} \frac{1}{K^{\Upsilon-1}} \sum_{i^\star_{-k}\in \arms^{\Upsilon-1}}\frac{1}{K} \sum_{i=1}^K\EEempty_{(i^\star_{-k},i)}[N_{i}^k]\,.
\]
Using the contraction of the entropy for the bounded random variable $N_{i}^k/\tau_k$ then the Pinsker inequality (see \citet{garivier2018explore}) we get
\[
2\left(\frac{1}{\tau_k K} \sum_{i=1}^K\EEempty_{(i^\star_{-k},i)}[N_{i}^k] -\frac{1}{\tau_k K} \sum_{i=1}^K\EEempty_{(i^\star_{-k},0)}[N_{i}^k] \right)^2 \leq \frac{1}{K} \sum_{i=1}^K \EEempty_{(i^\star_{-k},0)}[N_{i}^k] \frac{\Delta^2}{2\sigma^2}\CommaBin
\]
since problems $(i^\star_{-k},i)$ and $(i^\star_{-k},0)$ differ only by a gap $\Delta$ on the arm $i$ during epoch $k$. Thanks to the fact that  $\sum_i N_i^k \leq \tau_k$ we get 
\[
\frac{1}{K} \sum_{i=1}^K\EEempty_{(i^\star_{-k},i)}[N_{i}^k] \leq \frac{\tau_k}{K} + \frac{\Delta}{2\sigma \sqrt{K}}\tau_k^{\nicefrac{3}{2}}\,.
\]
Putting all together we have for $K\geq 2$
\begin{align*}
    \frac{1}{K^{\Upsilon}} \sum_{i^\star\in \arms^{\Upsilon}}  \EEempty_{i^\star}\big[R_T(\pi)\big]  \geq \left(\frac{T}{2} -  \sum_{k=1}^{\Upsilon} \frac{\tau_k^{\nicefrac{3}{2}} \Delta}{2\sigma \sqrt{K}}\right)\Delta\,.
\end{align*}
We have $\tau_k= \floor{\frac{T}{\Upsilon}}$ or $\tau_k= \ceil{\frac{T}{\Upsilon}}$  such that $\sum_{k=1}^{\Upsilon} \tau_k=T$. Hence, we have that $\tau_k \leq 2T/\Upsilon$ which leads to 
\[
 \frac{1}{K^{\Upsilon}} \sum_{i^\star\in \arms^{\Upsilon}}  \EEempty_{i^\star}\big[R_T(\pi)\big]  \geq  \left(\frac{1}{2}T - \frac{\sqrt{2}T^{\nicefrac{3}{2}}\Delta}{\sigma \sqrt{K\Upsilon}}\right)\Delta\,.
\]

Choosing $\Delta = \frac{1}{4}\sqrt{\frac{\sigma^2K\Upsilon}{2T}}$, we get 
\[
 \frac{1}{K^{\Upsilon}} \sum_{i^\star\in \arms^{\Upsilon}}  \EEempty_{i^\star}\big[R_T(\pi)\big]  \geq  \frac{1}{4}\sqrt{\frac{\sigma^2K\Upsilon}{2T}}\left(\frac{1}{4}T\right) \geq \frac{\sqrt{\sigma^2KT\Upsilon}}{32}\cdot
\]
We can conclude by noticing that the average expected regret across the problem set is lesser or equal to the maximum across the same problem set.
\end{proof}
\restapiecewiselb*
\begin{proof}
This result directly follows from Lemma~\ref{lemma:lb} by choosing $\Upsilon = \Upsilon_T$. Indeed, the set of problems $\left\{i^\star \in \left(\left\{0\right\} \cup \arms\right)^{\Upsilon_T} \right\}$ satisfy Assumptions~\ref{assum:piece-wise} and~\ref{assum:restless_rotting} as soon as $\Upsilon_T\Delta \leq V$, \ie $\Upsilon_T \leq \pa{\frac{32V^2T }{K\sigma^2}}^{\nicefrac{1}{3}}$.
\end{proof}

\restavariationlb*
\begin{proof}
\sloppy
We want to use Lemma~\ref{lemma:lb} but we need to make the set of problems $\left\{i^\star \in \left(\left\{0\right\} \cup \arms\right)^{\Upsilon_T} \right\}$ comply with Assumption~\ref{assum:variation}. First, the function are bounded by $-V_T$. Hence, we need : 
\begin{equation}
\label{eq:bounded_condition}
  \Upsilon \Delta \leq V_T.  
\end{equation}

Second, the total variation is bounded according to Equation~\ref{eq:defbudget}. When $t$ is not a breakpoint, the variation is null. At each breakpoint, the maximal variation across the arm is $2\Delta$. For $\Upsilon-1$ breakpoint, we have that 


\begin{equation}
\label{eq:totalvar_condition}
  2\Delta \pa{\Upsilon-1}  \leq V_T.  
\end{equation}

Since $ 2\Delta \pa{\Upsilon-1} \leq \frac{\sigma}{2}\sqrt{\frac{K}{2T}}\Upsilon^{\nicefrac{3}{2}} $, we choose 
\begin{equation}
\label{eq:set_upsilon}
\Upsilon = \min\pa{\max\pa{\floor{ 2\left(\frac{V_T^{2}T}{K\sigma^{2}}\right)^{\nicefrac{1}{3}}},1},T}.
\end{equation}

By construction, \ref{eq:set_upsilon} satisfies \ref{eq:totalvar_condition}. Moreover, when $\Upsilon >1$, \ref{eq:totalvar_condition} is more restrictive than \ref{eq:bounded_condition}. For $\Upsilon = 1$, we simply assume $\Delta \leq V_T$, \textit{i.e.} $V_{T} \geq \sigma \sqrt{\frac{K}{8 T}}$.

Plugging \ref{eq:set_upsilon} in Lemma~\ref{lemma:lb} allows us to conclude 
\[
    \mathbb{E}\left[R_T(\pi)\right] \geq \frac{1}{16\sqrt{2}} V_T^{\nicefrac{1}{3}}\sigma^{\nicefrac{2}{3}}K^{\nicefrac{1}{3}}T^{\nicefrac{2}{3}}.
\]
\end{proof}
%!TEX root = ../main.tex
\section{Analysis of adaptive window policies on restless rotting bandits.} 
\label{sec:restless-theory}
In Chapter~\ref{ch:rested}, we presented four adaptive window policies (\FEWA, \RAWUCB, \EFFFEWA, \EFFRAW). In this section, we will show that the exact same policies are able to match interesting upper bounds on the restless problems.  The proof of the regret upper bounds in the rested case uses three main steps. First, we design one favorable event per round on which all the constructed statistics concentrate on a well-chosen confidence region, such that it holds with sufficiently high probability. This part does not use that we faced a rested non-stationary environment; it only uses the concentration of independent subgaussian variables which remains true in our restless problem due to Doob's optional skipping. Hence, we restate Propositions~\ref{prop:prb_favorable_event} and~\ref{prop:prb_favorable_event_eff},
\begin{proposition}
\label{prop:prb_favorable_event_full}
We recall that, for any round $t$ and confidence $\delta_{t} \triangleq 2t^{-\alpha}$, we define
%
\begin{align*}
&\!\HPevent\! \triangleq\! \Big\{ \forall i\!\in\!\arms,\ \forall n \!\leq\! t\!-\!1 ,\ \forall h \!\leq\! n, \big| \hmu^h_i(t, \pi) - \bmu^h_i(t, \pi) \big| \!\leq\! c(h, \delta_{t}) \!\Big\}\\
&\!\HPeff\! \triangleq\! \Big\{ \forall i\!\in\!\arms, \forall n \!\leq\! t\!-\!1 , \forall h_j \!\in\! \Him(n), \big| \hmueff(t,\pi) - \bmueff(t,\pi) \big| \!\leq\! c(h_j, \delta_{t}) \!\Big\}
\end{align*}
with  $c(h,\delta_{t}) \triangleq \sqrt{2 \subgaussian^2\log(2/\delta_t)/h}$. Then, for a policy $\pi$ which pulls each arms once at the beginning, and for all $t>K$,
\[
\PPempty\Big[\bar{\HPevent}\Big] \leq \frac{Kt^2\delta_{t}}{2}=Kt^{2-\alpha} \text{ and } \PPempty\Big[\bar{\HPeff}\Big] \leq 3Kt\delta_t= 6Kt^{1-\alpha}.
\]
\end{proposition} 

Then, we use the mechanics of the algorithms to relate the average past performance of the selected arm with the current best value of the arms. As we noticed in the proofs (see e.g. the proof of Lemma~\ref{lem:core-FEWA}), we do not use the rested aspect of the problem. In fact, these results hold for a more general reward function $\mu_i(t,n)$ which is non-increasing with both $t$ and $n$. Therefore, we also restate Lemmas~\ref{lem:core-FEWA}, \ref{lem:core-RAWUCB} and~\ref{lem:core-eff},
\begin{lemma}
\label{lem:core-full}
At any round $t$ on favorable event $\HPevent$ (respectively, $\HPtwo$), if arm~$i_{t}$ is selected by $\pi \in \left\{\piF, \piR\right\}$ (respectively, $\pi \in \left\{\piEF, \piER\right\} $ tuned with $m=2$), for any $h \leq \Nitmone$,  the average of its $h$ last pulls cannot deviate significantly from the best available arm at that round, i.e.,
\begin{equation*}
\bmu^{h}_{i_t}(t,\pi) \geq \max_{i \in \arms} \mu_{i}(t)- \frac{C_\pi}{\sqrt{2\alpha}} c(h, \delta_t) \quad \text{with } 
\begin{cases}
C_{\piR} = 2\sqrt{2\alpha} \text{ and } C_{\piER} = \frac{4\sqrt{\alpha}}{\sqrt{2}-1}\\
C_{\piF} = 4\sqrt{2\alpha} \text{ and }C_{\piEF} = \frac{8\sqrt{\alpha}}{\sqrt{2}-1}
\end{cases}\cdot
\end{equation*}
\end{lemma}
Last, we use a specific rested regret decomposition to show that our algorithms are near-optimal both problem-dependent and problem-independent wise on rested rotting bandits. Unfortunately, this part cannot be used for the restless analysis. However, with a specific restless regret decomposition (see the proof in Subsection~\ref{ss:restless-proof}), we can show that our policies matches the two aforementioned lower bounds up to poly-logarithmic terms without any knowledge of the horizon $T$ nor $\Upsilon_T$ or $V_T$.
%
\begin{tBox}
\begin{restatable}{theoremeT}{restapiecewisetheorem}
\label{th:piecewise-minimax}
Let $\pi \in \left\{ \piF, \piR\right\}$ tuned with $\alpha \geq 4$ or $\pi \in \left\{ \piEF, \piER\right\}$ tuned with $\alpha \geq 3$ and $m=2$. For any piece-wise stationary bandit scenario with means $\{\mu_i(t)\}_{i,t}$ satisfying Assumptions~\ref{assum:piece-wise} and~\ref{assum:restless_rotting}  with $\Upsilon_T-1$ change-points, $\pi$  suffers an expected regret\,
\[
\EE{R_T(\pi)} \leq C_\pi \sigma \sqrt{\log{T}} \pa{ \sqrt{\Upsilon_T KT} + \Upsilon_T K} + 6KV.
\]
\end{restatable}
\end{tBox}
\vspace{-2em}
\begin{tBox}
\begin{restatable}{theoremeT}{restabudgettheorem}
\label{th:variation-minimax}
Let $\pi \in \left\{ \piF, \piR\right\}$ tuned with $\alpha \geq 4$ or $\pi \in \left\{ \piEF, \piER\right\}$ tuned with $\alpha \geq 3$ and $m=2$. For any variation budget bandit scenario with means $\{\mu_i(t)\}_{i,t}$ satisfying Assumptions~\ref{assum:variation} and~\ref{assum:restless_rotting}  with variation budget $V_T$, $\pi$ suffers an expected regret\,
\[
\mathbb{E}\left[R_T(\pi)\right] \leq 4\pa{C_\pi^2 \sigma^2 V_T K T^2\log{T}}^{\nicefrac{1}{3}} \!+ 2\Big(C_\pi \sigma V_T^2  K^2  T \sqrt{\log{T}}\Big)^{\nicefrac{1}{3}} \!+ 6 V_T K.
\]
\end{restatable}
\end{tBox}
%
The remaining terms are of second-order when $KV_T \leq \cO{\pa{T}}$, which is a necessary condition for the problem to be learnable (see Proposition~\ref{prop:variation_lb2}). 
%
\paragraph{Are rotting restless bandits easier?} Learning at the minimax rate without knowing $\Upsilon_T$ or $V_T$ was achieved in the non-rotting setup by significantly more complex algorithms. For instance, \citet{auer2019adaptively} use a combination of filtering on the set of potentially good arms, forced exploration planning on identified bad arms, and full restart of the algorithm when a change is detected. This algorithmic complexity has a performance cost, as \ADSWITCH is guaranteed to achieve 56 times the leading term in Theorem~\ref{th:piecewise-minimax}. Moreover, these algorithms rely on doubling trick when the horizon is unknown, which also has a regret cost compared to intrinsically anytime algorithms \citep{besson2018doubling}.

Yet, Proposition~\ref{prop:piecewise_lb2}  and~\ref{prop:variation_lb2} show that the rotting assumption do not improve the minimax rate for the two considered setups. Interestingly both these lower bounds are matched by (tuned) \EXPS \citep{auer2002nonstochastic}, an algorithm originally designed for switching best arm in adversarial sequences of rewards. This is comparable to the fixed best arm world:  adversarial and stochastic bandits share the same minimax rate which is matched in both setups by \EXP. The main interest of the stochastic assumption is to allow for \textit{problem dependent analysis}. For the stochastic stationary bandits, it leads to a stronger $\cO{\pa{\log\pa{T}}}$ bounds. In the (non-rotting) piece-wise stationary setting, we argued in Subsection~\ref{subsec:piecewise} that the learner has to maintain $\cO\pa{\sqrt{T}}$ exploratory pulls to shield against increase of currently suboptimal arm (see Proposition~\ref{prop:garivier-lb} and Corollary~\ref{cor:garivier-lb}).

The decreasing Assumption~\ref{assum:restless_rotting} excludes the problems where suboptimal arms increases to become optimal from the set of possible problems. Theorem~\ref{th:piecewise_pd} shows that not only \RAWUCB is able to recover the $\cO\pa{\log\pa{T}}$ on stationary problems but also recovers the same rate on each batch of a rotting piece-wise stationary problem. 
\begin{tBox}
\begin{restatable}{theoremeT}{restapiecewisetheorempd}
\label{th:piecewise_pd}
Let $\pi \in \left\{ \piF, \piR\right\}$ tuned with $\alpha \geq 4$ or $\pi \in \left\{ \piEF, \piER\right\}$ tuned with $\alpha \geq 3$ and $m=2$. For any piece-wise stationary bandit scenario with means $\{\mu_i(t)\}_{i,t}$ satisfying Assumptions~\ref{assum:piece-wise} and~\ref{assum:restless_rotting}  with $\Upsilon_T-1$ change-points, $\pi$ suffers an expected regret\,
\[
    \mathbb{E}\left[R_T(\pi)\right] \leq \sum_{k=0}^{\Upsilon_T-1} \sum_{i\in\arms} \frac{C_\pi^2 \sigma^2\log{T}}{\Delta_{i,k}} +  C_\pi \sigma \Upsilon_T K \sqrt{ \log{T}} + 6KV. 
\]
\end{restatable}
\end{tBox}
%TODO talk about tuning ?
%When $\Upsilon_T = 1$ (no changepoint), \RAWUCB recovers the same guarantee than \UCBone. However, \UCB with a more careful tuning $\delta_t \sim \frac{1}{t\log{t}^2}$ match \citet{lai1985asymptotically} asymptotic factor for Gaussian bandits \citep{lattimore2019bandit}. We leave the two following questions for future analysis: 1) Is there a tuning of $\delta_t$ such that \RAWUCB is asymptotic for stationary problems? 2) If yes, can \RAWUCB with this asymptotic tuning recovers some (rested or restless) non-stationary guarantees?

Notice that \citet{mukherjee2019distribution} use a different assumption to recover a similar problem-dependent bound. Indeed, they assume that all the arms change at the same time. In the counter-example displayed on Figure~\ref{fig:garivier-lb}, it is important that the arm 1 changes its value while arm 2 is stationary. Indeed, in that case, the learner cannot infer the increase on arm 2 by sampling arm 1. Hence, the assumption of \citet{mukherjee2019distribution} excludes this counter-example of the set of possible problems. That is why they were able to provide a logarithmic problem-dependent bound.  

%Therefore, \RAWUCB is near minimax optimal \emph{and} reaches the asymptotic rate for stationary bandits $O(\sum_{i \in \arms} \frac{\log(T)}{\Delta_i})$ (see Corollary~\ref{dependent_theorem}) without knowing the number of change points nor the horizon. 

\subsection{Proofs}
\label{ss:restless-proof}

\subsubsection*{Sketch.} 

We start by separating the regret on the bad events $\bar{\HPevent}$ from the good events $\HPevent$. According to Proposition~\ref{prop:prb_favorable_event}, the bad events $\bar{\xi}_t$ have low probability for appropriate $\alpha$. For $\alpha = 4$, they weigh at most $\cO{\pa{KV}}$ in the expected regret.  On the good events, we write:
\vspace{-4pt}
\begin{equation}
\label{eq:restless-regret-decompo}
R_T(\pi)= \sum_{t=1}^T \mu_{i_t^\star}(t) - \bar{\mu}_{i_t}^{h_t}(t, \pi) + \bar{\mu}_{i_t}^{h_t}(t, \pi) - \mu_{i_t}(t).   
\end{equation}

Notice that Lemma~\ref{lem:core-full} can bound the first difference for any $h_t$. When the reward is piece-wise stationary, we can select $h_t$ such that we include all the pulls of arm $i_t$ from the current stationary batch. If there is none, then it is the first pull of arm $i_t$ in this batch. We handle these $\cO{\pa{K\Upsilon_T}}$ rounds separately (see Lemma~\ref{lem:FP}). In the other cases, we note that the second difference is null because $\bar{\mu}_{i_t}^{h_t}(t, \pi) = \mu_{i_t}(t) = \mu_i^k$ by the piece-wise stationary assumption. The remaining of the proofs of Theorem~\ref{th:piecewise-minimax} and~\ref{th:piecewise_pd} are then very similar to the analysis of \cite{auer2002finite} on each stationary batch. Indeed, Lemma~\ref{lem:core-full} is similar to the two confidence bounds guarantee of \UCBone's guarantee.

In the variation budget setting, there is no stationary batches. Hence, we cannot choose an $h_t$ which cancels the second difference in Equation~\ref{eq:restless-regret-decompo}. Yet, we still decompose the rounds in $\Upsilon$ batches of equal length for the analysis. We choose $h_t$ such that we include all the pulls of arm $i_t$ from the current batch. For the sum of the first differences in Equation~\ref{eq:restless-regret-decompo}, there is no difference with the piece-wise stationary case and we can bound
\vspace{-4pt}
\begin{equation}
\label{eq:variance_bound}
    \sum_{t=1}^T \mu_{i_t^\star}(t) - \bar{\mu}_{i_t}^{h_t}(t, \pi)\leq \tcO{\pa{\sqrt{K\Upsilon T}}}.
\end{equation}
We call $\Delta_i^k \triangleq \mu_i(t_k) - \mu_i(t_{k+1})$, the total variation of arm $i$ in batch $k$. The sum of second differences in Equation~\ref{eq:restless-regret-decompo} can be bounded as follows: on each batch of $T\Upsilon^{-1}$ rounds, each second difference is bounded by $\max_{i\in \arms} \Delta_i^k$. When we sum over the batches, we get
\vspace{-4pt}
\begin{equation}
\label{eq:bias_bound}
  \sum_{t=1}^T  \bar{\mu}_{i_t}^{h_t}(t, \pi) - \mu_{i_t}(t)\leq \frac{T}{\Upsilon}\sum_{k=0}^{\Upsilon-1}\max_{i \in \arms}\Delta_i^k  \leq \frac{TV_T}{\Upsilon}\, .  
\end{equation}
Indeed, in the middle term, we have a maximum on the summed variation of arm $i$ in batch $k$. On the right-hand side, we have $V_T$ which bounds the sum over the rounds of maximal variation of the arms (see Equation~\ref{eq:defbudget}). Thus, the right-hand side is larger because the maximum of sums is smaller than the sum of maximums. We can then choose $\Upsilon = \tcO{\pa{T^{\nicefrac{1}{3}}V_T^{\nicefrac{2}{3}}K^{\nicefrac{-1}{3}}}}$ to minimise the sum of Equations~\ref{eq:variance_bound} and ~\ref{eq:bias_bound}. It leads to the leading term of our Theorem~\ref{th:variation-minimax}. Notice that we still have to handle the first pull of each arm in each batch. If we bound roughly each first pull by $V_T$, we would get $K\Upsilon V_T \sim \tcO{\pa{V_T^{\nicefrac{5}{3}}}}$ which would be the leading term for large $V_T$. Our Lemma~\ref{lem:FP} is more careful such that it leads to a second order term when $KV_T \leq o\pa{T}$.

\subsubsection*{Full proof}
\begin{lemma}[Bound on unfavorable events. Decomposition in unspecified batches. Bound on the first pull of each arm in each batch] %TODO
\label{lem:FP}
Let an integer $\Upsilon \in\left\{ 1, \dots,T\right\}$.\\
Let $\mu_i : \NN^\star \rightarrow \left[0, -V\right]$, the $K$ decreasing reward functions.\\ 
Let $\left\{t_k\in\left\{ 1, \dots,T\right\} \right.\allowbreak\left. |\, t_k > t_{k-1}\right\}_{k\in \left\{ 1, \dots,\Upsilon-1\right\}}$ a set of $\Upsilon - 1$ distinct rounds delimiting $\Upsilon$ batches. We set $t_0=0$ and $t_\Upsilon = T$. \\
We call $h_{i}^{k} \triangleq \sum_{t=t_k +1}^{t_{k+1}} \mathbbm{1}\left(i_{t} = i\right)$ the number of pulls of arm $i$ in batch $k$ and $t_i^k(h)$ the time at which arm $i$ is pulled for the $h$-th time since $t_k + 1$. We also call $\arms_k \triangleq  \left\{ i \in \arms | h_i^k \geq 1\right\}$ the set of pulled arms in batch $k$. 

Then, $\pi \in \left\{\piR, \piF \right\}$ run with $\alpha \geq 4$, or $\pi \in \left\{\piER, \piEF \right\}$ run with $m=2$ and $\alpha \geq 3$, suffers an expected regret of
\begin{align*}
\EE{R_T(\pi)} \leq &  \, \EE{\sum_{k=0}^{\Upsilon-1} \sum_{i\in\arms_k}\sum_{t=t_k +1 }^{t_{k+1}}\sum_{h=2}^{h^k_{i}}\mathbbm{1}\pa{ t = t_i^k(h) \land \HPevent} \Big(\mu_{\star}(t) - \mu_{i}(t)\Big)} \\
&+   C_\pi \sigma \Upsilon K\sqrt{\log{T}} + 6KV.
\end{align*}
\end{lemma}
\begin{proof}
We start by separating the favorable events from the unfavorable events:
\begin{equation}
\label{eq:event_sep}
    R_T(\pi) = \underbrace{\sum_{t=1}^T \mathbbm{1}\pa{\HPevent} \pa{\mu_{\star}(t) - \mu_{i_t}(t)}}_{R_T(\pi | \HPevent)} + \underbrace{\sum_{t=1}^T \mathbbm{1}\big(\bar{\HPevent}\big) \pa{\mu_{\star}(t) - \mu_{i_t}(t)}}_{{R_T(\pi | \bar{\HPevent})}} \,,
\end{equation}
with $\mu_\star(t) \triangleq \max_{i\in\arms}\mu_i(t)$. For $\alpha \geq 4$, we can bound the cost of the unfavorable events thanks to Proposition~\ref{prop:prb_favorable_event_full},
\begin{equation}
\label{eq:bad_event}
    \EE {R_T(\pi | \bar{\HPevent})} \leq \sum_{t=1}^T \PP{\bar{\HPevent}} V  \leq \sum_{t=1}^T \frac{KV}{t^2} = \frac{KV\pi^2}{6} \leq 2KV.
\end{equation}

On the favorable events, given any ordered set of $\Upsilon -1$ breakpoints $\left\{t_k\right\}$, we divide the horizon in $\Upsilon$ batches $\left\{t_k+1, \dots, t_{k+1} \right\}_{k \leq \Upsilon-1}$, 
\[
R_T(\pi | \HPevent) \leq \sum_{k=0}^{\Upsilon-1} \sum_{t=t_{k} +1 }^{t_{k+1}} \mathbbm{1}\pa{\HPevent} \big(\mu_{\star}(t) - \mu_{i_t}(t)\big).
\]
We define $h_{i}^{k}$ the number of pulls of arm $i$ in batch $k$, \textit{i.e.}  $h_{i}^{k} = \sum_{t=t_k +1}^{t_{k+1}} \mathbbm{1}\left(i_{t} = i\right)$. We use $t_i^k(h)$ to designate the time at which arm $i$ is pulled for the $h$-th time since $t_k$.
\[
R_T(\pi | \HPevent) \leq \sum_{k=0}^{\Upsilon-1} \sum_{t=t_k +1 }^{t_{k+1}} \sum_{i\in\arms_k} \sum_{h=1}^{h^k_{i}} \mathbbm{1}\pa{t_i^k(h) = t \land \HPevent} \Big(\mu_{\star}(t) - \mu_{i}(t)\Big).
\]
We split the regret on the first pulls of each batch,
\begin{align}
\label{eq:fp_op}
\begin{split}
    R_T(\pi | \HPevent) = & \underbrace{\sum_{k=0}^{\Upsilon-1}\sum_{t=t_{k} +1 }^{t_{k+1}} \sum_{i\in\arms_k} \mathbbm{1}\pa{t = t_i^k(1) \land \HPevent}\Big(\mu_{\star}(t) - \mu_{i}(t)\Big)}_{FP} \\ & +  \underbrace{\sum_{k=0}^{\Upsilon-1}\sum_{t=t_{k} +1 }^{t_{k+1}} \sum_{i\in\arms_k} \sum_{h=2}^{h^k_{i}}\mathbbm{1}\left( t = t_i^k(h) \land \HPevent \right)\Big( \mu_{\star}(t) - \mu_{i}(t)\Big)}_{OP} .
\end{split}
\end{align}

\paragraph{Analysis of the first pulls.}

We call $k_i^1$, the index of the batch at which arm $i$ is pulled for the first time (we assume that $T\geq K$).  We call $\arms_k^2 \triangleq \left\{ i \in \arms_k | k > k_i^1\right\}$, the set of arms pulled at least once during batch $k$ and at least once in a batch before $k$. We split the regret due to the very first pull each arm from the other first pulls in each batch,
\begin{align*}
FP  = &\sum_{k=0}^{\Upsilon-1}\sum_{i\in\arms_k}\sum_{t=t_{k} +1 }^{t_{k+1}}  \mathbbm{1}\pa{ t = t_i^k(1) \land \HPevent}\Big(\mu_{\star}(t) - \mu_{i}(t)\Big)\\
\leq& \sum_{i \in \arms}  \Big(0- \mu_i(t_i^{k_i^1}(1))\Big) +  \sum_{k=1}^{\Upsilon -1} \sum_{i\in \arms_k^2}\sum_{t=t_k +1 }^{t_{k+1}}  \mathbbm{1}\pa{t = t_i^k(1) \land \HPevent}\Big(\mu_{\star}(t) - \mu_{i}(t)\Big)\\
 =& \sum_{i \in \arms} \Big(0- \mu_i(t_i^{k_i^1}(1))\Big) \\
& + \sum_{k=1}^{\Upsilon -1} \sum_{i\in \arms_k^2} \sum_{t=t_k +1 }^{t_{k+1}}  \mathbbm{1}\pa{ t = t_i^k(1) \land \HPevent}\Big(\mu_{\star}(t) - \bar{\mu}^1_i(t, \pi) + \bar{\mu}^1_i(t,\pi) - \mu_{i}(t)\Big).
\end{align*}

The inequality is justified because $\mu_i(t) \leq 0$ for all $t$. In the last equation, we simply introduce $\bar{\mu}^1_i(t,\pi)$, the last pulled sample of arm $i$, which is well defined after the first pull of each arm.
According to Lemma~\ref{lem:core-full}, the first difference is bounded on the high-probability event $\HPevent$,
\begin{equation}
    \label{eq:fp_lemma1}
    \sum_{t=t_k +1 }^{t_{k+1}} \mathbbm{1}\pa{t = t_i^k(1) \land \HPevent}\pa{\mu_{\star}(t) - \bar{\mu}^1_i(t, \pi)} \leq \frac{C_\pi}{\sqrt{2\alpha}} c(1,2T^{-\alpha}) = C_{\pi} \sigma \sqrt{\log{T}}.
\end{equation}


We will show that we can telescope the second sum. First, we notice that we can collapse the sum on $t$ using $ \mathbbm{1}\pa{t = t_i^k(1)}$. Moreover, $\HPevent$ will not be needed: hence we can drop $\mathbbm{1}\pa{\HPevent} \leq 1 $.
\begin{equation}
\label{eq:fp_collapse}
 \sum_{t=t_k +1 }^{t_{k+1}} \mathbbm{1}\pa{ t = t_i^k(1) \land \HPevent}\pa{\bar{\mu}^1_i(t,\pi) - \mu_{i}(t)} \leq \bar{\mu}^1_i(t_i^k(1), \pi) - \mu_{i}(t_i^k(1)).
\end{equation}

For a given batch $k$ on which arm $i$ is pulled, the precedent reward sample has a mean $\bar{\mu}_i^1\pa{t_i^k\pa{1}, \pi}$. This sample is the last pull of the last batch $k'$ before $k$ on which arm $i$ is pulled. Hence, its mean is smaller than the mean of the first pull on this same batch $k'$ because the reward is decreasing. Hence, the sum can telescope
\begin{align}
\label{eq:fp_telescoping} 
\sum_{i \in \arms} \Big(0 -& \mu_i(t_i^{k_i^1}(1))\Big) + \sum_{k=1}^{\Upsilon -1}  \sum_{i\in \arms_k^2} \sum_{t=t_k +1 }^{t_{k+1}}\mathbbm{1}\pa{ t = t_i^k(1) \land \HPevent}\pa{ \bar{\mu}^1_i(t,\pi) - \mu_{i}(t)} \nonumber\\
& \leq \sum_{i \in \arms}\left\{ 0- \mu_i(t_i^{k_i^1}(1)) + \sum_{k=k_i^1 + 1}^{\Upsilon-1 } \mathbbm{1}\pa{h^k_{i} \geq 1  }\pa{\bar{\mu}^1_i(t_i^k(1), \pi) - \mu_{i}(t_i^k(1))} \right\} \nonumber\\
& \leq \sum_{i \in \arms} \Big(0-\mu_i(T)\Big) \leq KV\,.  
\end{align}
The first inequality uses the definition of $\arms_k^2$ along with Equation~\ref{eq:fp_collapse}. The second inequality follows from the telescoping argument presented above. The third inequality uses that $\mu_i(T) \geq -V$. Gathering Equation~\ref{eq:fp_lemma1} and  ~\ref{eq:fp_telescoping}, we can bound the term $FP$ (defined in Equation~\ref{eq:fp_op}) 
\begin{equation}
\label{eq:fp_bound}
FP \leq  KV + \sum_{k=1}^{\Upsilon - 1} \sum_{i\in \arms_k^2} C_\pi \sigma \sqrt{\log{T}} \leq KV + C_\pi \sigma \Upsilon K\sqrt{\log{T}} .    
\end{equation}



\paragraph{Conclusion.} From Equation~\ref{eq:event_sep}, we can bound the expected regret on the unfavorable events thanks to Equation~\ref{eq:bad_event}. On the favorable events, we can split the rounds in batches on which we isolate the first pull of each arm on each batch thanks to Equation~\ref{eq:fp_op}. Finally, we bound the regret due to these first pulls thanks to Equation~\ref{eq:fp_bound}, and for $\alpha \geq 4$,
\begin{align*}
\EE{R_T(\pi)} \leq &  \, \EE{\sum_{k=0}^{\Upsilon-1} \sum_{i\in\arms_k}\sum_{t=t_k +1 }^{t_{k+1}}\sum_{h=2}^{h^k_{i}}\mathbbm{1}\pa{ t = t_i^k(h) \land \HPevent} \Big( \mu_{\star}(t) - \mu_{i}(t)\Big)} \\
&+  C_\pi \sigma \Upsilon K \sqrt{\log{T}} + 3KV.
\end{align*}

For the efficient algorithms, we can use the same proof with $\HPtwo$ and get for $\alpha \geq 3$, 
\begin{align*}
\EE{R_T(\pi)} \leq &  \, \EE{\sum_{k=0}^{\Upsilon-1} \sum_{i\in\arms_k}\sum_{t=t_k +1 }^{t_{k+1}}\sum_{h=2}^{h^k_{i}}\mathbbm{1}\pa{ t = t_i^k(h) \land \HPevent} \Big(  \mu_{\star}(t) - \mu_{i}(t)\Big)} \\
&+  C_\pi \sigma \Upsilon K \sqrt{\log{T}} + 6KV.
\end{align*}

\end{proof}

\begin{lemma}[Analysis of the second pulls in each batch under the favorable events.]\label{lem:OP}
Let $\Delta_i^k \triangleq \mu_i(t_k+1) - \mu_i(t_{k+1})$, the decrement of arm $i$ in batch $k$. For any arm $i$ and any consecutive rounds $\left\{t_k+1, \dots , t_{k+1}\right\}$ such that $i$ is pulled $h_i^{k} \geq 1$ times, the regret due to the pulls after the first one can be bounded under the favorable events, 
\begin{multline*}
\sum_{t=t_{k} +1 }^{t_{k+1}} \sum_{h=2}^{h^k_{i}}\mathbbm{1}\left(t = t_i^k(h) \land \HPevent \right)\Big(  \mu_{\star}(t) - \mu_{i}(t)\Big) \\ \leq  \pa{h_i^k-1}\Delta_i^k + \sum_{h=2}^{h^k_{i}}\mathbbm{1}\left(\HPt{t_i^k(h)}\right)\pa{  \mu_{\star}(t_i^k(h)) - \bar{\mu}_i^{h-1}(t_i^k(h),\pi)} .
\end{multline*}
\end{lemma}
\begin{proof}
We call $\Delta_{i}(t,t')\triangleq \mu_i(t) - \mu_i(t')$ the variation of arm $i$ between times $t$ and $t'$.
As a short notation, we refer to $\Delta_i^k \triangleq \Delta_{i}(t_k+1,t_{k+1})$ for the variation of arm $i$ in batch $k$.

\begin{equation}
\label{eq:batch_delta_var}
    \forall h\leq h_i^k, \quad \mu_i( t_i^k(h)) \geq \mu_i(t_{k+1}) =  \mu_i(t_k +1 ) - \Delta_i^k \geq \bar{\mu}_i^{h-1}( t_i^k(h), \pi) - \Delta_i^k\,.
\end{equation} 
The two inequalities are justified by the rewards decay. Indeed, any pull in batch $k$ has a higher reward than the value of arm $i$ at the end of the batch $t_{k+1}$. Moreover, the value at the beginning of the batch is higher that any average of $h$ value in this batch. The middle equality follows from the definition of $\Delta_i^k$.

Then, we plug Equation~\ref{eq:batch_delta_var} in the left hand side of our claim,
\begin{align*}
\sum_{t=t_{k} +1 }^{t_{k+1}}
\sum_{h=2}^{h^k_{i}}\mathbbm{1}&\left(t = t_i^k(h) \land \HPevent\right)\Big(  \mu_{\star}(t) - \mu_{i}(t) \Big) \\
& = \sum_{h=2}^{h^k_{i}}\mathbbm{1}\left(\HPt{t_i^k(h)}\right)\pa{  \mu_{\star}(t_i^k(h)) - \mu_{i}(t_i^k(h))} \\
&\leq \sum_{h=2}^{h^k_{i}}\mathbbm{1}\left(\HPt{t_i^k(h)}\right)\pa{  \mu_{\star}(t_i^k(h)) - \bar{\mu}_i^{h-1}( t_i^k(h),\pi) + \Delta_i^k}\\
&\leq  \pa{h_i^k-1}\Delta_i^k + \sum_{h=2}^{h^k_{i}}\mathbbm{1}\left(\HPt{t_i^k(h)}\right)\pa{  \mu_{\star}(t_i^k(h)) - \bar{\mu}_i^{h-1}(t_i^k(h),\pi)} .
\end{align*}
The last inequality is justified by $\mathbbm{1}\left(\HPt{t_i^k(h)}\right)\leq 1$.
\end{proof}

\subsubsection*{Piecewise stationary rotting bandits.}
\sloppy
Let $\left\{t_k\right\}_{\left\{k \leq \Upsilon_T\right\}}$ be the set of breakpoints with $t_0=0$ and $t_{\Upsilon_T} = T$. For all $t \in \left\{t_k\! +\!1 , \dots , t_{k\!+\!1}\right\}$, $\mu_i(t) = \mu_i^k$. We denote $i^\star_k \in \argmax_{i\in \arms} \mu_i^k$ (one of) the best arm(s) in batch $k$, and $\mu_{\star}^k \triangleq \max_{i\in \arms} \mu_i^k$, the corresponding best value. We also call $\Delta_{i,k} \triangleq \mu_{\star}^k - \mu_i^k$ the gap between arm $i$ and optimal arm in batch $k$.

\begin{lemma}
\label{lem:OP-piecewise}
For an arm $i$ and a stationary batch $k$, we call $h_{i,\xi}^k \triangleq \max\left(h \leq h_i^k \text{ s.t. } \HPt{t_i^k(h)} \text{ holds}\right)$ the last pull of arm $i$ in batch $k$ under the favorable events (possibly 0). If $h_{i,\xi}^k \geq 1$, the regret due to the second pulls on the favorable events is bounded by,
\[
\sum_{t=t_{k}+1}^{t_{k+1}} \sum_{h=2}^{h_{i}^{k}} \!\mathbbm{1}\!\pa{\! t=t_{i}^{k}(h) \land \HPevent \!}\!\Big(\!\mu_{\star}(t)\!-\!\mu_{i}(t)\!\Big) \leq\pa{\!h_{i, \xi}^{k}\!-\!1 \!} \Delta_{i, k} \leq C_\pi\sigma \sqrt{\pa{\!h_{i,\xi}^k\!-\!1\!}\log{T}}.
\]
\end{lemma}
\begin{proof}
We apply Lemma~\ref{lem:OP} on each stationary batch. Hence, $\Delta_i^k =0$ and we can write,
%
\begin{equation*}
\!\sum_{t=t_{k} +1 }^{t_{k+1}}\! \sum_{h=2}^{h^k_{i}} \! \mathbbm{1}\left(\!t = t_i^k(h) \land \HPevent \!\right)\Big(\! \mu_{\star}(t) - \mu_{i}(t)\!\Big) \leq   \sum_{h=2}^{h^k_{i}} \!\mathbbm{1}\left(\!\HPt{t_i^k(h)}\!\right)\pa{\!  \mu_{\star}(t_i^k(h)) - \bar{\mu}_i^{h-1}(t_i^k(h),\pi)\!}\! .
\end{equation*}

We notice that $\mu_{\star}(t_i^k(h)) = \mu_{\star}^{k}$. We call $h_{i,\xi}^k \triangleq \max\left(h \leq h_i^k\, \,\text{ s.t. } \HPt{t_i^k(h)} \text{ holds} \right)$. Hence,
\begin{align*}
\sum_{h=2}^{h^k_{i}}\!\mathbbm{1}\left(\HPt{t_i^k(h)}\right)\!\pa{ \! \mu_{\star}(t_i^k(h)) \!-\! \bar{\mu}_i^{h\!-\!1}(t_i^k(h),\pi)\!}  & \!=\! \sum_{h=2}^{h^k_{i,\xi}}\!\mathbbm{1}\left(\HPt{t_i^k(h)}\right)\!\pa{  \mu^k_{\star} \!-\! \bar{\mu}_i^{h\!-\!1}(t_i^k(h), \pi)} \\
 &\leq \sum_{h=2}^{h^k_{i, \xi}} \mu_{\star}^{k} - \bar{\mu}_i^{h-1}(t_i^k(h), \pi)\\
 &= \sum_{h=2}^{h^k_{i, \xi}} \mu_{\star}^{k} - \mu_i^k\\
 & = \pa{h_{i,\xi}^k - 1 }\Delta_{i,k}\, .  
\end{align*}

The first equality follows from $\forall h > h_{i,\xi}^k, \, \mathbbm{1}\left(\HPt{t_i^k(h)}\right) =0$ by definition of $h_{i,\xi}^k$. The first inequality follows by dropping $\mathbbm{1}\left(\HPt{t_i^k(h)}\right) \leq 1$. The second equality uses that the function is stationary in batch $k$ : $\forall h \leq h_{i,\xi}^k, \bar{\mu}_i^{h-1}(t_i^k(h), \pi) = \mu_{i}^k.$ The last equality follows by definition of $\Delta_{i,k}$ (which does not depend on the summand index $h$).

Then, we apply Lemma~\ref{lem:core-full} at time $t_i^k\pa{h_{i,\xi}^k}$. By definition of $h_{i,\xi}^k$, $\mathbbm{1}\!\left(\!\HPt{t_i^k(h_{i,\xi}^k)\!}\right) = 1$.
\begin{gather*}
 \pa{h_{i,\xi}^k - 1 }\Delta_{i,k}  \leq \frac{C_\pi}{\sqrt{2\alpha}}\pa{h_{i,\xi}^k - 1 }c(h_{i,\xi}^k\!-\!1, 2T^{-\alpha}) = C_\pi \sigma \sqrt{\pa{h_{i,\xi}^k-1}\log{T}}.\qedhere
\end{gather*} 
\end{proof}

\begin{tBox}\restapiecewisetheorem*
\end{tBox}
\begin{proof}

We apply Lemma~\ref{lem:OP-piecewise},
\[\sum_{k=0}^{\Upsilon_T-1} \! \sum_{i \in \mathcal{K}_{k}} \!\sum_{t=t_{k}+1}^{t_{k+1}} \! \sum_{h=2}^{h_{i}^{k}} \!\mathbbm{1}\!\left( \!t\!=\!t_{i}^{k}(h)\!\land\!\HPevent\!\right)\Big(\mu_{\star}(t)\!-\!\mu_{i}(t)\Big) \leq  \sum_{k=0}^{\Upsilon_T-1} \!\sum_{i\in\arms_k}\!  C_\pi \sigma \sqrt{h_{i,\xi}^k\log{T}} .\]

We notice that $ \sum_{k=0}^{\Upsilon_T -1}\sum_{i\in\arms_k} h_{i,\xi}^k\leq T$. Hence, thanks to Jensen's inequality, 
\[
\sum_{k=0}^{\Upsilon_T-1} \sum_{i\in\arms_k} C_\pi\sigma \sqrt{h_{i,\xi}^k\log{T}} \leq  C_\pi \sigma \sqrt{ \Upsilon_T K T\log{T}}.
\]

We use Lemma~\ref{lem:FP} with the last equation and conclude,
\[
\EE{R_T(\pi)} \leq C_\pi \sigma \sqrt{\log{T}} \pa{ \sqrt{\Upsilon_T KT} + \Upsilon_T K} + 6KV.
\]
\end{proof}
\begin{tBox}
\restapiecewisetheorempd*
\end{tBox}
\begin{proof}
Let $\arms_k \triangleq \left\{ i \in \arms | \Delta_{i,k} > 0\right\}$, the set of sub-optimal arms in batch $k$.
We apply Lemma~\ref{lem:OP-piecewise} to bound the number of wrong pull (under the favorable events) of arm $i\in \arms_k$ during batch $k$,
\begin{align*}
     \Delta_{i,k} \pa{h^k_{i, \xi} -1} & \leq C_\pi \sigma \sqrt{\pa{h_{i,\xi}^k-1}\log{T}} \implies h_{i,\xi}^k \leq 1 + \frac{C_\pi^2 \sigma^2\log{T}}{\Delta_{i,k}^2}\, \cdot
\end{align*}

Then, we apply Lemma~\ref{lem:OP-piecewise} again to bound the regret due to second pulls of any sub-optimal arm $i\notin \argmax_{i \in \arms} \mu_i^k$ in any batch $k$,
\begin{align*}
OP\pa{i,k} &\triangleq\! \sum_{t=t_{k}+1}^{t_{k+1}} \sum_{h=2}^{h_{i}^{k}} \mathbbm{1}\!\left(\! t\!=\!t_{i}^{k}(h) \land \HPevent \!\right)\left(\mu_{\star}(t)\!-\!\mu_{i}(t)\right) \\
&\leq C_\pi \sigma \sqrt{\pa{h_{i,\xi}^k \!-\! 1}\log{T}}\\
 &\leq \frac{C_\pi^2\sigma^2\log{T}}{\Delta_{i,k}}\cdot
 \end{align*}

We apply Lemma~\ref{lem:FP} on the set of $\Upsilon_T -1$ breakpoints and we conclude thanks to the precedent equation,
\begin{align*}
\EE{R_T(\pi)} & \leq \EE{\sum_{k=0}^{\Upsilon_T-1} \sum_{i\in\arms_k}OP\pa{i,k} }+ C_\pi \sigma \Upsilon_T K\sqrt{\log{T}} + 6KV  \\
&\leq \sum_{k=0}^{\Upsilon_T-1} \sum_{i\in\arms} \frac{C_\pi^2 \sigma^2\log{T}}{\Delta_{i,k}} +  C_\pi \sigma \Upsilon_T K \sqrt{ \log{T}} + 6KV\,.
\end{align*}
\end{proof}
\subsubsection*{Variation budget rotting bandits.}
\begin{tBox}
\restabudgettheorem* 
\end{tBox}
\begin{proof}
Let $\Upsilon \in \left\{1, \dots, T\right\}$ a number of evenly spaced batches that we will specify later. We define the length of these batches $\left\{\tau_{k} \triangleq\left\lceil\frac{T}{\Upsilon}\right\rceil \text { if } k \leq T \bmod \Upsilon \text { else }\left\lfloor\frac{T}{\Upsilon}\right\rfloor\right\}_{k \leq \Upsilon}$. Note that $\sum_{k=1}^{\Upsilon} \tau_k = T$. Let $t_k = \sum_{k'=0}^k \tau_{k'}$ the last round of each batch and $t_0 = 0$. On each of these batches, we apply Lemma~\ref{lem:OP} for the set of arms which have been pulled in this batch,
\begin{multline}
\label{eq:op_decomposition}
    \sum_{k=0}^{\Upsilon_T-1} \sum_{i\in\arms_k}\sum_{t=t_{k}+1}^{t_{k+1}} \sum_{h=2}^{h_{t}^{k}} \mathbbm{1}\left( t=t_{i}^{k}(h) \land \HPevent\right)\Big(\mu_{\star}(t)-\mu_{i}(t)\Big)
\leq \sum_{k=0}^{\Upsilon-1} \sum_{i\in\arms_k} \pa{h_i^k-1}\Delta_i^k\\
+ \sum_{k=0}^{\Upsilon-1} \sum_{i\in\arms_k}\sum_{h=2}^{h^k_{i}}\mathbbm{1}\left(\HPt{t_i^k(h)}\right)\pa{  \mu_{\star}(t_i^k(h)) - \bar{\mu}_i^{h-1}(t_i^k(h), \pi)}.
\end{multline}

The first sums can be handled using Assumption~\ref{assum:variation} and the evenly spaced property of $\tau_k$,
\begin{equation}
\label{eq:use_evenly_spaced}
\sum_{k=0}^{\Upsilon-1} \sum_{i\in\arms} \pa{h_i^k-1}\Delta_i^k \leq \sum_{k=0}^{\Upsilon-1} \max_{j\in \arms} \Delta_j^k\sum_{i\in\arms} \pa{h_i^k-1} = \sum_{k=0}^{\Upsilon-1} \max_{j\in \arms} \Delta_j^k \pa{\tau_k-K} \leq \frac{T}{\Upsilon}\sum_{k=0}^{\Upsilon-1} \max_{j\in \arms} \Delta_j^k.
\end{equation}
%
The first inequality is justified by definition of the maximum. The second equality states that the total number of pulls in batch $k$ is $\tau_k$. The third inequality uses that $\tau_k - K \leq \ceil{\frac{T}{\Upsilon}} -K \leq \ceil{\frac{T}{\Upsilon}} -K \leq \frac{T}{\Upsilon}$. Now, we need to relate $\max_{j\in \arms} \Delta_j^k$ and $V_T$,
\begin{equation}
\label{eq:use_assum_variation}
   \sum_{k=0}^{\Upsilon\!-\!1}\max_{j\in \arms} \Delta_j^k \!=\! \sum_{k=0}^{\Upsilon\!-\!1}\max_{j\in \arms} \!\sum_{t = t_k\!+\!1}^{t_{k\!+\!1}\!-\!1}\! \Delta_j\! \pa{t,t\!+\!1} \!\leq\! \sum_{k=0}^{\Upsilon\!-\!1} \sum_{t = t_k\!+\!1}^{t_{k\!+\!1}\!-\!1} \! \max_{j\in \arms} \Delta_j\!\pa{t,t\!+\!1} \!\leq\!  \sum_{t = 1}^{T}\max_{j\in \arms} \Delta_j\!\pa{t,t\!+\!1}\!\leq\! V_T .
\end{equation}
%
The first inequality is justified because the maximum of a sum is smaller than the sum of the maximums. In the second inequality, we add positive terms which are the maximum of the decay among the arms at the boundary between batches. The last inequality is justified by Assumption~\ref{assum:variation}. Therefore, we can bound the first sums using Equation~\ref{eq:use_evenly_spaced} and ~\ref{eq:use_assum_variation},
\begin{equation}
\label{eq:bound_sum1}
\sum_{k=0}^{\Upsilon-1} \sum_{i\in\arms} \pa{h_i^k-1}\Delta_i^k \leq \frac{V_T T }{\Upsilon}\cdot    
\end{equation}


The second sums can be bounded using Lemma~\ref{lem:core-full} on the high probability event $\HPt{t_i^k(h)}$ and Jensen's inequality,
\begin{align}
    \sum_{k=0}^{\Upsilon-1} \!\sum_{i\in\arms_k}\!\sum_{h=2}^{h^k_{i}}\!\mathbbm{1}\!\left(\!\HPt{t_i^k(h)}\!\right)\pa{\! \mu_{\star}(t_i^k(h)) \!- \!\bar{\mu}_i^{h-1}(t_i^k(h),\pi)\!} \!&\leq \sum_{k=0}^{\Upsilon-1} \sum_{i\in\arms_k} \sum_{h=2}^{h^k_{i}} \frac{C_\pi c\pa{\!h\!-\!1, 2T^{-\alpha}\!}}{\sqrt{2\alpha}} \nonumber\\
&= \sum_{k=0}^{\Upsilon-1} \sum_{i\in\arms_k} \sum_{h=2}^{h^k_{i}}C_\pi\sigma \sqrt{\frac{\log{T}}{h -1}}\nonumber\\
&\leq \sum_{k=0}^{\Upsilon-1} \sum_{i\in\arms_k} 2 C_\pi \sigma \sqrt{h_i^k \log{T}}\nonumber\\
&\leq  2 C_\pi \sigma \sqrt{\Upsilon K T\log{T}} .
\label{eq:bound_sum2}
\end{align}

We remark that the bound in Eq.~\ref{eq:bound_sum1} is decreasing with $\Upsilon$ and the bound in Eq.~\ref{eq:bound_sum2} is increasing with $\Upsilon$. We will choose $\Upsilon$ in order to minimize the sum of these two bounds (which will be our leading term). Therefore, we set,
\begin{equation}
    \label{eq:set_upsilon_variation}
    \Upsilon \triangleq \ceil{\pa{\frac{V_T^2 T}{C_\pi^2 \sigma^2 K\log{T}}}^{\nicefrac{1}{3}}}.
\end{equation}

We have that $\Upsilon\leq T$ when $V_T \leq  C_\pi \sigma T\sqrt{ K\log{T}}$. Moreover, we will use that $ \Upsilon \leq 2 \pa{\frac{V_T^2 T}{C_\pi^2 \sigma^2 K\log{T}}}^{\nicefrac{1}{3}} $ which is true when $V_T \geq \sqrt{\frac{C_\pi^2 \sigma^2 K\log{T}}{8T}}$. 

Finally, we use Lemma~\ref{lem:FP} where we replace the inner sums thanks to Equations~\ref{eq:op_decomposition}, \ref{eq:bound_sum1} and~\ref{eq:bound_sum2}. Then, we plug $\Upsilon$ set in \ref{eq:set_upsilon_variation} and conclude,
\begin{align*}
\EE{R_T\pa{\pi}} & \leq \frac{V_T T}{\Upsilon} +  2C_\pi\sigma \sqrt{\Upsilon K T\log{T}} +  C_\pi \sigma \Upsilon  K\sqrt{\log{T}} + 6 V_T K\\
&\leq  4\pa{C_\pi^2 \sigma^2 V_T K T^2\log{T}}^{\nicefrac{1}{3}} \!+ 2\Big(C_\pi \sigma V_T^2  K^2  T \sqrt{\log{T}}\Big)^{\nicefrac{1}{3}} \!+ 6 V_T K.
\end{align*}

When $V_T\leq  \sqrt{\frac{C_\pi^2 \sigma^2 K \log{T}}{8T}}$, the regret of any policy can be bounded , 
\begin{align*}
\mathbb{E}\left[R_T(\pi)\right] &\leq T V_T = V_T^{\nicefrac{1}{3}} T^{\nicefrac{2}{3}} V_T^{\nicefrac{2}{3}} T^{\nicefrac{1}{3}}\\
&\leq V_T^{\nicefrac{1}{3}} T^{\nicefrac{2}{3}} \left(\frac{ C_\pi^2 \sigma^2 K \log{T}}{8T}\right)^{\nicefrac{1}{3}} T^{\nicefrac{1}{3}}\\ 
&= \frac{1}{2} \pa{C_\pi^2\sigma^2 V_T K T^2 \log{T}}^{\nicefrac{1}{3}}\\
&\leq 4 \pa{C_\pi^2 \sigma^2 V_T K T^2 \log{T}}^{\nicefrac{1}{3}}.
\end{align*}

For completion, we also consider $V_T \geq  C_\pi \sigma T\sqrt{ K\log{T}}$. Yet, notice that in that case the leading term is $\cO\pa{KV_T}$. We start back from Lemma~\ref{lem:FP},
\begin{align*}
\EE{R_T(\pi)} \leq &  \, \EE{\sum_{k=0}^{\Upsilon-1} \sum_{i\in\arms_k}\sum_{t=t_k +1 }^{t_{k+1}}\sum_{h=2}^{h^k_{i}}\mathbbm{1}\pa{ t = t_i^k(h) \land \HPevent} \Big(\mu_{\star}(t) - \mu_{i}(t)\Big)} \\
&+   C_\pi \sigma \Upsilon K\sqrt{\log{T}} + 6KV_T.
\end{align*}
In fact, this result can be slightly improved at no cost, 
\begin{align*}
\EE{R_T(\pi)} \leq &  \, \EE{\sum_{k=0}^{\Upsilon-1} \sum_{i\in\arms_k}\sum_{t=t_k +1 }^{t_{k+1}}\sum_{h=2}^{h^k_{i}}\mathbbm{1}\pa{ t = t_i^k(h) \land \HPevent} \Big(\mu_{\star}(t) - \mu_{i}(t)\Big)} \\
&+   C_\pi \sigma \min\pa{\Upsilon K, T}\sqrt{\log{T}} + 6KV_T,
\end{align*}
because there are at most $\min\pa{\Upsilon K, T}$ first pulls (see the proof of Lemma~\ref{lem:FP}). Now, we choose $\Upsilon = T$. Hence, there is no second pulls and we have,
\begin{equation*}
\EE{R_T(\pi)} \leq   C_\pi \sigma T\sqrt{\log{T}} + 6KV_T,
\end{equation*} 

Now, we use that $C_\pi \sigma T\sqrt{\log{T}} \leq \frac{V_T}{\sqrt{K}} \leq KV_T$,
\begin{align*}
\EE{R_T(\pi)} &\leq   \pa{C_\pi \sigma T\sqrt{\log{T}}}^{\nicefrac{2}{3}}\! \pa{C_\pi \sigma T\sqrt{\log{T}}}^{\nicefrac{1}{3}}\! + 6KV_T \\
& \leq  \pa{C_\pi^2 \sigma^2 V_T K T^2\log{T}}^{\nicefrac{1}{3}} \!+  6 KV_T\\
&\leq  4\pa{C_\pi^2 \sigma^2 V_T K T^2\log{T}}^{\nicefrac{1}{3}} \!+ 2\Big(C_\pi \sigma V_T^2  K^2  T \sqrt{\log{T}}\Big)^{\nicefrac{1}{3}} \!+ 6 K V_T.
\end{align*} 
\end{proof}


%!TEX root = ../main.tex
\section{Real-word data experiment on Yahoo! Front Page}
\label{sec:yahoo}
\paragraph{R6A - Yahoo! Front page today module user click log dataset.} 
This dataset was used for the Exploration and Exploitation Challenge\footnote{\url{http://explochallenge.inria.fr/}} at ICML 2012 and inspired new algorithms. Among them, we mention the work of \citet{traca2015regulating} who noticed the non-stationary trend and took advantage of it. Since then the dataset continues to be a benchmark\footnote{As it allows for offline evaluations as the actions were samples uniformly.} for non-stationary bandits \citep{liu2018change-detection,cao2019nearly}. It contains the history of clicks on news articles of 45 million users in the first ten days of May 2009. We use three features in this dataset: \textit{timestamp} (rounded every 5 minutes), \textit{article$\!\_\!$id}, and \textit{click}. 
 
\paragraph{A real decaying scenario.} Every day, between 6 pm and 6 am EST (12 hours), we notice a decreasing trend in click probability. It suggests that people in the US read less and less news during the evening and night. For each day, we keep all the articles that have been recommended at every timestamp during the 12 hours. For these articles, we use a rolling average window of 30000 in order to estimate the probability of click for each article at each timestamp \footnote{For each timestamp, we average the values given by rolling average. These values are close to each other because the number of click opportunities per article in the same timestamp is small compared to 30000.}. We use the \underline{real} total traffic for each timestamp. We highlight that \textit{we do not enforce any of our assumptions} to create reward functions to be aligned with our setup. In particular, we do not enforce them to be piecewise constant nor to be decreasing. At each round, the learner receives 10 reward samples in order to reduce the cost of computation.

\paragraph{Algorithms and Parameters.} We include two versions of \FEWA and \RAWUCB: with the theoretical tuning $\alpha =4$; and with the empirical tuning $\alpha_{\mathrm{R}} = 1.4$ and $\alpha_{\mathrm{F}} = 0.06$. These two values were selected on the rested benchmark (c.f. Section~\ref{sec:rested-experiment}). This benchmark has 30 different problems (for different $L$) but the best tuning of $\alpha$ is the same for all the considered problems. We replace \RAWUCB and \FEWA with their efficient versions because of the longer horizon. 

We also include \EXPS\citep{auer2002nonstochastic} and \GLRUCB \citep{besson2019generalized}.  For \EXPS, we use the theoretical tuning which requires the knowledge of $T$ and $V_T$. \GLRUCB has two parameters: a confidence level $\delta$ for its change-point detector and an active exploration rate $\alpha$. We set $\alpha$ to zero. Indeed, the active exploration of change-detection algorithms is only useful in the increasing case (as argued by \citet{cao2019nearly}). We tune $\delta$ by its theoretical value, which requires the knowledge of $T$. Last, we only restart the history of the changed arm as our setup does not assume that all the rewards change simultaneously. For a fair comparison, we only use the subgaussian version of the algorithm. Indeed, KL-UCB indexes are expensive to compute. Instead, for all the confidence bound algorithms, we rather tune $\sigma^2 = 1$ in the rested benchmark and $\sigma^2 = 0.29$ in the restless benchmark (the variance of a binomial $\mathcal{B}\pa{10,0.03)}$.  

We do not include \SWA \citep{levine2017rotting} which was shown to be less consistent than \FEWA and \RAWUCB on rested rotting bandits. We do not include \SWUCB and \DUCB as they were shown to be unable to learn in the rested setting  \citep{levine2017rotting, seznec2019rotting}. We also do not include \CUSUMUCB \citep{liu2018change-detection} and \MUCB \citep{cao2019nearly}, as 1) they were shown to under-perform against \GLRUCB \citep{besson2019generalized}; and 2) their change-detector is harder to tune.

Note that our goal is to compare algorithms with the same tuning in the rested and restless benchmark. 

\paragraph{Results.} We display the results for eight different days in Figure~\ref{fig:restless-exp}.%TODO add day 10?
We will comment day~2 and day~7. On day~2, there are several switches of optimal arms with many near-optimal ones: tracking the best arm is a "hard" problem. On day~7, one arm consistently dominates the others by far. Hence, it is an "easy" case where good algorithms should have a logarithmic regret rate. We also display the running time of each algorithm in Table~\ref{tab:restless-time}.

 \begin{figure*}[p!]
\caption{\textbf{Left:} reward functions from the Yahoo! dataset \\ \textbf{Right:} average regret of policies over 500 runs}
\label{fig:restless-exp}
\includegraphics[clip, width= 0.495\textwidth]{2.2Restless/fig/reward_plot_day2.pdf}
\includegraphics[clip, width= 0.495\textwidth]{2.2Restless/fig/DAY2.pdf}
\includegraphics[clip, width= 0.495\textwidth]{2.2Restless/fig/reward_plot_day3.pdf}
\includegraphics[clip, width= 0.495\textwidth]{2.2Restless/fig/DAY3.pdf}
\includegraphics[clip, width= 0.495\textwidth]{2.2Restless/fig/reward_plot_day4.pdf}
\includegraphics[clip, width= 0.495\textwidth]{2.2Restless/fig/DAY4.pdf}
\end{figure*}

\begin{figure*}[p!]
\includegraphics[clip, width= 0.495\textwidth]{2.2Restless/fig/reward_plot_day5.pdf}
\includegraphics[clip, width= 0.495\textwidth]{2.2Restless/fig/DAY5.pdf}
\includegraphics[clip, width= 0.495\textwidth]{2.2Restless/fig/reward_plot_day6.pdf}
\includegraphics[clip, width= 0.495\textwidth]{2.2Restless/fig/DAY6.pdf}
\includegraphics[clip, width= 0.495\textwidth]{2.2Restless/fig/reward_plot_day7.pdf}
\includegraphics[clip, width= 0.495\textwidth]{2.2Restless/fig/DAY7.pdf}
\end{figure*}
\begin{figure*}[p!]
\includegraphics[clip, width= 0.495\textwidth]{2.2Restless/fig/reward_plot_day8.pdf}
\includegraphics[clip, width= 0.495\textwidth]{2.2Restless/fig/DAY8.pdf}
\includegraphics[clip, width= 0.495\textwidth]{2.2Restless/fig/reward_plot_day9.pdf}
\includegraphics[clip, width= 0.495\textwidth]{2.2Restless/fig/DAY9.pdf}
\includegraphics[clip, width= 0.495\textwidth]{2.2Restless/fig/reward_plot_day10.pdf}
\includegraphics[clip, width= 0.495\textwidth]{2.2Restless/fig/DAY10.pdf}
\end{figure*}
\begin{table*}[ht!]
\begin{center}
\begin{tabular}{|@{\hskip3pt}c@{\hskip3pt}|@{\hskip5pt}c@{\hskip5pt}|@{\hskip5pt}c@{\hskip5pt}|@{\hskip5pt}c@{\hskip5pt}|@{\hskip5pt}c@{\hskip5pt}|@{\hskip5pt}c@{\hskip5pt}|@{\hskip5pt}c@{\hskip5pt}|@{\hskip5pt}c@{\hskip5pt}|@{\hskip5pt}c@{\hskip5pt}|@{\hskip5pt}c@{\hskip5pt}|}
\hline
\textbf{Day} & \textbf{2} & \textbf{3} & \textbf{4} & \textbf{5} & \textbf{6} & \textbf{7} & \textbf{8} & \textbf{9}              & \textbf{10}               \\ \hline
\!\EFFRAW\! \footnotesize{$\!(\alpha\!=\!1.4, m\!=\!1.1)\!$} \!& 67         & 66         & 90         & 86         & 91         & 74         & 88         & 64 & 48 \\  
\!\EFFRAW\!  {\footnotesize$(\alpha\!=\!1.4, m\!=\!2)$} \!  & 35         & 33         & 43         & 47         & 46         & 41         & 44         & 34   & 45 \\  
\!\EFFRAW \!{\footnotesize $(\alpha\!=\!4, m\!=\!1.1)$}\!   & 65         & 65         & 90         & 88         & 91         & 74         & 89         & 63 & 48  \\ \hline
\EFFFEWA \footnotesize{$(\alpha\!=\!0.06)$}           & 143        & 175        & 223        & 159        & 183        & 115        & 193        & 116        &       165       \\ 
\EFFFEWA \footnotesize{$(\alpha\!=\!4)$}              & 337        & 308        & 391        & 473        & 487        & 380        & 428        & 341     &  388               \\ \hline 
\EXPS   & 56         & 53         & 67         & 77         & 75         & 69         & 71         & 55      &   78 \\ \hline
\GLRUCB  & 560        & 613        & 683        & 2421       & 707        & 1529       & 957        & 971  & 4017\\ \hline
\end{tabular}
  \caption{Average computational time in seconds for each algorithm in each experiment.}
  \label{tab:restless-time}
\end{center}
\end{table*}

\paragraph{{\RAWUCB} vs {\FEWA}.} The two algorithms compute the same statistics and share most of their analysis. Yet, {\RAWUCB} consistently outperforms {\FEWA} as it was the case on the rested benchmark. The difference between the two is even more significant in the restless case. Its theoretical tuning $\alpha = 4$ gets reasonable results, while theoretical {\FEWA} is impractical. Finally, its empirical tuning $\alpha_{\mathrm{R}} =1.4$ is similar to the asymptotic optimal tuning of {\UCB} and shows good performance on both rested and restless problems. By contrast, {\FEWA} with $\alpha_{\mathrm{F}} = 0.06$ shows worse performance with larger deviation on the restless benchmark. 

\paragraph{{\RAWUCB} vs {\EXPS}.} {\EXPS} has good performances on the restless benchmark, on which it has theoretical guarantees. Yet, it is consistently outperformed by {\RAWUCB} when we tune the confidence bounds. It is particularly true in easy instances, e.g. on day 7. Indeed, in these cases, we expect a logarithmic regret rate for {\RAWUCB}.

\paragraph{{\RAWUCB} vs {\GLRUCB} (no active exploration).}  On the restless benchmark, {\GLRUCB} shows similar results than {\RAWUCB}. Yet, we highlight that 1) {\GLRUCB} needs the knowledge of the horizon to tune its change-detector; 2) we use an efficient version of {\RAWUCB} which runs $\sim 10$ times faster than {\GLRUCB}. In fact, the two algorithms are similar: they are UCB index policies, they recover logarithmic rate on easy restless rotting bandits problems and hence they would both suffer near-linear worst-case regret rate in the general restless setting (when active exploration is turned off for {\GLRUCB}). The main difference is that {\RAWUCB} scans its history to select its rotting UCB's window, while {\GLRUCB} scans its history to detect significant changes and restart. 


%!TEX root = ../main.tex
\section{Restless and rested rotting bandits}
\label{sec:general_decreasing_MAB_framework}
\subsection{The general case}
\begin{assumption}\label{assum:general}
For each arm $i$, any number of pulls $n$, and time $t$, the functions $\mu_i(t,\cdot)$ and $\mu_i(\cdot,n)$ are non-increasing.
\end{assumption}

In Section~\ref{sec:restless-theory}, we highlight that the main guarantee of our algorithms - Lemma~\ref{lem:core-full} -  holds in the general case of Assumption~\ref{assum:general}. Is it enough to show that our algorithms are near-optimal in this extended setup?

Like in Chapter~\ref{ch:rested} and~\ref{ch:restless}, we define the regret with respect to the best oracle.
\begin{equation*}
R_T(\pi, \mu) \triangleq \argmax_{\pi^\star_T \in \PiO}  J_T(\pi^\star_T, \mu) - J_T(\pi).
\end{equation*}

Like in the linear rested rotting bandits (Section~\ref{sec:linear-rotting}), we can show that not only the greedy oracle suffers linear regret but no learning policy can get a sublinear regret rate in the worst-case. 

\begin{proposition}
\label{prop:general-rotting-unlearnable}
In the no noise setting ($\sigma = 0$), there exists a rotting 2-arms bandits problem (satisfying Assumption~\ref{assum:general}) with reward value in $\left[0,1\right]$, with one rested arm and one restless arm, and with at most one change-point before $T$ each, such that the greedy oracle strategy $\pi_O$ suffers a regret 
\[R_T\pa{\pi_O} \geq \floor{\frac{T}{4}}.\]
Moreover, for any learning strategy  $\pi_S$, there exists a rotting 2-arms bandits problem (satisfying Assumption~\ref{assum:general}) with reward value in $\left[0,1\right]$, with one rested arm and one restless arm, and with at most one change-point before $T$ each, such that 
\[R_T\pa{\pi_S} \geq \floor{\frac{T}{8}}.\]
\end{proposition}

Notice that the two reward functions of the constructed difficult problems are simple: either rested or restless, bounded, and with at most one break-point. If we consider a 2-arm setup with one rested arm and one restless arm, a good strategy may be to select the restless arm even when its current value is the worst. Indeed, this value is only available now, while the good value of the rested arm will still be available in the future. Whether the restless rewards are interesting to the learner depends on the future behavior of the (currently best) rested arm. On the first hand, if it decays below the current value of the restless arm before $T$ pulls, then the learner should profit from the restless reward available right now. On the other hand, if the rested arm stays optimal until the end of the game then the learner should ignore the restless arm and follows the greedy oracle strategy. However, the learner does not know in advance if (and how much) an arm will decay and any anticipation she makes will turn to be bad in the worst case. We formalize these ideas in the proof at the end of the section.

\subsection{Rested rotting bandits with a restless envelope}
\begin{assumption}
\label{assum:envelop}
We consider the following reward functions, 
\[
\mu_i(t,n) = P(t) f_i(n) + S(t),
\]
where $P: \NN^* \rightarrow \R_+$, $\left\{f_i : \NN \rightarrow \R\right\}_{i \in \arms}$ and $S: \NN^* \rightarrow \R$ are non-increasing functions. 
\end{assumption}

Notice that all the arms have the same product $P$ and sum $S$ functions, the only difference is the rested evolution $f_i$. That is why we call this setup the rested rotting bandits with a restless envelope.

With this assumption, we can show that the greedy oracle is optimal.
\begin{proposition}
\label{prop:envelop}
For any reward functions $\left\{\mu_i\right\}_{i \in \arms}$ verifying Assumption~\ref{assum:envelop} and any horizon $T$, $\GO \in \argmax_{\pi \in \PiO} J_T(\pi)$.
\end{proposition}

We leave as an open problem to analyze the aforementioned algorithms in this setup. A first step would be to characterize the performance of the greedy bandit policy in the absence of noise (as we did for the rested problem, see Subsection~\ref{ss:rested-noiseless-online}). We may not recover the $\cO\pa{K}$ bound as in the rested setup. Indeed, the adversary can use the variation of $P$ and $S$ to trick the greedy bandit policy several times for each arm. Moreover, the order of the pull do matter in this problem: the cumulative reward is not a function of $\left\{\NiT\right\}_{i \in \arms}$ anymore.

\subsection{Proofs}
\label{ss:rested-restless-proofs}
\begin{proof}[Proof of Proposition~\ref{prop:general-rotting-unlearnable}]
Let $\mu^{0}$ and $\mu^{1}$, two decreasing 2-arms bandits problems such that:
\begin{align*}
 &\mu^{0}_1(t,n) = \mu_1(n) = 1 \text{ if } n<\frac{T}{2} \text{ else } 0\,,\\
 &\mu^{1}_1(t,n) = 1\,, \\
 &\mu^{0}_2(t,n) = \mu^{1}_2(t,n) = \mu_2(t) = 1/2 \text{ if } t<\frac{T}{2} \text{ else } 0.
\end{align*}
Problem $\mu^{1}$ only evolves according to time. Hence, the oracle greedy policy $\pi_O$ is optimal for this problem and collects
\begin{equation}
\label{eq:regret1-piO}
J_T\pa{\pi_O, \mu^1} = T.
\end{equation}
On $\mu^{0}$, $\pi_O$ selects arm $1$ during $\floor{\frac{T}{2}}$ rounds and then both arms yield $0$ reward. Thus, $\pi_O$ collects 
\[J_T\pa{\pi_O, \mu^0} = \floor{\frac{T}{2}}.\]
However, let $\pi_0$ the policy which selects arm 2 for $\floor{\frac{T}{2}}$ rounds and arm 1 afterwards. Thus, $\pi_0$ collects
\begin{equation}
\label{eq:regret0-pi0}
  J_T\pa{\pi_0, \mu^0} = \pa{3/2} \floor{\frac{T}{2}}.  
\end{equation}
Hence, we conclude the first part of our proposition, 
\[R_T\pa{\pi_O, \mu^0} = J_T\pa{\pi^\star_T, \mu^0} - J_T\pa{\pi_O, \mu^0} \geq J_T\pa{\pi_0, \mu^0} - J_T\pa{\pi_O, \mu^0} \geq  \floor{\frac{T}{4}}.\]
Now, we consider any learning policy $\pi_S$ and we call $\EEempty_j\big[N_{i,t}(\pi_S)\big]$ the (expected, if the policy is random) number of pulls of arm $i$ at any round $t$ by $\pi_S$ on problem $j$. Note that the leaner will receive the same rewards for both problems until at least $\floor{\frac{T}{2}}$. Therefore, we have that 

\[ \forall t \leq \floor{\frac{T}{2}}, \pi\big(\mathcal{H}_t\pa{\mu^0}\big) = \pi\big(\mathcal{H}_t\pa{\mu^1}\big) \implies \EEempty_0\Big[N_{2,\floor{\frac{T}{2}}}(\pi_S)\Big] = \EEempty_1\Big[N_{2,\floor{\frac{T}{2}}}(\pi_S)\Big] \triangleq n_2.\]

On problem $\mu^{1}$, $\pi_S$ collects a reward of at most,
\begin{equation}
\label{eq:regret1-piS}
    J_T\pa{\pi_S, \mu^1} = \EEempty_1[N_{1,T}(\pi_S)] + \frac{n_2}{2} = T - \EEempty_1[N_{2,T}(\pi_S)] + \frac{n_2}{2} \leq T - \frac{n_2}{2}\CommaBin
\end{equation}
because $n_2 = \EEempty_1\Big[N_{2,\floor{\frac{T}{2}}}(\pi_S)\Big] \leq \EEempty_1[N_{2,T}(\pi_S)]$. Using Equations~\ref{eq:regret1-piO} and~\ref{eq:regret1-piS}, we can lower bound the regret of $\pi_S$, 
\[ R_T\pa{\pi_S, \mu^1} = J_T\pa{\pi_O, \mu^1} - J_T\pa{\pi_S, \mu^1} \geq  \frac{n_2}{2}\cdot \]


On problem $\mu^{0}$, $\pi_S$ collects a reward of at most,
\begin{equation}
\label{eq:regret0-piS}
    J_T\pa{\pi_S, \mu^0} = \min\pa{\EEempty_1[N_{1,T}(\pi_S)],\floor{\frac{T}{2}}} + \frac{n_2}{2} \leq \floor{\frac{T}{2}} + \frac{n_2}{2}\cdot
\end{equation}
Using Equations~\ref{eq:regret0-pi0} and~\ref{eq:regret0-piS}, we can lower bound the regret of $\pi_S$, 
\[ R_T\pa{\pi_S, \mu^0} = J_T\pa{\pi_O, \mu^0} - J_T\pa{\pi_S, \mu^0} \geq  \frac{\floor{T/2} - n_2}{2}\cdot \]

Hence, the worst case regret on the two setups is bounded by 
\[R_T(\pi_S) \geq \max\pa{\frac{n_2}{2}, \frac{\floor{\frac{T}{2}} - n_2}{2}} \geq \floor{\frac{T}{8}}\!\cdot \]
\end{proof}

\begin{proof}[Proof of Proposition~\ref{prop:envelop}]
At any round $t$, we have,
\[
\GO(t) \in \argmax_{i \in \arms} \pa{P(t)f_i(\Nit) + S(t)} = \argmax_{i \in \arms} f_i \pa{\Nit}.
\] 
Therefore, at round $t$, collects the $t$ largest values of $\ev{f_i(n)}_{i\in \arms, n\leq T}$, \textit{i.e.} 
\[
\forall i \in \arms, \ \forall n_i \geq \Nit, \ \mu_{\GO(t)}\pa{N_{\GO(t),\,t}} \geq\mu_{i}\pa{\Nit}  \geq \mu_i(n_i).
\]

The first inequality is due to the selection rule of the policy; the second is due to the decreasing reward functions. 

A direct consequence is that, at the round $t$, $\GO$ selects the $t$-th largest value of $\left\{f_i(n)\right\}_{i \in \arms, n \leq T}$. Hence, at the round $T$, it has selected the $T$ largest value in the decreasing order. Since $P(t)$ is non-increasing and positive, an other policy which selects smaller values of $\left\{f_i(n)\right\}_{i \in \arms, n \leq T}$, or the same values but in an other order, have a smaller or equal cumulative reward than $\GO$.
\end{proof}

 %!TEX root = ../main.tex
\part{Beyond rotting bandits}
\chapterimage{chapter_head/6_546ours.jpg} 
\chapter{Master topics as soon as possible}
\label{ch:pomdp}
\vspace{-3cm}
\begin{flushright}
\emph{Turn your head left and blink twice. You'll see a bandit in a POMDP.}
\end{flushright}
\vspace{0.85cm}
\section{Beyond rotting bandits: some motivations}
\label{sec:beyond}
Our motivation for studying the rested rotting bandits was the ability to target the least known topic. This educational strategy can be interesting before an exam, when we assume that all the topics should be at least understandable by the student. However, during the curriculum, targeting the most difficult subject can demotivate the student and could result in no learning (the wheel-spinning effect, \cite{beck2013wheel}).

\RAWUCB (or \FEWA)  keeps switching between topics either because the confidence intervals are reduced through the pulls or because the student gains proficiency on the topics. On Afterclasse, a student before the exam needs to study $\sim10$ chapters divided into $\sim3$ topics. It makes up to 30 potential arms. If the student answers two hundreds of exercises (which is a lot compared to the average student), \RAWUCB is barely different than round-robin. 

Moreover, rotting bandits do not take into account the difficulty levels. If we consider that each difficulty level is a different arm, then \RAWUCB will focus on difficult questions before focusing on the easiest questions. Arguably, this is not a good educational strategy. If we consider that different difficulty levels are in the same arm, then we should select the difficulty uniformly at random to not bias the averages that \RAWUCB constructs. Another possibility is to choose the difficulty with a subroutine and correct the bias with specific computations (e.g. importance sampling). 

For all these reasons, \RAWUCB is hard to test on students in a relevant educational scenario. However, \RAWUCB does not have only disadvantages: it is quite interesting to take educational decisions based on pessimistic estimates of the student's proficiencies. Indeed, if we stop learning a topic because its estimate is high enough, it is important to be sure that this estimate is not high just by chance. 

In this chapter, we describe a setup where the goal is to validate topics as soon as possible. We show that, under relevant assumptions, the best thing to do is to first focus on the simplest topic and then switch to the more difficult ones promptly. In an online setting, we don't know which topic is the simplest, so we design an exploration strategy that outputs a topic among the easiest and then we focus on this arm until we are sure the topic is validated. This algorithm makes good use of the aforementioned pessimistic estimates to both select a simple topic and to be sure that the topic is validated at the end of the session. Finally, we discuss design improvements to switch our theoretical algorithm in a practical ITS. 

\section{Setup}
\label{sec:setup}
We model the student-ITS interaction as a formal Partially Observable Markov Decision Process (POMDP). 
\paragraph{State, actions and feedback}
The agent faces a set of $K$ tasks. Each task $i$ has a state $\mu_{i,t}\in \R$ at the round $t$ with initial value $\mu_{i,1}$. At each round $t$, the agent selects a task $i_t$ to allocate resource (e.g. time). He receives a noisy observation of its current state,
\[ 
o_{t} \triangleq \mu_{i,t} + \noise_t,
\]
where $\left\{\noise_t\right\}_{t\leq T}$ is an independent sequence of $\sigma$-subgaussian variables, \ie

\[
\EE{ \noise_t | \historyt }= 0 \;\; \text{and} \; \forall \lambda \in \R, \; \EE{ e^{\lambda\noise_t}} \leq e^{\frac{\subgaussian\lambda^2}{2}},
\]

with $\mathcal{H}_t \triangleq \left\{ \left\{i_s,o_s   \right\}\right\}_{s < t}$, the history of the agent at the beginning of the round $t$. We call $\statet \triangleq \ev{\mu_{i,t}}_{i \in \arms}$.

In the context of Intelligent Tutoring Systems, a task is to learn a topic. The state is the average level of the student on that topic. The action is to give a student a question related to that topic, and the observation is the grade associated with the answer to that question.

\paragraph{Transition.}
Between consecutive rounds, the state $\statet$ is randomly modified following transition probabilities which depend on the selected arm and the current state. It contrasts with the rotting bandits we studied so far where the evolution was deterministic. We discuss the meaning of this random evolution concerning our ITS application at the end of the section.  We use two assumptions that we already studied in the rested rotting bandits framework (Chapter~\ref{ch:rested}): the rested and monotone evolution of the arms' states.
\begin{assumption}
\label{assum:rested}
The transitions are rested, which means that selecting a task only modifies the state of this particular task. Hence, we have that,
\[
\mu_{i,t} = \mu_i(N_{i,t-1}),
\]
with $\left\{\mu_i(n)\right\}_{n\in \NN}$ a Markov Chain with transition operator $\Tp_i$ and $N_{i,t}\triangleq \sum_{s\!=\!1}^{t} \mathbb{I}\{i_s \!=\! i\}$.
\end{assumption}
\begin{assumption}
\label{assum:increase}
The state of a task can only increase with pulls. Hence, the transition operators $\ev{\Tp_i}_{i \in \arms}$ are triangular inferior.
\end{assumption}
Rotting bandits were considering non-increasing sequences of rewards while we consider now non-decreasing sequences of states. Yet, it can correspond to the same situation where the reward is the opposite of the state. This is not only a formal remark. It is indeed the case for Intelligent Tutoring System motivation: the student is progressing so the associated need to learn the topic is decreasing.

We now make two Assumptions on the transition operators $\ev{\Tp_i}_{i \in \arms}$.

\begin{assumption}
\label{assum:symmetric}
The transition operator is the same for all the tasks,
\[ \forall i \in \arms, \Tp_i = \Tp.\]
\end{assumption}

For a random variable $X$ with density $p$, we call $F_{p}(z) \triangleq \EE{\ind{X\leq z}}$ the cumulative distribution function. We define the first-order stochastic dominance of a variable X (drawn with probability $p_x$) over a random variable $Y$ (drawn with probability $p_y$),
\begin{equation}
\label{eq:stoch-dominance}
   X \succeq Y \iff p_x \succeq p_y \iff \forall z \in \R,\  F_{p_x}(z) \leq F_{p_y}(z). 
\end{equation}

\begin{assumption}
\label{assum:stochastic-monotone}
The transition operator $\Tp$ is stochastically monotone, \ie \ with $(\Tp\delta_x)(y) \triangleq \Tp(x,y)$,
\[
\forall (x_1,x_2)\in \R^2, x_1 \leq x_2 \implies  \Tp\delta_{x_1} \preceq \Tp\delta_{x_2}.
\]
\end{assumption}

In other words, the larger the starting point, the larger the probability to reach any threshold at the next step. This assumption was first studied by \citet{daley1968stochastically}. We restate their two main results,

\begin{lemma}[\cite{daley1968stochastically}]
\label{lem:daley}
Assumption~\ref{assum:stochastic-monotone} is equivalent with 
\[
\forall (p,q), p \preceq q \implies  \Tp p \preceq \Tp q.
\]
\end{lemma}

\begin{corollary}[\cite{daley1968stochastically}]
\label{cor:daley}
The larger the starting state, the larger the probability of reaching any threshold after a given number of steps $n\in \NN$, \ie, 
\[
\forall (x_1,x_2)\in \R^2, x_1 \leq x_2 \implies  \Tp^n\delta_{x_1} \preceq \Tp^n\delta_{x_2}.
\]
\end{corollary}
For Intelligent Tutoring Systems, Assumption~\ref{assum:symmetric} means that the student progresses in the same way for all the topics. It may not be true if the topics are completely different subjects (e.g. maths and history) but if it is two topics in the same chapter (e.g. Pythagore and Thales theorems), it is likely that the progression of the student will be similar. Assumption~\ref{assum:stochastic-monotone} is a smoothness assumption on the progression of the student. If a student is quite good on a first topic and quite bad on another one, it is unlikely (yet possible) that after a single question on each topic, s/he masters the second one and not the first one. 

\paragraph{Objective}
We consider a task as being completed when
$\mu_{i,t} \geq \mu$ for a given threshold $\mu$. We will consider two related objectives. First, the \textit{simple} objective is to maximize the number of completed tasks after the horizon $T$, 
\[
r_T(\pi) \triangleq \sum_{i \in \arms} \mathbbm{1}\left[\mu_{i,T+1} \geq \mu \right].
\]

With respect to this objective, we can define a reward for our POMDP associated to the transition from state $x$ to $y$ $\rho(x, y) \triangleq \mathbbm{1}\left[x<\mu \land y \geq \mu\right]$. $r_T(\pi)$ is the sum of the reward,
\begin{equation}
\label{eq:link-r-rho}
r_T(\pi) = \sum_{t=1}^T \rho(\mu_{\pi(t), t},\mu_{\pi(t), t+1}) + \sum_{i\in \arms} \mathbbm{1}\left[\mu_{i,1} \geq \mu \right].
\end{equation}

Notice that the second sum is the sum of the arms which are initially above the threshold: it does not depend on the agent's action. Second, the \textit{cumulative} objective is to optimize,
\[
J_T(\pi) = \sum_{t=1}^T r_t(\pi).
\]

The reward at each round is $r_t(\pi)$. Thus, a validated task at the round $t$ yields cumulatively a reward equals to the remaining number of rounds $T-t$. For Intelligent Tutoring Systems, a completed topic may trigger new teaching actions such as starting new topics. The sooner we can trigger these actions, the better it is. It suggests that it is not only important to master topics at the end of the studying session, but also to master them as fast as possible. 

\begin{remark}
For both objectives, the reward at any round $t$ is a function of the state (current or previous) which is itself partially observable. One can hardly reconstruct the reward at the round $t$ from the unique observation sample at this same round. Indeed, if a student answers correctly to a question, we may have chosen a topic which is already mastered by the student (no reward for the action) or we may have chosen a topic which will be mastered very soon (good reward for the action). It contrasts with the cumulative reward in the multi-armed bandits paradigm, where the relationship between observation and reward is more straightforward.
\end{remark}

\begin{remark}
Our objectives $r_T$ and $J_T$ are random quantities. In the following, we will aim at maximizing their expected values, where the expectation is on the random evolution of the Markov Chains, the random noise in the observation, and the potential randomization of the agent's strategy. In particular, we highlighted that Assumption~\ref{assum:stochastic-monotone} is a "smooth in probability" assumption. Hence, even if abrupt progression is possible, these paths will weigh little in the expected regret compared to smooth ones.
\end{remark}
We give a consequence of Assumption~\ref{assum:stochastic-monotone} in terms of the number of rounds to reach the threshold $\mu$,

\begin{definition}
Let $\pa{\mu_i\pa{n}}_{ n \in \NN}$ a Markov chain with transition probabilities $\Tp$. We define the stopping time,
\[
\tau_{i}  \triangleq \min\left\{\tau \in \NN \, | \, \mu_i\pa{\tau} \geq \mu \right\},
\]
the number of pulls to reach the threshold $\mu$. We also define, 
\[
\tau_{i,t}  \triangleq \max\pa{\tau_{i} - \Nitmone, 0}.
\]
the remaining number of pulls at a round $t$ after $\Nit$ pulls. We notice that $\tau_{i} = \tau_{i,0}$.

Let $(X_n)_{n\in \NN}$ a Markov Chain with $X_0=x$ a transition probabilities $\Tp$. We call, 
\[
\tau(x) \triangleq \min\left\{\tau \in \NN \, | \, X_\tau \geq \mu \right\}.
\]
\end{definition}
\begin{lemma}
\label{lem:taux-taui}
For any arm $i$, $k\in \NN$ and $t\in \ev{1, \dots, T}$, 
\[\PP{\tau_{i,t} = k| \Ft} =  \PP{\tau\pa{\mu_{i,t}}=k|\mu_{i,t}}.\]
\end{lemma}
\begin{proof}
It is equivalent to show that for all $k$, 
\[\PP{\tau_{i,t} \geq k| \Ft} =  \PP{\tau\pa{\mu_{i,t}}\geq k|\mu_{i,t}}.\]
Notice that $\ind{\tau_{i,t} \geq k} \iff \mu_i(\Nitmone +k)< \mu $. Hence, 
\[\PP{\tau_{i,t} \geq k| \Ft}  = \PP{\mu_i(\Nitmone +k)< \mu| \Ft} = F_{\Tp^k\delta_{\mu_{i,t}}}(\mu)  .
\]
We also have, 
\[ \PP{\tau\pa{\mu_{i,t}}\geq k|\mu_{i,t}}= \PP{X_k <\mu | X_0 = \mu_{i,t}} = F_{\Tp^k\delta_{\mu_{i,t}}}(\mu).  \]
\end{proof}
\begin{lemma}
\label{lem:stopping-time}
If $x \leq y$, $\tau(x) \succeq \tau(y)$.

It further implies $\EE{\tau\pa{x}} \geq \EE{\tau\pa{y}}$.
\end{lemma}
\begin{proof}
For any $x\in \R$, 
\begin{equation}
\label{eq:cdf-tau}
\PP{\tau(x) \geq n} = \PP{X_{n-1} < \mu| X_0=x} = F_{\Tp^{n-1}\delta_{x}}(\mu)
\end{equation}
where the first equality is justified by the definition of $\tau(x)$ and Assumption~\ref{assum:increase}. Using Corollary~\ref{cor:daley}, we show the stochastic dominance,  
\[
x \leq y \implies \Tp^{n-1}\delta_{x} \preceq \Tp^{n-1}\delta_{y} \implies \tau(x) \succeq \tau(y).
\]
The last implication uses that $\PP{\tau(x) \geq n} = 1-F_{\tau(x)}(n) \geq \PP{\tau(y) \geq n} = 1-F_{\tau(y)}(n)$, where the inequality comes from Equation~\ref{eq:cdf-tau}. It implies that $F_{\tau(x)}(n) \leq F_{\tau(xy}(n)$ which is the definition of stochastic dominance. 

For the expectation, we use the layer-cake representation together with Equation~\ref{eq:cdf-tau}, 
\begin{align*}
\EE{\tau(x)} &=  \sum_{n=1}^{+\infty} \PP{\tau(x) \geq n} = \sum_{n=0}^{+ \infty} F_{\Tp^{n}\delta_{x}}(\mu).
\end{align*}
Hence, if $x \leq y$,
\[
\EE{\tau(x) - \tau(y)} =  \sum_{n=0}^{+ \infty} \pa{F_{\Tp^{n}\delta_{x}}(\mu) - F_{\Tp^{n}\delta_{y}}(\mu)} \geq 0,
\]
where we used Corollary~\ref{cor:daley}. 
\end{proof}
\section{Optimal Oracle: Focus on the largest under the threshold}
\label{sec:oracle}

\subsection{The {\FLUT} oracle }
An oracle policy $\Tilde{\pi}$ is a policy which has access to the current and past values of all arms $\left\{\mu_{i,s}\right\}_{i \in \arms, s \leq t}$ and to the transition matrix $\Tp$. More precisely, we define the set of states at any round $t$ $\statet = \ev{\mu_{i,t}}_{i \in \possibleArms}$ and the random variables known by an oracle at $t$, 
\[\Ft = \ev{\ev{\state_{\boldsymbol{s}}}_{1 \leq s\leq t}, \ev{i_s}_{1 \leq s \leq t-1}}.\]
 Notice that the oracle does not have access to the future of the Markov Chain, and can only make projections based on $\Tp$.

We define the sets of arms under and above the threshold before the round $t$: 
\begin{align*}
\armsbelow \triangleq \left\{i \in \arms | \mu_{i,t} < \mu \right\}\\
\armsabove\triangleq \left\{i \in \arms | \mu_{i,t} \geq \mu \right\}.
\end{align*}

We describe Focus on the Largest Under the Threshold (\FLUT) in Algorithm~\ref{alg:flut}, an oracle policy $\oracle$ which selects at each round the arm with the largest state below the threshold $\mu$.
\begin{minipage}{\textwidth}
\renewcommand*\footnoterule{}
\begin{savenotes}
\begin{algorithm}[H]
\caption{Focus on the Largest Under the Threshold}% (\FLUT or $\oracle$)}
\label{alg:flut}
\begin{algorithmic}[1]
\Require $\mu$
	\For{$t \gets 1, 2, \dots \do $}
		\State \textsc{Receive} $\statet \leftarrow \ev{\mu_{i,t}}_{i \in \arms}$
		\State $\armsbelow \leftarrow \left\{i \in \arms | \mu_{i,t} < \mu \right\}$
		\If {$\armsbelow \neq \ev{}$}
		\textsc{Pull} $i_t \in \argmax_{i \in \armsbelow} \mu_{i,t}$\footnote{One can choose the tie break selection rule arbitrarily, e.g. by selecting the arm with the smallest index.}; 
		\Else {\textsc{ Pull at random} $i_t \in \arms$}
		\EndIf
	\EndFor
\end{algorithmic}
\end{algorithm}
\end{savenotes}
\end{minipage}
\begin{remark}
We note that $\oracle$ is an oracle policy which does not use the knowledge of $\Tp$. It is similar to the optimal oracle for rotting bandits. It is an interesting feature as one can hope to approximate $\Tp$ by simply estimating the state of the arms like in bandits, and without caring about the transitions.
\end{remark}

\subsection{Optimality}
\begin{theorem}
\label{th:FLUT-opt}
For any oracle policy $\Tilde{\pi}$ and any round $t$, 
\[
r_t(\oracle) \succeq r_t(\Tilde{\pi}).
\]
\end{theorem}

\begin{corollary}
$\oracle$ maximizes $\EE{r_t(\pi)}$ without the knowledge of the round $t$. Therefore, it maximizes $\EE{J_T(\pi)}$ for any horizon $T$.
\end{corollary}

%TODO

\subsection{Proof of Theorem~\ref{th:FLUT-opt}}
\subsubsection{Sketch}
The proof is quite technical. We give here a sketch highlighting the main difficulties. In the spirit of the Bellman Equation \citep{bellman1966dynamic}, our proof shows recursively from the end that selecting the largest arm under the threshold is the best thing to do concerning $r_{t:T}$, the future reward collected from the round $t$. More precisely, "the best thing to do" means that $\oracle$ maximizes 
$\PP{r_{t:T}(\Tpi) \geq r| \Ft}$ for any reward objective $r$.

The initialization at the last round is rather straightforward given our assumptions. Indeed, according to Assumption~\ref{assum:symmetric}, all the arms have the same transition operator. Moreover, according to Assumption~\ref{assum:stochastic-monotone}, the probability to reach any threshold in one step increases with the value of the starting point.  Hence, following \FLUT maximizes the probability to reach the threshold for the selected arm. Because the transitions are rested, we cannot bring more than $r=1$ arm above the threshold. Hence, \FLUT maximizes $\PP{r_{T:T}(\cdot) \geq 1| \FT}= \PP{r_{T:T}(\cdot) = 1| \FT}$.

Then, we consider a round $t$ such that \FLUT is the best thing to do from $t+1$. Hence, we compare \FLUT which policies which follow any rule at $t$ and then \FLUT from $t+1$.  We split the possibilities in three: (1) either $i_t \in \armsabove$, or (2) $i_t$ is in the $r$ largest value below the threshold at the round $t$, or (3) $i_t$ is below this $r$-th value.

(1) Arguably, selecting an arm $i_t \in \armsabove$ is totally useless because this arm is already above the threshold and the transitions are rested (Assumption~\ref{assum:rested}). 

(2) In order to pass $r$ arms above the threshold until the end of the game, $\oracle$ first selects repetitively the largest arm in $\armsbelow$ until it reaches the threshold, then the second largest, etc., until the $r$-th.  Hence, $\PP{r_{t:T}(\oracle) \geq r| \Ft}$ is equal to $\PP{\!\sum_{x \in \Ort} \!\tau(x) \leq T - t + 1 \Bigg| \Ft}$, that is, the probability that the sum of the remaining pulls to reach the threshold for the $r$ largest arms below the threshold\footnote{$\Ort$ is the set of $r$ largest values below the threshold at any round $t$.} is smaller than the remaining rounds (Lemma~\ref{lem:r-tau}). Since the transitions are rested and Markov, the order of the pulls does not matter: it is necessary to advance the $r$ Markov chains to get at least $r$ rewards. Hence, selecting any arm among the $r$ largest values below the threshold and then follow $\oracle$ from $t+1$ achieves the same $\PP{r_{t:T}(\cdot) \geq r| \Ft}$ than $\oracle$.

(3) Comparing \FLUT with the case where we pull an arm below the $r$-th value of $\armsbelow$ is the most difficult part of the proof. However, it seems quite intuitive with our Assumption~\ref{assum:stochastic-monotone} that pulling an arm that is among the furthest to the threshold is not optimal.

According to Lemma~\ref{lem:r-tau} and Corollary~\ref{cor:f-non-decreasing}, if we follow \FLUT after $t+1$, the only thing that matter with respect to $\PP{r_{t:T}(\cdot) \geq r| \Ft}$ is the $r$ largest states of $\arms^-_{t+1}$ (or the $r-1$ largest states of $\arms^-_{t+1}$ if the arm that we select at the round $t$ reaches the threshold). The larger are those states, the higher is $\PP{r_{t+1:T}(\cdot) \geq r| \Ftpone}$. After the $t$-th round, we move two different values below the threshold if we follow \FLUT or if we take an other arm. It is hard to compare these two states in terms of potential reward. The trick is to use the last result: $\oracle$ performs the same than the policy which selects the $r$-th value of $\armsbelow$ (with respect to $\PP{r_{t:T}(\cdot) \geq r| \Ft}$).  

If we compare to this policy instead of \FLUT, the $r$ largest states of $\arms^-_{t+1}$ are the $r-1$ largest states of $\armsbelow$ and an other value. If we pull an arm $i_r$ with the $r$-th value below the threshold at the round $t$, then this other value is $\mu_{i_r,t+1}$. If we pull an arm $i_t$ below $\mu_{i_r,t}$,  then the other value is $\max(\mu_{i_r,t}, \mu_{i_t, t+1})$. We can compare the distributions associated to these two random variables, and see that the first one stochastically dominates the other one (thanks to Assumption~\ref{assum:stochastic-monotone}). 

In the two cases, $\PP{r_{t:T}(\cdot) \geq r| \Ft}$ is the expectation of $\PP{r_{t+1:T}(\oracle) \geq r| \Ftpone}$ over these random variables. Since $\PP{r_{t+1:T}(\oracle) \geq r| \Ftpone}$ is non-decreasing with the $r$ largest values in $\armsbelow$, we can show that $\PP{r_{t:T}(\oracle) \geq r| \Ft} \geq \PP{r_{t:T}(\Tpi) \geq r| \Ft}$ thanks to Lemma~\ref{lem:stoch-dom-monotone}. 

It concludes the induction as it shows that for any arm choice at the round $t$, $\oracle$ maximizes $\PP{r_{t:T}(\cdot) \geq r| \Ft}$ for any $r$. 
\begin{proof}
\subsubsection{Introduction}
According to the definition of the first order stochastic dominance (Equation~\ref{eq:stoch-dominance}), we want to show that for all $r\in \NN$ and for any oracle policy $\Tpi$, 
\[
\PP{r_T(\oracle) \geq r| \F_1} \geq \PP{r_T(\Tpi) \geq r|\F_1}.
\]

$\F_1$ represents indeed the information available to the oracle at the beginning of the game. We recall the definition of $\rho(x, y) \triangleq \mathbbm{1}\left[x<\mu \land y \geq \mu\right]$. We define,
\[
r_{s:t}(\pi) =  \sum_{t'=s}^t \rho(\mu_{\pi(t'), t'},\mu_{\pi(t'), t'+1}). 
\]
Using Equation~\ref{eq:link-r-rho}, we can write, 
\[
r_T(\Tpi) =  r_{1:T}(\Tpi) + \sum_{i \in \arms} \ind{ \mu_{i,1} \geq \mu}.
\]
Since the above sum does not depend on the policy $\Tpi$, we will show recursively from the end $t=T$ that for all $r\in \NN$ that, 
\[
\PP{r_{t:T}(\oracle) \geq r| \Ft} \geq \PP{r_{t:T}(\Tpi) \geq r| \Ft}.
\]

\subsubsection{Last round}
At the last round $t=T$, the $r_{T:T}$ is equal to 1 if the selected arm is above the threshold and else to 0. For $r > 1$ and $r=0$, we have the trivial equalities,
\begin{align*}
    &\PP{r_{T:T}(\Tpi) \geq 0| \FT} = 1,\\
    &\PP{r_{T:T}(\Tpi) \geq 2| \FT} = 0.
\end{align*}

For $r=1$, if $\Tpi(T) = i_T \in \arms_T^+$, the probability of reaching $\mu$ with a new arm is null because the arm is already above the threshold. Hence, 
\[\PP{r_{T:T}(\Tpi) \geq 1| \FT \land i_T \in \arms_T^+} = 0 \leq \PP{r_{T:T}(\oracle) \geq 1| \FT}.\]

If $\Tpi(T) = i_T \in \arms_T^-$, we can use Assumption~\ref{assum:stochastic-monotone},
\begin{align*}
    \PP{r_{T:T}(\Tpi) \geq 1| \FT \land i_T \in \arms_T^-} &= \PP{\mu_{i_T,T+1}\geq \mu |\FT \land i_T \in \arms_T^-} \\
    &\leq \PP{\mu_{\isT,T+1}\geq \mu |  \FT \land i_T = \isT}\\
    &= \PP{r_{T:T}(\oracle) \geq 1| \FT}.
\end{align*}

Indeed, by definition of $\isT$, $\mu_{\isT,T} \geq \mu_{i_T,T}$ if $i_T \in \arms_T^-$. Therefore, we do have for all $r$ and any oracle policy $\Tpi$, 
\[
\PP{r_{T:T}(\oracle) \geq r| \FT} \geq \PP{r_{T:T}(\Tpi) \geq r| \FT}.
\]
\subsubsection{Backward induction}
Now, we consider a round $t$ such that, for any $\Tpi$ and $r$,
\begin{equation}
\label{eq:recursivity}
  \PP{r_{t+1:T}(\oracle) \geq r| \Ftpone} \geq \PP{r_{t+1:T}(\Tpi) \geq r| \Ftpone}.  
\end{equation}

We want to show that this relation is still true at the round $t$, 
\begin{equation*}
  \PP{r_{t:T}(\oracle) \geq r| \Ft} \geq \PP{r_{t:T}(\Tpi) \geq r| \Ft}.  
\end{equation*}


We consider the policy $\Tpi_{t}$ which follows $\oracle$ except at the round $t$ where it follows $\Tpi$. Thus, for any $r \in \NN$,
\begin{equation}
\label{eq:optimal-oracle-t+1}
    \PP{r_{t+1:T}(\Tpi_{t}) \geq r| \Ftpone} = \PP{r_{t+1:T}(\oracle) \geq r| \Ftpone} \geq \PP{r_{t+1:T}(\Tpi) \geq r| \Ftpone}.
\end{equation}
The first equality is justified by the fact that the two policies behave the same from $t+1$, hence they collect the same reward. The inequality follows from Equation~\ref{eq:recursivity}. 
\begin{align}
 \PP{r_{t:T}(\Tpi) \geq r| \Ft} = &\EE{ \ind{\mu_{i_t,t+1}\geq \mu}\PP{r_{t+1:T}(\Tpi) \geq r-1| \Ftpone}|\Ft \land i_t \sim \Tpi(t)} \nonumber\\
 &+ \EE{ \ind{\mu_{i_t,t+1}< \mu}\PP{r_{t+1:T}(\Tpi) \geq r| \Ftpone}|\Ft \land i_t \sim \Tpi(t)} \nonumber    \\
 \leq& \EE{ \ind{\mu_{i_t,t+1}\geq \mu}\PP{r_{t+1:T}(\Tpi_{t}) \geq r-1| \Ftpone}|\Ft \land i_t \sim \Tpi(t)} \nonumber\\
 &+ \EE{ \ind{\mu_{i_t,t+1}< \mu}\PP{r_{t+1:T}(\Tpi_{t}) \geq r| \Ftpone}|\Ft \land i_t \sim \Tpi(t) }\nonumber\\
 =  &\PP{r_{t:T}(\Tpi_{t}) \geq r| \Ft}.
 \label{eq:Tpit-Tpi}
\end{align}

The inequality follows from Equation~\ref{eq:optimal-oracle-t+1}: following the $\oracle$ (or equivalently $\Tpi_{t}$) is optimal after round $t$. The equalities mean that either arm $i_t$ reaches the threshold at the round $t$ and we still need $r-1$ arms to reach the threshold after the round $t$, or arm $i_t$ do not reach the threshold and we need $r$ arms to reach the threshold after $t$. To conclude the proof, we need to show that,
\[
\PP{r_{t:T}(\Tpi_{t}) \geq r| \Ft} \leq \PP{r_{t:T}(\oracle) \geq r| \Ft}.\]  

We call $\Ort$ the $r$ largest states below the threshold at the round $t$ (for $r\leq |\armsbelow|$). We call $\mu^r_{t} \triangleq \min {\Ort}$, the $r$-th largest value below the threshold. We call $\statet^{\boldsymbol{i}}$, the set of states at $t$ excluding the state of $i$. Hence, $\Or\pa{\statet^{\boldsymbol{i}}}$ is the set of the $r$ largest states below the threshold excluding the state of $i$.   We  distinguish three cases: when $\mu_{i_t, t} \in \left] - \infty,  \mu^r_{t}\right[$, $ \mu_{i_t, t} \in \left[ \mu^r_{t}, \mu \right[$ and $\mu_{i_t, t} \in \left[ \mu, + \infty\right[$. We call $\Krt \triangleq \ev{i\in \arms | \mu_{i,t} \in \Ort} \subset \armsbelow$ such that the three aforementioned cases corresponds to respectively $i_t\in \armsbelow \setminus \Krt$, $i_t\in \Krt$ and $i_t \in \armsabove$.


\subsubsection{Backward induction: the selected arm is in the $r$ largest values below the threshold}
We will start by considering the case $i_t \in \Krt$. It is equivalent to $\mu_{i_t, t-1}\in \Ort$. Lemma~\ref{lem:r-tau} becomes, 
\begin{equation*}
\PP{r_{t:T}(\Tpi_{t}) \geq r| \Ft \land i_t \in \Krt} = \PP{ \sum_{x \in \Ort} \tau(x) \leq T - t +1 \Bigg| \Ft}.
\end{equation*}
%The first equality uses that $\mu_{i_t, t-1} \geq \mu^r_{t-1}$ together with Lemma~\ref{lem:r-tau}. The second equality uses that, since $\mu_{i_t, t-1} \in  \Ortmone$ and the transitions are rested, $\Ortmone = \Ormonetmoneit \cup \left\{ \mu_{i_t, t-1} \right\} $. The third equality follows from Lemma~\ref{lem:tau+1}. 
Notice that this expression is independent of $i_t \in \Krt$. Therefore, since $\ist \in \Krt$ for any $r$, we have that,
\begin{equation}
\PP{r_{t:T}(\Tpi_{t}) \geq r| \Ft \land i_t \in \Krt}  = \PP{r_{t:T}(\oracle) \geq r| \Ft }.  \label{eq:muitt-middle}
\end{equation}

\subsubsection{Backward induction: the selected arm is below the $r$-th largest value below the threshold}
We consider the case $i_t \in \armsbelow \setminus \Krt$. Hence, $\mu_{i_t, t} \notin \Ormonet$, which implies $\Ormonetit = \Ormonet$.  Moreover, since the setup is rested, $\Ormonetponeit = \Ormonetit$ if $i_t$ is selected at the round $t$.  Hence, $\Ormonetponeit =\Ormonet$. Thus, we can rewrite Lemma~\ref{lem:r-tau}, 
\begin{multline}
     \PP{r_{t:T}(\Tpi_{t}) \geq r| \Ft \land i_t \in \armsbelow \setminus \Krt} \\
     = \PP{\tau(\max\pa{\mu^r_{t}, \mu_{i_t, t+1}}) +\!  \sum_{x \in \Ormonet}\! \tau(x) \leq T - t \Bigg| \Ft \land i_t \in \armsbelow \setminus \Krt}.\label{eq:suboptimal-f}
\end{multline}

Let $i_r\in \arms$, an arm with value $\mu^r_{t-1}$ at the beginning of the round $t$. We have that,  
\begin{align}
\label{eq:optimal-f}
    \PP{r_{t:T}(\oracle) \geq r| \Ft } &= \PP{r_{t:T}(\Tpi_{t}) \geq r| \Ft \land i_t =i_r } \nonumber\\
    &= \PP{\tau(\mu_{i_r, t+1}) +  \sum_{x \in \Ormonet} \tau(x) \leq T - t \Bigg| \Ft \land i_t = i_r}.
\end{align}

The first equation follows from Equation~\ref{eq:muitt-middle}. The second uses Lemma~\ref{lem:r-tau} with $\mu_{i_r,t+1} \geq \mu_{i_r, t} =\mu^{r}_{t}$. We also use that  with the same argument $\Ormonetir =\Ormonet$ because $i_r$ corresponds to the $r$-th value below the threshold. Hence, both $\PP{r_{t:T}(\oracle) \geq r| \Ft}$ and $\PP{r_{t:T}(\Tpi_{t}) \geq r| \Ft}$ can be written as the mean of the function, 
\[
f(y) = \PP{\tau(y) +  \sum_{x \in \Ormonet} \tau(x) \leq T - t
\Bigg | \Ft }, \]

according to different probability densities. Because $f$ is non-decreasing (Corollary~\ref{cor:f-non-decreasing}), we only have to show that the probability density associated to $\oracle$ stochastically dominates the probability density associated $\Tpi_{t}$ (Lemma~\ref{lem:stoch-dom-monotone}). The probability density associated to $\oracle$ in Equation~\ref{eq:optimal-f} is,
\[
p_\star(y) =  \PP{\mu_{i_r,t+1} = y | \Ft \land i_t = i_r } = \Tp \delta_{\mu_{i_r,t}}.
\]

The probability density associated to arm $i_t$ in 
Equation~\ref{eq:suboptimal-f} is,
\[
p_{i_t}(y) =  \PP{\max\pa{\mu_{i_t,t+1}, \mu^{r}_{t}} = y | \Ft \land i_t \in \armsbelow \setminus \Krt }.
\]

In order to prove the stochastic dominance,  we want to show that $F_{p_\star} \leq  F_{p_{i_t}}$. Notice that $p_{i_t}$ is the rectified probability density of $\mu_{i_t,t+1}$, where the mass below $\mu^r_{t}$ is transferred at $\mu^r_{t}$. Hence, we can write its CDF as,
\begin{equation}
\label{eq:Fpit}
F_{p_{i_t}}(x) = \begin{cases}
      0, & \text{if}\ x < \mu^{r}_{t} \\
      F_{\Tp\delta_{\mu_{i_t,t}}}(x), & \text{otherwise}.
    \end{cases}
\end{equation}

For $x < \mu^r_{t}$, 
\begin{equation}
\label{eq:dominance-small-x}
F_{p_{\star}}(x) = 0 = F_{p_{i_t}}(x).
\end{equation}

The first equality comes from Assumption~\ref{assum:increase}: since the reward is non-decreasing we have  $\PP{\mu_{i_r, t+1} < \mu_{i_r,t} | \Ft \land i_t =i_r} =0 $. The second equality comes from Equation~\ref{eq:Fpit}. For $x \geq \mu^r_{t}$, 
\begin{equation}
\label{eq:dominance-large-x}
\forall x \geq \mu^r_{t}, F_{p_{\star}}(x) = F_{\Tp\delta_{\mu_{i_r,t}}}(x) \leq F_{\Tp\delta_{\mu_{i_t,t}}}(x) =  F_{p_{i_t}}(x).
\end{equation}
where we use Assumption~\ref{assum:stochastic-monotone} and the fact that $\mu_{i_t,t} \leq \mu^r_{t}$. According to Equations~\ref{eq:dominance-small-x} and~\ref{eq:dominance-large-x}, we do have $F_{p_\star}(x) \leq  F_{p_{i_t}}(x)$ for all $x$ which is the definition of stochastic dominance: $p_\star \succeq  p_{i_t}$. Therefore, because $f$ is non decreasing (see Corollary~\ref{cor:f-non-decreasing} and Lemma~\ref{lem:stoch-dom-monotone}), we conclude,
\begin{equation}
    \PP{r_{t:T}(\oracle) \geq r| \Ft} = \E_{p_\star}\left[f\right] \geq \E_{p_{i_t}}\left[f\right] = \PP{r_{t:T}(\Tpi_t) \geq r| \Ft \land i_t\in \armsbelow \setminus \Krt }.
    \label{eq:muitt-low}
\end{equation}

\subsubsection{Backward induction: the selected arm is above the threshold}
We consider the case $\mu_{i_t,1} > \mu$. Intuitively, selecting such arm is useless, because it does not bring any new arm above or closer to the threshold. We write formally this argument,
\begin{align}
    \PP{r_{t:T}(\Tpi_{t}) \geq r| \Ft \land i_t \in \armsabove} &= \PP{\PP{r_{t+1: T}\left(\Tpi_{t}\right) \geq r | \Ftpone}|\Ft} \nonumber\\
    &=\PP{\PP{\sum_{x \in \Ortpone} \tau(x) \leq T-t | \Ftpone}|\Ft}\nonumber\\
    &=\PP{\sum_{x \in \Ort} \tau(x) \leq T-t | \Ft}\nonumber\\
    &\leq  \PP{\sum_{x \in \Ort} \tau(x) \leq T-t+1 | \Ft}\nonumber \\
    &=    \PP{r_{t:T}(\oracle) \geq r| \Ft } .
    \label{eq:muitt-high}
\end{align}
The first equation means that no arm goes above the threshold at the round $t$. The second equation follows from Lemma~\ref{lem:r-tau}. The third equation follows because by the rested assumption all the arm $i \in \armsbelow$ keep their value between $t$ and $t+1$. Hence, $\Ort = \Ortpone$ for all $r$. The inequation follows because the event in the RHS probability include the event in the LHS probability. Finally, we use again Lemma~\ref{lem:r-tau}.
\subsubsection{Conclusion}
Putting together Equations~\ref{eq:muitt-middle}, \ref{eq:muitt-low} and~\ref{eq:muitt-high}, we can write,
\[
  \PP{r_{t:T}(\oracle) \geq r| \Ft} \geq \PP{r_{t:T}(\Tpi_t) \geq r| \Ft \land i_t}. 
\]
Hence, if we average the RHS on $i_t \sim \Tpi_t(t)$ (notice that $\Tpi_t(t)$ is a $\Ft$-measurable distribution by definition of $\Ft$, we have,
\[
  \PP{r_{t:T}(\oracle) \geq r| \Ft} \geq \PP{r_{t:T}(\Tpi_t) \geq r| \Ft }. 
\]
Now, we can use Equation~\ref{eq:Tpit-Tpi} to conclude the induction, 
\[
  \PP{r_{1:T}(\oracle) \geq r| \Ft} \geq \PP{r_{t:T}(\Tpi) \geq r | \Ft }. 
\]

Hence, using the induction,
\[
  \PP{r_{1:T}(\oracle) \geq r| \F_1} \geq \PP{r_{1:T}(\Tpi) \geq r | \F_1 }. 
\]
This statement concludes the proof, as we noticed in the Introduction.
\end{proof}

\subsection{Technical Lemmas}
\begin{lemma}
\label{lem:tau+1}
Let $A$ a random variable $\pa{\Ft \land i_t}$-measurable. Let $i_t$ the selected arm by $\Tpi$ at a round $t$. Then,
\[
\PP{ \tau\pa{\mu_{i_t,t+1}} \leq A| \Ft \land i_t}  = \PP{\tau\pa{\mu_{i_t,t}} \leq A + 1| \Ft\land i_t}.
\]
\end{lemma}
\begin{proof}
According to Lemma~\ref{lem:taux-taui},
\[
\PP{ \tau\pa{\mu_{i_t,t+1}} \leq A| \Ft \land i_t} = \PP{ \tau_{i_t,t+1} \leq A| \Ft \land  i_t}.
\]
If arm $i_t$ is selected at a round $t$, 
\[\tau_{i_t,t+1} \triangleq \tau_{i_t} - N_{i_t,t+1} = \tau_{i_t} - \pa{N_{i_t,t} + 1} = \tau_{i_t,t}-1 .\]

Hence, we can write, 
\begin{align*}
\PP{ \tau_{i_t,t+1} \leq A| \Ft \land i_t} &= \PP{ \tau_{i_t,t} \leq A+1| \Ft \land  i_t}\\
&=\PP{ \tau\pa{\mu_{i_t,t}} \leq A+1| \Ft \land i_t}.
\end{align*}
\end{proof}
\begin{lemma}
\label{lem:r-tau}
We define the number of arms which passes the threshold between rounds $t$ and $T$ (included) when we follow policy $\pi$, 
\begin{equation}
\label{eq:def-rtT}
r_{t:T}(\pi) \triangleq \sum_{i \in \arms} \mathbbm{1} \left[\mu_{i,T+1} \geq \mu \land \mu_{i,t} < \mu\right].
\end{equation}
Let $\Tpi_{t}$ the policy which follows $\oracle$ except at the round $t$ where it uses any decision rule such that $i_t \in \armsbelow$. We call $\Ort$, the set of the $r \in \left\{1, \dots, |\armsbelow| \right\}$ largest arm below the threshold at the round $t$.  We call $\mu^{r}_{t} = \min \mathcal{O}_r(\statet)$, the $r$-th value below the threshold. We call $\Ormonetit$, the set of the $r-1$ largest values below $\mu$ at the round $t$ excluding $\mu_{i_t,t}$. Then,
 \begin{multline*}
 \PP{r_{t:T}(\Tpi_t) \geq r| \Ft\land i_t \in \armsbelow} \\= \PP{\tau(\max\pa{\mu_{i_t,t+1}, \mu^{r}_{t}})+\!\sum_{x \in \Ormonetit} \!\tau(x) \leq T - t  \Bigg| \Ft\land i_t \in \armsbelow}.
 \end{multline*}
Let $\Krt \triangleq \ev{i\in \arms | \mu_{i,t-1} \in \Ort} \subset \armsbelow$, the set of arms below the threshold with a state larger or equal than $\mu^{r}_{t-1}$. In the special case where $i_t \in \Krt$ (e.g. $\oracle$), we have, 
  \begin{equation*}
 \PP{r_{t:T}(\Tpi_t) \geq r| \Ft\land i_t \in \Krt} = \PP{\!\sum_{x \in \Ort} \!\tau(x) \leq T - t + 1 \Bigg| \Ft}.
 \end{equation*}
\end{lemma}
\begin{proof}
We will prove this claim by induction from $t=T$. For $r=1$, we have,
\begin{align*}
&\PP{r_{T:T}(\Tpi_{T}) \geq 1| \FT \land i_T \in \arms_T^-} \\
&\qquad\qquad= \PP{r_{T:T}(\Tpi_{T}) = 1| \FT\land i_T \in \arms_T^-} \\
&\qquad\qquad= \PP{\mu_{i_T,T+1} \geq \mu | \FT \land i_T \in \arms_T^-} \\
&\qquad\qquad= \PP{\tau\pa{\max\pa{\mu_{i_T,T+1}, \mu^1_{T}}} = 0  | \FT \land i_T \in \arms_T^-} \\
&\qquad\qquad= \PP{ \tau\pa{\max\pa{\mu_{i_T,T+1}, \mu^1_{T}}} + \sum_{x \in \OO_{0}\pa{\state_{\boldsymbol{T}}^{\boldsymbol{i_T}}}} \tau(x)\leq 0 \Bigg | \FT \land i_T \in \arms_T^-}. 
\end{align*}
The first equality is justified because, by the rested Assumption~\ref{assum:rested}, at most one arm can pass above the threshold during a single round. The only arm which can go above the threshold is the selected one, that is $i_T$, which leads to the second equation. The third equation uses that $\tau\pa{\max\pa{\mu_{i_T,T+1}, \mu^1_{T}}}=0 \iff \max\pa{\mu_{i_T,T+1}, \mu^1_{T}}\geq \mu \iff \mu_{i_T,T+1} \geq \mu$ because $\mu^1_{T} < \mu$ by definition of $\mu^r_t$.  Last, we use that $\OO_{0}\pa{\state_{\boldsymbol{T}}^{\boldsymbol{i_T}}}  = \left\{ \right\}$ and that $\tau(\cdot) \geq 0$ by definition of $\tau$.

For $\oracle$, which is the special case where $i_T = \isT \in \arms^1_T$, we can write,
\begin{align*}
\PP{r_{T:T}(\oracle) \geq 1| \FT }& = \PP{r_{T:T}(\Tpi_{T}) \geq 1| \FT \land i_T = \isT} \\
&=  \PP{\sum_{x \in \OO_{1}\pa{\state_{\boldsymbol{T+1}}}} \tau(x)\leq 0 \Bigg | \FT \land i_T=\isT} \\
& = \PP{\sum_{x \in \OO_{1}\pa{\state_{\boldsymbol{T}}}} \tau(x)\leq 1 \Bigg | \FT }. 
\end{align*}
The second equation follows from $\OO_{1}\pa{\state_{\boldsymbol{T+1}}} = \ev{\mu_{\isT,T}}$. The third equation uses Lemma~\ref{lem:tau+1} since there is only one element in the sum. 

Last, we notice that for $r>1$, 
\begin{align*}
&\PP{r_{T:T}(\Tpi_{T}) > 1| \FT \land i_T \in \arms_T^-} = 0,\\
& \PP{ \tau\pa{\max\pa{\mu_{i_T,T+1}, \mu^1_{T}}} + \sum_{x \in \OO_{r-1}\pa{\state_{\boldsymbol{T}}^{\boldsymbol{i_T}}}} \tau(x)\leq 0 \Bigg | \FT \land i_T \in \arms_T^-} =0,\\
 &\PP{\sum_{x \in \OO_{r}\pa{\state_{\boldsymbol{T}}}} \tau(x)\leq 1 \Bigg | \FT }=0.
\end{align*}
First, because $\Tpi_T$ cannot bring more than one arm above the threshold in one round. The second and third equations follows because $r>1$ and, for any $r'$ and $\bX$,
\[\sum_{x \in \OO_{r'}\pa{\bX}} \tau(x) \geq | \OO_{r'}\pa{\bX}| = r' .\] 
 Indeed, notice that $\tau(x) \geq 1$ when $x<\mu$, which is the case by definition of $\Or(\cdot)$. Therefore, we have the desired equations for all $r\leq | \arms^-_T|$ at the round $T$.


By induction, we assume a round $t$ such that $\oracle$ verifies for all $r\leq |\arms_{t+1}^-|$, 
\begin{equation*}
\PP{r_{t+1:T}(\oracle) \geq r| \Ftpone} =  \PP{\sum_{x \in \Ortpone} \tau(x) \leq T - t  \Bigg| \Ftpone}. 
\end{equation*}

Since $\Tpi_{t}$ follows the oracle after the round $t$, we have,
\begin{equation}
\label{eq:induction-lemma}
\PP{r_{t+1:T}(\Tpi_t) \geq r| \Ftpone} =  \PP{\sum_{x \in \Ortpone} \tau(x) \leq T - t  \Bigg| \Ftpone}. 
\end{equation} 

We decompose the probability at the round $t$ on either arm $i_t$ reaches the threshold at the round $t$ or not, 
 \begin{align}
 &\PP{r_{t:T}(\Tpi_t) \geq r| \Ft\land i_t \in \armsbelow} \nonumber\\
 &\qquad \quad = \EE{ \ind{\mu_{i_t,t+1} \geq \mu} \PP{r_{t+1:T}(\Tpi_t) \geq r-1| \Ftpone}| \Ft \land i_t\in \armsbelow}\nonumber \\
& \qquad \qquad + \EE{ \ind{\mu_{i_t,t+1} < \mu} \PP{r_{t+1:T}(\Tpi_t) \geq r| \Ftpone} | \Ft\land i_t\in \armsbelow}. 
\label{eq:split-on-muit}
 \end{align}
%
We start with the first term in the sum. When $\mu_{i_t,t+1} \geq \mu$, we can write,
\begin{align}
&\PP{r_{t+1:T}(\Tpi_t) \geq r-1| \Ftpone} \\
&\qquad \qquad= \PP{\sum_{x \in \Ormonetpone} \tau(x) \leq T - t \Bigg| \Ftpone}\nonumber\\
&\qquad \qquad=  \PP{ \tau(\max\pa{\mu_{i_t,t+1}, \mu^{r}_{t}})+ \sum_{x \in \Ormonetpone} \tau(x) \leq T -t \Bigg| \Ftpone}\nonumber \\
&\qquad \qquad=  \PP{ \tau(\max\pa{\mu_{i_t,t+1}, \mu^{r}_{t}})+ \sum_{x \in \Ormonetit} \tau(x) \leq T - t \Bigg| \Ftpone}.
\label{eq:muit-above-mu}
\end{align}
The first equality follows from Equation~\ref{eq:induction-lemma}. The second equality follows because $\tau(\max{\mu_{i_t,t+1}, \mu^{r}_{t}}) = \tau(\mu_{i_t,t+1}) = 0$ when $\mu_{i_t,t+1} \geq \mu > \mu^{r}_{t}$. The last equality follows because since $\mu_{i_t,t+1} \geq \mu \implies \mu_{i_t,t+1} \notin\Ormonetpone$. Hence,  $\Ormonetpone = \Ormonetponeit$. Moreover, because the transitions are rested, $\Ormonetponeit = \Ormonetit$. 

For the second term in the sum - when $\mu_{i_t, t+1} < \mu$ - we can write, 
\begin{align}
&\PP{r_{t+1:T}(\Tpi_t) \geq r| \Ftpone}\nonumber \\
&\qquad \qquad = \PP{\sum_{x \in \Ortpone} \tau(x) \leq T - t \Bigg| \Ftpone} \nonumber \\
&\qquad \qquad= \PP{\tau(\max\pa{\mu_{i_t,t+1}, \mu^{r}_{t}})+\sum_{x \in \Ormonetit} \tau(x) \leq T - t \Bigg| \Ftpone}.
\label{eq:muit-below-mu}
\end{align}
Again, we use Equation~\ref{eq:induction-lemma}. Then, we cut $\Ortpone$ in two: On the one hand, the $r-1$ largest values below the threshold excepted $\mu_{i_t,t}$, that is $\Ormonetponeit$. It is equal to $\Ormonetit$ by the rested assumption. On the other hand, the remaining value which is $\mu_{i_t,t+1}$ if $\mu_{i_t,t+1} \geq \mu^{r}_{t+1}$, or else $\mu^{r}_{t+1}$. In that second case, we have that $\mu^{r}_{t+1} > \mu_{i_t,t+1} \geq \mu_{i_t,t}$. Therefore, by the rested assumption, the $r$ largest values below the threshold remain the same between $t$ and $t+1$. Hence, we have that $\mu^{r}_{t+1} = \mu^r_{t}$.

Notice that Equations~\ref{eq:muit-above-mu} and~\ref{eq:muit-below-mu} leads to the same result, independently on whether $\mu_{i_t,t+1} \geq \mu$ or not. Hence, we can rewrite Equation~\ref{eq:split-on-muit} to conclude the first part of the Lemma,
 \begin{multline}
 \PP{r_{t:T}(\Tpi_t) \geq r| \Ft\land i_t \in \armsbelow} \\= \PP{\tau(\max\pa{\mu_{i_t,t+1}, \mu^{r}_{t}})+\!\sum_{x \in \Ormonetit} \!\tau(x) \leq T - t \Bigg| \Ft\land i_t \in \armsbelow}.
 \label{eq:conclusion-general}
 \end{multline}

Now, we look at the special case where $\Tpi_t$ selects an arm in $\Krt$. This is for instance the case of $\oracle$. We can rewrite Equation~\ref{eq:conclusion-general}, 
 \begin{align*}
 &\PP{r_{t:T}(\Tpi_t) \geq r| \Ft\land i_t \in \Krt} \\
 &\qquad \qquad= \PP{\tau(\mu_{i_t,t+1})+\!\sum_{x \in \Ormonetit} \!\tau(x) \leq T - t \Bigg| \Ft\land i_t \in \Krt}\\
 &\qquad \qquad= \PP{\tau(\mu_{i_t,t})+\!\sum_{x \in \Ormonetit} \!\tau(x) \leq T - t +1 \Bigg| \Ft\land i_t\in \Krt}\\
 &\qquad \qquad= \PP{\!\sum_{x \in \Ort} \!\tau(x) \leq T - t +1 \Bigg| \Ft}.
 \end{align*}
 The first equation follows from Equation~\ref{eq:conclusion-general} with $i_t \in \Krt \implies \mu_{i_t,t+1} \geq \mu_{i_t,t} > \mu^r_{t}$. The second equation follows by Lemma~\ref{lem:tau+1}. Indeed, $T-(t+1)- \!\sum_{x \in \Ormonetit} \!\tau(x)$ is a $\pa{\Ft \land i_t}$-measurable random variable. The last equation is justified by $\mu_{i_t,t} \in \Ort \implies \Ort = \Ormonetit \cup \ev{\mu_{i_t,t}}$. Last, we notice that $\sum_{x \in \Ort} \tau(x)$ is $\Ft$-measurable and does not depend on which $i_t\in \Krt$, thus we drop the $i_t$ dependency.
\end{proof}



\begin{corollary}
\label{cor:f-non-decreasing}
$\PP{r_{t:T}(\oracle) \geq r| \Ft} = f(\Ort)$, with \\$f(\bx) \triangleq \PP{\!\sum_{j=1}^r \!\tau(x_j) \leq T-t+1}$ a non-decreasing function of its $r$ variables. 
\end{corollary}
\begin{proof}
According to Lemma~\ref{lem:r-tau},
  \begin{align*}
 \PP{r_{t:T}(\oracle) \geq r| \Ft} &= \PP{\!\sum_{x \in \Ort} \!\tau(x) \leq T - t + 1 \Bigg| \Ft}\\
 &= \PP{\!\sum_{x \in \Ort} \!\tau(x) \leq T - t + 1 \Bigg| \Ort}.
 \end{align*}
Indeed $\ind{\sum_{x \in \Ort} \!\tau(x) \leq T - t + 1}$ is independent of $\Ft$ given $\Ort$. Hence,  $\PP{r_{t:T}(\oracle) \geq r| \Ft} $ is a function of the $r$ largest states below $\mu$. We study the CDF of the random variable $\sum_{x \in \Ort} \!\tau(x)$ given $\Ort$ that is, 
   \begin{align*}
 f_m(x_1, \dots, x_r) &= \PP{\!\sum_{j=1}^r \!\tau(x_j) \leq m}.
 \end{align*}
 
Notice that $\PP{r_{t:T}(\oracle) \geq r| \Ft} = f_{T-t+1}(\Ort)$. We want to show that $f_m$ is non-decreasing with each variable. Let's consider the $i$-th variable. We use that the probability of the sum of independent variables is the convolution of probabilities,
  \[
  f_m(x_1, \dots, x_r)  = \sum_{k=0}^m \PP{\!\sum_{j\neq i}^r \!\tau(x_j) =k  } * \PP{\tau(x_i) \leq m-k}.
  \]
Hence, $f_m$ is the sum of non-decreasing functions of $x_i$. Hence, $\PP{r_{t:T}(\oracle) \geq r| \Ft}$ is non decreasing with respect to any variable $\mu_t^i \in \Ort$ (the others being fixed).
\end{proof}
\begin{lemma}
\label{lem:stoch-dom-monotone}
Let f a non-decreasing function. Let $X$ and $Y$ two random variables with probability densities $\ev{p_x, p_y}$ such that $p_x \succeq p_y$. Then, 
\[
\EE{f(X)} \geq \EE{f(Y)}.
\]
\end{lemma}
\begin{proof}
This is a standard result for stochastic dominance that we show for completion. The key argument is that stochastic dominance implies a monotone coupling between the two distributions. Indeed, let 
$X(z)= F_{p_x}^{-1}(z)$ and $Y(z)= F_{p_y}^{-1}(z)$ with $z$ a random variable uniformly drawn in $\bra{0,1}$. We do have that $X$ and $Y$ are drawn with respective probability $p_x$ and $p_y$. Moreover, since $F_{p_x}\leq F_{p_y}$ (stochastic dominance), we can write that $X(z)\geq Y(z)$. 

Now we consider the expectation of a non decreasing function $f$, 
\[ \EE{f(X) -f(Y)} = \int_0^1 f(X(z)) - f(Y(z)) dz \geq 0.\]
\end{proof}

\section{What does random progression mean? }
One of the main differences with rotting bandits is that evolution is not deterministic anymore. From the formal point of view, it is an extension of the setup, as deterministic evolution is a special case of stochastic evolution. Yet, it is not clear what is the meaning of a random progression of the student. In this section, we give different interpretations of the transition operator $\Tp$ and we discuss the possibility to measure it and test our different assumptions. 

Uncertainty is often modeled with classical tools from the probability theory. As noticed by \citet{lavenant2019how}, this theory does not provide a meaning to what probabilities mean. In fact, it does not even provide a procedure to assign probabilities in practice. It is merely a theory of how we can compute together probabilities to determine other probabilities. \citet{lavenant2019how} review three ways to assign a probability: the classical one, the frequentist one, and the subjectivist one. 

The classical conception uses the indifference principle, which assumes that there exist some base events - the issues - which are equiprobable, and hence, one should count the number of issues that realize an event and divide by the total number of issues to get its probability. A classical example is the throw of a dice where we assume that each outcome has a probability $\nicefrac{1}{6}$. Notice that the characterization of what are the equiprobable issues does \emph{not} come from the probability theory. For instance, we can assume that the 11 outcomes of the sum of two dices are equiprobable and accurately use the probability theory. Yet, such theory will lead to wrong predictions when we compare to what happens in the real world \footnote{It is not clear what is a "wrong prediction" in an uncertain world. Indeed, when we try to relate probabilities to facts, the probability theory always says that facts are possible. Hence, to make a probabilistic theory testable (or refutable), it is philosophically necessary to interpret very likely / unlikely events as certain / uncertain. This is what Emile Borel calls the "loi unique du hasard" (unique law of chance).}. The fact that the correct equiprobable issues for two dices are the product ensemble of the ensembles of issues of each dice comes from external considerations: the symmetry in the geometry of one dice, the chaotic movement of rolling dices which "compensates" the fact that the dices are thrown together, etc. Can we use this classical conception to assign our probabilities $\Tp$? The example of the sum of two dices tells us that it is not because we don't know that we should assume uniform probability. The power of the classical method comes from the potential power of the indifference principle for the specific setup. It is not because physicists have no idea about how atoms in a gas are dispatched that they can make accurate predictions. It is in fact because they have a very accurate idea - all the micro configurations of atoms in a gas are equally likely -  that statistical physics can make accurate predictions. 

The frequentist conception - arguably the most well known in the bandits' community - defines the probability empirically as the limit of the observed frequency of the outcome of a given protocol. Notice that this is a definition, not a Theorem (we refer to the discussion about the status of the law of large numbers with respect to the frequentist interpretation by \citet{lavenant2019how}). It is only the repetition of the protocol which gives a sense - and value - to a probability. 

In our context, we do have a protocol: each incoming student on the website is a new realization of the protocol. In the frequentist interpretation, the transition probabilities $\Tp(x,x')$ should be interpreted as the fraction of students that reach level $x'$ after one question on the topic where there were at level $x$. Hence, maximizing our objectives in \emph{expectation} means that we maximize the objective on average across the population of students.

The frequentist interpretation is often believed to be the most scientific due to the elegant way it arranges facts, experimental setup, and theory. However, relating probabilistic models to objective reality is not always straightforward. Assuming that the states $x$ and $x'$ is observable (they are not) and that we do have many students at each level $x$ (we don't, since there is an infinite number of $x$), we would be able to estimate $\Tp(x,x')$ by simply measuring the fraction of incoming students. If we want to actually test our assumptions on $\Tp$ for the frequentist interpretation, one should be able to evaluate the transition model under partial observability and continuous state space. This is not a straightforward operation \citep{shani2005model}. In fact, before the listed Assumptions~\ref{assum:rested} to~\ref{assum:stochastic-monotone}, we assume that the future is independent of the past given the present (the Markov property). While this assumption is popular, it is rarely tested \citep{bickenbach2001markov}, and, again, partial observability and continuous state space make tests more challenging.

There also exist subjective interpretations of probabilities. For instance, \citet{lavenant2019how} advertise the interpretation of \citet{definetti1972probability}: probabilities are the amount an individual is ready to bet on an event if they are rewarded by one if the event occurs. This interpretation is \emph{antirealistic} as a probability does not have to match the facts in any way: there is no need for the gambler to be good. The interest of this interpretation is that we can recover the rules of probability calculus by assuming rational gambler. For instance, the fact that probabilities are normalized to one is a consequence that no one wants to accept a bet where he or she is sure to lose (for a well-chosen weighting scheme on the different bets). In this interpretation, probability theory is a way to enforce coherence in one's system of belief. For instance, our Theorem~\ref{th:FLUT-opt} states that if the bets we are ready to accept on the student progression satisfies our different assumptions then we should accept a better odds for the bet "policy $\oracle$ will achieve at least $r$ rewards" than for "any other policy $\Tpi$ will achieve at least $r$ rewards". 
\section{Learning Perspectives}
\subsection{Regret}
Like in the previous chapters, we define the cumulative regret with respect to the optimal policy, 
\[
R^c_T(\pi) = \EE{J_T(\oracle)} - J_T(\pi).
\]
We also define the simple regret, 
\[
R^s_T(\pi) = \EE{r_T(\oracle)} - r_T(\pi).
\]
\subsection{Counter-examples and a new learning assumption}
\subsubsection{Counter-example 1: Stagnating arms near the threshold}
We consider a two-arm game with deterministic transitions: each pull add a quantity $\epsilon>0$ to the arm's state, 
\[
\forall i \in \arms,\  \forall n \in \ev{1, \dots, T},\ \mu_i(n+1) = \mu_i(n) + \epsilon.
\]
It verifies all our Assumptions~\ref{assum:rested} to~\ref{assum:stochastic-monotone}. We consider two sets of initial conditions, 
\begin{align}
&\mu_1^1(0) = \mu, \qquad &\mu_2^1(0) = \mu - \frac{2T\epsilon}{3} \CommaBin\label{eq:pb1}\\
&\mu_1^2(0) = \mu - \frac{(T+1)\epsilon}{3}\CommaBin \qquad &\mu_2^2(0) = \mu - \frac{2T\epsilon}{3}\cdot\label{eq:pb2}
\end{align}
On the problem 1 (Eq.~\ref{eq:pb1}), $\oracle$ pulls arm 2 $\ceil{\nicefrac{2T}{3}}$ times. Then, all the arms are above the threshold and $\oracle$ plays randomly. Hence, $r_T(\oracle)=2$ and $J_T(\oracle) = T +  \floor{\nicefrac{T}{3}}$. On the problem 2 (Eq.~\ref{eq:pb2}), $\oracle$ pulls arm 1 $\ceil{\nicefrac{T+1}{3}}$ times and then arm $2$ until the end of the game. Hence, $r_T(\oracle)=1$ and $J_T(\oracle) = T -  \floor{\nicefrac{T+1}{3}}$.

We consider the case $\epsilon \rightarrow 0$.  If $N_{1,T} \geq \ceil{\nicefrac{T+1}{3}}$ on problem 1, $R^c_T(\pi)= \floor{\nicefrac{T}{3}}$ and $R^s_T(\pi)= 1$. Moreover, if $N_{1,T} < \ceil{\nicefrac{T+1}{3}}$ on problem 2, $R^c_T(\pi)\geq \ceil{\nicefrac{2T}{3}} - \ceil{\nicefrac{T+1}{3}}$ ($R^s_T(\pi)$ can take the value $0$ or $1$). For $\epsilon$ small enough, the two problems cannot be distinguished in the presence of noise ($\sigma >0$) and hence, any algorithm would suffer linear regret in the worst-case.

\subsubsection{Counter-example 2: Infinitely close arms with diverging behaviors}
We consider a two-arm game with deterministic transitions:
\[
\forall i \in \arms,\  \forall n \in \ev{1, \dots, T},\ \mu_i(n+1) = f(\mu_i(n)) \text{ with } f(x) = \begin{cases}
0 &\text{if } x = 0\\
x +\epsilon &\text{if } x \leq \frac{3T\epsilon}{4} \\
\mu &\text{otherwise}
\end{cases}\cdot
\]
We consider the following initial states,
\[
\mu_1(0) = 0, \qquad \mu_2(0) = \epsilon.
\]
Hence, arm 1 is stationary and arm 2 will need $\ceil{\frac{3T}{4}}-1$ pulls to reach the threshold. Hence, $J_T(\oracle) \sim \nicefrac{T}{4}$ and $r_T(\oracle) =1$. 

For $\epsilon$ small enough, the two arms cannot be distinguished with reasonable confidence before arm 2 reaches the threshold. Since we cannot pull both arms $\sim \nicefrac{3T}{4}$, we cannot do much better than betting on one of the arms, and suffering $R_T(\pi) \sim \nicefrac{T}{4}$ in half of the cases.


\subsubsection{A new assumption}
The two counter-examples show that stagnation is a problem for learning. We make a new assumption to limit this kind of behavior,

\begin{assumption}
\label{assum:min-increase}
Let $\epsilon>0$. Let $y \leq x +\epsilon$. Then, $\Tp(x,y) =0$.
\end{assumption}

Notice that this assumption is quite strong as it assumes that the selected state is always progressing by at least $\epsilon$. Instead, we could assume that the state progresses by at least $\epsilon$ in expectation. 

We hope to derive an $\epsilon$-dependent lower bound by adapting the previous counter-examples. However, we can already state that small values of $\epsilon$ correspond to the hardest cases. For $\epsilon \sim T^{\nicefrac{-3}{2}}$, the worst-case regret is linear. 

\subsection{Focus on the Largest Under the Threshold with Exploration ({\FLUTE})}
\subsubsection{Upper and lower confidence bounds on an increasing sequence }
In Subsection~\ref{ss:rawucb}, we use the fact that the rewards were decreasing in rotting bandits to compute an upper-confidence bound on the value of the next pull. Following the same idea, we use the increasing Assumption~\ref{assum:increase} to derive a lower-confidence bound on the value of the last pull at a round $t$, and an upper-confidence bound on the value of the first pull of each arm.
\paragraph{Estimators}
As in Subsection~\ref{ss:SWA}, we define the average of the last $h$ observations of arm $i$ at time $t$ for learning policy $\pi$ as
\begin{equation*}
\widehat{\mu}_i^h(t,\pi) \triangleq \frac{1}{h}\sum_{s=1}^{t-1} \mathbbm{1}\pa{\pi\pa{s}\! =\! i \land N_{i,s}\!>\! N_{i,t-1}\! -\! h } o_{s},
\end{equation*}
and the average of the associated means as
\begin{equation*}
\bar{\mu}_i^h(t,\pi) \!\triangleq\! \frac{1}{h}\sum_{s=1}^{t-1} \mathbbm{1}\pa{\pi\pa{s}\! =\! i \land N_{i,\,s}\!>\! N_{i,\,t-1}\! -\! h } \mu_{i}(N_{i,s-1})\,.
\end{equation*}
 
Similarly, we also define the average of the first $h$ observations of arm $i$ at time $t$ for policy $\pi$ as,
\begin{equation*}
\widehat{\mu}_i^{1:h}(t,\pi) \triangleq \frac{1}{h}\sum_{s=1}^{t-1} \mathbbm{1}\pa{\pi\pa{s}\! =\! i \land N_{i,s} \!\leq\! h } o_{s},
\end{equation*}
and the average of the associated means as
\begin{equation*}
\bar{\mu}_i^{1:h}(t,\pi) \!\triangleq\! \frac{1}{h}\sum_{s=1}^{t-1} \mathbbm{1}\pa{\pi\pa{s}\! =\! i \land N_{i,s} \!\leq\! h} \mu_{i}(N_{i,s-1})\,.
\end{equation*}


We recall that $c(h,\delta) = \sqrt{2\sigma^2\log\pa{2/\delta}/h}$. We define the $\ucb$ and $\lcb$ statistics,
\begin{align}
&\ucb(i,\delta) = \min_{h\leq N_{i,t-1}} \widehat{\mu}_i^{1:h}(t,\pi) +c (h, \delta),\nonumber\\
&\lcb(i,\delta) = \max_{h\leq N_{i,t-1}} \widehat{\mu}_i^{h}(t,\pi) - c (h, \delta).\label{eq:lcbucb}
\end{align}

\subsubsection{{\FLUTE} algorithm}
We present the Focus on the Largest Under the Threshold with Exploration (\FLUTE) in Algorithm~\ref{alg:flute}. During each phase $p$, \FLUTE (1) explore to find a good (hopefully, the best) arm below the threshold; and (2) focus on this arm until we are sure enough that its value is above the threshold. 

At Line~\ref{algline:flute-armsbelow}, the algorithm estimates $\armsbelow$. More precisely, it returns all the arms whose $\lcb$ on their last value is above $\mu$. It corresponds to the arms for which we are not sufficiently sure that they are in $\armsabove$. Since we want to stop pulling the arms in $\armsabove$, it is important to be confident that they indeed reach the threshold. That is why we discard $i_p$ if it is not in $\hat{\armsbelow}$ (Line~\ref{algline:flute-phase-cond}). We also increase the phase counter $p$ by the number of arms which are detected above the threshold between $t-1$ and $t$ (Line~\ref{algline:flute-phase-increase}).

At Line~\ref{algline:flute-ip-select}, we select (if it exists) $i_p$, an arm whose $\lcb$ on the last value is at most at a distance $\Delta$ from the best $\ucb$  among arms in $\hat{\armsbelow}$. By doing so, we guarantee that the selected arm is not too far from the best arm under the threshold selected by \FLUT.

When this arm does not exist, we continue our round-robin exploration (Line~\ref{algline:flute-pull-it}). In practice (or maybe in theory), it may be interesting to filter out the arms for which we are sure that they are not among the best ones. For instance, we suggest restricting the round-robin exploration to the arms in $\ev{i \in \hat{\armsbelow} | \ucb(i,\delta_T) \geq \max_{j \in \hat{\armsbelow}} \lcb(j, \delta_T )}$. Notice that when there are no arms in this set, then there is at least one candidate for arm $i_p$ (Line~\ref{algline:flute-ip-select}).

\begin{figure*}[ht]
\begin{minipage}{\textwidth}
\renewcommand*\footnoterule{}
\begin{savenotes}
\begin{algorithm}[H]
\caption{Focus on the Largest Under the Threshold with Exploration (\FLUTE)}
\label{alg:flute}
\begin{algorithmic}[1]
\Require $\mu$, $\Delta$, $\delta_T$
\State $p \leftarrow 1$
\State $i_p \leftarrow \Null$ 
\State $\hat{\arms_{0}^-} \leftarrow \arms$
\For{$t \gets 1, 2, \dots, K \do $}{\footnotesize \Comment{\emph{Pull each arm once}}}
	\State \textsc{Pull}  $i_t \gets t$; \textsc{Receive} $o_{t}$
\EndFor
\For{$t \gets K+1, K+2, \dots \do $}
		\State \textsc{Compute} $\ev{\lcb(i,\delta_T), \ucb(i,\delta_T)}_{i,\in \arms}$ {\footnotesize \Comment{\emph{Equation~\ref{eq:lcbucb}}}}
		\State $\hat{\armsbelow} \leftarrow \left\{i \in \arms | \lcb(i,\delta_T) \leq \mu \right\}$\label{algline:flute-armsbelow}
		\State $p \leftarrow p + | \hat{\arms_{t-1}^-} \setminus \hat{\armsbelow}|$ 	\label{algline:flute-phase-increase}
		\If{$i_p$ is not \Null  {\bf{ and}} $i_p \notin \armsbelow$}\label{algline:flute-phase-cond}
		\State $i_p \leftarrow \Null$ 
	\EndIf
	\If{$i_p$ is \Null}
	\State $i_p \in \ev{i \in \hat{\armsbelow} | \max_{j \in \hat{\armsbelow}} \ucb(j,\delta_T) - \lcb(i,\delta_T) \leq \Delta}$\footnote{One can choose the tie break selection rule arbitrarily, e.g. by selecting the arm with the smallest index.};\label{algline:flute-ip-select}
	\EndIf
	\If{$i_p$ is \Null}
	\State $i_t \in \argmin_{i \in \hat{\armsbelow}} N_{i,t}$\label{algline:flute-pull-it}
	\Else 
	\State $i_t \leftarrow i_p$ \label{algline:flute-pull-ip}
	\EndIf
	\State \textsc{Pull} $i_t $ \textsc{Receive} $o_{t}$
\EndFor
\end{algorithmic}
\end{algorithm}
\end{savenotes}
\end{minipage}
\end{figure*}



\subsection{Regret upper bound perspectives}
For simplicity, we consider the case where all the arms are below the threshold at the beginning. Without loss of generality, we assume that arms are ordered by their starting value. 

Like in the previous chapters, we can design a high probability event such that our estimators are well concentrated. On this high probability event, we know that arms which are not in $\hat{\armsbelow}$ are above the threshold. It can be interesting to upper-bound the expected duration of the phases of \FLUTE compared to the ones of \FLUT. The phase $p$ of \FLUT is simply the number of rounds it takes to reach the threshold from $\mu_{p}(0)$. There are three sources of overhead for the duration of the phase $p$ of \FLUTE. 

First, \FLUTE spends pulls in its exploration phase. The exploration stops when an arm $i_p$ is found. Even in the case where the arms are near-stationary, the condition at Line~\ref{algline:flute-ip-select} will be fulfilled when the confidence bound $c(N_{expl},\delta_T)$ becomes comparable with $\Delta$. (We conjecture : $4c(N_{expl}, \delta_T) \leq \Delta$). It gives an upper-bound on the number of exploration pulls $N_{expl}$ and finally on the delay $KN_{expl}$.

Second, the arm $i_p$ which is selected is not the best below the threshold as in \FLUT. Yet, we conjecture that $\mu_{i_p}(\Nit) \geq \mu_{p}(0) -\Delta$. With the Assumption~\ref{assum:min-increase}, an imprecision of size $\Delta$ costs $\nicefrac{\Delta}{\epsilon}$ in number of pulls. Notice that $\Delta$ is a parameter of our algorithm, and we should tune its value to balance the two aforementioned costs.

Third, in the learning setup, there is a delay to detect when an arm is above the threshold. Thanks to our Assumption~\ref{assum:min-increase}, the state cannot stay near $\mu$ for too long. After $N_{detect}$ pulls, we conjecture that
$\mu +\nicefrac{N_{detect}\epsilon}{2} - 2c(N_{detect}, \delta_T) \leq \lcb(i, \delta_T)$ due to this minimal increase. Hence, the condition at Line~\ref{algline:flute-phase-cond} will be necessary fulfilled when $N_{detect}\epsilon$ has the same order of magnitude than $c(N_{detect}, \delta_T)$.

We are still quite far to get an upper bound on $R^{c}_T$ (or $R^{s}_T$). Indeed, we need to characterize precisely how these overheads add together when we evaluate the total reward at round $T$.

\section{Practical considerations for ITS applications}
The fact that \FLUTE focuses on a given topic after the exploration phase is an interesting feature for ITS applications. Yet, the initial exploration phase may be very long if we try to learn from scratch for each student. 

\subsection{Including prior knowledge}
If one topic is often easier than the others for students, we would like to use this prior information to speed up the exploration. Using Bayesian statistics instead of frequentist tools is a natural way to work with prior information (see Subsection~\ref{ss:bayes} for the stationary bandits case).

How can we learn the prior? Knowledge Tracing \citep{desmarais2012review} is the application of (often online) supervised learning to the prediction of the student's answer given the question and past interactions. \citet{wilson2016back}  design shallow models which outperform deep networks \citep{piech2015deep, khajah2016how, xiong2016going} in their experiments. These models use some variations of a classical student model - the item response theory - with a Bayesian learning method. In its simplest form, Item Response Theory \citep{hambleton2013item} associates to each student a proficiency $\theta_s$ and to each exercise a difficulty $d_i$ such that the difference $\theta_s - d_i$ is fed in a logistic model to output the probability of success. \citet{wilson2016back} add a hierarchical Bayesian structure: each item difficulty have a prior which depends on the topic difficulty which is a parameter drawn from an uninformative Gaussian prior.

We could replace the frequentist confidence levels in \FLUTE by Bayesian credible intervals on the parameters of a similar model. Indeed, we can estimate credible intervals with MCMC sampling, which is often used in Bayesian learning \citep{andrieu2003introduction}. This approach would incorporate prior knowledge (learned from the other students' data) and enable shared knowledge between arms.


\subsection{The exercises population is finite}
On Afterclasse, there are roughly 20 questions per couple topic-difficulty. Notice that our confidence band is quite large for this number of samples: $c(h=20,\delta_t = 10\%)\sim 0.16$. Hence, even if a student answers the 20 questions correctly, its $\lcb$ on the topic will be smaller than $0.85$. It is a problem if the targeted $\mu$ is above 0.85. We suggest using the ratio of answered questions as a multiplicative factor in front of the confidence band in the $\lcb$ / $\ucb$ definition (Equation~\ref{eq:lcbucb}). Hence, when a window $h$ includes all the questions, the associated confidence level becomes the empirical average (no uncertainty). 

\subsection{Tuning $\Delta$ with $\epsilon$}
We have already noticed that $\Delta$ should be tuned theoretically according to $\epsilon$: the smaller the $\epsilon$, the more accurate we need to be in the exploration phase. However, in practice, we do not know $\epsilon$. We can estimate from data the average progression of students per question, but (1) it is not the minimum progression, and (2) it is not student-specific. 

We suggest to overestimate $\epsilon$ (and, hence, $\Delta$) at the beginning of the game. It would reduce the exploration phase and with the usage of prior information, it is even possible that the algorithm directly starts to exploit an arm. We can decrease the value of $\Delta$ if the student starts to wheel-spin, that is when a phase lasts for too long. We believe that for an ITS application it is indeed better to take a guess and start focusing on a topic, and explore only if the student shows unexpected difficulties.

\subsection{Managing difficulty with a Zone of Proximal Development}
In Section~\ref{sec:beyond}, we mention that the Rotting Bandits framework can hardly take different difficulties into account. In the current framework, we can use \FLUTE together with the Zone of Proximal Development paradigm \citep{luckin2001designing, clement2015multi}. The arms are the different topic-difficulty pairs, but we locked the advanced difficulties at the beginning. We unlock them when the student validates the easier difficulty associated with that topic, that is when the easier arm is not in $\hat{\armsbelow}$.

\chapterimage{chapter_head/7_782phare.jpg} 
\part{References}
\chapter*{References}
\vspace{-3cm}
\emph{– Tu n’as même pas appris le métier de tailleur ? dit-elle.\\
– Jamais, répondit K.\\
– Quelle est ta profession ?\\
– Arpenteur.\\
– Qu’est-ce là ? \\
K. le lui expliqua, l’explication la fit bâiller. } \\ \vspace{-1.2cm}
\begin{flushright}
\emph{Franz Kafka,} Le Château, \emph{Dernier Chapitre}.
\end{flushright}
\markboth{References}{}
\addcontentsline{toc}{chapter}{\textcolor{myBlue}{References}}
\begin{refcontext}[sorting=nyt]
\printbibliography[heading=bibempty]
\end{refcontext}



\end{document} 
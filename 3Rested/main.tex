%!TEX root = ../main.tex
\part{Rotting bandits}
\section*{Decreasing reward}
In Subsection~\ref{ss:less-known}, we presented a line of work that aims at asking questions from the least known subject to a student. In the multi-armed bandits' formulation, it associates positive reward to failed questions. Yet, none of these works consider the impact of the questions on the knowledge of the student. When the answer is given to the student after her trial, questions are a powerful learning tool. Therefore, the more the student work on a topic, the better he becomes, and the smaller is the reward for this topic.

Other situations can be modeled with decreasing rewards caused by the repetition of an action. For instance, the more we recommend an item to a user in a recommender system, the more he might get bored \citep{warlop2018fighting}. In medicine, the efficiency of antibiotics is diminishing with the overall use due to bacteria's mutation \citep{ventola2015antibiotic, ventola2015antibiotic2}.

In microeconomics, the law of diminishing marginal utility states that the utility associated with each unit of goods is decreasing with the number of goods a consumer holds. It is an \emph{ad hoc} explanation to justify that rational consumers, who maximize their total utility, may select different goods. In production theory, the law of diminishing returns \citep{canan1892origin} states that the increment of production caused by the increment of a factor of production (labor, capital) by one unit is decreasing. Again, there is the idea that repeating always the same action - buying one good, investing in a project - may become suboptimal even though the returns were high at the beginning. 

Motivated by these broad applications, \citet{heidari2016tight, levine2017rotting} study this non-stationarity with bandits feedback.  \citet{heidari2016tight} study the noise-less case under the name \emph{decaying bandits}. \citet{levine2017rotting} study the problem with noise under the name \emph{rotting bandits}. In the following, we call this problem the \emph{rested rotting bandits} to emphasize that actions cause the rewards' decay. We also mention the works of \citet{warlop2018fighting, immorlica2018recharging, pikeburke2019recovering} which model boredom effects in recommender systems as a rested decaying bandit problem but with restless recharging effects. 

In Chapter~\ref{ch:rested}, we synthesized our contributions to the rested rotting bandits problem\citep{seznec2019rotting, seznec2020single}: we present new algorithms and we prove that for an unknown horizon $T$, and without any knowledge on the decreasing behavior of the $K$ arms, these algorithms achieve problem-dependent regret bound of $\tcO(\log{(T)}),$ and a problem-independent one of $\tcO(\sqrt{KT})$. Our result substantially improves over the algorithm of~\citet{levine2017rotting}, which suffers regret $\tcO(K^{1/3}T^{2/3})$. These bounds are at a polylog factor of the optimal bounds on the stationary problem; hence our conclusion: rotting bandits are not harder than stationary ones. 

Another decaying setup is when the reward decreases no matter what the agent is doing. It models different situations such as the aging of content in recommender systems. \citet{louedec2016algorithme} models obsolescence of appearing arms (e.g. piece of news) with a known exponential rate. \cite{komiyama2014time-decaying} study a parametric decay in restless bandits where rewards are linear combinations of known decaying functions. However, the rotting assumption was not studied in the well-studied non-parametric restless bandit setting\citep{garivier2011upper-confidence-bound, auer2019adaptively,chen2019new, cheung2019new, russac2019weighted, besson2019generalized, liu2018change-detection, cao2019nearly, besbes2014stochastic}. That is why we consider the  \emph{restless rotting bandits} problem in Chapter~\ref{ch:restless} which is adapted from \citet{seznec2020single}.  We show that the rotting algorithms designed for the rested case match the problem-independent lower bound and a problem-dependent $\cO(\log{T})$. The latter was shown to be unachievable in the general case where rewards can increase. We conclude: the rotting assumption makes the restless bandits easier. 

Since the same algorithms work in both setups, we investigate in Section~\ref{sec:general_decreasing_MAB_framework} the joint setup where the reward can decrease with the number of pulls and the rounds. Yet, we show that the optimal oracle policy cannot be approached at a nontrivial rate by a learning policy.

\chapterimage{chapter_head/4_94tag.jpg} 
\chapter{Rested rotting bandits are not harder than stationary ones}
\label{ch:rested}
\vspace{-3cm}
\begin{flushright}
\emph{This rested rotting bandit seems quite stationary to me.}
\end{flushright}
\vspace{.85cm}
%!TEX root = ../main.tex
\section{Rested rotting bandit : model and preliminaries}
\label{sec:Model}
\subsection{Model}
\subsubsection*{Feedback loop}
At each round $t$, an agent chooses an arm $i_t \in \possibleArms \triangleq \left\{ 1, ... , K\right\} $ and receives a noisy reward $o_t$. The reward associated to each arm $i$ is a $\subgaussian^2$-sub-Gaussian random variable with expected value of $\mu_i(n)$, which depends on the number of times $n$ it was pulled before; $\mu_i(0)$ is the initial expected value.
We use $\mu_i(n)$ for the expected value of arm~$i$ \textit{after $n$ pulls} instead of when it is pulled \textit{for the $n$-th time}. 
Let $\historyt \triangleq \left\{ \left\{ i_s, o_s \right\}, \forall s < t \right\}$ be the sequence of arms pulled and rewards observed until round $t$, then 
%
\begin{equation}
\label{eq:rested-feedback}
o_{t} \triangleq \mu_{i_t}(N_{i_t,t-1}) + \noise_t
 \;\; \text{with}\; \EE{ \noise_t | \historyt }= 0 \;\; \text{and} \; \forall \lambda \in \R, \; \EE{ e^{\lambda\noise_t}} \leq e^{\frac{\subgaussian\lambda^2}{2}},
\end{equation}
%
where $N_{i,t}\triangleq \sum_{s\!=\!1}^{t} \mathbb{I}\{i_s \!=\! i\}$ is the number of times arm $i$ is pulled after round $t$.
\begin{definition}\label{def:rew-bounded-decay} 
We introduce $\rewardSet$, the set of non-increasing reward functions with bounded decay $L$,
\[ 
\rewardSet \triangleq \left\{ \mu : \NN \rightarrow \left[- \infty,  L\right] \;\big{|}\; 0 \leq \mu(n) - \mu(n+1)  \leq L \text{ and } \mu(0) \in \left[0,  L\right] \right\}.
\]
\end{definition}
%
\begin{remark}
\label{rem:stationary-is-rotting}
We define the set of constant reward functions in $\left[0, L\right]$ : 
\[ 
\stationarySet \triangleq \left\{ \mu : \NN \rightarrow \left[0,  L\right] \;\big{|}\;  \mu(n) = \mu_i  \right\}.
\]
We have that $\stationarySet \subset \rewardSet$. Hence, we can conclude that the rotting bandits model includes all the stationary bandits problems.
\end{remark}
%
\subsubsection*{Online and offline objectives}
In this chapter, we will only consider deterministic agents which output an arm $i$ at each round $t$. They are degenerate cases of probabilistic agent, which outputs a probability distribution over arm at each round. For the sake of simplicity, we present only the deterministic formalism.   

We will distinguish two types of policies. On the one hand, an offline (or oracle) policy~$\pi \in \PiO$ is a function which maps the round $t$ and the set of reward functions $\mathbb{\mu} \triangleq \left\{ \mu_i \right\}_{i \in \possibleArms}$ to arms, i.e. $\pi(t, \mathbb{\mu}) \in \possibleArms$.  On the other hand, an online (or learning) policy~$\pi \in \PiL$ is a function from the history of observations at time $t$ (which includes the knowledge of the round $t$) to arms, i.e., $\pi(\mathcal{H}_t) \in \mathcal{K}$. For both types of policies, we often use the shorter notation $\pi(t)$, where the dependencies on $\mu$ or $\mathcal{H}_t$ is implicit. 

For a policy $\pi$, let $N_{i,t}^\pi \triangleq \sum_{s=1}^{t} \mathbb{I}\{\pi(s) = i\}$ be the number of pulls of arm $i$ at the end of round $t$. The performance of a policy $\pi$ is measured by the (expected) rewards accumulated over time, 
%
\begin{equation}
\label{eq:cumul-reward}
J_T(\pi) \triangleq \sum_{t=1}^T \mu_{\pi(t)}\pa{N_{\pi(t),t-1}} = \sum_{i \in \possibleArms} \sum_{n=0}^{N_{i,T}^\pi-1} \mu_i(n).
\end{equation}
%
\begin{remark}
\label{rem:pull-allocation}
The cumulative reward depends only on the number of pull of each arm at the horizon $T$: it does not depend on the specific pulling order of the arms. Hence, two distinct policies with the same pulling allocation at the horizon $T$, \emph{i.e.} $N_{i,T}^{\pi_1} = N_{i,T}^{\pi_2}$ for all $i$, have the same cumulative reward.
\end{remark}
%
We notice that $\pi \in \PiL$ depends on the (random) history observed over time, and $J_T(\pi)$ is also random for learning policies. The goal of the learning agent is to maximize the expected reward $\EE{J_T(\pi)}$.

On the contrary,  oracle policies do not depend on the (random) history. They can be computed entirely before the start of the game. Hence, finding $\pi^\star \in \argmax_{\pi \in \PiO} J_T(\pi)$ is called the \textit{offline problem}. For a given problem $\mathbbm{\mu}$, there is a finite number ($K^T$) of policies, hence the maximum always exists and it could be found by brute-force with infinite computational power.

We set a policy $\pi^\star\in \argmax_{\pi \in \PiO} J_T(\pi)$. Calling $J_T^\star = J_T(\pi^\star)$ the largest cumulative reward achievable, one can measure the regret of any policy (learning or oracle) compared to the optimal one, 
\begin{align}\label{eq:regret}
\regret(\pi) \triangleq J^\star - J_T(\pi).
\end{align}
%
Let $N_{i,T}^\star \triangleq N_{i,T}^{\pi^\star}$ be the number of times that arm~$i$ is pulled by the oracle policy $\pi^\star$ up to time~$T$ (excluded).  Using Equation~\ref{eq:cumul-reward},  we can conveniently rewrite the regret as
%
\begin{align}
\!\regret(\pi) &= \sum_{i\in\possibleArms}\left( \sum_{n=0}^{N_{i,T}^\star-1}  \reward_{i}(n)  - \sum_{n=0}^{N_{i,T}^\pi-1}  \mu_{i}(n) \right) \nonumber\\ 
& = \sum_{i \in \underpullSet}\sum_{n=N_{i, T}^{\pi}}^{N_{i, T}^{\star}-1} \mu_i(n) - \sum_{i \in \overpullSet} \sum_{n=N_{i, T}^{\star}}^{N_{i, T}^{\pi}-1} \mu_i(n),\label{eq:regret2}
\end{align}
%
where we define $\underpullSet \triangleq \left\{ \arm \in \possibleArms | N_{i, T}^{\star} > N_{i, T}^{\pi} \right\}$ and likewise $\overpullSet \triangleq \left\{ i\in \possibleArms | N_{i, T}^{\star} < N_{i, T}^{\pi}\right\}$ as the sets of arms that are respectively under-pulled and over-pulled by~$\pi$ with respect to the optimal policy.
%
\begin{remark}
The regret is measured against an optimal allocation over arms rather than a fixed-arm policy as it is a case in adversarial and stochastic bandits. Therefore, even the adversarial algorithms that one could think of applying in our setting (e.g., \EXP of \citet{auer2002finite}) are not known to provide any guarantee for our definition of regret. Moreover, for constant $\mu_i(n)$-s, our problem and definition of regret reduce to the one of stationary stochastic bandits. 
\end{remark}
%
We give an upperbound on the regret that holds for any policy and will be used in the analysis of all the presented learning policies. First, we upper-bound all the rewards in the first double sum - the underpulls - by their maximum $\reward^+_T(\pi) \triangleq \max_{i\in\possibleArms} \reward_i(N_{i,T}^\pi)$. Indeed, for any overpulls $\mu_i(n_i) $ (with  $n_i \geq N_{i,T}^\pi$), we have that
\[
\mu_i(n_i) \leq \mu_i(N_{i,T}^\pi) \leq \mu^+_T(\pi)  \triangleq \max_{i\in\possibleArms} \reward_i(N_{i,T}^\pi),
\]
where the first inequality follows by the non-increasing property of $\mu_i$s; and the second by the defintion of the maximum operator. Second, we notice that there are as many underpulls than overpulls (terms of the second double sum) because there both policies $ \pi$ and $\pi^\star$ pull $T$ arms. Notice that this does \emph{not} mean that for each arm $i$, the number of overpulls equals to the number of underpulls, which cannot happen anyway since an arm cannot be simultaneously underpulled and overpulled. Therefore, we keep only the second double sum,
\begin{equation}
\label{eq:regret-first-bound}
\regret(\pi) \leq \sum_{i\in \overpullSet}   \sum_{n=N_{i,T}^\star}^{N_{i,T}^\pi-1} \pa{\mu^+_T(\pi) - \mu_i(n)}.
\end{equation}
%
The \textit{online problem} is to find a learning policy which maximizes the expected cumulative reward (or equivalently minimizes the expected regret). In the next sections, we will present the main results of \citet{heidari2016tight}, which has solved the offline problem and the online problem in the absence of noise, and \citet{levine2017rotting}, which has presented the first learning policy with non trivial guarantees for rotting bandits with noise. 
%
\subsection{The offline problem \citep{heidari2016tight}}
We consider the greedy policy $\GO$ (Alg.~\ref{alg:greedy-oracle}) which at each round selects the arm with the best value.

\begin{minipage}{\textwidth}
\renewcommand*\footnoterule{}
\begin{savenotes}
\begin{algorithm}[H]
\caption{Greedy Oracle $\GO$ (or $\Azero$, \citet{heidari2016tight})}
\label{alg:greedy-oracle}
\begin{algorithmic}[1]
	\Require $\left\{\mu_i\right\}_{i \in \possibleArms}$
	\State Initialize $N_i \leftarrow 0$ for all $i \in \possibleArms$
	\For{$t \gets 1, 2, \dots \do $}
		\State \textsc{Pull}  $i_t \in \argmax_{i \in \possibleArms} \mu_i(N_{i})$\footnote{One can choose the tie break selection rule arbitrarily, e.g. by selecting the arm with the smallest index.}
		\State $N_{i_t} \leftarrow N_{i_t} + 1$
	\EndFor
\end{algorithmic}
\end{algorithm}
\end{savenotes}
\end{minipage}

\begin{proposition}[\citet{heidari2016tight}]
For any reward functions $\mu \in \rewardSet^K$ and any horizon $T$, $\GO \in \argmax_{\pi \in \PiO} J_T(\pi)$.
\end{proposition}%
\begin{proof}

At each round $t$, $\GO$ collects the largest reward that can be available in the future, \textit{i.e.} 
\[
\forall i \in \possibleArms, \ \forall n_i \geq \Nit^{\GO}, \ \mu_{\GO(t)}\pa{N_{\GO(t),\,t}^{\GO}} \geq\mu_{i}\pa{\Nit^{\GO}}  \geq \mu_i(n_i).
\]

The first inequality is due to the selection rule of the policy; the second is due to the decreasing reward functions. 

A direct consequence is that, at round $T$, $\GO$ has selected the $T$ largest reward sample among the $KT$ possible ones. Therefore, any other policy which would select other reward samples can only have worse or equal cumulative reward. 
\end{proof}

According to Remark~\ref{rem:pull-allocation}, for a given horizon $T$, all the policies with the same number of pulls of each arm than $\GO$ at round $T$ have the optimal cumulative reward. Yet, we show in the following Proposition that $\GO$ is the only \emph{anytime} optimal policy.\\

\begin{proposition}
Let $\pi$ such that $\pi(t) \notin \argmax_{i\in \possibleArms} \mu_i(\Nitpi)$.
\[\text{Then, } J_t(\pi) < \max_{\pi \in \PiO} J_t(\pi).\]
\end{proposition}
\begin{proof}
Let $i^\star_t \in \argmax_{i\in \possibleArms} \mu_i(N_{i,t}^\pi)$. We consider the policy $\pi^+$ which selects the same arm than $\pi$ during the $t-1$ first rounds and selects $i^\star_t$ at round $t$. Therefore, the two policies $\pi$ and $\pi^+$ collects the same rewards except the last one. Notice that before the last round $t$, the two policies have the same pulling allocation $N_{j,\,t-1}^\pi = N_{j,\,t-1}^{\pi^+}$ for all $j \in \possibleArms$.  Hence, there is only a difference between the two last reward samples,
\[ 
J_t(\pi^+) - J_t(\pi) =  \mu_{i^\star_t}(N_{i^\star_t,\,t-1}^{\pi^+}) - \mu_{\pi(t)}(N_{\pi(t),\,t-1}^{\pi}) = \mu_{i^\star_t}(N_{i^\star_t,\,t-1}^{\pi}) - \mu_{\pi(t)}(N_{\pi(t),\,t-1}^{\pi}) > 0.
\]
%
The inequality follows from $\pi(t) \notin \argmax_{i\in \possibleArms} \mu_i(N_{i,t}^\pi)$ and $i^\star_t \in \argmax_{i\in \possibleArms} \mu_i(N_{i,t}^\pi)$.
\end{proof}
\begin{remark}
%
\textbf{Complexity.} We have already highlighted that the offline problem is a computational problem. Indeed, the optimal solution can always be computed by brute force by iterating all the possible policies, i.e. with exponential time complexity per round $\cO(K^T)$. By contrast, $\GO$ can be computed with space complexity $\cO(K)$ and time complexity per round $\cO(\log{K})$. Indeed, at each round one should find the maximum among $K$ values. Yet, from one round to another, there is only one value which changes : the value of the last selected arm. Thus, one can store a sorted list of the $K$ arm's value and change one element at each round which costs $\cO(\log{K})$. Then, accessing the first element of the sorted list is a $\cO(1)$ operation.
\end{remark}
%
To conclude, $\GO$ solves the offline problem in the sense that it provides a cheap way to compute the optimal policy without any knowledge of the horizon $T$. Interestingly, $\GO$ takes the optimal decision by being greedy on the current values. It shows that there is no planning aspect in this problem : the learner never has to sacrifice rewards in the present to get more reward in the future.
%
\subsection{The noise-free online problem \citep{heidari2016tight}}
In the online problem, the learner does not have access to the current value of the arms. Can they track the best current value using only the observed past values ?  \citet{heidari2016tight} first studied the simpler noise-free problem ($\sigma =0$), where the learner observes the true value of an arm after selecting it (instead of a noisy sample). They suggested the greedy bandit $\GB$ (Alg.~\ref{alg:greedy-bandit}), a policy which selects greedily the arm with the largest last observed value. Indeed, instead of looking at the (unavaible) current values as $\GO$, $\GB$ looks at the closest past. 

\begin{minipage}{\textwidth}
\renewcommand*\footnoterule{}
\begin{savenotes}
\begin{algorithm}[H]
\caption{Greedy Bandit $\GB$ (or $\Atwo$, \citet{heidari2016tight})}
\label{alg:greedy-bandit}
\begin{algorithmic}[1]
\Require
\State Initialize $\hat{\mu}_{i}^1 \leftarrow + \infty$ for all $i \in \possibleArms$
	\For{$t \gets 1, 2, \dots \do $}
		\State \textsc{Pull} $i_t \in \argmax_{i \in \possibleArms} \hat{\mu}_{i}^1$\footnote{One can choose the tie break selection rule arbitrarily, e.g. by selecting the arm with the smallest index.}; \textsc{Receive} $o_{t}$
		\State $\hat{\mu}_{i_t}^1 \leftarrow o_{t}$
	\EndFor
\end{algorithmic}
\end{algorithm}
\end{savenotes}
\end{minipage}


\begin{proposition}[\citet{heidari2016tight}]
\label{prop:GB-ub}
For any problem $\mu \in \rewardSet^K$ and any horizon $T$, 
\[\regret (\GB) \leq (K-1)L. \]
\end{proposition}
Surprisingly, the worst case regret is upper-bounded by a constant with respect to $T$. %TODO EXAMPLE.
\begin{proof}
We start from Equation~\ref{eq:regret-first-bound} applied to policy $\GB$,
\begin{equation}
\label{eq:regret-first-bound-GB}
\regret(\GB) \leq \sum_{i\in \overpullSet}   \sum_{n=\NiT^\star}^{\NiT^{\GB}-1} \pa{\mu^+_T(\GB) - \mu_i(n)}.
\end{equation}
%
Let $i \in \arms$ an arm which is pulled at least twice at the end of the game $\NiT^{\GB} \geq 2$. We call $t_i \triangleq \min\left\{t\leq T\; |\; N_{i,t} = N_{i,T}\right\}$ the last round at which $i$ is pulled. For any arm  $j \in \arms$ pulled at least once at the end of the game $\NjT^{\GB} \geq 1$, and for all $n_i \leq \NiT^{\GB} -2$, 
\begin{equation}
\label{eq:overpull-GB1}
\mu_i(n_i) \geq \mu_i(\NiT^{\GB} - 2 ) = \mu_i(N_{i,\,t_i -1}^{\GB} - 1 ) \geq \mu_j(N_{j,\,t_i-1}^{\GB} - 1).
\end{equation}
The first inequality follows by the non-increasing hypothesis on the reward function. The equality follows by definition of $t_i$. The last inequality is by definition of the policy : at time $t_i$, $\GB$ selects $i \in \argmax_{j \in \arms} \mu_j(N_{j,\,t_i-1}^{\GB}-1)$, the largest last observed sample. 

We choose $j$ such that $ \mu_j(\NjT^{\GB}) = \mu^+_T(\GB) \pa{\triangleq \max_{j'\in \possibleArms} \mu_{j'}(N_{j',\,T}^{\GB}}$. \\Since $t_i \leq T$, $N_{j,\,t_i-1}^{\GB} - 1 < \NjT^{\GB}$. By the rotting assumption, 
\begin{equation}
\label{eq:overpull-GB2}
 \mu_j(N_{j,\,t_i-1}^{\GB} - 1) \geq \mu_j(\NjT^{\GB}) = \mu^+_T(\GB).
\end{equation}
%
Gathering Equations~\ref{eq:overpull-GB1} and~\ref{eq:overpull-GB2}, we have that 
\begin{equation}
\label{eq:overpull-GB3}
\forall n_i \leq \NiT^{\GB} \!-\! 2, \;\;  \mu(n) \geq \mu^+_T(\GB).
\end{equation}
Therefore,  we can upper-bound all the before last terms in each second sum in Equation~\ref{eq:regret-first-bound-GB} by zero. Hence, 
\begin{align*}
\regret(\GB) &\leq \sum_{i\in \overpullSet} \pa{\mu^+_T(\GB) - \mu_i(\NiT^{\GB}-1)}\\
&\leq \sum_{i\in \overpullSet} \pa{\mu^+_T(\GB) - \pa{\mu_i(\NiT^{\GB}-2) - L}}\\
&\leq |\overpullSet| L \\
&\leq \pa{K-1} L
\end{align*}
In the second inequality, we used $\mu_i \in \rewardSet$ (see Definition~\ref{def:rew-bounded-decay}). The third inequality follows from Equation~\ref{eq:overpull-GB3}. We can conclude by noticing that they are at most $K-1$ overpulled arm. Indeed, there are as many overpulls than underpulls since the two policies $\pi^\star$ and $\GB$ both pull $T-1$ sample. Hence, if there is at least one overpulled arm, there is necessary at least one underpulled arm. 
\end{proof}

In the next proposition, we state that this rate is minimax optimal at the first order in $\frac{K}{T}$.

\begin{proposition}[\citet{heidari2016tight}]
\label{prop:lb-noisefree}
For any policy $\pi\in \PiL$ and any horizon $T \geq K-1$, there exists a stationary problem $\mu \in \stationarySet \subset \rewardSet$ (see Remark~\ref{rem:stationary-is-rotting}) , 
\[\regret (\pi) \geq (K-1)L \pa{1-\frac{K-1}{T}}. \]
\end{proposition}
We highlight that our proposition is more precise than the one of \citet{heidari2016tight}. Indeed, while they show only a $\cO(K)$ worst case rate, we show that $\pi_G$ is minimax optimal up to a second order term in $\cO\pa{\frac{K}{T}}$. Moreover, we show that this lower bound holds for the easier stationary problem. Hence, it shows that, without noise, rotting bandits are not harder than stationary ones.

\begin{proof}
We consider a set of $K$ problems where 
\begin{itemize}
\item the first arm has always a constant value equals to $L\pa{1- \frac{K-1}{T}}$;
\item problem $p =1$ has all the other arms with a value $0$;
\item problem $p \in \left\lbrace 2, \dots, K \right\rbrace$ has arm $p$ with value $L$ and the other arms $i\in \possibleArms \smallsetminus \left\lbrace 1, i \right\rbrace$ with a value $0$.
\end{itemize}
The learner can distinguish between problem $p \in \left\lbrace 2, \dots, K \right\rbrace$ and problem $1$ only by pulling arm $p$ once. If the learner $\pi \in \PiL$ pulls every arm $i \in \left\lbrace 2, \dots, K \right\rbrace$ once, it suffers on problem $1$,
\[\regret^1\pa{\pi} \geq \pa{K-1}L\pa{1- \frac{K-1}{T}}.\]
If there exists an arm $i \in \left\lbrace 2, \dots, K \right\rbrace$ which is never pulled, $\pi$ suffers on problem $i$,
\[\regret\pa{\pi}^i \geq T \pa{L - L\pa{1- \frac{K-1}{T}}}= L\pa{K-1}.\]
Therefore, we have that for any $\pi$, there exists a stationary problem $\mu \in \stationarySet$ such that,
\[\regret\pa{\pi} \geq \pa{K-1}L\pa{1- \frac{K-1}{T}}\]
\end{proof} 
\begin{remark}
\citet{heidari2016tight} have also studied rested bandits with increasing and concave reward functions (without noise).The offline analysis shows that the optimal policy selects always the same arm. This is very different from the rotting case, where the optimal allocation may pull several arms. They suggest an online policy which plays Round-robin on an active set of arms. An arm is excluded from this active set if the optimistic projection of its total available reward until the end of the game (which can be computed thanks to the concavity assumption) is lower than the pessimistic projection of any other arm (i.e. the arm stays constant). They prove a $o\pa{T}$ regret bound (in the noise-less case !) for this algorithm. While they do not provide a lower bound, it suggests that this problem is harder than the rotting case, where the minimax rate is only in $\cO\pa{KL}$.
\end{remark}

\subsection{\citet{levine2017rotting} : {\wSWA}, a first policy for the noisy problem}
\subsubsection{Sliding-Window Average ({\SWA})}
When the feedback is noisy ($\sigma > 0$), selecting greedily on the last observed reward may be very risky. Indeed, a sample from an optimal pull could be underestimated by $\cO(\sigma)$. $\GB$ may not pull this good underestimated arm for a long time, because it only estimates the value of the arm with the last sample. This behavior may cause a regret of $\cO(\sigma T)$ which can be much larger than the noise-free rate $\cO(KL)$.

\citet{levine2017rotting} suggested to use the Sliding-Window Average (\SWA) policy, a policy which selects the arm with the largest average of its $h$ last sample. Averaging in the presence of noise is a straightforward idea. Yet, it is unclear how the learner should choose $h$. Before going through the detailed analysis, we give the high-level idea. First, we notice that when $h=1$, \SWA reduces to $\GB$. Indeed, intuitively, the smaller the noise, the less averaging we need. On the one hand, with a window $h$, the learner should expect to do $\cO(h)$ overpulls for an arm which abruptly decays at $N_{i,T}^\star$ with drop size $B$. Indeed, its estimator $\hat{\mu}_i^h$ will be positively bias during the next $h$ pulls. Hence, the learner may suffer up to $\cO(KBh)$ due to this bias. On the other hand, the learner takes slighlty wrong decisions due to the variance of their estimators $\cO(\frac{\sigma}{\sqrt{h}})$ which can cost up to $\tcO(\frac{\sigma T}{\sqrt{h}})$ on the long run. Choosing $h = \tcO\pa{\frac{ \sigma T}{KB}}^{2/3}$, we get the regret rate of $\tcO\pa{B^{1/3} \sigma^{2/3} K^{1/3} T^{2/3}}$. 

\begin{minipage}{\textwidth}
\renewcommand*\footnoterule{}
\begin{savenotes}
\begin{algorithm}[H]
\caption{\SWA \citep{levine2017rotting} }
\label{alg:SWA}
\begin{algorithmic}[1]
\Require $h$
\State Initialize $\hat{\mu}_{i}^h \leftarrow + \infty$ for all $i \in \possibleArms$
\State Initialize $\history(i) \leftarrow []$ for all $i \in \possibleArms$
	\For{$t \gets 1, 2, \dots, Kh \do $}
	 	\State \textsc{Pull Round-Robin}  $i_t \gets t \% h $; \textsc{Receive} $o_{t}$
	 	\State $\history(i_t) \leftarrow \history(i_t)\text{.append}(o_{t})$
	\EndFor
	\For{$t \gets Kh + 1, Kh + 2, \dots \do $}
		\State \textsc{Pull}  $i_t \in \argmax_{i \in \possibleArms} \hat{\mu}_{i}^h$\footnote{One can choose the tie break selection rule arbitrarily, e.g. by selecting the arm with the smallest index.}; \textsc{Receive} $o_{t}$
		\State $\history(i_t) \leftarrow \history(i_t)\text{.append}(o_{t})$
		\If{$\text{len}(\history(i_t)) \geq h$}
		\State $\hat{\mu}_{i_t}^h \leftarrow \textsc{Mean}(\history(i_t)[-h:])$
		\EndIf
	\EndFor
\end{algorithmic}
\end{algorithm}
\end{savenotes}
\end{minipage}
\begin{remark}
\SWA uses a rested sliding-window mechanism. Indeed, the window of arm $i$ slides only when arm $i$ is selected. Notice the difference with the restless sliding-window of \SWUCB \citep{garivier2011upper-confidence-bound}, which slides for all arms at every round.
\end{remark}
%
\subsubsection*{Analysis}

The analysis of \citet{levine2017rotting} uses the set of bounded decaying function instead of $\rewardSet$. 

\begin{definition}\label{def:rew-bounded} 
Let $\BBxSet$, the set of non-increasing reward functions with bounded amplitude $B$,
\[ 
\BBxSet \triangleq \left\{ \mu : \NN \rightarrow \left[x , x +B\right] \;\big{|}\; \mu(n) \geq \mu(n+1)  \right\}.
\]
The choice of origin $x$ is not important. Without loss of generality, we will carry the analysis on $\BBSet \triangleq \BSet_{B,0}$. 
\end{definition}
\begin{remark}
\label{rem:BBvsLL}
We have that $\BSet_L \subset \rewardSet$. Hence, any guarantee of any algorithm on $\rewardSet$ applies on $\BBSet$ by setting $L := B$. We also have that $\rewardSet \subset \BSet_{LT, -L\pa{T-1}}$. Hence, any guarantee of any algorithm on $\BBxSet$ applies on $\rewardSet$ by setting $B := LT$.
\end{remark}

\paragraph{Estimators}  
For policy $\pi$, we define the average of the last $h$ observations of arm $i$ at time $t$ as
\begin{equation}
\label{eq:def-hmu}
\widehat{\mu}_i^h(t,\pi) \triangleq \frac{1}{h}\sum_{s=1}^{t-1} \mathbbm{1}\pa{\pi\pa{s}\! =\! i \land N_{i,s}^\pi\!>\! N_{i,t-1}^\pi\! -\! h } o_{s}
\end{equation}
and the average of the associated means as
\begin{equation}
\label{eq:def-bmu}
\bar{\mu}_i^h(t,\pi) \!\triangleq\! \frac{1}{h}\sum_{s=1}^{t-1} \mathbbm{1}\pa{\pi\pa{s}\! =\! i \land N_{i,\,s}^\pi\!>\! N_{i,\,t-1}^\pi\! -\! h } \mu_{i}(N_{i,s-1}^\pi)\,.
\end{equation}
%
We notice that $\bar{\mu}_i^h(t,\pi) = \frac{1}{h}\sum_{h'=1}^{h} \mu_i(N_{i,\,t-1}^\pi-h') = \bar{\mu}_i^h(N_{i,\,t-1}^\pi)$ . With a slight abuse of notation, we will also use $\hat{\mu}_i^h(\Nit^\pi) \triangleq \hat{\mu}_i^h(t,\pi)$. Indeed, the average of the observations depends on the realization of the noise $\epsilon_t$ at time $t$. Yet, these $h$ samples of noise are i.i.d.\,and thus do not perturb the analysis. 

\paragraph{A favorable event}
\begin{proposition}
\label{prop:prb_favorable_event_SWA}
For a confidence level $\delta_{T} \triangleq T^{-2}$
, let
\begin{equation}
\!\HPSWA \! \triangleq\! \Big\{\forall i\!\in\!\mathcal{K},\ \forall n\in \left\{h, \dots, t-1\right\}, \ \big| \ \hmu_i^h(n) - \bmu_i^h(n) \big| \leq c(\window, \delta_{T}) \Big\}
\label{eq:def_favorable_event_SWA}
\end{equation}
be the event under which all the possible estimates constructed at round $t$ are all accurate up to $c(h,\delta_{T}) \triangleq \sqrt{2 \subgaussian^2\log(2/\delta_T)/h}$. Then, for a policy which pulls every arm $h$ times at the beginning (like \SWA),
\[
\PPempty\Big[\bar{\HPSWA}\Big] \leq\frac{K}{T}\,\cdot
\]
\end{proposition} 
%
\begin{proof}
We want to upper bound the probability
\[
\PP{\bar{\HPSWA}} = \PP{\exists i \!\in\! \arms,\,\exists n \!\in\! \left\{h, \dots, t\!-\!1\right\}, \big| \hmu_i^h(n) - \bmu_i^h(n)\big|>c(h,\delta_T) }.
\]
%
For $N_{i,\,t-1}^{\piSWA} = n$, we have that, 
\[
 \hmu_i^h(n) - \bmu_i^h(n)= \frac{1}{h} \sum_{s=1}^{T}\mathbbm{1}\pa{i_s = i \, |\, n -h < N_{i,s} \leq  n}\epsilon_s\,.
\]
By Doob's optional skipping (e.g. see \citet{chow1997probability}, Section 5.3) there exists a sequence of random independent variables $(\epsilon'_l)_{l\in\NN}$ , $\sigma^2$ sub-Gaussian such that 
\[\hmu_i^h(n) - \bmu_i^h(n)= \frac{1}{h} \sum_{s=1}^{T-1}\mathbbm{1}\pa{i_s = i \, |\, N_{i,s}> n - h }\epsilon_s=  \frac{1}{h} \sum_{l=n-h+1}^n \epsilon'_l \triangleq \hepsilon^h_n. \]
%
Hence, 
\begin{align*}
    &\PP{\exists i \in \arms,\,\exists n \in \left\{h, \dots, T-1\right\}, \big| \hmu_i^h(n) - \bmu_i^h(n)\big|>c(h,\delta_T) }\\
    &\qquad= \PP{\exists i \in \arms,\,\exists n \in \left\{h, \dots, T-1\right\},|\hepsilon^h_n|>c(h,\delta_T) }\\
    &\qquad\leq \sum_{i \in \arms} \sum_{n=h}^{T-1} \PP{|\hepsilon^h_n|>c(h,\delta_T)} \\
    &\qquad\leq  KT \delta_T  \\
    &\qquad \leq \frac{K}{T}\,,
\end{align*}
where we used the Chernoff inequality at the before last line and $\delta_{T} = T^{-2}$ at the last one. 
\end{proof}
%TODO correct this proof with \delta_t = 1/T^2
%
\paragraph{Regret upper-bound}
%
\begin{proposition}[\citet{levine2017rotting}]
\label{prop:SWA}
 For a problem $\Bmu \in \BBSet^K$, the expected regret of \SWA tuned with $h$ is bounded as
 \[
\EE{\regret(\piSWA)} \leq 4\sigma T\cdot\sqrt{\frac{\log\pa{\sqrt{2}T}}{h}} + K\pa{h+1}B
 \]
\end{proposition}
%
\begin{proof}
If $T \leq Kh$, we can bound the regret by the maximum regret ($T$ errors of magnitude $B$)
\[
\EE{\regret(\piSWA)} \leq TB \leq KhB \leq 4\sigma T\cdot\sqrt{\frac{\log\pa{\sqrt{2}T}}{h}} + K\pa{h+1}B.
\]
%
If $T > Kh$, we notice that any arm $i$ is pulled at least $h$ times, \ie $\NiT^{\piSWA} \geq h$. We split the regret on the events $\HPSWA$ and $\bar{\HPSWA}$, 
\[
\EE{\regret(\piSWA)} \leq \EE{\mathbbm{1} \Big[\HPSWA\Big] \regret(\piSWA)} + \EE{\mathbbm{1} \left[\bar{\HPSWA}\right] \regret(\piSWA)}.
\]
%
The regret on the unfavorable event $\mathbbm{1} \left[\bar{\HPSWA}\right]$ can be bounded by the maximal regret $BT$ (since $\mu \in \BBSet^K$), 
\[
\EE{\regret(\piSWA)} \leq  \EE{\mathbbm{1} \Big[\HPSWA\Big] \regret(\piSWA)} + \PP{\bar{\HPSWA}} BT.
\]
%
Using Proposition~\ref{prop:prb_favorable_event_SWA}, we get,
\begin{equation}
\label{eq:regret-unfav-event-SWA}
\EE{\regret(\piSWA)} \leq  \EE{\mathbbm{1} \Big[\HPSWA\Big]  \regret(\piSWA)} + KB.
\end{equation}
%
We will now bound the regret on the favorable event,
\[
\regret(\piSWA | \HPSWA) \triangleq \mathbbm{1} \Big[\HPSWA\Big]  \regret(\piSWA)
\]
%
We start from Equation~\ref{eq:regret-first-bound} applied to policy $\SWA$,
\begin{equation}
\label{eq:regret-first-bound-SWA}
\regret(\piSWA| \HPSWA) \leq  \mathbbm{1} \Big[\HPSWA\Big] \sum_{i\in \overpullSet}    \sum_{n=\NiT^\star}^{\NiT^{\piSWA}-1} \pa{\mu^+_T(\piSWA) - \mu_i(n)}.
\end{equation}
%
The remaining of the proof is similar to the proof of Proposition~\ref{prop:GB-ub} about algorithm $\GB$. Instead of showing that the before last terms in the sums are equals to zeros, we will show that the terms before the $h$ last one cost less than $2c(h, \delta_T)$. Let $i \in \arms$ an arm which is pulled at least $h+1$ times at the end of the game $\NiT^{\piSWA} \geq h+1$. We call $t_i \triangleq \min\left\{t\leq T\; |\; \Nit^{\piSWA} = \NiT^{\piSWA}\right\}$ the last round at which $i$ is pulled. Notice that $t_i > Kh$ because the $Kh$ first pulls corresponds to the round-robin period. Hence, for any arm $j \in \arms$, $N_{j,\,t_i -1}^{\piSWA} \geq h$. For all $n_i \leq \NiT^{\piSWA} -(h+1)$, 
\begin{align}
\label{eq:overpull-SWA1}
\mu_i(n_i) &\geq \mu_i(\NiT^{\piSWA} - (h+1) )\nonumber\\
 &\geq \bmu_i^h(N_{i,\,t_i -1}^{\piSWA}) \nonumber\\
 &\geq \hmu_i^h(N_{i,\,t_i -1}^{\piSWA}) - c(h, \delta_T) \nonumber\\
& \geq \hmu_j^h(N_{j,\,t_i -1}^{\piSWA}) - c(h, \delta_T)  \nonumber\\
& \geq \bmu_j^h(N_{j,\,t_i -1}^{\piSWA}) - 2c(h, \delta_T). 
\end{align}
%
The first inequality follows by the non-increasing hypothesis on the reward function. The second inequality is because $\bmu_i^h(N_{i,\,t_i -1}^{\piSWA})$ is the average of $h$ reward sample of arm $i$ after the $\NiT^{\piSWA} - (h+1)$-th (according to the definition of $t_i$). The third and fifth one use the concentration of all the constructed estimates on the event $\HPSWA$.  The fourth  inequality follows by definition of the policy : at time $t_i$, $\piSWA$ selects $i \in \argmax_{j \in \arms} \hmu_j^h(N_{j,\,t_i -1}^{\piSWA})$.

We choose $j$ such that $ \mu_j(\NjT^{\piSWA}) = \mu^+_T(\piSWA) \pa{\triangleq \max_{j'\in \possibleArms} \mu_{j'}(N_{j',\,T}^{\piSWA})}$. \\Since $t_i \leq T$, by the rotting assumption, 
\begin{equation}
\label{eq:overpull-SWA2}
 \bmu_j^h(N_{j,\,t_i -1}^{\piSWA}) \geq \mu_j(\NjT^{\piSWA}) = \mu^+_T(\piSWA).
\end{equation}
%
Gathering Equations~\ref{eq:overpull-SWA1} and~\ref{eq:overpull-SWA2}, we have that 
\begin{equation}
\label{eq:overpull-SWA3}
\forall n_i \leq \NiT^{\piSWA} \!-\! \pa{h+1}, \;\;   \pa{\mu^+_T(\piSWA) - \mu_i(n_i)} \leq 2c(h, \delta_T).
\end{equation}
Therefore,  in Equation~\ref{eq:regret-first-bound-SWA}, we can split the sum on $\NiT^{\piSWA} \!-\! h$. Hence, 
\begin{align}
\regret(\piSWA| \HPSWA) \leq&  \mathbbm{1} \Big[\HPSWA\Big] \sum_{i\in \overpullSet}    \sum_{n=\NiT^\star}^{\NiT^{\piSWA}-1} \pa{\mu^+_T(\piSWA) - \mu_i(n)}\nonumber\\
=&  \mathbbm{1} \Big[\HPSWA\Big] \sum_{i\in \overpullSet}    \sum_{n=\NiT^\star}^{\NiT^{\piSWA}-\pa{h+1}} \pa{\mu^+_T(\piSWA) - \mu_i(n)} \nonumber\\ 
&+ \mathbbm{1} \Big[\HPSWA\Big] \sum_{i\in \overpullSet} \sum_{n=\NiT^{\piSWA} - h }^{\NiT^{\piSWA}-1} \pa{\mu^+_T(\piSWA) - \mu_i(n)}\nonumber\\
\leq& 2Tc\pa{h,\delta_T} + KhB.
\label{eq:regret-fav-event-SWA}
\end{align}
%
In the last inequality, we used Equation~\ref{eq:overpull-SWA3} and that there is less than $T$ overpulls in the first sums. We also use $\mu \in \BBSet$ to bound each term in the second sum by $B$. Finally,  we can conclude by plugging Equation~\ref{eq:regret-fav-event-SWA} in Equation~\ref{eq:regret-unfav-event-SWA} and by using the definition of $c\pa{h,\delta_T}$ and $\delta_T= T^{-2}$ in Proposition~\ref{prop:prb_favorable_event_SWA},
\[
\EE{\regret(\piSWA)} \leq 4\sigma T\cdot\sqrt{\frac{\log\pa{\sqrt{2}T}}{h}} + K\pa{h+1}B
\]
\end{proof}
\begin{corollary}[\citet{levine2017rotting}]
\label{cor:SWA}
\newcommand*{\Scale}[2][4]{\scalebox{#1}{$#2$}}
For $C$ such that $h:= \ceil{\Scale[0.90]{C \pa{\frac{\sigma T}{KB}}^{2/3}\log\pa{\sqrt{2}T}^{1/3}}}$,
\[
\regret(\piSWA) \leq \pa{\frac{4}{C^{1/2}} + C} \pa{\sigma^2 B K T^2 \log\pa{\sqrt{2}T}}^{1/3} + 2KB. 
\]
%
Hence, if the learner knows $T$ and the ratio $\frac{\sigma}{B}$, they can set $h:= \ceil{\Scale[0.90]{\pa{\frac{2\sigma T}{KB}}^{2/3}\log\pa{\sqrt{2}T}^{1/3}}}$ (\ie $C=2^{2/3}$) and be guaranteed to perform,
\[
\regret(\piSWA) \leq 5 \pa{\sigma^2 B K T^2 \log\pa{\sqrt{2}T}}^{1/3} + 2KB. 
\]
%
\end{corollary}

\subsubsection*{Empirical evaluation of the anytime version {\wSWA}}
The theoretical window choice require the knowledge of the horizon $T$, the subgaussian parameter $\sigma$ and the reward range $B$ (or at least the ratio $\frac{\sigma}{B}$). \citet{levine2017rotting} suggest \wSWA (Alg. ~\ref{alg:wSWA}), which wraps \SWA  with the doubling trick. The algorithm is initialized with a first (small) guess of the horizon. When the horizon is reached, the algorithm is fully reinitialized and restarted with a doubled horizon. This is a classic trick in the literature : it is known to recover the problem-independent rate of a given algorithm (with a worse constant factor), but the empirical performance is often significantly reduced \citep{besson2018doubling}. In the case of \wSWA, the doubling trick erases all the history $\history_t$ and increases the window. 

\begin{minipage}{\textwidth}
\renewcommand*\footnoterule{}
\begin{savenotes}
\begin{algorithm}[H]
\caption{\wSWA \citep{levine2017rotting} }
\label{alg:wSWA}
\begin{algorithmic}[1]
\Require $\alpha$, $\sigma$, $T \gets 1$
\State $h \gets \ceil{\alpha\pa{\frac{4\sigma T}{K}}^{2/3}\pa{\log\pa{\sqrt{2}T}}^{1/3}}$
\For{$t \gets 1, 2, \dots, T \do $}
		\State \textsc{Run} \SWA(h)
	\EndFor
\State \textsc{Clean \SWA's \textsc{Memory}}
\State \wSWA($\alpha$, $\sigma$, $2T$) 
\end{algorithmic}
\end{algorithm}
\end{savenotes}
\end{minipage}

We notice that the parameter $\alpha$ of \wSWA hides the dependency on $B$. Indeed, the best theoretical tuning corresponds to $\alpha := \pa{2B}^{-2/3}$. In their experimental section, \citet{levine2017rotting} select $\alpha:= 0.2$ by grid-search on one problem. Yet, the reader should not forget that the tuning of $\alpha$ depends on the rotting magnitude $B$.
%TODO figures

\subsection{Open problems}
\subsubsection{Minimax rate}
We report existing regret bounds for two special cases. First, in Proposition~\ref{prop:lb-noisefree}, \citet{heidari2016tight} show that in the absence of noise, the regret is lower bounded by $\cO\pa{KL}$. Second, we recall the minimax regret lower bound for stochastic stationary bandits.

\begin{proposition}[\cite{auer2002nonstochastic}][Thm.\,5.1]
\label{stochastic-LB}
For any learning policy $\policy$ and any horizon $T$, there exists a stochastic stationary problem $\left\{ \mu_i (n) \triangleq \mu_i\right\}_i$ with $K$ $\sigma$-sub-Gaussian arms such that $\pi$ suffers a regret
\begin{equation*}
%\max_{\left\{ \mu_i \in [0,L] \right\}_i}
 \mathbb{E}[\regret(\policy)] \geq \frac{\sigma}{10}\min\pa{\sqrt{\narms\timeEnd},\timeEnd}.
\end{equation*}
where the expectation is w.r.t.\ both the randomization
over rewards and algorithm's internal randomization.
\end{proposition}

Any problem in the two settings above is a rotting problem with parameters ($\sigma$, $L$). Therefore, the performance of any algorithm on the general rotting problem is also bounded by these two lower bounds. For reward functions in $\BBSet$, \SWA is guaranteed to achieve $\cO\pa{T^{2/3}}$ regret rate. Yet, \citet{levine2017rotting} do not provide a lower bound while they suggest it could be an interesting future work direction.

\subsubsection{Problem-dependent rate}
\SWA starts by pulling every arm $h$ times. It means that even for simple stationary problem with large difference $\Delta_i > \sigma$ between suboptimal and optimal arms, \SWA does $h = \cO\pa{T^{2/3}}$ mistakes per suboptimal arms which is much more than the stationary optimal pulling rate $\cO\pa{\frac{\sigma\log\pa{T}}{\Delta_i^2}}$.


\subsubsection{Agnostic algorithm}
\SWA requires the knowledge of the horizon $T$, the subgaussian parameter $\sigma$ and the reward range $B$  to tune the window $h$. We showed empirically that the doubling trick leads to large regret increases at each restart. We also showed that the tuning of $h$


\subsubsection{Global budget or Budget per round}
The guarantee

%!TEX root = ../main.tex 

\section{{\FEWA} and {\RAW} : Two adaptive window algorithms}

\subsection{}
Since the expected rewards $\mu_i$ change from one pull to another, the main difficulty in the rested rotting bandits is that we cannot rely on all samples observed until time~$t$ to predict which arm is likely to return the highest reward in the future. In fact, the older a sample, the less representative it is for future rewards. This suggests constructing estimates using the more recent samples. Nonetheless, discarding older rewards reduces the number of samples used in the estimates, thus increasing their variance. 

\SWA chooses a window which balances the cost due to variance and the cost due to bias. %TODO Comment figure 1 and explain why it is stupid.


%TODO : A Favorable event 

\subsection{{\FEWA}: Filtering on expanding window average}%\label{Algorithm}

 In Alg.\,\ref{EWA} we introduce \myAlgorithm (or~$\EWA$) that at each round $t$, relies on estimates using windows of increasing length to filter out arms that are suboptimal with high probability and then pulls the least pulled arm among the remaining arms. 

We first describe the subroutine {\small\textsc{Filter}} in Alg.\,\ref{filter}, which receives a set of active arms $\mathcal{K}_h$, a window~$h$, and a confidence parameter $\delta$ as input and returns an updated set of arms $\mathcal{K}_{h+1}$. For each arm~$i$ that has been pulled~$n$ times, the algorithm constructs an estimate $\estReward^\window_\arm(n)$ that averages the $h \leq n$ most recent rewards observed from~$i$. %The estimator is well defined only for  and the construction of the set $\mathcal{K}_h$ and the stopping condition at Line~\ref{algline:condition} in Alg.\,\ref{EWA} guarantee that $\estReward^\window_\arm(\armCount_{\arm,\currentTime})$ are always well defined for the arms in $\mathcal{K}_h$. 
The subroutine {\small\textsc{Filter}} discards all the arms whose mean estimate (built with window~$h$) from $\mathcal{K}_h$  is lower than the empirically best arm by more than twice a threshold $c(\window, \delta_\currentTime)$ constructed by standard Hoeffding's concentration inequality (see Prop.\,\ref{prop:heoffding}). %TODO Update reference


\begin{algorithm}[t]
\caption{\myAlgorithm}
\label{EWA}
\begin{algorithmic}[1]
\REQUIRE $\subgaussian$, $\possibleArms$, $\delta_0$, $\alpha$
	\STATE pull each arm once, collect reward, and initialize $N_{\arm,K} \leftarrow 1$ 
	\FOR{$\currentTime \gets K+1, K+2, \dots \do $}
		\STATE $\delta_t \leftarrow \delta_0/(t^\alpha)$
		\STATE $\window \leftarrow 1$ 
		{\footnotesize \COMMENT{\emph{initialize bandwidth}}}
		\STATE $\possibleArms_1 \leftarrow \possibleArms$ 
		{\footnotesize \COMMENT{\emph{initialize with all the arms}}}
		\STATE $\arm(t) \gets {\tt none}$
		\WHILE{$\arm(t)$ is  ${\tt none}$}
			\STATE $\possibleArms_{\window+1} \leftarrow {\textsc{Filter}}(\possibleArms_{\window} ,\window, \delta_\currentTime)$
			\STATE $\window \leftarrow \window+1$ \label{algline:window}
			\IF{$\exists \arm \in \possibleArms_{\window}$ such that $\armCount_{\arm, t}=h$}
			\label{algline:condition}
			\STATE $\arm(t) \leftarrow \arg\min_{i\in\possibleArms_{\window}} N_{i,t}$
			\ENDIF
		\ENDWHILE
		\STATE  receive $\obs_\arm(\armCount_{\arm,\currentTime +1 }) \leftarrow \obs_{\arm(\currentTime),\currentTime}$
		\STATE $\armCount_{\arm(\currentTime),\currentTime} \leftarrow \armCount_{\arm(\currentTime),\currentTime-1} +1$
		\STATE $\armCount_{j,\currentTime} \leftarrow \armCount_{j, \currentTime-1}, \quad \forall j \neq \arm(\currentTime)$
	\ENDFOR
\end{algorithmic}

\end{algorithm}

%\end{minipage}
%
%\hfill
%\begin{minipage}{0.5\textwidth}
\begin{algorithm}[t]
\caption{{\textsc{Filter}}}
\label{filter}
\begin{algorithmic}[1]
\REQUIRE $\possibleArms_{\window}$, $\window$, $\delta_\currentTime$
\STATE $c(\window, \subgaussian, \delta_\currentTime) \leftarrow \sqrt{(2\subgaussian^2/\window) \log{(1/\delta_\currentTime)}}$
\FOR{$ \arm \in \possibleArms_{\window}$}
\STATE $\estReward^\window_\arm(\armCount_{\arm,\currentTime}) \leftarrow \frac{1}{\window} \sum_{j=1}^\window \obs_\arm(\armCount_{\arm,\currentTime} -j)$
\ENDFOR
\STATE $\estReward^\window_{\max,\currentTime}  \leftarrow \max_{\arm \in \possibleArms_{\window}}\estReward^\window_\arm(\armCount_{\arm,\currentTime})$
\FOR{$ \arm \in \possibleArms_{\window}$}
	\STATE $\Delta_\arm \leftarrow  \estReward^\window_{\max,\currentTime}  - \estReward^\window_\arm(\armCount_{\arm,\currentTime})$
	\IF{$\Delta_\arm \leq 2c(\window, \subgaussian, \delta_\currentTime) $}
	\STATE add $\arm$ to $\possibleArms_{\window+1}$
	\ENDIF
\ENDFOR
\ENSURE $\possibleArms_{\window+1}$
\end{algorithmic}
\end{algorithm}
%\end{minipage}



The {\small\textsc{Filter}} subroutine is used in \myAlgorithm to incrementally refine the set of active arms, starting with a window of size $1$, until the condition at Line~\ref{algline:condition} is met. As a result, $\mathcal{K}_{h+1}$ only contains arms that passed the filter for all windows from $1$ up to $h$. Notice that it is important to start filtering arms from a small window and to keep refining the previous set of active arms. % instead of completely recomputing them for every new window $h$. 
In fact, the estimates constructed using a small window use recent rewards, which are closer to the future value of an arm. As a result, if there is enough evidence that an arm is suboptimal already at a small window $h$, it should be directly discarded. On the other hand, a suboptimal arm may pass the filter for small windows as the threshold $c(\window, \subgaussian, \delta_\currentTime)$ is large for small $h$ (i.e., as few samples are used in constructing $\estReward^\window_\arm(\armCount_{\arm,\currentTime})$, the estimation error may be high). Thus, \myAlgorithm keeps refining $\mathcal{K}_{h}$ for larger windows in the attempt of constructing more accurate estimates and discard more suboptimal arms. This process stops when we reach a window as large as the number of samples for at least one arm in the active set $\mathcal{K}_{h}$ (i.e., Line~\ref{algline:condition}). At this point, increasing $h$ would not bring any additional evidence that could refine $\mathcal{K}_{h}$ further (recall that $\estReward^\window_\arm(\armCount_{\arm,\currentTime})$ is not defined for $h > \armCount_{\arm,\currentTime}$). Finally,  \myAlgorithm selects the active arm $i(t)$ whose number of samples matches the current window, i.e., the least pulled arm in $\mathcal{K}_{h}$. The set of available rewards and the number of pulls are then updated accordingly. 

\paragraph{Active set} We derive an important lemma that provides support for the arm selection process obtained by a series of refinements through the {\small \textsc{Filter}} subroutine. Recall that at any round $t$, after pulling arms $\{ \armCount^{\EWA}_{\arm,\currentTime} \}_i$ the greedy (oracle) policy would select an arm 
%
\begin{align*}
\arm^\star_\currentTime \pa{\left\{ \armCount^{\EWA}_{\arm,\currentTime} \right\}_i}  \in  \argmax_{\arm \in \possibleArms} \reward_\arm \left( \armCount^{\EWA}_{\arm,\currentTime}\right).
\end{align*}
%
We denote by $\reward^+_t(\EWA) \triangleq \max_{\arm \in \possibleArms} \reward_\arm ( \armCount^{\EWA}_{\arm,\currentTime}),$ the reward obtained by pulling~$\arm^\star_\currentTime.$ The dependence on $\EWA$ in the definition of $\reward^+_t(\EWA)$ stresses the fact that we consider what the oracle policy would do at the state reached by $\EWA$.
%In the following, we drop the dependency on the number of pulls and we use $i^\star_t$ to denote the greedy arm at round $t$. 
While \myAlgorithm cannot directly match the performance of the oracle arm, the following lemma shows that the reward averaged over the last $h$ pulls of any arm in the active set is close to the performance of the oracle arm up to four times $c(\window,  \delta_\currentTime)$.

\begin{restatable}{lemma}{restafundamentallemma}
\label{fundamental-lemma}
On the favorable event $\HPevent_t$, if an arm~$\arm$ passes through a filter of window $\window$ at round $\currentTime$, i.e., $i\in\ \mathcal{K}_h$, then the average of its $\window$ last pulls satisfies
%
\begin{equation}\label{eq:fundamental.eq}
\expestReward^{\window}_\arm(\armCount_{\arm,\currentTime}^{\EWA} ) \geq  \reward^+_t(\EWA) - 4 c(\window, \delta_\currentTime).
\end{equation}
%
\end{restatable}
This result  relies heavily on the non-increasing assumption of rotting bandits. In fact, for any arm $i$ and any window $h$, we have
%
\begin{equation*}
\wb\mu_i^h(N_{i,t}^{\EWA}) \geq \wb\mu_i^1(N_{i,t}^{\EWA}) \geq \mu_i(N_{i,t}^{\EWA}).
\end{equation*}
%
While the inequality above for $i_t^*$ trivially satisfies Eq.\,\ref{eq:fundamental.eq}, Lem.\,\ref{fundamental-lemma} is proved by integrating the possible errors introduced by the filter in selecting active arms due to the error of the empirical estimates.

\begin{restatable}{corollary}{restafundamentalcorrelary}\label{fundamental-correlary}
	%\begin{corollary}
	Let $\arm \in \overpullSet$ be an arm overpulled by {\FEWA} at round $t$ and $\window_{\arm,t} \triangleq \armCount_{\arm, t}^{\EWA} - \armCount_{\arm, t}^{\policy^\star} \geq 1$ be the difference in the number of pulls w.r.t.\,the optimal policy $\pi^\star$ at round $t$. On the favorable event $\HPevent_t$,  we  have
	\begin{align}
	\reward^+_t(\EWA) - \expestReward^{\window_{\arm,t}}_i(\armCount_{\arm,t}) \leq  4 c(\window_{\arm,t}, \delta_t).
	\end{align}
\end{restatable}


\subsection{The {\EUCB} algorithm}
\label{sec:algo}


\paragraph{A general index policy}
We will study a single class of policies which select at each round $t$ the arm with the maximal index of the form
\vspace{-4pt}
\begin{align}
\label{eq:xindex}
\operatorname{ind}(i,t, \delta_{t}) \triangleq \min_{h\leq N_{i,t-1}} {\widehat{\mu}_i^h(t-1,\pi) + c(h,\delta_{t})}.
\end{align}
We set $\delta_{t} \triangleq \frac{1}{t^\alpha}$ and  call this algorithm Rotting Adaptive Window UCB (\EUCB). There is  a bias-variance trade-off for the window choice: more variance for smaller size of the window $h$ and more bias for larger $h$. The goal of \XUCB is to adaptively select the right window to compute the tightest UCB. \XUCB uses the indexes of \UCBone computed on all the slices of each arm's history which include the last pull. When the rewards are rotting, all these indexes are upper confidence bounds on the \textit{next value}.  Thus, \XUCB simply selects the tightest (minimum) one as index of the arm: it is a pure UCB-index algorithm. By contrast, when reward can increase, the learner can only derive upper-confidence bound on past values which are loosely related to the next value. Hence, all the UCB-index algorithms in the restless non-stationary literature need to add change-detection sub-routine, active random exploration or passive forgetting mechanism. 

\begin{figure}
\vspace{-10pt}
\bookboxx{
\begin{algorithmic}[1]
\REQUIRE$(\delta_t)_{t\geq 1}$
	\STATE pull each arm once
	\STATE initialize $N_{\arm,K} \leftarrow 1$
	\FOR{$\currentTime \gets K+1, K+2, \dots \do $}
		\STATE $\arm_t \gets \argmax_i\operatorname{ind}(i,t, \delta_{t})$ 
		{\footnotesize \COMMENT{\emph{cf.\,\eqref{eq:xindex}}}}
		\STATE  receive reward $\obs_t$%$\obs_\arm(\armCount_{\arm,\currentTime +1 }) \leftarrow \obs_{\arm(\currentTime),\currentTime}$
		\STATE $\armCount_{\arm_\currentTime,\currentTime} \leftarrow \armCount_{\arm_\currentTime,\currentTime-1} +1$
		\STATE $\armCount_{j,\currentTime} \leftarrow \armCount_{j, \currentTime-1}, \quad \forall j \neq \arm_\currentTime$
	\ENDFOR
\end{algorithmic}
\caption{The \EUCB algorithm} \label{algo:xucb}}
\end{figure}

\begin{restatable}{lemma}{restafundamentallemma}
\label{fundamental-lemma}
At round $t$ on favorable event $\HPevent_t$, if arm~$i_{t}$ is selected, for any $h \leq N_{i,t-1}$,  the average of its $\window$ last pulls cannot deviate significantly from the best available arm at that round, i.e.,
%
\vspace{-4pt}
\begin{equation*}
\bar{\mu}^{h}_{i_{t}}(t-1,\pi) \geq \max_{i \in \possibleArms}\mu_i(t,N_{i,t-1}) - 2 c(h, \delta_{t}).
\end{equation*}
\end{restatable}

This fundamental guarantee is comparable with Corollary~ %TODO corollary link
about the algorithm \FEWA. \FEWA uses the same statistics than \XUCB but in a rather complex expanding filtering mechanism. \EUCB has tighter guarantees than \FEWA (2 versus 4 confidence bands), which is the benefit of upper confidence bounds index policies over confidence bound filtering policies. 

%!TEX root = ../main.tex 
\section{Regret Analysis}\label{sec:theory}

In the last section, we presented two algorithms which have very different behaviours. Yet, they show two main similarities. First, for each arm they compute several statistics $\hmu_i^h(\Nitmone)$ for different windows $h\leq \Nitmone$. Second, on the same favorable events $\HPevent$ (on which all these aforementioned statistics are well concentrated around their means, see Prop.~\ref{prop:prb_favorable_event}), we have shown that both algorithms share a guarantee with similar shape that we restate in a general form,
\begin{corollary}[Lemmas~\ref{lem:core-FEWA} and~\ref{lem:core-RAWUCB}]
\label{cor:core-RAW-FEWA}
At a round $t$, on favorable event $\HPevent$, if arm~$i_{t}$ is selected by $\pi(\alpha) \in \left\{ \piR, \piF\right\}$, for any $h \leq N_{i,t-1}$,  the average of its $h$ last pulls cannot deviate significantly from the best available arm at that round, i.e.,
%
\begin{multline*}
\bar{\mu}_{i_t}^{h}(N_{i_t,\, t-1}) \geq \max_{i \in \arms} \mu_{i}(\Nitmone) - \frac{C_\pi}{\sqrt{2\alpha}} c(h, \delta_t) = \max_{i \in \arms} \mu_{i}(\Nitmone) - C_\pi \sigma\sqrt{\frac{\log\pa{t}}{h}}\,,\\
\text{with } C_{\piR} = 2\sqrt{2\alpha} \text{ and } C_{\piF} = 4\sqrt{2\alpha}.
\end{multline*}
\end{corollary}

We will see that this Corollary is the only characterization we need in our Analysis. We first give problem-independent regret bound for \FEWA and \RUCB and sketch its proof in Subsection~\ref{ss:rested-PI}. Then, we discuss problem-dependent guarantees in Subsection~\ref{ss:rested-PD}. Finally, we give a detailed analysis in Subsection~\ref{ss:rested-proof}.


\subsection{Problem-independent bound}
\label{ss:rested-PI}
\begin{restatable}{theorem}{restaalgoindepub}
\label{th:rested-PI}
For any rotting bandit scenario with means $\{\mu_i\}_{i} \in \rewardSet^K$ and any time horizon $T$, $\pi \in \left\{\piR, \piF \right\}$ run with $\alpha \geq 5$ suffers an expected regret of
\begin{equation*}
\mathbb{E}[\regret(\pi)] \leq C_\pi \sigma \sqrt{\log\pa{T}}\pa{\sqrt{KT} +K} + 3KL\, \;\; \text{with } 
\begin{cases}
C_{\piR} \!=\! 2\sqrt{2\alpha}\\
C_{\piF} \!=\! 4\sqrt{2\alpha}
\end{cases}\!\cdot
\end{equation*}
\end{restatable}
\paragraph{Comparison to \citet{levine2017rotting}} The regret of \SWA is bounded by $\tcO(B^{1/3}K^{1/3} T^{2/3})$ for bounded rotting functions in $\BBSet$. According to Subsection~\ref{subsec:rested-open}, the regret guarantee translate in $\cO(T)$ for rotting functions in $\rewardSet$.  Thus, according to its original analysis, \SWA may not be able to learn for our general setting. On the other hand, we could use \FEWA or \RUCB with rotting functions in $\BBSet$ and recover the same regret bound with $L := B$. In this case, our two algorithms suffer a regret of $\tcO(\sqrt{KT})$, thus significantly improving over \SWA. 

The improvement is mostly because \FEWA and \RUCB use adaptive window mechanisms to smoothly track changes in the value of each arm.  Indeed, \SWA relies on a fixed exploratory phase where all arms are pulled in a round-robin way and the tracking is performed using averages constructed with a fixed window. According to Proposition~\ref{prop:SWA}, this fixed window trades off between the cost of biased estimates $\cO\pa{KBh}$ - for scenarios where the arms abruptly decay and their values are overestimated during at most $h$ rounds - and the cost of the variance of estimators $\tcO\pa{{\nicefrac{\sigma T}{\sqrt{h}}}}$ - for scenarios where the arms keep their value close to each other for $\cO\pa{T}$ rounds. In Theorem~\ref{th:rested-PI}, the regret of \FEWA and \RUCB is also bounded by an additive decomposition between the terms depending on the noise level $\sigma$ and the terms depending on the rotting level $L$. Yet, adaptive window algorithms do not need to trade-off: their regret is bounded by $\cO\pa{KL} +\tcO\pa{\sigma\sqrt{KT}}$. It evidence that our algorithms can take decision based on a relevant $h \in \left\{1, \dots, \Nitmone\right\}$ depending on the scenarios.

Last, our algorithms are anytime and agnostic to $L$ (or $B$), while the tuning of \SWA requires to know $B$ and $T$ (or to resort to a doubling trick, which performs poorly in practice). 

\paragraph{Comparison to stationary stochastic bandits}
The regret upper bounds of \FEWA and \RUCB match the worst-case optimal regret bound of the standard stochastic bandits (i.e., $\mu_i(n)$s are constant) up to a logarithmic factor. Whether an algorithm can achieve $\cO(\sqrt{KT})$ regret bound is an open question. On one hand, our analysis needs confidence bounds to hold for different windows at the same time, which requires an additional union bound and thus larger confidence intervals w.r.t.\,\UCBone. On the other hand, our worst-case analysis shows that some of the difficult problems that reach the worst-case bound of Thm.\,\ref{th:rested-PI} are realized with constant functions, which is the standard stochastic bandits, for which \MOSS-like~\citep{audibert2009minimax} algorithms achieve regret guarantees without the $\sqrt{\log T}$ factor. Thus, the necessity of the extra $\sqrt{\log T}$ factor for the worst-case regret of rotting bandits remains an open problem.

\subsection{Problem-dependent bound}
\label{ss:rested-PD}
Since our setting generalizes the stationary stochastic bandit setting, a natural question is whether we pay any price for this generalization. While the result of~\citet{levine2017rotting} suggested that learning in rotting bandits could be more difficult, in Thm.\,\ref{th:rested-PI} we actually proved that \FEWA and \RUCB nearly match the problem-independent regret rate $\tcO(\sqrt{KT})$. We may wonder whether this is true for the \emph{problem-dependent} regret as well.
%
\begin{remark}
Consider a stationary stochastic bandit setting with expected rewards $\{\mu_i\}_i $ and $\mu_\star \triangleq \max_i \mu_i$. For $\pi \in \piFRSet$, on the favorable event $\HPevent$ with $\delta_t \geq 2/T^\alpha$,  we can apply Corollary~\ref{cor:core-RAW-FEWA}  at the last time arm $i$ is pulled (\ie after $\NiT\!-\!1$ pulls) for $h = \NiT\!-\!1$, 
\begin{align}
\mu_\star - \mu_i \leq \frac{C_\pi}{\sqrt{2\alpha}} c\pa{\NiT-1,  \delta_t} = C_\pi\sigma \frac{\sqrt{\log(T)}}{\NiT -1} \CommaBin\nonumber \\
\text{\ or equivalently,\ }
\NiT \leq 1+ \frac{C_\pi^2 \sigma^2 \log(T)}{(\mu_{\star} - \mu_i) ^2}\cdot\label{eq:LaiRob}
\end{align}
Therefore, for $\alpha > 4$\footnote{$\alpha$ should be large enough to control the cost of the unfavorable events, see Lemma~\ref{lem:rested-B}.}, our algorithms match the lower bound of~\citet{lai1985asymptotically} up to a constant factor $C_\pi^2/2$.
\end{remark}
%
With a similar argument, we can show a similar bound on the number of overpulls $\hiT$  of arm $i$ in the general rested rotting bandits case. Indeed, we show in Lemma~\ref{lem:UB-OP-PD} that $\hiT$ is smaller than a problem-dependent quantity $\hiT^+$ which is itself smaller by construction than a function of "gaps" $\Delta_{i,\hiT^+-1}$,
%
\begin{equation}
\label{eq:hit+}
\hiT^+ \triangleq \max \left\{ h \leq 
1 \! + \! \frac{C_\pi^2 \sigma^2 \log \pa{T}}{\Delta_{i,h-1}^2} \right\}\text{ with  } \Delta_{i,h} \triangleq \min_{j \in \arms} \mu_j\pa{N_{j,T}^\star \!-\!1} - \bar{\mu}_i^h\left( \NiT^\star \!+\!h \right). 
\end{equation}
\begin{remark}
Notice that for stationary bandits, we have for all $h$, $\Delta_{i,\,h} = \Delta_i = \mu_\star - \mu_i$. In fact, $ \Delta_{i,\,h} $ extends the notion of gap to our non-stationary setting: it is the average gap between the smallest value pulled by the optimal policy and the average value of the $h$ first overpulls of arm $i$.  We also highlight that $\hit^+$ is always defined because $h=1$ always verify the self-bounding property. 
\end{remark}

Moreover,  on the favorable event $\HPevent$, we can show that the regret of $\hiT$ overpulls of arm $i$ is bounded by $\cO (\sqrt{\hiT})$ (see Lemma~\ref{lem:rested-A}, in Subsection~\ref{ss:rested-proof}). Hence, we bound $\hiT$ by $\hiT^+$ and we use the self-bounding property in the definition of $\hiT^+$ (Equation~\ref{eq:hit+}) to get a $\cO\pa{\log\pa{T}}$ problem-dependent bound for our algorithms on any rotting bandit scenario. 

\begin{restatable}{theorem}{restaalgoub}\label{th:rested-PD}
For any rotting bandit scenario with means $\{\mu_i\}_{i} \in \rewardSet^K$ and any time horizon $T$, $\pi \in \left\{\piR, \piF \right\}$ run with $\alpha \geq 5$ suffers an expected regret of
\begin{align*}
\EE{R_T(\pi)} \leq \sum_{i\in \arms} \pa{\frac{C_\pi^2\sigma^2\log\pa{T}}{\Delta_{i,\hiT^+-1}} + C_\pi \sigma \sqrt{ \log\pa{T}} +3L } \CommaBin \\
\text{with } 
\begin{cases}
C_{\piR} \!=\! 2\sqrt{2\alpha}\\
C_{\piF} \!=\! 4\sqrt{2\alpha}\\
\text{$\Delta_{i,h}$ and $\hiT^+$ defined in Equation~\ref{eq:hit+}.}
\end{cases}
\end{align*}
\end{restatable}
\begin{remark}
The problem-dependent guarantee of \RUCB is 4 times smaller than the guarantee of \FEWA: this is the benefits of upper-confidence bound index policies over confidence bound filtering ones. However, for $\alpha = 5$, our guarantee for \RUCB is still at a factor $C_{\piR}^2 /2 = 20$ of the lower bound of~\citet{lai1985asymptotically} for stationary bandits.

This is mostly due to our proof technique. Indeed, \citet{auer2002finite} also use a similar high-probability proof for \UCBone and also get a large factor compared to the lower bound and an over-conservative tuning of the confidence bounds\footnote{To make the results comparable to the one of~\citet{auer2002finite}, we need to replace $\sigma^2$ by $\nicefrac{1}{4}$ for sub-Gaussian noise.}. Yet, even compared to \UCBone, we have to use a more conservative tuning of the confidence bounds. On the first hand, we use a larger number of estimators at each round: $Kt^2$ instead of $Kt$ for \UCB. Hence, after taking the union bound, we need to increase $\alpha$ by one to have the same probability of the unfavorable event as for \UCBone (see Prop.~\ref{prop:prb_favorable_event_SWA}). On the other hand, for reward functions in $\rewardSet$, the maximal possible regret at a round $t$ is bounded by $Lt$ which is larger than the constant cost $L$ for the stationary case. Thus, we have to increase $\alpha$ by one to control the cost of the unfavorable event. Notice that it is a consequence of our extended setting: we would not need to increase $\alpha$ for reward functions in $\BBSet$.

While we presented our Theorems~\ref{th:rested-PI} and~\ref{th:rested-PD} with $\alpha \geq 5$, we could have similar results for $\alpha > 4$ by replacing the additive term $3KL$ by $\pa{1 + \zeta(\alpha-3)}KL$\footnote{$\zeta(x) \triangleq \sum_n n^{-x}$}. For bounded reward functions, we can further reduce $\alpha >3$. It is still a larger confidence interval than with $\delta_t \sim \frac{1}{t\log{t}^2}$, which is used in \UCB with asymptotic-optimal tuning for sub-gaussian stationary bandits  \citep{lattimore2020banditbook}.  We further discuss the notion of asymptotic optimality and confidence level tuning in rotting bandits in Section~\ref{sec:howhard}. 
\end{remark}
%
\subsection{Proof}
\subsubsection*{Sketch of the proof}
In Lemma~\ref{lem:regret-decompo}, we split the regret decomposition according to whether the overpulls has been done on the favorable event $\HPevent$ or not. 

In Lemma~\ref{lem:rested-B}, we show that the part of the expected regret due to pulls under $\bar{\HPevent}$ is bounded by a constant with respect to $T$ for $\alpha > 4$. Indeed, while we have only trivial bounds on the quality of the pulls on these events, we can control their probabilities thanks to Proposition~\ref{prop:prb_favorable_event}.

In Lemma~\ref{lem:rested-A}, we show that for $\hiT$ overpulls of arm $i$, we suffer no more than $\tcO\pa{\sqrt{\hiT}}$ on the favorable event. Indeed, thanks to Corollary~\ref{cor:core-RAW-FEWA}, we know that the cost of the $h$ before last pulls is bounded by $h \cdot c(h, \delta_t) = \tcO\pa{\sqrt{h}}$.

The proof of Theorem~\ref{th:rested-PI} follows by noticing that $\sum_{i \in \arms} \hiT \leq T$ which leads to the $\tcO\pa{\sqrt{KT}}$ rate. Indeed, thanks to the concavity of the $\sqrt{\cdot}$ and to Jensen's inequality, we find that the worst allocation is $\hiT = \frac{T}{K}$.

In Lemma~\ref{lem:UB-OP-PD}, we construct a problem-dependent bound of $\hiT$ which extends the notion of gaps for rotting bandits using Corollary~\ref{cor:core-RAW-FEWA}.

The proof of Theorem~\ref{th:rested-PD} follows by plugging this bound in the result of Lemma~\ref{lem:rested-A}.
%
\subsubsection*{Full proof}
\label{ss:rested-proof}
Let $t_i^\pi(n)$ the function such that $t_i^\pi(n) = t$ when policy $\pi$ selects arm $i$ at time $t$ for the $n$-th time. We call $\mu_T^+(\pi) \triangleq \max_{i \in \arms} \mu_i\left(\NiT\right)$, \textit{i.e.} the largest available reward for $\pi$ at the round T+1.  
\begin{lemma}
\label{lem:regret-decompo}
 Let $\hiT \triangleq | \NiT - \NiT^{\star}|$. For any policy $\pi$, the regret at the round T is no bigger than
\begin{equation*}
R_T(\pi) \leq \sum_{i \in \overpullSet} \sum_{h=0}^{\hiT-1}\left[\xi^\alpha_{t_i^\pi(\NiT^\star + h)} \right]\left(\mu^+_T(\pi) - \mu_i(\NiT^{\star} + h ) \right) + \sum_{t=1}^T \Big[\bar{\HPevent}\Big]Lt.
\end{equation*}
We refer to the the first sum above as to $A_\pi$ and to the second sum as to $B$.
\end{lemma}
\begin{proof}
We consider the regret at the round $T$. We start from the upper bound in Equation~\ref{eq:regret-first-bound}, 
\begin{equation}
%\label{eq:RegretDecompo}
\regret(\pi) \leq \sum_{i\in \overpullSet}   \sum_{h=0}^{\hiT-1} \pa{\mu^+_T(\pi) - \mu_i(\NiT^{\star} + h)}.
\end{equation}
Then, we need to separate overpulls that are done under $\HPevent$ and under $\bar{\HPevent}$. We introduce $t_i^{\pi}(n)$, the round at which $\pi$ pulls arm $i$ for the $n$-th time. We now make the round at which each overpull occurs  explicit,
\begin{align*}
\regret(\pi) & \leq \sum_{i\in \overpullSet}   \sum_{h=0}^{\hiT-1} \sum_{t=1}^T \left[ t_i^{\pi}\pa{\NiT^{\star} + h} = t \right]  \pa{\mu^+_T(\pi) - \mu_i(\NiT^{\star} + h)}\\
& \leq \underbrace{\sum_{i\in \overpullSet}   \sum_{h=0}^{\hiT-1} \sum_{t=1}^T \left[ t_i^{\pi}\pa{\NiT^{\star} + h} = t \land \HPevent \right] \pa{\mu^+_T(\pi) - \mu_i(\NiT^{\star} + h)}}_{A_\pi}\\
&+ \underbrace{\sum_{i\in \overpullSet}\sum_{h=0}^{\hiT-1} \sum_{t=1}^T \left[ t_i^{\pi}\pa{\NiT^{\star} + h} = t \land \bar{\HPevent} \right]\pa{\mu^+_T(\pi) - \mu_i(\NiT^{\star} + h)}}_B.
\end{align*}
For the analysis of the pulls done under $\HPevent$ we do not need to know at which round it was done. Therefore, 
\[
A_\pi \leq \sum_{i\in \overpullSet}   \sum_{h=0}^{\hiT-1}  \left[ \xi^\alpha_{t(\Nit^\star + h)} \right] \pa{\mu^+_T(\pi) - \mu_i(\NiT^{\star} + h)}.
\]
For \FEWA or \RUCB, it is not easy to directly guarantee the low probability of overpulls (the second sum). Thus, we upper-bound the regret of each overpull at a round $t$ under $\bar{\HPevent}$ by its maximum value $Lt$. While this is done to ease \myAlgorithm analysis, this is valid for any policy $\pi$. Then, noticing that we can have at most 1 overpull per round $t$, i.e., $\sum_{i\in \overpullSet}\sum_{h=0}^{\hiT-1}\left[ t_i^{\pi}\pa{\NiT^{\star} + h} = t  \right] \leq 1$, we get
\[
B \leq  \sum_{t=1}^T \Big[\bar{\HPevent}\Big] Lt\pa{\sum_{i\in \overpullSet}\sum_{h=0}^{\hiT-1}\left[ t_i^{\pi}\pa{\NiT^{\star} + h} = t  \right]} \leq  \sum_{t=1}^T \Big[\bar{\HPevent}\Big] Lt.
\]
Therefore, we conclude that
\[
\regret(\pi) \leq \underbrace{\sum_{i\in \overpullSet} \sum_{h=0}^{\hiT-1} \left[ \xi^\alpha_{t_i^\pi(\Nit^\star + h)} \right]\pa{\mu^+_T(\pi) - \mu_i(\NiT^{\star} + h)}}_{A_{\pi}} + \underbrace{\sum_{t=1}^T \Big[\bar{\HPevent}\Big] Lt}_B. 
\]
\end{proof}
\begin{lemma}
\label{lem:rested-B}
Let $\zeta(x) = \sum_n n^{-x}$. Thus, with $\delta_t = 2t^{-\alpha}$ and $\alpha > 4$, we can use Proposition~\ref{prop:prb_favorable_event} and get
\[
\EE B \triangleq \sum_{t=1}^T p\pa{\bar{\HPevent}}Lt \leq \sum_{t=1}^T KLt^{3-\alpha}\leq KL \zeta(\alpha -3)\, .
\]
In particular, for $\alpha \geq 5$, we have:
\[
\EE B \leq KL\zeta(2) \leq 2KL\, .
\]
\end{lemma}

\begin{lemma}
\label{lem:rested-A}
We define $\hiT^\xi \triangleq \max\left\{ h \leq \hiT | \ \xi^\alpha_{t_i^{\pi}(\Nit^\star + h)}\right\}$, the largest number of overpulls of arm $i$ pulled under $\HPevent$ at the round $t = t_i^{\pi}(\Nit^\star + \hiT^\xi) \leq T$. We also define $\overpullSet_\xi \triangleq \left\{ i \in \overpullSet | \  \hiT^\xi \geq 1 \right\}.$ For policy $\pi \in \left\{ \piR, \piF\right\}$ with parameter $\alpha$, $A_{\pi}$ defined in Lemma~\ref{lem:regret-decompo} is upper-bounded by
\begin{align*}
A_{\pi} &\triangleq  \sum_{i\in \overpullSet}   \sum_{h=0}^{\hiT-1}  \left[ \xi^\alpha_{t_i^{\pi}(\NiT^\star + h) }\right] \pa{\mu^+_T(\pi) - \mu_i(\NiT^{\star} + h)} \\
& \leq \sum_{i\in \overpullSet_\xi} \pa{C_\pi \sigma \sqrt{\left(\hiT^\xi -1\right)\log\pa{T}} +C_\pi \sigma\sqrt{\log{\pa{T}}}+  L}.
\end{align*}
\end{lemma}
\begin{proof}

First, we define $\hiT^\xi \triangleq \max\left\{ h \leq \hiT | \ \xi^\alpha_{t_i^{\pi}(\Nit^\star + h)}\right\}$, the largest number of overpulls of arm $i$ pulled at the round $t_i \triangleq t_i^{\pi}(\Nit^\star + \hiT^\xi) \leq T$ under $\HPevent$. Now, we upper-bound $A_{\pi}$ by including all the overpulls of arm $i$ until the $\hiT^\xi$-th overpull, even the ones under $\bar{\HPevent}$,
\begin{align*}
A_{\pi} &\triangleq  \sum_{i\in \overpullSet}   \sum_{h=0}^{\hiT-1}  \left[ \xi^\alpha_{t_i^{\pi}(\Nit^\star + h) }\right] \pa{\mu^+_T(\pi) - \mu_i(\NiT^{\star} + h)} \\
&
\leq \sum_{i\in \overpullSet_\xi}   \sum_{h=0}^{\hiT^\xi-1}  \pa{\mu^+_T(\pi) - \mu_i(\NiT^{\star} + h)},
\end{align*}
where $\overpullSet_\xi \triangleq \left\{ i \in \overpullSet | \  \hiT^\xi \geq 1 \right\}.$ We can split the second sum of~$\hiT^\xi$ terms above into two parts: on the one hand, the first $\hiT^\xi-1$ (possibly zero) terms (overpulling differences); and on the other hand, the last  $(\hiT^\xi-1)$-th one. Recalling that at the round $t_i$, arm $i$ was selected under $\xi^\alpha_{t_i}$, we apply
Corollary~\ref{cor:core-RAW-FEWA} to bound the regret caused by the first $\hiT^\xi-1$ overpulls of $i$ (possibly none),
\begin{align}
A_{\pi} &\leq  \sum_{i \in \overpullSet_\xi}   \mu^+_T(\pi) - \mu_i\pa{N_{i, T}^\star + \hiT^\xi  -1} + \frac{C_\pi}{\sqrt{2\alpha}}\pa{\hiT^\xi - 1}c\!\pa{\hiT^\xi-1, \delta_{t_i }} \label{eq:cor1-use1}\\
&\leq \sum_{i \in \overpullSet_\xi}   \mu^+_T(\pi) - \mu_i\pa{N_{i, T}^\star + \hiT^\xi  -1} + \frac{C_\pi}{\sqrt{2\alpha}}\pa{\hiT^\xi - 1}c\!\pa{\hiT^\xi-1, \delta_{T}}\\
&\leq \sum_{i \in \overpullSet_\xi}   \mu^+_T(\pi) - \mu_i\pa{N_{i, T}^\star + \hiT^\xi  -1} + C_\pi \sigma\sqrt{\pa{\hiT^\xi - 1}\log{\pa{T}}}.
\label{eq:lasttimepdq}
\end{align}
 The second inequality is obtained because $\delta_t$ is decreasing and $c(\cdot,\delta)$ is decreasing as well. The last inequality is the definition of confidence interval in Proposition~\ref{prop:prb_favorable_event} with $\delta_T = 2T^{-\alpha}$. 
 If  $\NiT^{\star} = 0$ and $\hiT^\xi = 1$, then,
\[ \mu^+_T(\pi) - \mu_i(\NiT^{\star} + \hiT^\xi - 1) =  \mu^+_T(\pi) - \mu_i(0) \leq L,\] 
since $\mu^+_T (\pi) \leq \max_{j\in\arms}\mu_j(0)$ and  $\max_{j\in\arms}\mu_j(0) - \mu_i(0) \leq L$ because $\left\{\mu_i\right\}_{i \in \arms} \in \rewardSet^K$ (Def.~\ref{def:rew-bounded-decay}).
Otherwise, we can decompose 
\begin{align*}
\mu^+_T(\pi) - \mu_i(\NiT^{\star} + \hiT^\xi - 1) %\\
%&
= &\underbrace{\mu^+_T(\pi) - \mu_i(\NiT^{\star} + \hiT^\xi-2)}_{A_1} \\&+ 
\underbrace{\mu_i(\NiT^{\star} + \hiT^\xi-2) -  \mu_i(\NiT^{\star} + \hiT^\xi - 1)}_{A_2}.
\end{align*}
For term $A_1$, since this $\hiT^\xi$-th overpull is done under $\xi^\alpha_{t_i}$, by Corollary~\ref{cor:core-RAW-FEWA} we have that
\[
A_1 = \mu^+_T(\pi) - \bmu_i^1(\NiT^{\star} + \hiT^\xi-1) \leq 1c(1, \delta_{t_i}) \leq 2c(1,\delta_{T}) \leq C_\pi \sigma \sqrt{\log\left(T\right)} .
\] 
The second difference, 
$A_2 = \mu_i(\NiT^{\star} + \hiT^\xi-2) -  \mu_i(\NiT^{\star} + \hiT^\xi - 1 )$  
cannot exceed $L$, since by the assumptions of our setting  (Def.~\ref{def:rew-bounded-decay}), the maximum decay in one round is bounded.
Therefore, we further upper-bound Equation~\ref{eq:lasttimepdq} as
\begin{align}
A_{\pi} \leq \sum_{i\in \overpullSet_\xi} \pa{C_\pi \sigma \sqrt{\left(\hiT^\xi -1\right)\log\pa{T}} + C_\pi \sigma\sqrt{\log{\pa{T}}}+  L}.
\label{HPevent0}
\end{align}
\end{proof}

\label{proof1}
\restaalgoindepub*
\label{proof2} 

\begin{proof}
In Lemma~\ref{lem:regret-decompo}, we split the regret in two parts. The first one $B$ corresponds to the regret due to unfavorable events $\bar{\HPevent}$. We do not derive any guarantee of our algorithms on these events but their probabilities can be controlled thanks to parameter $\alpha$. Hence, for $\alpha > 4$, we show in Lemma~\ref{lem:rested-B} that the part of the expected regret due to unfavorable events can be bounded by a constant w.r.t. $T$. Yet, we choose $\alpha \geq 5$ to have a small constant.

The second one $A_\pi$ corresponds to the regret due to favorable events $\HPevent$ which can be bounded for our two algorithms (\FEWA and \RUCB) thanks to Lemma~\ref{lem:rested-A}. In order to get a problem-independent upper bound, we need to replace $\hiT^\xi$ by a problem-independent quantity. Starting from Lemma~\ref{lem:rested-A},

\begin{equation*}
A_{\pi} \leq \sum_{i\in \overpullSet_\xi} \pa{C_\pi \sigma \sqrt{\left(\hiT^\xi -1\right)\log\pa{T}} +C_\pi \sigma\sqrt{\log{\pa{T}}}+  L}.
\end{equation*}

Since $\overpullSet_\xi \subseteq \overpullSet$, we can upper-bound the number of terms in the above sum by  $K$.
Next, the total number of overpulls $\sum_{i\in\overpullSet} \hiT$ cannot exceed $T$. 
As square-root function is concave we can use Jensen's inequality. 
Moreover, we can deduce that the worst allocation of overpulls is the uniform one, i.e., $\hiT = T/K,$
\begin{align}
A_{\pi} &\leq K(C_\pi \sigma\sqrt{\log(T)} + L) + C_\pi \sigma\sqrt{\log(T)} \sum_{i\in \overpullSet} \sqrt{(\hiT - 1)}\nonumber\\ 
&\leq K (C_\pi \sigma\sqrt{\log(T)} + L) + C_\pi \sigma\sqrt{KT\log(T)}.
\label{eq:Abound-PI}
\end{align}

Therefore, using Lemma~\ref{lem:regret-decompo} together with Equations~\ref{eq:Abound-PI} and Lemma~\ref{lem:rested-B}, we bound the total expected regret as
\begin{equation}
\mathbb{E}[\regret(\pi)] \leq C_\pi \sigma\sqrt{\log\pa{T}}\pa{\sqrt{KT} +K} + 3KL\cdot
\end{equation}
\end{proof}

\begin{lemma}\label{lem:UB-OP-PD}
We define the smallest reward gathered by the optimal policy $\mu^-_T$ and the gap of the h first overpulls of arm $i$ with respect to that value $\Delta_{i,h}$.
\begin{align*}
&\mu^-_T\triangleq \min_{i \in \arms^\star} \mu_i \pa{\NiT^\star -1} \text{ with }\arms^\star \triangleq\left\{ i \in \arms | \NiT^\star \geq 1 \right\}, \\
&\Delta_{i,h} \triangleq \mu^-_T- \bar{\mu}_i^h\left( \Nit^\star+h \right).
\end{align*}
$\hiT^\xi$ defined in Lemma~\ref{lem:regret-decompo} is upper-bounded by a problem-dependent quantity,
\begin{equation*}
\hiT^\xi \leq   \hiT^+  \triangleq \max \left\{ h \leq T \big| \ h \leq  1 + \frac{C_\pi^2 \sigma^2 \log \pa{T}}{\Delta_{i,h-1}^2} \right\}  \leq  1 + \frac{C_\pi^2 \sigma^2 \log \pa{T}}{\Delta_{i,\hiT^+-1}^2}\cdot
\end{equation*}
\end{lemma}
\begin{proof}

We want to bound $\hiT^\xi $ with a problem dependent quantity $\hiT^+$. We remind the reader that for arm $i$, the $\hiT^\xi$-th overpull is pulled under $\xi^\alpha_{t_i}$ at the round $t_i$. Therefore, Corollary~\ref{cor:core-RAW-FEWA} applies and we have
\begin{align*}
\bmu_i^{\hiT^\xi  - 1} \left( \NiT^\star + \hiT^\xi   - 1 \right) &\geq \mu_T^+(\pi) - \frac{C_\pi}{\sqrt{2\alpha}} c\pa{\hiT^\xi   - 1, \delta_{t_i}}
\\& \geq \mu_T^+(\pi) - \frac{C_\pi}{\sqrt{2\alpha}} c\pa{\hiT^\xi   - 1, \delta_T}
\\& \geq \mu_T^+(\pi) - C_\pi \sigma \sqrt{\frac{\log\pa{T}}{\hiT^\xi-1}}\CommaBin
\end{align*}
Hence, we have that 
\begin{equation}
\label{eq:hiTxi-bound}
\hiT^\xi \leq 1 + \frac{C_\pi^2 \sigma^2\log\pa{T}}{\pa{\mu_T^+(\pi)- \bmu_i^{\hiT^\xi  - 1} \left( \NiT^\star + \hiT^\xi   - 1 \right) }^2 }\cdot
\end{equation}
%
Yet, this upper bound still depends on random quantities such as $\mu_T^+(\pi)$ or $\hiT^\xi$ on the denominator. 
Consider the smallest value collected by the optimal policy, 
\[
\mu^-_T\triangleq \min_{i \in \arms^\star} \mu_i \pa{\NiT^\star -1}\text{ with } \arms^\star \triangleq\left\{ i \in \arms | \NiT^\star \geq 1 \right\}.
\]
We recall that the greedy oracle $\GO$ selects the rewards in the decreasing order (see the proof of Proposition~\ref{prop:heidari-oracle}). Therefore, $\mu^-_T$ (the smallest value selected at the round $T$) is the $T$-th largest value among the $KT$ possible ones. Moreover, the overpulls - which are the values that are not among the $T$ largest ones selected by $\GO$ - are all smaller than $\mu^-_T$. Since $\bmu_i^{\hiT^\xi  - 1} \left( \NiT^\star + \hiT^\xi  - 1 \right)$ is an average of overpulls' values, we have,
\[\mu^-_T\geq \bmu_i^{\hiT^\xi  - 1} \left( \NiT^\star + \hiT^\xi   - 1 \right).\] 
%
Moreover, $\mu_T^- > \mu_T^+(\pi)$ implies that the regret is 0. Indeed, in that case $\mu_T^+(\pi)$ - the pull with the largest value among the remaining values at the end of the game for $\pi$ - is \emph{strictly smaller} than $\mu_T^-$ - the $T$-th largest reward sample.  Therefore, $\pi$ has collected the $T$ largest values and has zero regret. Hence, we focus on the case $\mu^-_T\leq \mu^+_T(\pi)$, for which the regret may not be zero.  In that case, we can upperbound the RHS term Equation~\ref{eq:hiTxi-bound} by replacing the random quantity $\mu_T^+(\pi)$ by the smaller quantity $\mu^-_T$. Hence, 
\[
\hiT^\xi \leq 1 + \frac{C_\pi^2 \sigma^2\log\pa{T}}{\pa{\mu_T^+(\pi)- \bmu_i^{\hiT^\xi  - 1} \left( \NiT^\star + \hiT^\xi   - 1 \right) }^2 }\ \leq 1 + \frac{C_\pi^2 \sigma^2\log\pa{T}}{\Delta_{i,\hiT^\xi  - 1}^2}\CommaBin
\] 
with $\Delta_{i,h} \triangleq \mu^-_T- \bar{\mu}_i^h\left( \Nit^\star+h \right)$, the difference between the lowest mean value of the arm pulled by $\pi^\star$ and the average of the $h$ first overpulls of arm~$i$. Yet, this self-bounding property of $\hiT^\xi $ is not a proper problem-dependent upper bound. We will consider the largest $h$ which satisfies this self-bounding property, 
\begin{equation*}
 \hiT^+  \triangleq \max \left\{ h \leq T \big| \ h \leq  1 + \frac{C_\pi^2 \sigma^2 \log \pa{T}}{\Delta_{i,h-1}^2} \right\}\cdot
\end{equation*}
We have that,
\begin{equation*}
\hiT^\xi \leq  \hiT^+  \leq  1 + \frac{C_\pi^2 \sigma^2 \log \pa{T}}{\Delta_{i,\hiT^+-1}^2}\cdot
\end{equation*}
\end{proof}
\restaalgoub*
\begin{proof}
We use Lemmas~\ref{lem:rested-A} and Lemma~\ref{lem:UB-OP-PD} to bound $A_\pi$ (see Lemma~\ref{lem:regret-decompo}). Indeed, since the square-root function is increasing, we can upper-bound the result in Lemma~\ref{lem:rested-A} by replacing $\hiT^\xi$ by its upper bound in Lemma~\ref{lem:UB-OP-PD}
\begin{align*}
A_{\pi} &\leq \sum_{i\in \overpullSet_\xi} \pa{C_\pi \sigma\sqrt{\log(T)} \left( 1 + \sqrt{\hiT^+ - 1}\right) + L}\\
& \leq \sum_{i\in \overpullSet_\xi} \pa{C_\pi \sigma\sqrt{\log(T)} \left( 1 + \frac{C_\pi \sigma\sqrt{\log(T)}}{\Delta_{i,\hiT^+-1}}\right) + L}. 
\end{align*}
Notice that the quantity $\overpullSet_\xi \subset \arms$. Therefore, we have 
\begin{equation}
\label{eq:Abound-PD}
A_{\pi} \leq \sum_{i\in \arms} \pa{\frac{C_\pi^2\sigma^2\log\pa{T}}{\Delta_{i,\hiT^+-1}} + C_\pi \sigma \sqrt{\log\pa{T}} +L }. 
\end{equation}
Using Lemmas~\ref{lem:regret-decompo}, \ref{lem:rested-B}, and Equation~\ref{eq:Abound-PD} we get
\begin{align*}
\EE{\regret(\pi)} &=\EE{A_{\pi}} + \EE B 
\\&
\leq \sum_{i\in \arms} \pa{\frac{C_\pi^2\sigma^2\log\pa{T}}{\Delta_{i,\hiT^+-1}} + C_\pi \sigma \sqrt{\log\pa{T}} +L } + 2KL \\
&\leq \sum_{i\in \arms} \pa{\frac{C_\pi^2\sigma^2\log\pa{T}}{\Delta_{i,\hiT^+-1}} + C_\pi \sigma \sqrt{\log\pa{T}} +3L } \cdot
\end{align*}
\end{proof}
%!TEX root = ../main.tex 
%!TEX root = ../main.tex 
\section{Efficient algorithms}
\label{app:efficient_alg}
\subsection{The numerical cost of adaptive windows}

In the three last sections, we presented two adaptive windows algorithms whose significantly improved over state-of-the-art algorithms, both theoretically and experimentally. Yet, we highlight that these improvements are computationally expensive. Indeed, at each round $t$, we store, update and compare $\cO\pa{t}$ statistics. 

The full update of the statistics can be done at a worst case cost of $\cO\pa{t}$. Indeed, each statistics $\hmu_i^h$ can be refreshed with a $\cO\pa{1}$ operation : 
\[\hmu_i^{h+1}(n+1) = \frac{h}{h+1}\hmu_i^{h}(n) + \frac{1}{h+1}o_t \,. \]

The comparison part in both \FEWA and \RUCB is also a $\cO\pa{t}$ operations. In \FEWA , we do a scan based on $\hmu_i^{h}$ for all $i \in \arms_h$ with increasing $h$. Hence, the total number of unitary operation is in $\cO\pa{t}$ in the worst case, as it scales with the number of statistics. \RUCB computes one UCB for each of the $\cO\pa{t}$ statistics. For each arm, it selects the minimum UCB as index, which can be done with complexity $\cO\pa{t}$. Finally, finding the largest index is an $\cO\pa{K}$ operations. Therefore, we can conclude,

\begin{proposition}
\FEWA and \RUCB have a $\cO\pa{t}$ worst-case complexity per round $t$ in time and memory.
\end{proposition}

\begin{remark}
\SWA($h$) has a $\cO\pa{h}$ worst-case complexity in time and memory because the sliding-window mechanism need to store and update $\cO\pa{h}$ statistics to always have the average of the $h$ last sample ready. Hence, when it is optimally tuned for the minimax bound, $\SWA$ has a $\cO\pa{T^{2/3}}$ per round complexity. As often in non-stationary bandits, it may be possible to replace sliding window statistics by discounted statistics. Such modification often leads to slightly worse theoretical regret rate but to a much better $\cO\pa{K}$ complexity. 
\end{remark}

Hence, handling a large number of windows, which is the main strength of our algorithms to achieve a lower regret, is a significant drawback when it comes to design fast algorithm. Therefore, it is an open question whether one can enjoy the benefits of adaptive windows without suffering large time and space complexity. 

\subsection{The efficient update trick}
We detail \EFF, an update scheme to handle efficiently statistics of different windows. A similar yet different approach has appeared independently in the context of streaming mining~\citep{bifet2007learning}. \EFF is built around two main ideas.

First, at any time $t$ we can avoid using $\left\{\hmu_i^h\right\}_{h}$ for all possible windows $h$ starting from 1 with an increment of 1. In fact, both statistics $\hmu_i^h$ and constructed confidence levels $c(h, \delta_t)$  have very close value for successive $h$ as $h$ becomes large : 
\begin{align*}
& \hmu_i^{h+1}(n) = \hmu_i^{h}(n) + \cO\pa{\frac{\sigma + L}{h}}\,,\\
& c(h+1, \delta_t) = c(h, \delta_t) + \cO\pa{\frac{\sigma }{h^{3/2}}} \,.
\end{align*}
Hence, in both \FEWA and \RUCB, we compute a lot of very similar quantities. Instead, we could use fewer statistics which are significantly different : $\left\{\hmu_i^h(\Nitmonepi)\right\}_{h\in \Him}$, where the window $h$ is dispatched on a geometric grid, 
 \[\Him\pa{\Nitmonepi} \triangleq \left\{ h_j \in  \left\{1, \dots , \Nitmonepi \right\} \;|\; h_{j+1} = \ceil{m \cdot h_j} \text{ and } h_1 = 1\right\}\quad \text{with } m > 1.\]

When there is no confusion, we drop the dependency in $\Nitmonepi$.  This modification alone is not enough to reduce both the time and space complexity. Indeed, updating $\hmu^h_{i}$ requires to replace the $h$-th last sample by the new one $o_t$. Hence, we need to store all the collected statistics to be able to update all the $\hmu^h_{i}$ for all $h$ with $\cO\pa{1}$ complexity. Therefore, in \EFF, we will use $\cO\pa{K\log\pa{t}}$ \emph{delayed} statistics that we can update with $\cO\pa{K\log\pa{t}}$ space and time complexity.

\EFF (Alg.~\ref{alg:effupdate}) takes as input the new observation $o_t$ that the learner gets at the $N_i$-th pull of arm $i$; the geometric window grid $\Him$ tuned with an hyperparameter $m>1$, and for each window $h_j$ in this grid, three different numbers $\hmueff,\; \peff, \; \neff$. $\left\{\hmueff\right\}_{i,h_j}$ represents the set of \emph{current} statistics of window size $h_j$ that will be used instead of $\left\{\hmu_i^h\right\}_{i,h}$ in our efficient algorithms. We also store a pending statistic $\peff$ and a count $\neff$  which are used in the sparse update procedure of $\hmueff$. \EFF outputs an updated set of statistics.  

\begin{minipage}{\textwidth}
\renewcommand*\footnoterule{}
\begin{savenotes}
\begin{algorithm}[H]
\caption{{\small\sc Eff\_Update}}
\begin{algorithmic}[1]
\label{alg:effupdate}
\Require $o_t$, \small $\Him \gets \left\{h_j \! <\! \ceil{m \cdot N_i} \; | \;  h_{j+1} \!=\! \ceil{m \cdot h_j}  \text{with } h_0 \!=\! 1\right\} $\normalsize, $\left\{ \{ \hmueff,\, \peff, \, \neff \}\right\}_{h_j \in \Him}$
\If{$N_i = \max\pa{\Him}$}\label{algline:effu-new-condition} \Comment{Create a new triplet with window $h_j = \ceil{m \cdot N_i}$}  
\State $\Him \gets \Him \cup \left\{ \ceil{m \cdot N_i} \right\}$\label{algline:effu-new-h}
\State $p_i^{\ceil{m \cdot N_i} } = p_i^{N_i} $\label{algline:effu-new-p}
\State $n_i^{\ceil{m \cdot N_i} } \gets n_i^{N_i} $\label{algline:effu-new-n}
\State $\hmu_{i, \, \tteff}^{\ceil{m \cdot N_i} }\leftarrow \texttt{None}$\label{algline:effu-new-mu}
\EndIf\label{algline:effu-new-end} 
\State $p_i^{1} \gets o_t$ \label{algline:effu-update-first-p} \Comment{Update the first triplet with $o_t$}
\State $n_i^{1} \gets 1$\label{algline:effu-update-first-n}
\State $\hmu_{i, \, \tteff}^{1}\leftarrow o_t$ \label{algline:effu-update-first-hmu}
\For{$h_j \in  \Him \smallsetminus \left\{ 1\right\} $}\label{algline:effu-update-start} \Comment{Update the other pending statistics $\peff$ and $\neff$}
\State $p_i^{h_j} \gets p_i^{h_j}  +o_t$\label{algline:effu-update-p}
\State $n_i^{h_j} \gets n_i^{h_j} + 1$\label{algline:effu-update-n}
\EndFor\label{algline:effu-update-end} 
\For{$h_j \in  $ \textsc{Sort\_Desc}$\pa{\Him \smallsetminus \left\{ 1\right\} }$}\label{algline:effu-refresh-start}
\If{$n_i^{h_j} = h_j$} \label{algline:effu-refresh-condition}
\State $\hmueff \leftarrow p_i^{h_j}/h_j$ \Comment{Replace the current statistic $\hmueff$}\label{algline:effu-refresh-hmu}
\State{$p_i^{h_{j}} = p_i^{h_{j-1}} $} \label{algline:effu-refresh-p}\Comment{Refresh the pending statistics}
\State $n_i^{h_{j}} \gets n_i^{h_{j-1}} $\label{algline:effu-refresh-n}
\EndIf
\EndFor \label{algline:effu-refresh-end}
\Ensure $\left\{\left\{  \hmueff,\; p_i^{h_j}, \; n_i^{h_j} \right\}\right\}_{h_j \in \Him}$
\end{algorithmic}
\end{algorithm}
\end{savenotes}
\end{minipage}




The core of \EFF is divided in four parts: 1) From Lines~\ref{algline:effu-new-condition} to~\ref{algline:effu-new-end}, we create new statistics at a logarithmic rate with respect to the growth of $N_i$; 2) From Lines~\ref{algline:effu-update-first-p} to~\ref{algline:effu-update-first-hmu}, we update the statistics of window $h_1=1$;
3) From Lines~\ref{algline:effu-update-start} to~\ref{algline:effu-update-end}, we update the other pending statistics and count;
4) From Lines~\ref{algline:effu-refresh-start} to~\ref{algline:effu-refresh-end}, we eventually update $\hmueff$ and refresh the correspounding pending statistic and count. The remaining details are quite technical. Thus, we first give the high-level properties that are ensured by the recursive usage of \EFF. Then, we prove them by going through the algorithm line by line.

\begin{proposition}
\label{prop:effu}
 $\left\{\left\{  \hmueff,\; p_i^{h_j}, \; n_i^{h_j} \right\}\right\}_{h_j \in \Him}$, constructed recursively with \EFF with initial value $\left\{\left\{  \hmu_{i,\,\tteff}^1 : \texttt{None},\; p_i^{1} :0 , \; n_i^{1}:0 \right\}\right\}$ have the following properties :
 \begin{enumerate}[topsep=0pt]
  \item $\hmueff$ is the average of exactly $h_j$ consecutive samples among the $2h_j -1$ last ones. \label{list:effu-hmu}
  \item The delay between two updates of $\hmueff$ is in $\left\{\ceil{\frac{m-1}{m} h_j}, \dots, h_j -1\right\}$.\label{list:effu-delay}
  \item When $m = 2$, $h_j = 2^{j}$. Moreover, for $j\geq1$,  the $k$-th update $\hmueff$ happens at pull $ \pa{k+1} \cdot 2^{j-1}$, \ie every $2^{j-1}$ pulls (and at every rounds for $j=0$).\label{list:effu-m2}
  \item $\peff$ is the sum of the $\neff$ last samples. \label{list:effu-p}
  \item $\neff < h_j$ for $j\geq 1$. Also, $n_i^1 \leq 1$.\label{list:effu-n1}
  \item $\left\{ \neff \right\}_{h_j}$ is an non-decreasing sequence with respect to $h_j$ (or $j$).\label{list:effu-n2}
 \end{enumerate}
\end{proposition}
\begin{proof}
The three last properties are trivially true at the initialization. Thus, we show by induction that they remain true after updates.
\paragraph{Proof of \ref{list:effu-p}. } At Lines~\ref{algline:effu-new-p} and~\ref{algline:effu-new-n}, we create a new pending statistics and count by initializing them with other statistics and counts. Hence, because of the recursion hypothesis, all the pending statistics $\peff$ (including the created one) contains the sum of the $\neff$ \emph{before last} pulls. At Lines~\ref{algline:effu-update-first-p} and~\ref{algline:effu-update-first-n}, we update $p_i^1$ with the last sample and set $n_i^1$ to $1$. At Lines~\ref{algline:effu-update-p} and~\ref{algline:effu-update-n}, we add the last sample to $\peff$ (which was containing the before last samples) and increase the count by $1$. Hence, at the end of Line~\ref{algline:effu-update-n}, all the $\peff$ contains the sum of the last $\neff$ samples. Thus, refreshing $\peff$ and $\neff$ with $ p_i^{h_{j-1}}$ and $n_i^{h_{j-1}}$ keeps this property true (Lines~\ref{algline:effu-refresh-p} and~\ref{algline:effu-refresh-n}). 

\paragraph{Proof of \ref{list:effu-n1}.}
For $j=0$, $n_i^1$, which is equal to $0$ at the initialization, is set at $1$ at every update (Line~\ref{algline:effu-update-first-n}). Hence, we have $n_i^{h_0} \leq h_0=1$.
For $j\geq 1$, $n_i^{\ceil{m \cdot N_i}}$ is initialized at Line~\ref{algline:effu-new-n} with the value $n_i^{N_i} < N_i < \ceil{m \cdot N_i}$ by the induction hypothesis and because $m>1$.  Then, $\neff < h_j$ ($j\geq1$) is increased by one at each update at Line~\ref{algline:effu-update-n}. Hence, we now have $\neff \leq  h_j$ for all $j\in\Him$. However, for $j\geq 1$, if $\neff = h_j$ (Line~\ref{algline:effu-refresh-condition}), it is replaced by the precedent count $n_i^{h_{j-1}}\leq h_{j-1} < h_j$ (Line~\ref{algline:effu-update-n}).Thus, at the end of the update, we do have $\neff < h_j$ for $j\geq1$.

\paragraph{Proof of \ref{list:effu-n2}.}
At Line~\ref{algline:effu-new-n}, we create a new pending count corresponding to the largest $h_j$ and we initialize it with the precedent largest count. At Lines~\ref{algline:effu-update-first-n} and~\ref{algline:effu-update-n}, we set $n_i^1 =1$ and increase all the other $\neff$ by one. This operation preserves the non-decreasing property of the ordered set. Last, at Line~\ref{algline:effu-refresh-n}, we set few counts $\neff$ to the precedent value $n_i^{h_{j-1}}$- which also preserves the non-decreasing property of the ordered set. 

\paragraph{Proof of \ref{list:effu-hmu} and \ref{list:effu-delay}.}
Thanks to Property~\ref{list:effu-p}, we know that $\peff$ is the sum of the $\neff$ last sample. It is still true at the end of Line~\ref{algline:effu-update-n} (see the proof). Then, at Line~\ref{algline:effu-refresh-hmu}, and given the condition in Line~\ref{algline:effu-refresh-condition}, we set $\hmueff$ with the average of the last $h_j$ sample. Then, $\hmueff$ is not updated untill the condition at Line~\ref{algline:effu-refresh-condition} is fulfilled again. 

$\neff$ is refreshed with a quantity larger or equal to $1$ and smaller or equal to $h_{j-1}$ at Line~\ref{algline:effu-refresh-n}. Then, it is increased by one at each update. we know that $\hmueff$ will be updated at least every $h_j-1$, and at most every $h_j -h_{j-1}$ round. Hence, considering the worst possible delay we can conclude : $\hmueff$ is the average of exactly $h_j$ consecutive samples among the $2h_j -1$ last ones. Last, considering that $h_{j-1}\leq h_j /m$, we conclude that the minimal delay is larger or equal to $\frac{m-1}{m}h_j$.

\paragraph{Proof of \ref{list:effu-m2}.}
When $m=2$, it is easy to find by induction that,
 \[
 h_{j+1} = \ceil{m\cdot h_j} = 2h_j = 2^{j+1}.
 \]
For $j=0$, $\hmu_{i,\,\tteff}^1$ is updated at every update at Line~\ref{algline:effu-update-first-hmu}.
By induction on $j\geq 1$, $\hmueff$ is initialized (Line~\ref{algline:effu-refresh-hmu}) for the first time after $h_j= 2^{j} = 4 \cdot 2^{j-2}$ pulls. Therefore, it is also an updating pull for $\hmu_{i,\,\tteff}^{h_{j-1}}$ (by the induction hypothesis) and $n_j$ is set with $n_{j-1} = 2^{j-1}$ at Line~\ref{algline:effu-refresh-n}. Notice that we sort $\Him$ in the decreasing order at Line~\ref{algline:effu-refresh-start}, hence $n_j$ is updated with $n_{j-1}$ before it is itself updated with $n_{j-2}$.  Hence, $\hmueff$ is updated again in $h_j - 2^{j-1} = 2^{j-1}$ pulls, \ie after $6 \cdot 2^{j-2}$ pulls of arm $i$. Again, $n_j$ is set with $n_{j-1} = 2^{j-1}$ (because it is an updating pull for $\hmu_{i,\,\tteff}^{h_{j-1}}$). By induction, we see that the $k$-th update happens at pull $ \pa{k+1} \cdot 2^{j-1}$, \ie every $2^{j-1}$ pulls.


\end{proof}

\begin{remark}
At Line~\ref{algline:effu-refresh-n}, we refresh $\neff$ with $n_i^{h_{j-1}}$ which is often larger than $1$. Indeed, we could refresh $\peff$ and $\neff$ at $0$. Yet, in order to reduce the delay in the update, we use the variable available in the memory which contains the sums of $h$ last sample, with the largest $h< h_j$. According to Properties~\ref{list:effu-p},~\ref{list:effu-n1} and~\ref{list:effu-n2}, this quantity is $p_i^{h_{j-1}}$. 

Notice that we also sort $\Him$ in the decreasing order at Line~\ref{algline:effu-refresh-start} to minimize the delay: if there is two consecutive updates of $\hmueff$ and $\hmu_{i,\, \tteff}^{h_{j+1}}$ at the same run of \EFF, doing a backward loop guarantees to refresh $n_i^{h_{j+1}}$ with a larger value than with a forward loop. 
\end{remark}
%Property~\ref{list:effu-delay} in Proposition~\ref{prop:effu} states that the delay in the update is at most $h_j-1$ in the worst case. Yet, notice that it is often much less: 


\subsection{{\EFFFEWA} and {\EFFRAW}}
{\EFFFEWA} and {\EFFRAW} are the two efficient versions of our initial algorithms. With an hyperparameter $m>1$, they use \EFF instead of \UPDATE (Lines~\ref{algline:fewa-update1} and~\ref{algline:fewa-update2} in \FEWA and Lines~\ref{algline:raw-update1} and~\ref{algline:raw-update2} in \RUCB). Therefore, they use $\left\{\hmueff\right\}_{i,h_j\in \Him}$ instead of $\left\{\hmu_i^h\right\}_{i,h \leq \Nitmone}$. 

More precisely, in \FEWA, we replace the increment $h\gets h+1$ by $h\gets\ceil{m\cdot h}$ at Line~\ref{algline:fewa-window}. Hence, the next set is not called $\arms_{h+1}$ but $\arms_{\ceil{m\cdot h}}$ (Line~\ref{algline:fewa-filter} in \FEWA and Line~\ref{algline:filter-add} in \FILTER). Finally, at Lines~\ref{algline:fewa-condition} and~\ref{algline:fewa-pull}, the condition is not $N_{i_t}=h$ but $N_{i_t} \leq h$. In the \FILTER procedure, we also change $\hmu_i^h$ by $\hmu_{i,\,\tteff}^h$ at Lines~\ref{algline:filter-max} and~\ref{algline:filter-delta}. In \RUCB, we only change the $h\leq N_i$ by $h_j \in \Him$ and $\hmu_i^h$ by $\hmu_{i,\,\tteff}^h$ in the index computation at Line~\ref{algline:raw-pull}.

\begin{proposition}
\EFFFEWA and \EFFRAW tuned with hyperparmaeter $m$ have a $\cO\pa{K\log_m\pa{t}}$ worst-case time and space complexity at round $t$.
\end{proposition}
\begin{proof}
The total number of statistics for each arm $i$ at round $t$ is bounded by $\cO\pa{\log_m\pa{t}}$. Indeed, 
\[t \geq \Nitmone \geq h_j \geq m^{j-1} \implies j \leq 1 + \log_m\pa{t}.\]
Moreover, in \EFF we use 3 numbers for each $\left\{\hmueff\right\}_j$. Hence, the space complexity scales with \[ \sum_{i\in \arms} | \Him| = \sum_{i\in \arms} \cO\pa{\log_m\pa{t}} = \cO\pa{K\log_m\pa{t}}.\]
The time complexity of $\EFF$ scales with the number of statistics in arm $i_t$, \ie at most $\cO\pa{\log_m\pa{t}}$. The indexes computation of \EFFRAW  find the minimum of $K$ sets with cardinality $\cO\pa{\log_m\pa{t}}$, while finding the maximum among these indexes is a $\cO\pa{K}$ operation.  Thus, the worst-case time complexity is $\cO\pa{K\log_m\pa{t}}$. \EFFFEWA uses at most $\cO\pa{\log_m\pa{t}}$ times the procedure \FILTER  whose inner complexity scales with $|\arms_h| \leq K$. Therefore, in the worst case, the time complexity of \EFFFEWA at round $t$ is bounded by $\cO\pa{K\log_m\pa{t}}$.
\end{proof}



\subsection{Analysis}


The analysis of \RUCB (respectively \FEWA) only uses Proposition~\ref{prop:prb_favorable_event} and Lemma~\ref{lem:core-RAWUCB} (respectively~\ref{lem:core-FEWA}). We will derive analoguous results for \EFFRAW and \EFFFEWA, which allows us to reproduce very similar upper-bounds on the regret. 
\paragraph{A favorable event for efficiently updated adaptive windows}
\begin{proposition}
\label{prop:prb_favorable_event_eff}
For any round $t$ and confidence $\delta_{t} \triangleq 2t^{-\alpha}$, let 
%
\begin{equation*}
\!\HPeff\! \triangleq\! \Big\{ \forall i\!\in\!\arms,\ \forall n \!\leq\! t\!-\!1 ,\ \forall h_j \in \Him(n), \big| \hmueff(n) - \bmueff(n) \big| \!\leq\! c(h_j, \delta_{t}) \!\Big\}
\end{equation*}
 be the event under which the estimates at round $t$  are all accurate up to $c(h,\delta_{t}) \triangleq \sqrt{2 \subgaussian^2\log(2/\delta_t)/h}$. Then, for a policy $\pi$ which pulls each arms once at the beginning, and for all $t>K$,
\[
\PPempty\Big[\bar{\HPeff}\Big] \leq 3Kt\delta_t= 6Kt^{1-\alpha}\,\cdot
\]
\end{proposition} 
\begin{remark}
The probability of the unfavorable event $\bar{\HPeff}$ scales with $\cO\pa{t^{1-\alpha}}$ compared to $\cO\pa{t^{2-\alpha}}$ for $\bar{\HPevent}$ because the efficient algorithms construct less statistics. It means that our theory will hold for a wider range of $\alpha$. Yet, this benefits is only theoretical : in practice, we know that the union bound technique leads to conservative tuning of the confidence bounds. 
\end{remark}

\begin{proof}
As in Propositions~\ref{prop:prb_favorable_event_SWA} and~\ref{prop:prb_favorable_event}, we have to count the number of statistics that are required to hold in the confidence region. Calling $u_j(t)$ the number of update of statistics $\hmueff$ after $t$ pulls, we have
\begin{align*}
    \PPempty\Big[\bar{\HPeff}\Big] &\leq \sum_{i \in \arms} \sum_{j=0}^{\floor{\log_2\pa{t}}-1} u_j(t) \delta_t \\
    &\leq \sum_{i \in \arms} \pa{t - 1  + \sum_{j=1}^{\floor{\log_2\pa{t}}-1} \frac{t-1}{2^{j-1}}} \delta_t \\
    &\leq 3Kt\delta_t
\end{align*}
In the second inequality, we use Property~\ref{list:effu-m2} in Proposition~\ref{prop:effu}: statistics $\hmueff(n)$ is only updated every $2^{j-1}$ pulls for $j\geq 1$ (and every pull for $j=0$).
\end{proof}

\begin{lemma}
\label{lem:core-eff}
At round $t$ on favorable event $\HPeff$, if arm~$i_{t}$ is selected by $\pi \in \left\{\piEF, \piER\right\}$ tuned with $m=2$, for any $h \leq \Nitmone$,  the average of its $h$ last pulls cannot deviate significantly from the best available arm at that round, i.e.,
\begin{equation*}
\bmu^{h}_{i_t}(t-1,\pi) \geq \min_{j \in \Hitm} \mu_{\ist}(t,\Nitmone)- C_\pi c(h, \delta_t) \quad \text{with } 
\begin{cases}
C_{\piER} = \frac{2\sqrt{2}}{\sqrt{2}-1}\\
C_{\piEF} = \frac{4\sqrt{2}}{\sqrt{2}-1}
\end{cases}\cdot
\end{equation*}
\end{lemma}

\begin{proof}
Like for Lemma~\ref{lem:core-FEWA} (see its proof), our proof is done in a more general rotting framework that can be used in the next chapter. We denote by $\bar{\mu}^{hh'}_i(t-1,\pi)$ and $\hat{\mu}^{hh'}_i(t-1,\pi)$ the true mean and empirical average associated to the $h'-h$ samples between the $h$-th last one (included) and the $h'$-th last one (excluded). Let $j_h \in \NN^\star$ such that :
$2^{j_h} -1 \leq  h < 2^{j_h+1}$.
\begin{equation}
\label{eq:eff-decompo}
\bar{\mu}^{h}_{i_t}(t-1,\pi) \geq \bar{\mu}^{2^{j_h}-1}_{i_t}(t-1,\pi) = \sum_{j=0}^{j_h-1} \frac{2^j}{2^{j_h}-1} \bar{\mu}^{2^{j}2^{j+1}}_{i_t}(t-1,\pi).
\end{equation}
The inequality follows because the reward is decreasing and $h\geq 2^{j_h}-1$. Then, we decompose the average in a weighted sum of averages of geometricly expanding windows. Since the reward is decreasing we have that 
\begin{equation*}
\forall k \leq 2^j, \quad \bar{\mu}^{2^{j}2^{j+1}}_{i_t}(t-1,\pi) \geq \bar{\mu}^{k : k+2^{j}}_{i_t}(t-1,\pi).
\end{equation*}

$\hmuiteff$ contains $2^j$ samples among the $2^{j+1}-1$ last ones (see Proposition~\ref{prop:effu}). Because we are on $\HPeff$,  we can write
\begin{equation}
\label{eq:hmueff-link}
\bar{\mu}^{2^{j}2^{j+1}}_{i_t}(t-1,\pi) \geq \hmuiteff = \bar{\mu}^{k : k+2^{j}}_{i_t}(t-1,\pi)\geq  \hmuiteff - c(2^j, \delta_t),
\end{equation}
with $k\leq 2^j$ the current delay of the statistics $\hmuiteff$. Therefore, gathering Equations~\ref{eq:eff-decompo} and~\ref{eq:hmueff-link}, 
\begin{equation}
\label{eq:eff-general}
\bar{\mu}^{h}_{i_t}(t-1,\pi) \geq \sum_{j=0}^{j_h-1} \frac{2^j}{2^{j_h}-1} \pa{\hmuiteff - c(2^j, \delta_t)}.
\end{equation}
Now, we will use the mechanics of the two algorithms. On the first hand, for $\EFFRAW$, we make the index appear in the inequality,
\begin{align}
 \bar{\mu}^{h}_{i_t}(t-1,\piER) &\geq \sum_{j=0}^{j_h-1} \frac{2^j}{2^{j_h}-1} \pa{\hmuiteff - c(2^j, \delta_t)} \nonumber\\
 &=\sum_{j=0}^{j_h-1} \frac{2^j}{2^{j_h}-1} \pa{\hmuiteff + c(2^j, \delta_t) - 2c(2^j, \delta_t)}\nonumber\\
 &\geq \min_{j \in \Hitm} \pa{\hmuiteff + c(2^j, \delta_t)} - 2 \sum_{j=0}^{j_h-1} \frac{2^{j}}{2^{j_h}-1} c(2^j, \delta_t).
 \label{eq:effraw-index-appear}
 \end{align}
 
 Then, we can relate the left part of the sum to the best current value $\max_{i\in \arms} \mu_i(t, \Nitmone)$,
 \begin{equation}
\min_{j \in \Hitm} \pa{\hmuiteff + c(2^j, \delta_t)} \geq \min_{j \in H_{\ist,m}} \pa{\hmu_{\ist,\, \tteff}^{h_j} + c(2^j, \delta_t)}\geq  \bmu_{\ist,\, \tteff}^{h_{\min}} \geq \mu_{\ist}(t,\Nitmone).
 \label{eq:effraw-index-use}
 \end{equation}
 
where $h_{\min} \in \argmin_{h_j \in \Him} \pa{\hmueff + c(h_j, \delta_t)} $.The first inequality follows because \EFFRAW selects the arm with the largest index. In particular, the index of $i_t$ is larger or equal to the index of $i^\star_t \in \argmax_{i\in \arms} \mu_i(t, N_{\ist,\,t})$. The second inequality holds on $\HPeff$. The third inequality uses the decreasing of the reward. Putting Equations~\ref{eq:effraw-index-appear} and~\ref{eq:effraw-index-use}, we get,

\begin{equation}
\label{eq:effraw-result}
\bmu^{h}_{i_t}(t-1,\piER) \geq \min_{j \in \Hitm} \mu_{\ist}(t,\Nitmone)- 2 \sum_{j=0}^{j_h-1} \frac{2^{j}}{2^{j_h}-1} c(2^j, \delta_t).
\end{equation}

On the other hand, for \EFFFEWA, we know that the selected arm passes any filter of window $2^j \in \Him$. Therefore, with $i_{\max} \in \argmax_{i \in \arms_{h_j}} \bmueff$, we can write,
\begin{flalign}
\qquad\hmuiteff &\geq \max_{i\in \arms_{h_j}} \hmueff -2c\pa{h_j, \delta_t} \nonumber && \text{Filtering rule}\\
\qquad&\geq \hmu_{i_{\max}, \, \tteff}^{h_j}  -2c\pa{h_j, \delta_t} \nonumber  && i_{\max} \in \arms_{h_j} \\
\qquad&\geq \bmu_{i_{\max}, \, \tteff}^{h_j} - 3c(h_j, \delta_t)\nonumber && \text{ on }\HPeff \\
\qquad& = \max_{i\in \arms_{h_j}}  \bmueff -3c\pa{h_j, \delta_t}
\label{eq:efffewa-3c}
\end{flalign}

We relate $\bmueff$ to the largest available value at round $t$,
\begin{equation}
\label{eq:efffewa-ist-relation}
\max_{i\in \arms_{h_j}} \bmueff \geq \max_{i\in \arms_{1}}\bmu_{i,\tteff}^1 =  \max_{i\in \arms}\bmu_{i,\tteff}^1 \geq \bmu_{\ist,\tteff}^1 \geq  \mu_{\ist}(t, \Nitmone).
\end{equation}

The last inequality follows from the decreasing of the reward and the before last from the definition of the maximum operator. The first one uses a similar argument than in Lemma~\ref{lem:core-FEWA} : $\max_{i\in \arms_{h_j}} \bmueff$ increases with $h_j$.  Indeed, on $\HPeff$, $i_j \triangleq \argmax_{i\in \arms_{h_j}} \bmueff$ is in $\arms_{h_{j+1}}$ because it cannot be at more than two confidence bounds from the best empirical value during the filter $h_j$. Thus, we get, 

\[
\max_{i\in \arms_{h_j}} \bmueff = \bmu_{i_j,\tteff}^{h_j} \leq \bmu_{i_j,\tteff}^{h_{j+1}} \leq \max_{i\in \arms_{h_{j+1}}}\bmu_{i,\tteff}^{h_{j+1}}. 
\]

The first inequality follows because $\bmu_{i_j,\tteff}^{h_{j+1}}$ contains reward sample which are either in $\bmu_{i_j,\tteff}^{h_j}$ or are older than the ones in $\bmu_{i_j,\tteff}^{h_j}$. Indeed, when $m=2$, $\hmu_{i, \, \tteff}^{h_{j+1}}$ is updated synchronously with $\hmueff$ (see Property~\ref{list:effu-m2} in Proposition~\ref{prop:effu}). Hence,  at each update of $\hmu_{i, \, \tteff}^{h_{j+1}}$, it contains all the samples of $\hmueff$ and the $2^j$ precedent ones. Thus, because the reward is decreasing, we have $\bmu_{i_j,\tteff}^{h_{j+1}} \geq \bmu_{i_j,\tteff}^{h_{j}} $.  The second inequality uses that $i_j \in \arms_{h_{j+1}}$. Gathering Equations~\ref{eq:eff-general}, \ref{eq:efffewa-3c} and~\ref{eq:efffewa-ist-relation}, we get 
\begin{equation}
\label{eq:efffewa-result}
\bmu^{h}_{i_t}(t-1,\piEF) \geq \min_{j \in \Hitm} \mu_{\ist}(t,\Nitmone)- 4 \sum_{j=0}^{j_h-1} \frac{2^{j}}{2^{j_h}-1} c(2^j, \delta_t).
\end{equation}

With few lines of algebra, we reduce the sum,
\begin{flalign*}
\sum_{j=0}^{j_h-1} \frac{2^{j}}{2^{j_h}-1} c(2^j, \delta_t) &= \sum_{j=0}^{j_h-1} \frac{\sqrt{2}^{j}}{2^{j_h}-1} c(1, \delta_t) && c(2^j, \delta_t) = \frac{c(1,\delta_t)}{\sqrt{2^j}} \\
& = \frac{\sqrt{2}^{j_h} -1 }{\pa{\sqrt{2}-1}\pa{2^{j_h}-1}}c(1, \delta_t) && \sum_{n=0}^N q^n = \frac{q^{N+1}-1}{q-1}\\
& = \frac{1}{\pa{\sqrt{2}-1}\pa{\sqrt{2}^{j_h} +1}} c(1, \delta_t) && 2^{j_h}\!-\!1 \!=\!  \pa{\sqrt{2}^{j_h}\!-\!1}\!\pa{\sqrt{2}^{j_h}\!+\!1}\\
& \leq  \frac{\sqrt{2}}{\pa{\sqrt{2}-1}\sqrt{2^{j_h+1}}} c(1, \delta_t) &&\sqrt{2^{j_h}} +1 \geq \sqrt{2^{j_h}} = \frac{\sqrt{2^{j_h+1}} }{\sqrt{2}} \\
& = \frac{\sqrt{2}}{\sqrt{2}-1} c(2^{j_h+1}, \delta_t) &&  \frac{c(1,\delta_t)}{\sqrt{2^{j_h+1}}} = c(2^{j_h+1},\delta_t)\\
& \leq \frac{\sqrt{2}}{\sqrt{2}-1} c(h, \delta_t). && h \leq 2^{j_h+1} \text{ and }  c(\cdot, \delta) \text{ decreases}
\end{flalign*}

Plugging this last equation in Equations~\ref{eq:effraw-result} and~\ref{eq:efffewa-result} leads to the final result,
\[
\bmu^{h}_{i_t}(t-1,\pi) \geq \min_{j \in \Hitm} \mu_{\ist}(t,\Nitmone)- C_\pi c(h, \delta_t) \quad \text{with } 
\begin{cases}
C_{\piER} = \frac{2\sqrt{2}}{\sqrt{2}-1}\\
C_{\piEF} = \frac{4\sqrt{2}}{\sqrt{2}-1}
\end{cases}\cdot
\]
\end{proof}

Using Proposition~\ref{prop:prb_favorable_event_eff} and Lemma~\ref{lem:core-eff} instead of Prop.~\ref{prop:prb_favorable_event} and Lemmas~\ref{lem:core-FEWA} and~\ref{lem:core-RAWUCB}, we can obtain similar problem dependent and independent bounds than for \FEWA and \RUCB. The proof directly follows from the precedent analysis. 
\begin{restatable}{theorem}{restaalgoindepub}
\label{th:rested-PI}
For any rotting bandit scenario with means $\{\mu_i\}_{i} \in \rewardSet^K$ and any time horizon $T$, $\pi \in \left\{\piER, \piEF \right\}$ run with $\alpha \geq 4$ and $m=2$ suffers an expected regret of
\begin{equation*}
\mathbb{E}[\regret(\pi)] \leq C_\pi\sqrt{2\alpha\sigma^2\log\pa{T}}\pa{\sqrt{KT} +K} + 6KL\,.
\end{equation*}
\end{restatable}
\begin{restatable}{theorem}{restaalgoub}\label{th:rested-PD}
For any rotting bandit scenario with means $\{\mu_i\}_{i} \in \rewardSet^K$ and any time horizon $T$, $\pi \in \left\{\piER, \piEF \right\}$ run with $\alpha \geq 4$ and $m=2$ suffers an expected regret of
\begin{align*}
\mathbb{E}\left[R_T(\pi)\right]  \leq \sum_{i\in \arms} \pa{\frac{2\alpha C_\pi^2\sigma^2\log\pa{T}}{\Delta_{i,\hiT^+-1}} + \sqrt{2\alpha C_\pi^2\sigma^2\log\pa{T}} +6L } \cdot
%\\ 
%\text{\ with  and $\hiT^+$ defined in Equation~\ref{eq:hit+}.}
\end{align*}
\end{restatable}

Among the differences, we notice that our theory holds for a larger range of $\alpha \geq 4$ but the constant $C_\pi$ is $\frac{\sqrt{2}}{\sqrt{2}-1} \sim 3.4$ times larger than their original counter part. We will display multiple experiments where the regret ratio between the algorithms is much smaller than $3.4$. Indeed, to derive Lemma~\ref{lem:core-eff}, we consider that the statistics $\hmueff$ is delayed by at most $h_j$ rounds. According to Proposition~\ref{prop:effu}, the delay is at most $h_j/2$ for $m=2$ (and less than $h_j$ for any $m$). Moreover, these guarantees on the worst case delay are pessimistic: the delay could be much smaller. 

\begin{remark}
\textbf{Can we adapt the theory for $m\neq2$?} %TODO
\end{remark}

\subsection{Experimental Result}






%!TEX root = ../main.tex 
%!TEX root = ../main.tex 

\section{Linear rotting bandits are impossible to learn}
\label{Model} 



\subsection{Linear rested rotting bandits}
In this section, we present our rotting linear bandit framework which recovers 1) the linear model of %TODO name paper
 as soon as the reward is stationnary; and 2) the rotting multi-armed bandits model as soon as $\mathcal{X}$ contains exclusively canonical basis vectors. 

We introduce $d$ non-increasing and $L$-Lipschitz functions $\mu_i : \realset \rightarrow \realset$. These functions satisfies Assumption~\ref{assum-Lipschitz}, but while there were $K$ reward functions defined on $\NN$ in the rotting MAB model, we now have $d$ functions defined on $\realset$. Indeed, in the linear setup the number of reward parameter is $d$ and we expect this value to replace $K$ in the regret bound.

We call $N_{i,t} \triangleq \sum_{t'=0}^t (X_{t'})_i$, which quantifies the amount of pull of direction $i$. We then define the reward : 
\[
o_t(X) = \sum_{i\leq d} \int_{N_{i,t}}^{N_{i,t+1}} \mu_i(x)dx. +\eta_t 
 = \int_{\bm{N_{t}}}^{\bm{N_{t+1}}} \bm{\mu}(\bm{n})^\intercal  d\bm{n} + \eta_t\]

The total reward can thus be writen:  
\[ J(\pi, T) = \int_{\bm{0}}^{\bm{N_{T}}} \bm{\mu}(\bm{n})^\intercal d\bm{n} .\]

Hence, we found a model which extends both rotting MAB model (when the actions are encoded by canonical vectors) and linear bandit model (when the reward is stationnary, i.e. $\bm{\mu}$ is a constant vector function). Moreover, for any vector $\bm{X}$, the reward associated to $\bm{X}$ is decreasing along the pulls while the cumulative reward is totally determined by the number of pull $\bm{N_T}$ and the knowledge of $\bm{\mu}$.

However, one can note that the number of pulls $N_{i,t}$ in the rotting MAB setup has two meaningfull equivalent in the rotting linear setup : $\sum_t x_{i,t}$ and $\sum_t x_{i,t}^2$. The first one is useful in the integral to have linear dependence of the reward with X. The second is usefull from an information theoretic point of view (least square regression) to quantify how much we pulled each direction.

Bandits problems are often considered as the RL problems without state. However, in this setup we do have a state as the next reward depends on the matrix $A_T$. In the  rotting MAB framework, we overcome this issue by showing that the greedy oracle strategy is optimal. Hence, there is no need for planning and the stochastic learning problem is reduced to a pure exploration-exploitation problem where one needs to determine the action which currently performs the best. Therefore, we would like to show that the greedy oracle policy (ie. the policy which selects $\int_{\bm{N_{t}}}^{\bm{N_{t+1}}} \bm{\mu}(\bm{n})^\intercal  d\bm{n}$) is optimal in the rotting linear bandit problem. In the next section, we will show that the greedy oracle policy is not optimal and hence that there is no anytime optimal policy. 

\section{The non-optimality of the greedy oracle policy }
\label{Optimal}
\begin{theorem}
The greedy oracle strategy $\pi_G$ is not optimal. More precisely, for any horizon $T$, there exists a reward vector function $\vec{\mu}$ such as the performance compared to the optimal policy for horizon $T \geq 2$ $\pi_{O_T}$ is :
\[
J(\pi_{O_T}, T) - J(\pi_G, T) \geq \frac{L(T-1)}{8}
\]
\end{theorem}
\begin{proof}
We consider $d = 2$, $\mathcal{X} = \left\{ X_1, X_2 \right\}$ with $X_1 = (1,0)^\intercal$ and $X_2 = (\frac{1}{\sqrt{2}},\frac{1}{\sqrt{2}})^\intercal$. For any horizon $T$, we consider the following reward functions :

\[\mu_1(x) = L \text{ if } x < \frac{T}{2} \text{ else } 0 \qquad \text{and} \qquad  \mu_2(x) = \frac{L}{2}.
\]

The greedy strategy will therefore select $X_1$ until $\floor{\frac{T}{2}}$ and then $X_2$ untill the end of the game. Hence  :
\[J(\pi_G, T) = \int_0^{\floor{\frac{T}{2}} + \ceil{\frac{T}{2}}/2} \mu_1(x)dx + \int_0^{\ceil{\frac{T}{2}}/2}  \mu_2(x)dx = \frac{T}{2}  L + \ceil{\frac{T}{2}} \frac{L}{4} \leq \frac{5LT + L }{8}.\]
We now consider the policy $\pi_2$ which always selects arm 2. At the end of the game, it gathers the reward : 
\[
J(\pi_2, T) = \int_0^{\frac{T}{2}} \mu_1(x)dx + \int_0^{\frac{T}{2}} \mu_2(x)dx = \frac{T}{2} L + \frac{T}{2} \frac{L}{2} = \frac{3LT}{4}
\]

Hence, since optimal policy $\policy_T^\star$ has larger reward than  $\policy_2$ at horizon $T$ (by definition), we have that 
\[
J(\policy_T^\star, T) - J(\pi_G, T) \geq J(\policy_2, T) - J(\pi_G, T) \geq \frac{L(T-1)}{8}
\]

\end{proof}

Hence the greedy policy can be as bad as 8th the regret of the worst performance possible on the problem sets. This is surprising as the greedy oracle strategy was optimal for the rotting MAB problem.  One can note that the vectors used in the proof have the same $L_2$-norm and that the vector function $\vec{\mu}$ is bounded in $[0, L]^2$. The overall setup is simple. We do not need complex decays nor vectors with different "pulling amount" to have a suboptimal performance of the greedy policy. The suboptimality comes from the fact that we do not have access to all the canonical vectors. Hence, when the greedy algorithm has collected all the reward it can get from direction 1, it will start focusing on collecting reward in the second direction. When it pulls the second vector to take advantage of the second direction it also pulls the first direction which is now useless. Here comes some "regret" : the algorithm could have started directly collecting direction 2 as it would have got all the direction 1 benefits anyway. The following corollary underlines that the failure of the greedy oracle strategy implies the necessity of planning.

\begin{lemma}
For a cumulative reward exploration exploitation problem, the only possible anytime optimal oracle strategy is the greedy oracle one.
\end{lemma}
\begin{proof}
Let's assume $\pi_{O_a}$ an anytime optimal strategy which does not select the greedy action at time $T$.
Let's consider $\pi_{G_T}$ a strategy which copies $\pi_{O_a}$ for the $T-1$ round and is greedy at round $T$.
\[
J(\pi_{O_a}, T) - J(\pi_{G_T},T) = r_T(\pi_{O_a}(T)) - r_T(\pi_{G_T}(T)) < 0
\]
where the last inequality comes from the fact that $r_T(\pi_{O_a}(T))$ is below the best reward available for that time. 
\end{proof}

\begin{corollary}
There is no anytime optimal oracle strategy for the rotting linear bandit model. 
\end{corollary}

What is the regret of a short-sighted oracle strategy which sees F steps in the future (ie . knows $\mu_i$ from $0$ up to $a_{ii,t}^2 + F \max_d X_{d,i}^2$?)
\begin{theorem}
Any strategy which can anticipate the future up to $F$ steps in advance has a worst case regret which scales at least with $O(T-2F)$. More precisely : 
\[
\max_\mu R(\pi, T) \geq  \frac{L(T-2F)}{12} - \frac{L}{6}
\]
\end{theorem}
\begin{proof}
We still consider $d = 2$, $\mathcal{X} = \left\{ X_1, X_2 \right\}$ with $X_1 = (1,0)^\intercal$ and $X_2 = (\frac{1}{\sqrt{2}},\frac{1}{\sqrt{2}})^\intercal$. For any horizon $T$, we consider the following reward functions :

\[\mu_1^1(x) = L \quad \text{and} \quad\mu_1^2(x) = L \text{ if } x < \frac{T}{2} \text{ else } 0 \quad \text{and} \quad  \mu_2(x) = \frac{L}{2}.
\]

The optimal strategy associated to $\mu_1^1$ (respectively $\mu_1^2$) is $\pi_O \triangleq \pi_1$  (resp. $\pi_O \triangleq\pi_2$), the policy which always pulls the first (resp. second) arm, and it gathers the cumulative reward $J_1(\pi_O, T)$ (resp. $J_2(\pi_O, T)$). We have by simple calculations:
\[
J_1(\pi_O, T) = LT \quad \text{and} \quad J_2(\pi_O, T) = \frac{3TL}{4} 
\]

Depending on whether $\mu_1$ is $\mu_1^1$ or $\mu_1^2$, we can express the regret as a function of $N_{1,T}$ or $N_{2,t}$.
\begin{align}
R_1(\pi, T) \triangleq J_1(\pi_O, T) - J_1(\pi_{t_f},T) = LT -  L (T- N_{2,T}) -  N_{2,T} \frac{3L}{4}  = \frac{LN_{2,T}}{4}\\
R_2(\pi, T) \triangleq J_2(\pi_O, T) - J_2(\pi_{t_f},T) = \frac{3LT}{4} -  \frac{LT}{2}  -  (T- N_{1,T}) \frac{L}{4}  = \frac{LN_{1,T}}{4}
\end{align}


 
We call $t_f$ the first time such that $||\epsilon_1 ||_{A_{t_f}} \geq \frac{T}{2} - F$. After $t_f$ the learner entirely knows which reward functions she faced and before $t_f$ the two reward functions are undistinguishable to the learner.  Note that for any policy, $||\epsilon_1 ||_{A_{T}} \geq \frac{T}{2}  > \frac{T}{2} - F$, hence $t_f$ exists for any policy.  We have that
\begin{align}
& N_{1, t_f} + \frac{N_{2, t_f}}{2} \geq \frac{T}{2} - F  \\
& N_{1, t_f} + \frac{N_{2, t_f}}{2} \leq \frac{T}{2} - F + 1 \\
& N_{1, t_f} + N_{2, t_f} = t_f
\end{align}

Hence, we have the following lowerbound for $N_{i,T}$:
\begin{align}
& N_{1,T} \geq N_{1, t_f} \geq T - t_f - 2F  \\
& N_{2,T} \geq N_{2, t_f} \geq 2 t_f - T + 2(F-1) 
\end{align}

Hence, worst case regret is :
\[
\max_\mu R(\pi, T) \geq \max( R_1(\pi,T), R_2(\pi, T)) = \frac{L}{4} \max_{0 \leq t_f \leq T}(T - t_f - 2F, 2 t_f - T + 2(F-1) ) \geq  \frac{L(T-2F)}{12} - \frac{L}{6}
\]

\end{proof}

Note we can slightly modify the proof to get $O(T -\alpha F)$ for $\alpha >1$.

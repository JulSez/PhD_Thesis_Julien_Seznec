%!TEX root = ../main.tex 
\section{Efficient algorithms}
\label{app:efficient_alg}
\subsection{The numerical cost of adaptive windows}

In the three last sections, we presented two adaptive windows algorithms whose significantly improved over state-of-the-art algorithms, both theoretically and experimentally. Yet, we highlight that these improvements are computationally expensive. Indeed, at each round $t$, we store, update and compare $\cO\pa{t}$ statistics. 

The full update of the statistics can be done at a worst case cost of $\cO\pa{t}$. Indeed, each statistics $\hmu_i^h$ can be refreshed with a $\cO\pa{1}$ operation : 
\[\hmu_i^{h+1}(n+1) = \frac{h}{h+1}\hmu_i^{h}(n) + \frac{1}{h+1}o_t \,. \]

The comparison part in both \FEWA and \RUCB is also a $\cO\pa{t}$ operations. In \FEWA , we do a scan based on $\hmu_i^{h}$ for all $i \in \arms_h$ with increasing $h$. Hence, the total number of unitary operation is in $\cO\pa{t}$ in the worst case, as it scales with the number of statistics. \RUCB computes one UCB for each of the $\cO\pa{t}$ statistics. For each arm, it selects the minimum UCB as index, which can be done with complexity $\cO\pa{t}$. Finally, finding the largest index is an $\cO\pa{K}$ operations. Therefore, we can conclude,

\begin{proposition}
\FEWA and \RUCB have a $\cO\pa{t}$ worst-case complexity per round $t$ in time and memory.
\end{proposition}

\begin{remark}
\SWA($h$) has a $\cO\pa{h}$ worst-case complexity in time and memory because the sliding-window mechanism need to store and update $\cO\pa{h}$ statistics to always have the average of the $h$ last sample ready. Hence, when it is optimally tuned for the minimax bound, $\SWA$ has a $\cO\pa{T^{2/3}}$ per round complexity. As often in non-stationary bandits, it may be possible to replace sliding window statistics by discounted statistics. Such modification often leads to slightly worse theoretical regret rate but to a much better $\cO\pa{K}$ complexity. 
\end{remark}

Hence, handling a large number of windows, which is the main strength of our algorithms to achieve a lower regret, is a significant drawback when it comes to design fast algorithm. Therefore, it is an open question whether one can enjoy the benefits of adaptive windows without suffering large time and space complexity. 

\subsection{The efficient update trick}
We detail \EFF, an update scheme to handle efficiently statistics of different windows. A similar yet different approach has appeared independently in the context of streaming mining~\citep{bifet2007learning}. \EFF is built around two main ideas.

First, at any time $t$ we can avoid using $\left\{\hmu_i^h\right\}_{h}$ for all possible windows $h$ starting from 1 with an increment of 1. In fact, both statistics $\hmu_i^h$ and constructed confidence levels $c(h, \delta_t)$  have very close value for successive $h$ as $h$ becomes large : 
\begin{align*}
& \hmu_i^{h+1}(n) = \hmu_i^{h}(n) + \cO\pa{\frac{\sigma + L}{h}}\,,\\
& c(h+1, \delta_t) = c(h, \delta_t) + \cO\pa{\frac{\sigma }{h^{3/2}}} \,.
\end{align*}
Hence, in both \FEWA and \RUCB, we compute a lot of very similar quantities. Instead, we could use fewer statistics which are significantly different : $\left\{\hmu_i^h(\Nitmonepi)\right\}_{h\in \Him}$, where the window $h$ is dispatched on a geometric grid, 
 \[\Him\pa{\Nitmonepi} \triangleq \left\{ h_j \in  \left\{1, \dots , \Nitmonepi \right\} \;|\; h_{j+1} = \ceil{m \cdot h_j} \text{ and } h_1 = 1\right\}\quad \text{with } m > 1.\]

When there is no confusion, we drop the dependency in $\Nitmonepi$.  This modification alone is not enough to reduce both the time and space complexity. Indeed, updating $\hmu^h_{i}$ requires to replace the $h$-th last sample by the new one $o_t$. Hence, we need to store all the collected statistics to be able to update all the $\hmu^h_{i}$ for all $h$ with $\cO\pa{1}$ complexity. Therefore, in \EFF, we will use $\cO\pa{K\log\pa{t}}$ \emph{delayed} statistics that we can update with $\cO\pa{K\log\pa{t}}$ space and time complexity.

\EFF (Alg.~\ref{alg:effupdate}) takes as input the new observation $o_t$ that the learner gets at the $N_i$-th pull of arm $i$; the geometric window grid $\Him$ tuned with an hyperparameter $m>1$, and for each window $h_j$ in this grid, three different numbers $\hmueff,\; \peff, \; \neff$. $\left\{\hmueff\right\}_{i,h_j}$ represents the set of \emph{current} statistics of window size $h_j$ that will be used instead of $\left\{\hmu_i^h\right\}_{i,h}$ in our efficient algorithms. We also store a pending statistic $\peff$ and a count $\neff$  which are used in the sparse update procedure of $\hmueff$. \EFF outputs an updated set of statistics.  

\begin{minipage}{\textwidth}
\renewcommand*\footnoterule{}
\begin{savenotes}
\begin{algorithm}[H]
\caption{{\small\sc Eff\_Update}}
\begin{algorithmic}[1]
\label{alg:effupdate}
\Require $o_t$, \small $\Him \gets \left\{h_j \! <\! \ceil{m \cdot N_i} \; | \;  h_{j+1} \!=\! \ceil{m \cdot h_j}  \text{with } h_0 \!=\! 1\right\} $\normalsize, $\left\{ \{ \hmueff,\, \peff, \, \neff \}\right\}_{h_j \in \Him}$
\If{$N_i = \max\pa{\Him}$}\label{algline:effu-new-condition} \Comment{Create a new triplet with window $h_j = \ceil{m \cdot N_i}$}  
\State $\Him \gets \Him \cup \left\{ \ceil{m \cdot N_i} \right\}$\label{algline:effu-new-h}
\State $p_i^{\ceil{m \cdot N_i} } = p_i^{N_i} $\label{algline:effu-new-p}
\State $n_i^{\ceil{m \cdot N_i} } \gets n_i^{N_i} $\label{algline:effu-new-n}
\State $\hmu_{i, \, \tteff}^{\ceil{m \cdot N_i} }\leftarrow \texttt{None}$\label{algline:effu-new-mu}
\EndIf\label{algline:effu-new-end} 
\State $p_i^{1} \gets o_t$ \label{algline:effu-update-first-p} \Comment{Update the first triplet with $o_t$}
\State $n_i^{1} \gets 1$\label{algline:effu-update-first-n}
\State $\hmu_{i, \, \tteff}^{1}\leftarrow o_t$ \label{algline:effu-update-first-hmu}
\For{$h_j \in  \Him \smallsetminus \left\{ 1\right\} $}\label{algline:effu-update-start} \Comment{Update the other pending statistics $\peff$ and $\neff$}
\State $p_i^{h_j} \gets p_i^{h_j}  +o_t$\label{algline:effu-update-p}
\State $n_i^{h_j} \gets n_i^{h_j} + 1$\label{algline:effu-update-n}
\EndFor\label{algline:effu-update-end} 
\For{$h_j \in  $ \textsc{Sort\_Desc}$\pa{\Him \smallsetminus \left\{ 1\right\} }$}\label{algline:effu-refresh-start}
\If{$n_i^{h_j} = h_j$} \label{algline:effu-refresh-condition}
\State $\hmueff \leftarrow p_i^{h_j}/h_j$ \Comment{Replace the current statistic $\hmueff$}\label{algline:effu-refresh-hmu}
\State{$p_i^{h_{j}} = p_i^{h_{j-1}} $} \label{algline:effu-refresh-p}\Comment{Refresh the pending statistics}
\State $n_i^{h_{j}} \gets n_i^{h_{j-1}} $\label{algline:effu-refresh-n}
\EndIf
\EndFor \label{algline:effu-refresh-end}
\Ensure $\left\{\left\{  \hmueff,\; p_i^{h_j}, \; n_i^{h_j} \right\}\right\}_{h_j \in \Him}$
\end{algorithmic}
\end{algorithm}
\end{savenotes}
\end{minipage}




The core of \EFF is divided in four parts: 1) From Lines~\ref{algline:effu-new-condition} to~\ref{algline:effu-new-end}, we create new statistics at a logarithmic rate with respect to the growth of $N_i$; 2) From Lines~\ref{algline:effu-update-first-p} to~\ref{algline:effu-update-first-hmu}, we update the statistics of window $h_1=1$;
3) From Lines~\ref{algline:effu-update-start} to~\ref{algline:effu-update-end}, we update the other pending statistics and count;
4) From Lines~\ref{algline:effu-refresh-start} to~\ref{algline:effu-refresh-end}, we eventually update $\hmueff$ and refresh the correspounding pending statistic and count. The remaining details are quite technical. Thus, we first give the high-level properties that are ensured by the recursive usage of \EFF. Then, we prove them by going through the algorithm line by line.

\begin{proposition}
\label{prop:effu}
 $\left\{\left\{  \hmueff,\; p_i^{h_j}, \; n_i^{h_j} \right\}\right\}_{h_j \in \Him}$, constructed recursively with \EFF with initial value $\left\{\left\{  \hmu_{i,\,\tteff}^1 : \texttt{None},\; p_i^{1} :0 , \; n_i^{1}:0 \right\}\right\}$ have the following properties :
 \begin{enumerate}[topsep=0pt]
  \item $\hmueff$ is the average of exactly $h_j$ consecutive samples among the $2h_j -1$ last ones. \label{list:effu-hmu}
  \item The delay between two updates of $\hmueff$ is in $\left\{\ceil{\frac{m-1}{m} h_j}, \dots, h_j -1\right\}$.\label{list:effu-delay}
  \item When $m = 2$, $h_j = 2^{j}$. Moreover, for $j\geq1$,  the $k$-th update $\hmueff$ happens at pull $ \pa{k+1} \cdot 2^{j-1}$, \ie every $2^{j-1}$ pulls (and at every rounds for $j=0$).\label{list:effu-m2}
  \item $\peff$ is the sum of the $\neff$ last samples. \label{list:effu-p}
  \item $\neff < h_j$ for $j\geq 1$. Also, $n_i^1 \leq 1$.\label{list:effu-n1}
  \item $\left\{ \neff \right\}_{h_j}$ is an non-decreasing sequence with respect to $h_j$ (or $j$).\label{list:effu-n2}
 \end{enumerate}
\end{proposition}
\begin{proof}
The three last properties are trivially true at the initialization. Thus, we show by induction that they remain true after updates.
\paragraph{Proof of \ref{list:effu-p}. } At Lines~\ref{algline:effu-new-p} and~\ref{algline:effu-new-n}, we create a new pending statistics and count by initializing them with other statistics and counts. Hence, because of the recursion hypothesis, all the pending statistics $\peff$ (including the created one) contains the sum of the $\neff$ \emph{before last} pulls. At Lines~\ref{algline:effu-update-first-p} and~\ref{algline:effu-update-first-n}, we update $p_i^1$ with the last sample and set $n_i^1$ to $1$. At Lines~\ref{algline:effu-update-p} and~\ref{algline:effu-update-n}, we add the last sample to $\peff$ (which was containing the before last samples) and increase the count by $1$. Hence, at the end of Line~\ref{algline:effu-update-n}, all the $\peff$ contains the sum of the last $\neff$ samples. Thus, refreshing $\peff$ and $\neff$ with $ p_i^{h_{j-1}}$ and $n_i^{h_{j-1}}$ keeps this property true (Lines~\ref{algline:effu-refresh-p} and~\ref{algline:effu-refresh-n}). 

\paragraph{Proof of \ref{list:effu-n1}.}
For $j=0$, $n_i^1$, which is equal to $0$ at the initialization, is set at $1$ at every update (Line~\ref{algline:effu-update-first-n}). Hence, we have $n_i^{h_0} \leq h_0=1$.
For $j\geq 1$, $n_i^{\ceil{m \cdot N_i}}$ is initialized at Line~\ref{algline:effu-new-n} with the value $n_i^{N_i} < N_i < \ceil{m \cdot N_i}$ by the induction hypothesis and because $m>1$.  Then, $\neff < h_j$ ($j\geq1$) is increased by one at each update at Line~\ref{algline:effu-update-n}. Hence, we now have $\neff \leq  h_j$ for all $j\in\Him$. However, for $j\geq 1$, if $\neff = h_j$ (Line~\ref{algline:effu-refresh-condition}), it is replaced by the precedent count $n_i^{h_{j-1}}\leq h_{j-1} < h_j$ (Line~\ref{algline:effu-update-n}).Thus, at the end of the update, we do have $\neff < h_j$ for $j\geq1$.

\paragraph{Proof of \ref{list:effu-n2}.}
At Line~\ref{algline:effu-new-n}, we create a new pending count corresponding to the largest $h_j$ and we initialize it with the precedent largest count. At Lines~\ref{algline:effu-update-first-n} and~\ref{algline:effu-update-n}, we set $n_i^1 =1$ and increase all the other $\neff$ by one. This operation preserves the non-decreasing property of the ordered set. Last, at Line~\ref{algline:effu-refresh-n}, we set few counts $\neff$ to the precedent value $n_i^{h_{j-1}}$- which also preserves the non-decreasing property of the ordered set. 

\paragraph{Proof of \ref{list:effu-hmu} and \ref{list:effu-delay}.}
Thanks to Property~\ref{list:effu-p}, we know that $\peff$ is the sum of the $\neff$ last sample. It is still true at the end of Line~\ref{algline:effu-update-n} (see the proof). Then, at Line~\ref{algline:effu-refresh-hmu}, and given the condition in Line~\ref{algline:effu-refresh-condition}, we set $\hmueff$ with the average of the last $h_j$ sample. Then, $\hmueff$ is not updated untill the condition at Line~\ref{algline:effu-refresh-condition} is fulfilled again. 

$\neff$ is refreshed with a quantity larger or equal to $1$ and smaller or equal to $h_{j-1}$ at Line~\ref{algline:effu-refresh-n}. Then, it is increased by one at each update. we know that $\hmueff$ will be updated at least every $h_j-1$, and at most every $h_j -h_{j-1}$ round. Hence, considering the worst possible delay we can conclude : $\hmueff$ is the average of exactly $h_j$ consecutive samples among the $2h_j -1$ last ones. Last, considering that $h_{j-1}\leq h_j /m$, we conclude that the minimal delay is larger or equal to $\frac{m-1}{m}h_j$.

\paragraph{Proof of \ref{list:effu-m2}.}
When $m=2$, it is easy to find by induction that,
 \[
 h_{j+1} = \ceil{m\cdot h_j} = 2h_j = 2^{j+1}.
 \]
For $j=0$, $\hmu_{i,\,\tteff}^1$ is updated at every update at Line~\ref{algline:effu-update-first-hmu}.
By induction on $j\geq 1$, $\hmueff$ is initialized (Line~\ref{algline:effu-refresh-hmu}) for the first time after $h_j= 2^{j} = 4 \cdot 2^{j-2}$ pulls. Therefore, it is also an updating pull for $\hmu_{i,\,\tteff}^{h_{j-1}}$ (by the induction hypothesis) and $n_j$ is set with $n_{j-1} = 2^{j-1}$ at Line~\ref{algline:effu-refresh-n}. Notice that we sort $\Him$ in the decreasing order at Line~\ref{algline:effu-refresh-start}, hence $n_j$ is updated with $n_{j-1}$ before it is itself updated with $n_{j-2}$.  Hence, $\hmueff$ is updated again in $h_j - 2^{j-1} = 2^{j-1}$ pulls, \ie after $6 \cdot 2^{j-2}$ pulls of arm $i$. Again, $n_j$ is set with $n_{j-1} = 2^{j-1}$ (because it is an updating pull for $\hmu_{i,\,\tteff}^{h_{j-1}}$). By induction, we see that the $k$-th update happens at pull $ \pa{k+1} \cdot 2^{j-1}$, \ie every $2^{j-1}$ pulls.


\end{proof}

\begin{remark}
At Line~\ref{algline:effu-refresh-n}, we refresh $\neff$ with $n_i^{h_{j-1}}$ which is often larger than $1$. Indeed, we could refresh $\peff$ and $\neff$ at $0$. Yet, in order to reduce the delay in the update, we use the variable available in the memory which contains the sums of $h$ last sample, with the largest $h< h_j$. According to Properties~\ref{list:effu-p},~\ref{list:effu-n1} and~\ref{list:effu-n2}, this quantity is $p_i^{h_{j-1}}$. 

Notice that we also sort $\Him$ in the decreasing order at Line~\ref{algline:effu-refresh-start} to minimize the delay: if there is two consecutive updates of $\hmueff$ and $\hmu_{i,\, \tteff}^{h_{j+1}}$ at the same run of \EFF, doing a backward loop guarantees to refresh $n_i^{h_{j+1}}$ with a larger value than with a forward loop. 
\end{remark}
%Property~\ref{list:effu-delay} in Proposition~\ref{prop:effu} states that the delay in the update is at most $h_j-1$ in the worst case. Yet, notice that it is often much less: 


\subsection{{\EFFFEWA} and {\EFFRAW}}
{\EFFFEWA} and {\EFFRAW} are the two efficient versions of our initial algorithms. With an hyperparameter $m>1$, they use \EFF instead of \UPDATE (Lines~\ref{algline:fewa-update1} and~\ref{algline:fewa-update2} in \FEWA and Lines~\ref{algline:raw-update1} and~\ref{algline:raw-update2} in \RUCB). Therefore, they use $\left\{\hmueff\right\}_{i,h_j\in \Him}$ instead of $\left\{\hmu_i^h\right\}_{i,h \leq \Nitmone}$. 

More precisely, in \FEWA, we replace the increment $h\gets h+1$ by $h\gets\ceil{m\cdot h}$ at Line~\ref{algline:fewa-window}. Hence, the next set is not called $\arms_{h+1}$ but $\arms_{\ceil{m\cdot h}}$ (Line~\ref{algline:fewa-filter} in \FEWA and Line~\ref{algline:filter-add} in \FILTER). Finally, at Lines~\ref{algline:fewa-condition} and~\ref{algline:fewa-pull}, the condition is not $N_{i_t}=h$ but $N_{i_t} \leq h$. In the \FILTER procedure, we also change $\hmu_i^h$ by $\hmu_{i,\,\tteff}^h$ at Lines~\ref{algline:filter-max} and~\ref{algline:filter-delta}. In \RUCB, we only change the $h\leq N_i$ by $h_j \in \Him$ and $\hmu_i^h$ by $\hmu_{i,\,\tteff}^h$ in the index computation at Line~\ref{algline:raw-pull}.

\begin{proposition}
\EFFFEWA and \EFFRAW tuned with hyperparmaeter $m$ have a $\cO\pa{K\log_m\pa{t}}$ worst-case time and space complexity at round $t$.
\end{proposition}
\begin{proof}
The total number of statistics for each arm $i$ at round $t$ is bounded by $\cO\pa{\log_m\pa{t}}$. Indeed, 
\[t \geq \Nitmone \geq h_j \geq m^{j-1} \implies j \leq 1 + \log_m\pa{t}.\]
Moreover, in \EFF we use 3 numbers for each $\left\{\hmueff\right\}_j$. Hence, the space complexity scales with \[ \sum_{i\in \arms} | \Him| = \sum_{i\in \arms} \cO\pa{\log_m\pa{t}} = \cO\pa{K\log_m\pa{t}}.\]
The time complexity of $\EFF$ scales with the number of statistics in arm $i_t$, \ie at most $\cO\pa{\log_m\pa{t}}$. The indexes computation of \EFFRAW  find the minimum of $K$ sets with cardinality $\cO\pa{\log_m\pa{t}}$, while finding the maximum among these indexes is a $\cO\pa{K}$ operation.  Thus, the worst-case time complexity is $\cO\pa{K\log_m\pa{t}}$. \EFFFEWA uses at most $\cO\pa{\log_m\pa{t}}$ times the procedure \FILTER  whose inner complexity scales with $|\arms_h| \leq K$. Therefore, in the worst case, the time complexity of \EFFFEWA at round $t$ is bounded by $\cO\pa{K\log_m\pa{t}}$.
\end{proof}



\subsection{Analysis}


The analysis of \RUCB (respectively \FEWA) only uses Proposition~\ref{prop:prb_favorable_event} and Lemma~\ref{lem:core-RAWUCB} (respectively~\ref{lem:core-FEWA}). We will derive analoguous results for \EFFRAW and \EFFFEWA, which allows us to reproduce very similar upper-bounds on the regret. 
\paragraph{A favorable event for efficiently updated adaptive windows}
\begin{proposition}
\label{prop:prb_favorable_event_eff}
For any round $t$ and confidence $\delta_{t} \triangleq 2t^{-\alpha}$, let 
%
\begin{equation*}
\!\HPeff\! \triangleq\! \Big\{ \forall i\!\in\!\arms,\ \forall n \!\leq\! t\!-\!1 ,\ \forall h_j \in \Him(n), \big| \hmueff(n) - \bmueff(n) \big| \!\leq\! c(h_j, \delta_{t}) \!\Big\}
\end{equation*}
 be the event under which the estimates at round $t$  are all accurate up to $c(h,\delta_{t}) \triangleq \sqrt{2 \subgaussian^2\log(2/\delta_t)/h}$. Then, for a policy $\pi$ which pulls each arms once at the beginning, and for all $t>K$,
\[
\PPempty\Big[\bar{\HPeff}\Big] \leq 3Kt\delta_t= 6Kt^{1-\alpha}\,\cdot
\]
\end{proposition} 
\begin{remark}
The probability of the unfavorable event $\bar{\HPeff}$ scales with $\cO\pa{t^{1-\alpha}}$ compared to $\cO\pa{t^{2-\alpha}}$ for $\bar{\HPevent}$ because the efficient algorithms construct less statistics. It means that our theory will hold for a wider range of $\alpha$. Yet, this benefits is only theoretical : in practice, we know that the union bound technique leads to conservative tuning of the confidence bounds. 
\end{remark}

\begin{proof}
As in Propositions~\ref{prop:prb_favorable_event_SWA} and~\ref{prop:prb_favorable_event}, we have to count the number of statistics that are required to hold in the confidence region. Calling $u_j(t)$ the number of update of statistics $\hmueff$ after $t$ pulls, we have
\begin{align*}
    \PPempty\Big[\bar{\HPeff}\Big] &\leq \sum_{i \in \arms} \sum_{j=0}^{\floor{\log_2\pa{t}}-1} u_j(t) \delta_t \\
    &\leq \sum_{i \in \arms} \pa{t - 1  + \sum_{j=1}^{\floor{\log_2\pa{t}}-1} \frac{t-1}{2^{j-1}}} \delta_t \\
    &\leq 3Kt\delta_t
\end{align*}
In the second inequality, we use Property~\ref{list:effu-m2} in Proposition~\ref{prop:effu}: statistics $\hmueff(n)$ is only updated every $2^{j-1}$ pulls for $j\geq 1$ (and every pull for $j=0$).
\end{proof}

\begin{lemma}
\label{lem:core-eff}
At round $t$ on favorable event $\HPeff$, if arm~$i_{t}$ is selected by $\pi \in \left\{\piEF, \piER\right\}$ tuned with $m=2$, for any $h \leq \Nitmone$,  the average of its $h$ last pulls cannot deviate significantly from the best available arm at that round, i.e.,
\begin{equation*}
\bmu^{h}_{i_t}(t-1,\pi) \geq \min_{j \in \Hitm} \mu_{\ist}(t,\Nitmone)- C_\pi c(h, \delta_t) \quad \text{with } 
\begin{cases}
C_{\piER} = \frac{2\sqrt{2}}{\sqrt{2}-1}\\
C_{\piEF} = \frac{4\sqrt{2}}{\sqrt{2}-1}
\end{cases}\cdot
\end{equation*}
\end{lemma}

\begin{proof}
Like for Lemma~\ref{lem:core-FEWA} (see its proof), our proof is done in a more general rotting framework that can be used in the next chapter. We denote by $\bar{\mu}^{hh'}_i(t-1,\pi)$ and $\hat{\mu}^{hh'}_i(t-1,\pi)$ the true mean and empirical average associated to the $h'-h$ samples between the $h$-th last one (included) and the $h'$-th last one (excluded). Let $j_h \in \NN^\star$ such that :
$2^{j_h} -1 \leq  h < 2^{j_h+1}$.
\begin{equation}
\label{eq:eff-decompo}
\bar{\mu}^{h}_{i_t}(t-1,\pi) \geq \bar{\mu}^{2^{j_h}-1}_{i_t}(t-1,\pi) = \sum_{j=0}^{j_h-1} \frac{2^j}{2^{j_h}-1} \bar{\mu}^{2^{j}2^{j+1}}_{i_t}(t-1,\pi).
\end{equation}
The inequality follows because the reward is decreasing and $h\geq 2^{j_h}-1$. Then, we decompose the average in a weighted sum of averages of geometricly expanding windows. Since the reward is decreasing we have that 
\begin{equation*}
\forall k \leq 2^j, \quad \bar{\mu}^{2^{j}2^{j+1}}_{i_t}(t-1,\pi) \geq \bar{\mu}^{k : k+2^{j}}_{i_t}(t-1,\pi).
\end{equation*}

$\hmuiteff$ contains $2^j$ samples among the $2^{j+1}-1$ last ones (see Proposition~\ref{prop:effu}). Because we are on $\HPeff$,  we can write
\begin{equation}
\label{eq:hmueff-link}
\bar{\mu}^{2^{j}2^{j+1}}_{i_t}(t-1,\pi) \geq \hmuiteff = \bar{\mu}^{k : k+2^{j}}_{i_t}(t-1,\pi)\geq  \hmuiteff - c(2^j, \delta_t),
\end{equation}
with $k\leq 2^j$ the current delay of the statistics $\hmuiteff$. Therefore, gathering Equations~\ref{eq:eff-decompo} and~\ref{eq:hmueff-link}, 
\begin{equation}
\label{eq:eff-general}
\bar{\mu}^{h}_{i_t}(t-1,\pi) \geq \sum_{j=0}^{j_h-1} \frac{2^j}{2^{j_h}-1} \pa{\hmuiteff - c(2^j, \delta_t)}.
\end{equation}
Now, we will use the mechanics of the two algorithms. On the first hand, for $\EFFRAW$, we make the index appear in the inequality,
\begin{align}
 \bar{\mu}^{h}_{i_t}(t-1,\piER) &\geq \sum_{j=0}^{j_h-1} \frac{2^j}{2^{j_h}-1} \pa{\hmuiteff - c(2^j, \delta_t)} \nonumber\\
 &=\sum_{j=0}^{j_h-1} \frac{2^j}{2^{j_h}-1} \pa{\hmuiteff + c(2^j, \delta_t) - 2c(2^j, \delta_t)}\nonumber\\
 &\geq \min_{j \in \Hitm} \pa{\hmuiteff + c(2^j, \delta_t)} - 2 \sum_{j=0}^{j_h-1} \frac{2^{j}}{2^{j_h}-1} c(2^j, \delta_t).
 \label{eq:effraw-index-appear}
 \end{align}
 
 Then, we can relate the left part of the sum to the best current value $\max_{i\in \arms} \mu_i(t, \Nitmone)$,
 \begin{equation}
\min_{j \in \Hitm} \pa{\hmuiteff + c(2^j, \delta_t)} \geq \min_{j \in H_{\ist,m}} \pa{\hmu_{\ist,\, \tteff}^{h_j} + c(2^j, \delta_t)}\geq  \bmu_{\ist,\, \tteff}^{h_{\min}} \geq \mu_{\ist}(t,\Nitmone).
 \label{eq:effraw-index-use}
 \end{equation}
 
where $h_{\min} \in \argmin_{h_j \in \Him} \pa{\hmueff + c(h_j, \delta_t)} $.The first inequality follows because \EFFRAW selects the arm with the largest index. In particular, the index of $i_t$ is larger or equal to the index of $i^\star_t \in \argmax_{i\in \arms} \mu_i(t, N_{\ist,\,t})$. The second inequality holds on $\HPeff$. The third inequality uses the decreasing of the reward. Putting Equations~\ref{eq:effraw-index-appear} and~\ref{eq:effraw-index-use}, we get,

\begin{equation}
\label{eq:effraw-result}
\bmu^{h}_{i_t}(t-1,\piER) \geq \min_{j \in \Hitm} \mu_{\ist}(t,\Nitmone)- 2 \sum_{j=0}^{j_h-1} \frac{2^{j}}{2^{j_h}-1} c(2^j, \delta_t).
\end{equation}

On the other hand, for \EFFFEWA, we know that the selected arm passes any filter of window $2^j \in \Him$. Therefore, with $i_{\max} \in \argmax_{i \in \arms_{h_j}} \bmueff$, we can write,
\begin{flalign}
\qquad\hmuiteff &\geq \max_{i\in \arms_{h_j}} \hmueff -2c\pa{h_j, \delta_t} \nonumber && \text{Filtering rule}\\
\qquad&\geq \hmu_{i_{\max}, \, \tteff}^{h_j}  -2c\pa{h_j, \delta_t} \nonumber  && i_{\max} \in \arms_{h_j} \\
\qquad&\geq \bmu_{i_{\max}, \, \tteff}^{h_j} - 3c(h_j, \delta_t)\nonumber && \text{ on }\HPeff \\
\qquad& = \max_{i\in \arms_{h_j}}  \bmueff -3c\pa{h_j, \delta_t}
\label{eq:efffewa-3c}
\end{flalign}

We relate $\bmueff$ to the largest available value at round $t$,
\begin{equation}
\label{eq:efffewa-ist-relation}
\max_{i\in \arms_{h_j}} \bmueff \geq \max_{i\in \arms_{1}}\bmu_{i,\tteff}^1 =  \max_{i\in \arms}\bmu_{i,\tteff}^1 \geq \bmu_{\ist,\tteff}^1 \geq  \mu_{\ist}(t, \Nitmone).
\end{equation}

The last inequality follows from the decreasing of the reward and the before last from the definition of the maximum operator. The first one uses a similar argument than in Lemma~\ref{lem:core-FEWA} : $\max_{i\in \arms_{h_j}} \bmueff$ increases with $h_j$.  Indeed, on $\HPeff$, $i_j \triangleq \argmax_{i\in \arms_{h_j}} \bmueff$ is in $\arms_{h_{j+1}}$ because it cannot be at more than two confidence bounds from the best empirical value during the filter $h_j$. Thus, we get, 

\[
\max_{i\in \arms_{h_j}} \bmueff = \bmu_{i_j,\tteff}^{h_j} \leq \bmu_{i_j,\tteff}^{h_{j+1}} \leq \max_{i\in \arms_{h_{j+1}}}\bmu_{i,\tteff}^{h_{j+1}}. 
\]

The first inequality follows because $\bmu_{i_j,\tteff}^{h_{j+1}}$ contains reward sample which are either in $\bmu_{i_j,\tteff}^{h_j}$ or are older than the ones in $\bmu_{i_j,\tteff}^{h_j}$. Indeed, when $m=2$, $\hmu_{i, \, \tteff}^{h_{j+1}}$ is updated synchronously with $\hmueff$ (see Property~\ref{list:effu-m2} in Proposition~\ref{prop:effu}). Hence,  at each update of $\hmu_{i, \, \tteff}^{h_{j+1}}$, it contains all the samples of $\hmueff$ and the $2^j$ precedent ones. Thus, because the reward is decreasing, we have $\bmu_{i_j,\tteff}^{h_{j+1}} \geq \bmu_{i_j,\tteff}^{h_{j}} $.  The second inequality uses that $i_j \in \arms_{h_{j+1}}$. Gathering Equations~\ref{eq:eff-general}, \ref{eq:efffewa-3c} and~\ref{eq:efffewa-ist-relation}, we get 
\begin{equation}
\label{eq:efffewa-result}
\bmu^{h}_{i_t}(t-1,\piEF) \geq \min_{j \in \Hitm} \mu_{\ist}(t,\Nitmone)- 4 \sum_{j=0}^{j_h-1} \frac{2^{j}}{2^{j_h}-1} c(2^j, \delta_t).
\end{equation}

With few lines of algebra, we reduce the sum,
\begin{flalign*}
\sum_{j=0}^{j_h-1} \frac{2^{j}}{2^{j_h}-1} c(2^j, \delta_t) &= \sum_{j=0}^{j_h-1} \frac{\sqrt{2}^{j}}{2^{j_h}-1} c(1, \delta_t) && c(2^j, \delta_t) = \frac{c(1,\delta_t)}{\sqrt{2^j}} \\
& = \frac{\sqrt{2}^{j_h} -1 }{\pa{\sqrt{2}-1}\pa{2^{j_h}-1}}c(1, \delta_t) && \sum_{n=0}^N q^n = \frac{q^{N+1}-1}{q-1}\\
& = \frac{1}{\pa{\sqrt{2}-1}\pa{\sqrt{2}^{j_h} +1}} c(1, \delta_t) && 2^{j_h}\!-\!1 \!=\!  \pa{\sqrt{2}^{j_h}\!-\!1}\!\pa{\sqrt{2}^{j_h}\!+\!1}\\
& \leq  \frac{\sqrt{2}}{\pa{\sqrt{2}-1}\sqrt{2^{j_h+1}}} c(1, \delta_t) &&\sqrt{2^{j_h}} +1 \geq \sqrt{2^{j_h}} = \frac{\sqrt{2^{j_h+1}} }{\sqrt{2}} \\
& = \frac{\sqrt{2}}{\sqrt{2}-1} c(2^{j_h+1}, \delta_t) &&  \frac{c(1,\delta_t)}{\sqrt{2^{j_h+1}}} = c(2^{j_h+1},\delta_t)\\
& \leq \frac{\sqrt{2}}{\sqrt{2}-1} c(h, \delta_t). && h \leq 2^{j_h+1} \text{ and }  c(\cdot, \delta) \text{ decreases}
\end{flalign*}

Plugging this last equation in Equations~\ref{eq:effraw-result} and~\ref{eq:efffewa-result} leads to the final result,
\[
\bmu^{h}_{i_t}(t-1,\pi) \geq \min_{j \in \Hitm} \mu_{\ist}(t,\Nitmone)- C_\pi c(h, \delta_t) \quad \text{with } 
\begin{cases}
C_{\piER} = \frac{2\sqrt{2}}{\sqrt{2}-1}\\
C_{\piEF} = \frac{4\sqrt{2}}{\sqrt{2}-1}
\end{cases}\cdot
\]
\end{proof}

Using Proposition~\ref{prop:prb_favorable_event_eff} and Lemma~\ref{lem:core-eff} instead of Prop.~\ref{prop:prb_favorable_event} and Lemmas~\ref{lem:core-FEWA} and~\ref{lem:core-RAWUCB}, we can obtain similar problem dependent and independent bounds than for \FEWA and \RUCB. The proof directly follows from the precedent analysis. 
\begin{restatable}{theorem}{restaalgoindepub}
\label{th:rested-PI}
For any rotting bandit scenario with means $\{\mu_i\}_{i} \in \rewardSet^K$ and any time horizon $T$, $\pi \in \left\{\piER, \piEF \right\}$ run with $\alpha \geq 4$ and $m=2$ suffers an expected regret of
\begin{equation*}
\mathbb{E}[\regret(\pi)] \leq C_\pi\sqrt{2\alpha\sigma^2\log\pa{T}}\pa{\sqrt{KT} +K} + 6KL\,.
\end{equation*}
\end{restatable}
\begin{restatable}{theorem}{restaalgoub}\label{th:rested-PD}
For any rotting bandit scenario with means $\{\mu_i\}_{i} \in \rewardSet^K$ and any time horizon $T$, $\pi \in \left\{\piER, \piEF \right\}$ run with $\alpha \geq 4$ and $m=2$ suffers an expected regret of
\begin{align*}
\mathbb{E}\left[R_T(\pi)\right]  \leq \sum_{i\in \arms} \pa{\frac{2\alpha C_\pi^2\sigma^2\log\pa{T}}{\Delta_{i,\hiT^+-1}} + \sqrt{2\alpha C_\pi^2\sigma^2\log\pa{T}} +6L } \cdot
%\\ 
%\text{\ with  and $\hiT^+$ defined in Equation~\ref{eq:hit+}.}
\end{align*}
\end{restatable}

Among the differences, we notice that our theory holds for a larger range of $\alpha \geq 4$ but the constant $C_\pi$ is $\frac{\sqrt{2}}{\sqrt{2}-1} \sim 3.4$ times larger than their original counter part. We will display multiple experiments where the regret ratio between the algorithms is much smaller than $3.4$. Indeed, to derive Lemma~\ref{lem:core-eff}, we consider that the statistics $\hmueff$ is delayed by at most $h_j$ rounds. According to Proposition~\ref{prop:effu}, the delay is at most $h_j/2$ for $m=2$ (and less than $h_j$ for any $m$). Moreover, these guarantees on the worst case delay are pessimistic: the delay could be much smaller. 

\begin{remark}
\textbf{Can we adapt the theory for $m\neq2$?} %TODO
\end{remark}

\subsection{Experimental Result}





